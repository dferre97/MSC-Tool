There is a wide variety of message-passing communication models, ranging from synchronous "rendez-vous"
communications to fully asynchronous/out-of-order communications. For large-scale distributed systems, the
communication model is determined by the transport layer of the network, and a few classes of 
orders of message delivery (FIFO, causally ordered) have been identified in the early days of 
distributed computing. For local-scale message-passing applications, 
e.g., running on a single machine, the communication model may be determined by the actual implementation of 
message buffers and by how FIFO queues are used. While large-scale communication
models, such as causal ordering, are defined by logical axioms, local-scale models are often defined by an operational
semantics. In this work, we connect these two approaches, and we present a unified hierarchy of communication
models encompassing both large-scale and local-scale models, based on their non-sequential behaviors.
We also show that all the communication models we consider can be axiomatised in the monadic second order logic, 
and may therefore benefit from several bounded verification techniques based on bounded special treewidth.
