\documentclass{article}

% Language setting
% Replace `english' with e.g. `spanish' to change the document language
\usepackage[english]{babel}

% Set page size and margins
% Replace `letterpaper' with `a4paper' for UK/EU standard size
\usepackage[a4paper,top=2cm,bottom=2cm,left=3cm,right=3cm,marginparwidth=1.75cm]{geometry}

% Useful packages
\usepackage{amsmath}
\usepackage{graphicx}
\usepackage[colorlinks=true, allcolors=blue]{hyperref}

\title{Your Paper}
\author{You}

\begin{document}
\maketitle

\begin{abstract}
Your abs
\end{abstract}

\newpage
    
\section{Introduction}
This is the report of my Travail d'étude et de recherche (TER), which was carried out under the supervision of professor Enrico Formenti. The TER has been an opportunity to explore two fields that were almost brand new to me: Cellular Automata (CA) and Elliptic Curve Cryptography (ECC). The goal was to explore and investigate cellular automata and elliptic curves, to see if we could somehow combine the two to design a CA-based ECC algorithm. The interest in cellular automata is mainly due to their ability to exhibit complex behaviours, despite their simple design. Cellular automata have been employed in a plethora of fields, including cryptography. ECC is by far the most popular choice nowadays in public-key cryptography, having replaced RSA in several applications. However, the recommended key size for ECC algorithms is slowly increasing to keep up with technological advances. Finding better alternatives to ECC, or improving it, can be an important step for the future of public-key cryptography. \\

\section{Elliptic curve cryptography}
\label{sec:ecc}
    The introduction of asymmetric cryptography was a major milestone in the history of cryptography. The focus shifted from finding a way to exchange a common key to ensuring a secure communication between two parties, without worrying about a man in the middle. The idea of asymmetric cryptography revolves around pairs of public-private keys and problems involving \emph{one-way functions}, i.e. functions that are easy to compute, but hard to inverse.\\
    
    RSA is one of the oldest and most popular public-key cryptosystems, and it is based on the difficulty of factorizing big numbers into two primes. Over the years, the size of an RSA private key gradually increased, to ensure a proper level of security. Since 2015, NIST recommends a minimum of 2048-bit keys for RSA. Because of this, researchers explored alternative public-key cryptosystems, that should guarantee a similar level of security to RSA, but with smaller keys. Elliptic curve cryptography (ECC) was the most successful proposal, and it has gradually replaced RSA in many applications. The content of this section was mainly inspired by.

\end{document}