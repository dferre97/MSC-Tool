\newcommand{\cmark}{\ding{51}}%
\newcommand{\xmark}{\ding{55}}%
\begin{figure}[t]
		\begin{tabular}{| c | c | c|  c| c| }
			\hline
			& Weakly  & Weakly  & $\exists$k & $\forall$k  \\
			& sync & k-sync & bounded & bounded \\
			\hline \hline
			$\asy$ &  \xmark & \cmark & \cmark & \cmark \\
			\hline
			$\oneone$  & \xmark~[1] & \cmark~[1] & \cmark~[1] & \cmark~[1] \\
			\hline
			$\co$  & \xmark & \cmark & \cmark & \cmark \\
			\hline
			$\none$ & \cmark~[1] & \cmark~[1] & \cmark~[1] & \cmark~[1] \\
			\hline
			$\onen$ & \cmark & \cmark & \cmark & \cmark \\
			\hline
			$\nn$ & \cmark & \cmark & \cmark & \cmark \\
			\hline
		%	$\rsc$ & \cmark & \cmark & \cmark & \cmark & \cmark \\
		%	\hline
		\end{tabular}
		\caption{Table summarising the (un)decidability results for the synchronisability problems (each 
		combination of a communication model $\comsymb$ and a class $\aMSCclass$ of MSCs is a different 
		synchronisability problem). 
		The symbol \xmark\;stands for undecidability and unbounded special tree-width
		of $\comsymb\cap \aMSCclass$, whereas \cmark\;stands for decidability and bounded STW
		of $\comsymb\cap \aMSCclass$.  
		[1] indicates that the result was shown by Bollig~\emph{et al}~\cite{DBLP:conf/concur/BolligGFLLS21}.}
		\label{fig:stw-bound}
\end{figure}
