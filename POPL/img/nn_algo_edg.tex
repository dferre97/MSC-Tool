\begin{example}
	Fig.~\ref{fig:nn_algo_ex} shows an example of $\nn$ MSC and its EDG. We use it to show how the algorithm that builds a $\nn$ linearization works. Please note that, for convenience, not all the edges of the EDG have been drawn, but those missing would only connect events for which there is already a path is our drawing; these edges do not have any impact on the execution of the algorithm. We start by applying step 1 on the event $!5$, which has in-degree 0. The algorithm starts to build a linearization using $!5$ as the first event, and all the outgoing edges of $!5$ are removed from the EDG, along with the event itself. Now, $!1$ has in-degree 0 and we can apply again step 1. The partial linearization becomes $!5\;!1$. Similarly, we can then apply step 1 on $!2$ and $!3$ to get the partial linearization $!5\;!1\;!2\;!3$. At this point, step 1 and 2 cannot be applied, but we can use step 3 on $?5$, which gets added to linearization. We then apply step 3 also to $?1$ and $?2$, followed by step 1 on $!4$, step 2 on $!6$ (which is an unmatched send event), and step 3 on $?3$ and $?4$. Finally, all the events of the MSC have been added to our linearization, which is $!5\;!1\;!2\;!3\;?5\;?1\;?2\;!4\;!6\;?3\;?4$. Note that this is a $\nn$ linearization.
\end{example}

\begin{figure}[t] 
\begin{subfigure}[c]{0.45\textwidth}\centering
\begin{tikzpicture}[ scale = .7,every node/.style={transform shape}]
\newproc{0}{p}{-3.3}; 
\newproc{1.2}{q}{-3.3}; 
\newproc{2.4}{r}{-3.3}; 
\newproc{3.6}{s}{-3.3}; 
\newproc{4.8}{t}{-3.3};

\newmsgm{0}{2.4}{-0.6}{-2.4}{5}{0.8}{black};
\newmsgm{0}{1.2}{-1}{-1}{1}{0.65}{black};
\newmsgm{2.4}{1.2}{-1.4}{-1.4}{2}{0.5}{black};
\newmsgm{2.4}{3.6}{-1.8}{-1.8}{3}{0.5}{black};
\newmsgm{1.2}{3.6}{-2.8}{-2.8}{4}{0.2}{black};
\newmsgum{3.6}{4.8}{-1}{6}{0.5}{black}; 

\end{tikzpicture} 
\end{subfigure} 
\hfill
\begin{subfigure}[c]{0.45\textwidth}\centering
	\begin{tikzpicture}[scale = .6,every node/.style={transform shape}] 
\begin{scope} 
\node (r4) at (0,0) {$?4$};
     \node (s4) at (-1,-1) {$!4$};
     \node (r3) at (1,-1) {$?3$};
     \node (r2) at (-1,-2) {$?2$};
     \node (s3) at (1,-3) {$!3$};
     \node (s6) at (2,-2) {$!6$};
     \node (r1) at (-1,-3) {$?1$};
     \node (s1) at (-1,-4) {$!1$};
     \node (r5) at (-2,-4) {$?5$};
     \node (s5) at (-1,-5) {$!5$};
     \node (s2) at (-2,-6) {$!2$};	

     \draw[->] (s2) -- (r5); 
     \draw[->] (s5) -- (r5);
     \draw[->] (s5) -- (s1);
     \draw[->] (s1) -- (r1); 
     \draw[->, color = blue] (r5) -- (r1);
     \draw[->] (r1) -- (r2);
     \draw[->] (r2) -- (s4);
     \draw[->] (s4) -- (r4);
     \draw[->, color = Green, dashed] (s3) -- (s6);
     \draw[->] (s3) -- (r3);
     \draw[->] (s6) -- (r3); 
     \draw[->, color = red] (r3) -- (r4);
     \draw[->, color = blue ] (r2) -- (r3); 
 
     \draw[->] (s2) to[bend left= 40] (r2);  
     \draw[->] (s2) to[bend right= 70] (s3);
     \draw[->, color = Green, dashed] (s2) to[bend right= 70] (s6);
     \draw[->, color = red] (s1) to[bend left= 60] (s2); 
     \draw[->, color = Green, dashed] (s1) to[bend right= 40] (s6); 
     \draw[->, color = Green, dashed] (s4) to[bend right= 30] (s6); 
\end{scope} 
     \begin{scope}[shift = {(2.5,-4.5)}]
 	     \draw[->] (0,0) -- (.8,0); 
 	     \draw[->, color = blue] (0,-.5) -- (.8,-.5); 
 	     \draw[->, color = red] (0,-1) -- (.8,-1); 
 	     \draw[->, color = Green, dashed] (0,-1.5) -- (.8,-1.5); 
	\node[right] (1) at (.8,0)  {$= (\procrel / \lhd)$}; 
	\node[right] (2) at (.8,-.5) {$= (\onenrel)$}; 
	\node[right] (3) at (.8, -1) {$= (\mbrel)$}; 
	\node[right] (4) at (.8,-1.5) {$=$ (other edges)};
     \end{scope} 
\end{tikzpicture}
\end{subfigure}
    \caption{An MSC and its EDG. In the EDG, only meaningful edges are shown.}
    \label{fig:nn_algo_ex}
\end{figure} 

