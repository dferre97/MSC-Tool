\subsection{MSO-definable properties}

In this sections we give MSO formulas for some MSO-definable properties that are used throughout the paper.

\paragraph*{Transitive Closure}
Given a binary relation $\to$, we can express its reflexive transitive closure $\to^*$ in MSO as
\[
x \to^* y = \forall X.(x \in X \;\wedge\; forward\_closed(X)) \implies y \in X
\]
\[
forward\_closed(X) = \forall z.\forall t.(z \in X \;\wedge\; z \to t) \implies t \in X
\]
The (non-reflexive) transitive closure of $\to$ can be simply expressed as
\[
x \to^+ y = x \to^* y \;\wedge\; x \neq y
\]

\paragraph*{Acyclicity} 

Given a binary relation $\to$, we can use MSO to express the acyclicity of $\to$. Recall that, given a binary relation $\to$, it is acyclic if and only if its transitive closure $\to^+$ is antisymmetric. The MSO formula of acyclicity directly follows from this definition:
\[
\Phi_{acyclic} =  \neg \exists x.(x \to^+ y \;\wedge\; y \to^+ x).   
\]
or, equivalently:
\[
\Phi_{acyclic} =  \neg \exists x.(x \to^+ x).   
\]

\subsection{Missing proofs}

We show here that the two alternative definitions of $\none$ MSC that we gave are equivalent.

\begin{proposition}
    Definition~\ref{def:mb_msc} and Definition~\ref{def:n_one_alt} of $\none$ MSC are equivalent.
\end{proposition}
\begin{proof}
    ($\Rightarrow$)  We show that if $\msc$ is a $\none$ MSC, according to Definition~\ref{def:n_one_alt}, then it is also a $\none$ MSC, according to Definition~\ref{def:mb_msc}. By definition of $\mbpartial$, we must have 
    \begin{enumerate*}[label={(\roman*)}]
        \item $s \mbpartial s'$ for any two matched send events $s$ and $s'$ addressed to the same process, such that $r \procrel^+ r$, where $s \lhd r$ and $s' \lhd r'$, and
        \item $s \mbpartial s'$, if $s$ and $s'$ are a matched and an unmatched send event, respectively.
    \end{enumerate*} 
    If $\mbpartial$ is a partial order, we can find at least one linearization $\linrel$ such that $\mbpartial \;\subseteq\; \linrel$; such a linearization satisfies the conditions of Definition~\ref{def:mb_msc}.\newline
    ($\Leftarrow$) We show that if $\msc$ is not a $\none$ MSC, according to Definition~\ref{def:n_one_alt}, then it is also not a $\none$ MSC, according to Definition~\ref{def:mb_msc}. Since ${\mbpartial} = ({\procrel} \,\cup\, {\lhd} \,\cup\, {\mbrel})^\ast$ is not a partial order, $\mbpartial$ must be cyclic\footnote{$\mbpartial$ is reflexive and transitive by definition, if it were also acyclic it would be a partial order}. If $\mbpartial$ is cyclic, it means that we cannot find a linearization $\linrel$ such that $\mbpartial \;\subseteq\; \linrel$. In other words, we cannot find a linearization where      
    \begin{enumerate*}[label={(\roman*)}]
        \item $s \linrel s'$ for any two matched send events $s$ and $s'$ addressed to the same process, such that $r \procrel^+ r$, where $s \lhd r$ and $s' \lhd r'$, and
        \item $s \linrel s'$, if $s$ and $s'$ are a matched and an unmatched send event, respectively.
    \end{enumerate*} 
    It follows that $\msc$ is not a $\none$ MSC also according to Definition~\ref{def:mb_msc}.
\end{proof}

We show here that the two alternative definitions of $\onen$ MSC that we gave are equivalent.

\begin{proposition}
    Definition~\ref{def:one_n} and Definition~\ref{def:one_n_alt} of $\onen$ MSC are equivalent.
\end{proposition}
\begin{proof}
    ($\Rightarrow$)  We show that if $\msc$ is a $\onen$ MSC, according to Definition~\ref{def:one_n_alt}, then it is also a $\onen$ MSC, according to Definition~\ref{def:one_n}. By definition of $\onenpartial$, we must have 
    \begin{enumerate*}[label={(\roman*)}]
        \item $r \onenpartial r'$ for any two receive events $r$ and $r'$ whose matched send events $s$ and $s'$ are such that $s \procrel^+ s'$, and
        \item $s \onenpartial s'$, if $s$ and $s'$ are a matched and an unmatched send event executed by the same process, respectively.
    \end{enumerate*} 
    If $\onenpartial$ is a partial order, we can find at least one linearization $\linrel$ such that $\onenpartial \;\subseteq\; \linrel$; such a linearization satisfies the conditions of Definition~\ref{def:one_n}.\newline
    ($\Leftarrow$) We show that if $\msc$ is not a $\onen$ MSC, according to Definition~\ref{def:one_n_alt}, then it is also not a $\onen$ MSC, according to Definition~\ref{def:one_n}. Since ${\onenpartial} = ({\procrel} \,\cup\, {\lhd} \,\cup\, {\onenrel})^\ast$ is not a partial order, $\onenpartial$ must be cyclic. If $\onenpartial$ is cyclic, it means that we cannot find a linearization $\linrel$ such that $\onenpartial \;\subseteq\; \linrel$. In other words, we cannot find a linearization where      
    \begin{enumerate*}[label={(\roman*)}]
        \item $r \linrel r'$ for any two receive events $r$ and $r'$ whose matched send events $s$ and $s'$ are such that $s \procrel^+ s'$, and
        \item $s \linrel s'$, if $s$ and $s'$ are a matched and an unmatched send event executed by the same process, respectively.
    \end{enumerate*} 
    It follows that $\msc$ is not a $\onen$ MSC also according to Definition~\ref{def:one_n}.
\end{proof}