% !TEX root = ../popl-paper.tex

\subsection{MSO-definable properties}

In this sections we give MSO formulas for some MSO-definable properties that are used throughout the paper.

\paragraph*{Transitive Closure}
Given a binary relation $\abinrel$, we can express its reflexive transitive closure $\abinrel^*$ in MSO as
\[
x \abinrel^* y = \forall X.(x \in X \;\wedge\; forward\_closed(X)) \implies y \in X
\]
\[
forward\_closed(X) = \forall z.\forall t.(z \in X \;\wedge\; z \abinrel t) \implies t \in X
\]
The transitive (but not necessarily reflexive) 
closure of $\abinrel$ can also be expressed as
\[
    x \abinrel^+ y = \forall X.\ \big(
        \forall z,t\ (z\in X\cup\{x\}\wedge z \abinrel t)\implies t \in X\big) \implies y\in X
\]
        
\paragraph*{Acyclicity} 

Given a binary relation $\abinrel$, we can use MSO to express the 
acyclicity of $\abinrel$,
or equivalently, the fact that its transitive closure
$R^+$ is irreflexive.
\[
\Phi_{acyclic} =  \neg \exists x.(x \abinrel^+ x).   
\]

\subsection{Omitted proofs of Section~\ref{sec:MSO}}\label{app:sec-mso}

\paragraph*{\bf Mailbox}

We show here that the two alternative definitions of $\none$-MSC that we gave are equivalent.

\begin{proposition}
    Definition~\ref{def:mb_msc} and Definition~\ref{def:n_one_alt} of $\none$-MSC are equivalent.
\end{proposition}
\begin{proof}
    ($\Rightarrow$)  We show that if $\msc$ is a $\none$-MSC, according to Definition~\ref{def:n_one_alt}, then it is also a $\none$-MSC, according to Definition~\ref{def:mb_msc}. By definition of $\mbpartial$, we must have 
    \begin{enumerate*}[label={(\roman*)}]
        \item $s \mbpartial s'$ for any two matched send events $s$ and $s'$ addressed to the same process, such that $r \procrel^+ r$, where $s \lhd r$ and $s' \lhd r'$, and
        \item $s \mbpartial s'$, if $s$ and $s'$ are a matched and an unmatched send event, respectively.
    \end{enumerate*} 
    If $\mbpartial$ is a partial order, we can find at least one linearization $\linrel$ such that $\mbpartial \;\subseteq\; \linrel$; such a linearization satisfies the conditions of Definition~\ref{def:mb_msc}.\newline
    ($\Leftarrow$) We show that if $\msc$ is not a $\none$-MSC, according to Definition~\ref{def:n_one_alt}, then it is also not a $\none$-MSC, according to Definition~\ref{def:mb_msc}. Since ${\mbpartial} = ({\procrel} \,\cup\, {\lhd} \,\cup\, {\mbrel})^\ast$ is not a partial order, $\mbpartial$ must be cyclic\footnote{$\mbpartial$ is reflexive and transitive by definition, if it were also acyclic it would be a partial order}. If $\mbpartial$ is cyclic, it means that we cannot find a linearization $\linrel$ such that $\mbpartial \;\subseteq\; \linrel$. In other words, we cannot find a linearization where      
    \begin{enumerate*}[label={(\roman*)}]
        \item $s \linrel s'$ for any two matched send events $s$ and $s'$ addressed to the same process, such that $r \procrel^+ r$, where $s \lhd r$ and $s' \lhd r'$, and
        \item $s \linrel s'$, if $s$ and $s'$ are a matched and an unmatched send event, respectively.
    \end{enumerate*} 
    It follows that $\msc$ is not a $\none$-MSC also according to Definition~\ref{def:mb_msc}.
\end{proof}

\paragraph*{\bf $\onen$}

We show  that the two alternative definitions of $\onen$ MSC that we gave are equivalent.

\begin{proposition}
    Definition~\ref{def:one_n} and Definition~\ref{def:one_n_alt} of $\onen$ MSC are equivalent.
\end{proposition}
\begin{proof}
    ($\Rightarrow$)  We show that if $\msc$ is a $\onen$ MSC, according to Definition~\ref{def:one_n_alt}, then it is also a $\onen$ MSC, according to Definition~\ref{def:one_n}. By definition of $\onenpartial$, we must have 
    \begin{enumerate*}[label={(\roman*)}]
        \item $r \onenpartial r'$ for any two receive events $r$ and $r'$ whose matched send events $s$ and $s'$ are such that $s \procrel^+ s'$, and
        \item $s \onenpartial s'$, if $s$ and $s'$ are a matched and an unmatched send event executed by the same process, respectively.
    \end{enumerate*} 
    If $\onenpartial$ is a partial order, we can find at least one linearization $\linrel$ such that $\onenpartial \;\subseteq\; \linrel$; such a linearization satisfies the conditions of Definition~\ref{def:one_n}.\newline
    ($\Leftarrow$) We show that if $\msc$ is not a $\onen$ MSC, according to Definition~\ref{def:one_n_alt}, then it is also not a $\onen$ MSC, according to Definition~\ref{def:one_n}. Since ${\onenpartial} = ({\procrel} \,\cup\, {\lhd} \,\cup\, {\onenrel})^\ast$ is not a partial order, $\onenpartial$ must be cyclic. If $\onenpartial$ is cyclic, it means that we cannot find a linearization $\linrel$ such that $\onenpartial \;\subseteq\; \linrel$. In other words, we cannot find a linearization where      
    \begin{enumerate*}[label={(\roman*)}]
        \item $r \linrel r'$ for any two receive events $r$ and $r'$ whose matched send events $s$ and $s'$ are such that $s \procrel^+ s'$, and
        \item $s \linrel s'$, if $s$ and $s'$ are a matched and an unmatched send event executed by the same process, respectively.
    \end{enumerate*} 
    It follows that $\msc$ is not a $\onen$ MSC also according to Definition~\ref{def:one_n}.
\end{proof}

\paragraph*{\bf $\nn$}

We show here the missing proofs for the equivalence of the two definitions of $\nn$ MSC that we gave.

\begin{proposition}\label{prop:nn_first_prop}
	Let $\msc$ be an MSC. Given two matched send events $s_1$ and $s_2$, and their respective receive events $r_1$ and $r_2$, $r_1 \nnbowtie r_2 \implies s_1 \nnbowtie s_2$.
\end{proposition}
\begin{proof}
    Follows from the definition of $\nnbowtie$. We have $r_1 \nnbowtie r_2$ if either:
    \begin{itemize}%\itemsep=0.5ex
        \item $r_1 \nnrel r_2$. Two cases: either \begin{enumerate*}[label={(\roman*)}]
            \item $s_1 \nnrel s_2$, or 
            \item $s_1 \notnnrel s_2$.
        \end{enumerate*}
        The first case clearly implies $s_1 \nnbowtie s_2$, for rule 1 in the definition of $\nnbowtie$. The second too, because of rule 3.
        \item  $r_1 \notnnrel r_2$, but $r_1 \nnbowtie r_2$. This is only possible if rule 2 in the definition of $\nnbowtie$ was used, which implies $s_1 \nnrel s_2$ and, for rule 1, $s_1 \nnbowtie s_2$.
    \end{itemize}
\end{proof}

\nnsecondprop*
\begin{proof}
    According to Definition~\ref{def:n_n}, an MSC is $\nn$ if it has at least one $\nn$ linearization. Note that, because of how it is defined, any $\nn$ linearization is always both a $\none$ and a $\onen$ linearization. It follows that the cyclicity of $\nnrel$ (not $\nnbowtie$) implies that $\msc$ is not $\nn$, because it means that we are not even able to find a linearization that is both $\none$ and $\onen$. Moreover, since in a $\nn$ linearization the order in which messages are sent matches the order in which they are received, and unmatched send events can be executed only after matched send events, a $\nn$ MSC always has to satisfy the constraints imposed by the $\nnbowtie$ relation. If $\nnbowtie$ is cyclic, then for sure there is no $\nn$ linearization for $\msc$.
 %   \davide{This proof is not very formal... do you think it is fine?}
\end{proof}    

\nnalgotermination*
\begin{proof}
We want to prove that, if $\nnbowtie$ is acyclic, step 4 of the algorithm is never executed, i.e. it terminates correctly. Note that the acyclicity of $\nnbowtie$ implies that the EDG of $\msc$ is a DAG. Moreover, at every step of the algorithm we remove nodes and edges from the EDG, so it still remains a DAG. The proof proceeds by induction on the number of events added to the linearization.\newline
Base case: no event has been added to the linearization yet. Since the EDG is a DAG, there must be an event with in-degree 0. In particular, this has to be a send event (a receive event depends on its respective send event, so it cannot have in-degree 0). If it is a matched send event, step 1 is applied. If there are no matched send events, step 2 is applied on an unmatched send. We show that it is impossible to have an unmatched send event of in-degree 0 if there are still matched send events in the EDG, so either step 1 or 2 are applied in the base case. Let $s$ be one of those matched send events and let $u$ be an unmatched send. Because of rule 4 in the definition of $\nnbowtie$, we have that $s \nnbowtie u$, which implies that $u$ cannot have in-degree 0 if $s$ is still in the EDG.\newline
Inductive step: we want to show that we are never going to execute step 4. In particular, Step 4 is executed when none of the first three steps can be applied. This happens when there are no matched send events with in-degree 0 and one of the following holds:
\begin{itemize}%\itemsep=0.5ex
	\item \emph{There are still matched send events in the EDG with in-degree $>0$, there are no unmatched messages with in-degree 0, and there is no receive event $r$ with in-degree 0 in the EDG, such that $r$ is the receive event of the first message whose sent event was already added to the linearization}. Since the EDG is a DAG, there must be at least one receive event with in-degree 0. We want to show that, between these receive events with in-degree 0, there is also the receive event $r$ of the first message whose send event was added to the linearization, so that we can apply step 3 and step 4 is not executed. Suppose, by contradiction, that $r$ has in-degree $>0$, so it depends on other events. For any maximal chain in the EDG that contains one of these events, consider the first event $e$, which clearly has in-degree 0. In particular, $e$ cannot be a send event, because we would have applied step 1 or step 2. Hence, $e$ can only be a receive event for a send event that was not the first added to the linearization (and whose respective receive still has not been added). However, this is also impossible, since $r_e \nnbowtie r$ implies $s_e \nnbowtie s$, according to Proposition~\ref{prop:nn_first_prop}, and we could not have added $s$ to the linearization before $s_e$. Because we got to a contradiction, the hypothesis that $r$ has in-degree $>0$ must be false, and we can indeed apply step 3.
	\item \emph{There are still matched send events in the EDG with in-degree $>0$, there is at least one unmatched message with in-degree 0, and there is no receive event $r$ with in-degree 0 in the EDG, such that $r$ is the receive event of the first message whose sent event was already added to the linearization}. We show that it is impossible to have an unmatched send event of in-degree 0 if there are still matched send events in the EDG. Let $s$ be one of those matched send events and let $u$ be an unmatched send. Because of rule 4 in the definition of $\nnbowtie$, we have that $s \nnbowtie u$, which implies that $u$ cannot have in-degree 0 if $s$ is still in the EDG.
	\item \emph{There are no more matched send events in the EDG, there are no unmatched messages with in-degree 0, and there is no receive event $r$ with in-degree 0 in the EDG, such that $r$ is the receive event of the first message whose sent event was already added to the linearization}. Very similar to the first case. Since the EDG is a DAG, there must be at least one receive event with in-degree 0. We want to show that, between these receive events with in-degree 0, there is also the receive event $r$ of the first message whose send event was added to the linearization, so that we can apply step 3 and step 4 is not executed. Suppose, by contradiction, that $r$ has in-degree $>0$, so it depends on other events. For any maximal chain in the EDG that contains one of these events, consider the first event $e$, which clearly has in-degree 0. In particular, $e$ cannot be a send event, because by hypothesis there are no more send events with in-degree 0 in the EDG. Hence, $e$ can only be a receive event for a send event that was not the first added to the linearization (and whose respective receive still has not been added). However, this is also impossible, since $r_e \nnbowtie r$ implies $s_e \nnbowtie s$ (see Proposition~\ref{prop:nn_first_prop}), and we could not have added $s$ to the linearization before $s_e$. Because we got to a contradiction, the hypothesis that $r$ has in-degree $>0$ must be false, and we can indeed apply step 3.
\end{itemize}
We showed that, if $\nnbowtie$ is acyclic, the algorithm always terminates correctly and computes a valid $\nn$ linearization.
\end{proof}
