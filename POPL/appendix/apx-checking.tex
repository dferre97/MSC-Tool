\subsection{Communicating finite state machines\label{app:cfsm}}
We now recall the definition of communicating systems (aka communicating finite-state
machines or message-passing automata), which consist of finite-state machines $A_p$
(one for every process $p \in \Procs$) that can communicate through channels from $\Ch$.

\begin{definition}\label{def:cs}
A \emph{system of communicating finite state machines} over the set $\Procs$ of rocesses
and the set $\Msg$ of messages is a tuple
   $ \Sys = (A_p)_{p\in\procSet}$. For each
   $p \in \Procs$, $A_p = (Loc_p, \delta_p, \ell^0_p)$ is a finite transition system where
   $\Loc_p$ is a finite set of local (control) states, $\delta_p
   \subseteq \Loc_p \times \pAct{p} \times \Loc_p$ is the
   transition relation, and $\ell^0_p \in Loc_p$ is the initial state.
\end{definition}

Given $p \in \Procs$ and a transition $t = (\ell,a,\ell') \in \delta_p$, we let
$\tsource(t) = \ell$, $\ttarget(t) = \ell'$, $\tlabel(t) = a$, and
$\tmessage(t) = \msg$ if $a \in \msAct{\msg} \cup \mrAct{\msg}$.

% \smallskip

% There are in general two ways to define the semantics of a communicating system.
% Most often it is defined as a global infinite transition system that keeps track
% of the various local control states and all (unbounded) channel contents.
% As, in this paper, our arguments are based on a graph view of MSCs, we will define
% the language of $\Sys$ directly as a set of MSCs. These two semantic views are essentially
% equivalent, but they have different advantages depending on the context.
% We refer to \cite{CyriacG14} for a thorough discussion.

% \davidequestion{MSCs are formally defined in chapter 5... maybe move here in the preliminaries just the asynchronous definition?}

Let $\msc = (\Events,\procrel,\lhd,\lambda)$ be an MSC.
A \emph{run} of $\Sys$ on $\msc$ is a mapping
$\rho: \Events \to \bigcup_{p \in \Procs} \delta_p$
that assigns to every event $e$ the transition $\rho(e)$
that is executed at $e$. Thus, we require that
\begin{enumerate*}[label={(\roman*)}]
\item for all $e \in \Events$, we have $\tlabel(\rho(e)) = \lambda(e)$,
\item for all $(e,f) \in {\procrel}$, $\ttarget(\rho(e)) = \tsource(\rho(f))$,
\item for all $(e,f) \in {\lhd}$, $\tmessage(\rho(e)) = \tmessage(\rho(f))$,
and
\item for all $p \in \Procs$ and $e \in \Events_p$ such that there is no $f \in \Events$ with $f \procrel e$, we have $\tsource(\rho(e)) = \ell_p^0$.
\end{enumerate*}

We write $L_{\asy}(\Sys)$ to denote the set of MSCs $\msc$ that admit a run of $\Sys$.
Intuitively, $L_{\asy}(\Sys)$ is the set of all asynchronous behaviours of $\Sys$.

% Letting run $\Sys$ directly on MSCs is actually very convenient.
% This allows us to associate with $\Sys$ its p2p language and mailbox language
% in one go. The \emph{\pp language} of $\Sys$ is $\ppL{\Sys} = \{\msc \in \ppMSCs \mid$ there is a run of $\Sys$ on $\msc\}$.
% The \emph{mailbox language} of $\Sys$ is $\mbL{\Sys} = \{\msc \in \mbMSCs \mid$ there is a run of $\Sys$ on $\msc\}$.
%
% Note that, following \cite{DBLP:conf/cav/BouajjaniEJQ18,DBLP:conf/fossacs/GiustoLL20},
% we do not consider final states or final configurations, as our purpose is to
% reason about all possible
% traces that can be \emph{generated} by $\Sys$.
% %We will discuss this issue in more detail later in the paper. \todo{Do we still discuss this in the conference version or can we omit this sentence?}

\subsection{Conflict Graph}\label{app:conflict-graph}

\etienne{Copy-paste from a file, not sure how it fits with the rest of the appendix yet.}
\davidequestion{The conflict graph is needed only for some proofs on the cuncur paper. I think it makes sense not to include it here and eventually introduce it later in the MSO-treewidth chapter if we really need it.}

We now recall the notion of a conflict graph associated to an MSC defined in \cite{DBLP:conf/cav/BouajjaniEJQ18}. This graph is used to depict the causal dependencies between message exchanges.  Intuitively, we have a dependency whenever
two messages have a process in common. For instance, an $\xrightarrow{SS}$
dependency between message exchanges $v$ and $v'$ expresses the fact that
$v'$ has been sent after $v$, by the same process. This notion is of interest because it was seen in \cite{DBLP:conf/cav/BouajjaniEJQ18} that the notion of synchronizability in MSCs (which is studied in this paper) can be graphically characterized by the nature of the associated conflict graph.
It is defined in terms of linearizations
in \cite{DBLP:conf/fossacs/GiustoLL20}, but we equivalently express it
directly in terms of MSCs.

For an MSC $\msc = (\Events, \rightarrow, \lhd, \lambda)$ and
$e \in \Events$, we define the type $\type(e) \in \{\stype,\rtype\}$ of $e$ by $\type(e) = \stype$ if $e \in \SendEv{\msc}$
and $\type(e) = \rtype$ if $e \in \RecEv{\msc}$.
Moreover, for $e \in \Unm{\msc}$, we let $\mexch(e) = e$,
and for $(e,e') \in \lhd$, we let $\mexch(e) = \mexch(e') = (e,e')$.


\begin{definition}[Conflict graph]
	The \emph{conflict graph} $\cgraph{\msc}$ of an MSC $\msc = (\Events, \rightarrow, \lhd, \lambda)$ is the labeled graph $(\Nodes, \Edges)$, with $\Edges \subseteq \Nodes \times \{\stype,\rtype\}^2 \times \Nodes$, defined by
	$\Nodes = {\lhd} \cup \Unm{\msc}$ and $\Edges = \{(\mu(e),\type(e)\type(f),\mu(f)) \mid (e,f) \in {\to^+}\}$.
In particular, a node of $\cgraph{\msc}$ is either a single unmatched send event or a message pair $(e,e') \in {\lhd}$.
\end{definition}


\subsection{Proof of Theorem~\ref{thm:sync}}
\label{apx:sync}
In order to prove Theorem \ref{thm:sync}, we first need to introduce some concepts and give preliminary proofs. 

\begin{definition}[Prefix]
	Let $\msc = (\Events,\procrel,\lhd,\lambda) \in \MSCs$ and consider
	$E \subseteq \Events$ such that $E$ is ${\le}$-\emph{downward-closed}, i.e,
	for all $(e,f) \in {\le}$ such that $f \in E$, we also have $e \in E$.
	Then, the MSC $M' = (E,{\procrel} \cap (E \times E),{\lhd} \cap (E \times E),\lambda')$,
	where $\lambda'$ is the restriction of $\Events$ to $E$, is called a \emph{prefix}
	of $\msc$. 	
\end{definition}

If we consider a set $E$ that is ${\onenpartial}$-\emph{downward-closed}, we call $M'$ a \emph{$\onen$ prefix}.
If the set $E$ is ${\bowtie}$-\emph{downward-closed}, we call $M'$ a \emph{$\nn$ prefix}. Note that every $\onen$ or $\nn$ prefix is also a prefix, since $\le \subseteq {\onenpartial}$ and $\le \subseteq {\bowtie}$.

Note that the empty MSC is a prefix of $\msc$.
We denote the set of prefixes of $\msc$ by $\Pref{\msc}$, whereas $\Prefonen{\msc}$ and $\Prefnn{\msc}$ are used for the $\onen$ and the $\nn$ variants, respectively.
This is extended to sets $L \subseteq \MSCs$ as expected, letting
$\Pref{L} = \bigcup_{\msc \in L} \Pref{\msc}$.

\begin{proposition}
	\label{prop:prefixes}
	For $\comsymb \in \{\asy, \oneone, \co, \none\}$, every prefix of a $\comsymb$ MSC is a $\comsymb$ MSC.
\end{proposition}
\begin{proof}
    For $\comsymb = \asy$ it is true by definition. For $\comsymb = \{\oneone, \none\}$ it was already shown to be true in \cite{BolligFG21}, so we just consider $\comsymb = \co$. Let $\msc = (\Events, \procrel, \lhd, \lambda) \in \coMSCs$ and let $\msc_0 =
    (\Events_0, \procrel_0, \lhd_0, \lambda_0)$ be a prefix of $\msc$. By contradiction, suppose that $\msc_0$ is not a	causally ordered MSC. There must be two distinct $s,s' \in \Events_0$ such that $\lambda(s)=\pqsAct{\plh}{q}$, $\lambda(s')=\pqsAct{\plh}{q}$, $s \le_{\msc_0} s'$ and either
    \begin{enumerate*}[label={(\roman*)}]
        \item $r' \procrel^+ r$, where $r$ and $r'$ are two receive events such that $s \lhd r$ and $s' \lhd r'$, or
        \item $s \in \Unm{\msc_0}$ and $s' \in \Matched{\msc_0}$.
    \end{enumerate*}
    In both cases, $\msc$ would also not be a causally ordered $\MSCs$, since $\Events_0 \subseteq \Events$, ${\rightarrow_0} \subseteq {\rightarrow}$, and ${\lhd_0} \subseteq {\lhd}$. This is a contradiction, thus $\msc_0$ has to be causally ordered.
\end{proof}

Note that this proposition is not true for the $\onen$ and the $\nn$ communication models. Fig.\ref{fig:onen-prefix} shows an example of $\nn$ MSC with a prefix that is neither a $\nn$ MSC nor a $\onen$ MSC.

\begin{figure}[t]
	\captionsetup[subfigure]{justification=centering}
% \centering
\begin{subfigure}[t]{0.45\textwidth}\centering
	\begin{tikzpicture}[scale=0.7, every node/.style={transform shape}]
		\newproc{0}{p}{-1.5};
		\newproc{1}{q}{-1.5};
		\newproc{2}{r}{-1.5};

		\newmsgm{0}{1}{-0.5}{-0.5}{1}{0.3}{black};
		\newmsgm{0}{2}{-1.0}{-1.0}{2}{0.7}{black};

	\end{tikzpicture}
	\caption{A $\nn$ MSC $\msc$.}
\end{subfigure}
% \hfill
\begin{subfigure}[t]{0.45\textwidth}\centering
	\begin{tikzpicture}[scale=0.7, every node/.style={transform shape}]
		\newproc{0}{p}{-1.5};
		\newproc{1}{q}{-1.5};
		\newproc{2}{r}{-1.5};

		\newmsgum{0}{1}{-0.5}{1}{0.3}{black};
		\newmsgm{0}{2}{-1.0}{-1.0}{2}{0.7}{black};

	\end{tikzpicture}
	\caption{A prefix of $\msc$.}
\end{subfigure}
	\caption{A $\nn$ MSC with a prefix that is neither $\onen$ nor $\nn$.}
	\label{fig:onen-prefix}
\end{figure}

\begin{proposition}
	\label{prop:prefixes-onen}
	Every $\onen$ prefix of a $\onen$ MSC is a $\onen$ MSC.
\end{proposition}
\begin{proof}
    Let $\msc = (\Events, \procrel, \lhd, \lambda) \in \onenMSCs$ and let $\msc_0 =
    (\Events_0, \procrel_0, \lhd_0, \lambda_0)$ be a $\onen$ prefix of $\msc$, where $\Events_0 \subseteq \Events$. Firstly, the $\onenpartial$-downward-closeness of $\Events_0$ guarantees that ${\msc_0}$ is still an MSC. We need to prove that it is a $\onen$ MSC. By contradiction, suppose that $\msc_0$ is not a $\onen$ MSC. Then, there are distinct $e,f \in \Events_0$ such that $e \onenpartial_{\msc_0} f \onenpartial_{\msc_0} e$, where $\onenpartial_{\msc_0} = (\procrel_0 \cup \lhd_0 \cup \onenrel_{\msc_0})^\ast$. As $\Events_0 \subseteq \Events$, we have that ${\rightarrow_0} \subseteq {\rightarrow}$, ${\lhd_0} \subseteq {\lhd}$, ${\onenrel_{\msc_0}} \subseteq {\onenrel}$. Clearly, $\onenpartial_{\msc_0} \subseteq \onenpartial$, so $e \onenpartial f \onenpartial e$. This implies that $\msc$ is not a $\onen$ MSC, because $\onenpartial$ is cyclic, which is a contradiction. Hence $\msc_0$ is a $\onen$ MSC.
\end{proof}

\begin{proposition}
	\label{prop:prefixes-nn}
	Every $\nn$ prefix of a $\nn$ MSC is a $\nn$ MSC.
\end{proposition}
\begin{proof}
	Let $\msc = (\Events, \procrel, \lhd, \lambda) \in \nnMSCs$ and let $\msc_0 =
	(\Events_0, \procrel_0, \lhd_0, \lambda_0)$ be a $\nn$ prefix of $\msc$, where $\Events_0 \subseteq \Events$. Firstly, the $\bowtie_\msc$-downward-closeness of $\Events_0$ guarantees that ${\msc_0}$ is still an MSC. We need to prove that it is a $\nn$ MSC. By contradiction, suppose that $\msc_0$ is not a $\nn$ MSC. Then, there are distinct $e,f \in \Events_0$ such that $e \bowtie_{\msc_0} f \bowtie_{\msc_0} e$. As $\Events_0 \subseteq \Events$, we have that ${\rightarrow_0} \subseteq {\rightarrow}$, ${\lhd_0} \subseteq {\lhd}$, ${\nnrel_0} \subseteq {\nnrel}$. Clearly, $\bowtie_{\msc_0} \subseteq\; \bowtie_\msc$, so $e \bowtie_\msc f \bowtie_\msc e$. This implies that $\msc$ is not a $\nn$ MSC, because $\bowtie_\msc$ is cyclic, which is a contradiction. Hence $\msc_0$ is a $\nn$ MSC.
\end{proof}

The next lemma is about the prefix closure of a communicating system and it follows from Proposition \ref{prop:prefixes}.

\begin{proposition}\label{lem:co-prefix-closed}
	For all $\comsymb \in \{\asy, \ppsymb, \mbsymb, \cosymb\}$, $\cL{\Sys}$ is prefix-closed:
	$\Pref{\cL{\Sys}} \subseteq \cL{\Sys}$.
\end{proposition}

Similar results also hold for the $\onen$ and $\nn$ communication models.

\begin{proposition}\label{lem:onen-prefix-closed}
	$\onenL{\Sys}$ is $\onen$ prefix-closed:
	$\Prefonen{\onenL{\Sys}} \subseteq \onenL{\Sys}$.
\end{proposition}
\begin{proof}
	Given a system $\System$, we have that $\onenL{\System} = \ppL{\System} \cap \onenMSCs$. Note that, because of how we defined a $\onen$ prefix, we have that $\Prefonen{\onenL{\Sys}} = \Pref{\onenL{\Sys}} \cap \onenMSCs$. Moreover, $\Pref{\onenL{\Sys}} \subseteq \Pref{\ppL{\Sys}}$, and $\Pref{\onenL{\Sys}} \subseteq \ppL{\Sys}$ for Proposition~\ref{lem:prefix-closed}. Putting everything together, $\Prefonen{\onenL{\Sys}} \subseteq \ppL{\Sys} \cap \onenMSCs = \onenL{\System}$.
\end{proof}

\begin{proposition}\label{lem:nn-prefix-closed}
	$\nnL{\Sys}$ is $\nn$ prefix-closed:
	$\Prefnn{\nnL{\Sys}} \subseteq \nnL{\Sys}$.
\end{proposition}
\begin{proof}
	Given a system $\System$, we have that $\nnL{\System} = \ppL{\System} \cap \nnMSCs$. Note that, because of how we defined a $\nn$ prefix, we have that $\Prefnn{\nnL{\Sys}} = \Pref{\nnL{\Sys}} \cap \nnMSCs$. Moreover, $\Pref{\nnL{\Sys}} \subseteq \Pref{\ppL{\Sys}}$, and $\Pref{\nnL{\Sys}} \subseteq \ppL{\Sys}$ for Proposition~\ref{lem:prefix-closed}. Putting everything together, $\Prefnn{\nnL{\Sys}} \subseteq \ppL{\Sys} \cap \nnMSCs = \nnL{\System}$.
\end{proof}

In general, even simple verification problems, such
as control-state reachability, are undecidable for
communicating systems \cite{DBLP:journals/jacm/BrandZ83}.
However, they are decidable when we restrict to behaviors of
bounded special tree-width, which motivates the following
definition of a generic \emph{bounded model-checking} problem for $\{\asy, \oneone, \co, \none, \onen, \nn \}$ :\\
%
{\bf Input:} Two finite sets $\Procs$ and $\Msg$, a communicating system $\System$, an MSO sentence $\phi$, and $k \in \N$.\\
%
{\bf Question:} Do we have $\cL{\Sys} \cap \stwMSCs{k} \subseteq L(\phi)$?

\begin{theorem}
	\label{thm:bounded_model_checking}
	The bounded model-checking problem for $\comsymb \in \{$$\asy, $ $\oneone, $ $\co, $ $\none, $ $\onen, $ $\nn\}$ is decidable.
\end{theorem}
\davidequestion{Proof missing for $\asy$, Courcelle's theorem?}

\begin{proof}
    For $\comsymb = \{\oneone, \none\}$ refer to \cite{BolligFG21}. 
    Consider $\comsymb = \co$. As shown in Section~\ref{sec:MSO}, $\coMSCs=L(\coformula)$. Given a system $\System$, we then have that $\coL{\System} = \ppL{\System} \cap L(\coformula)$. We can rewrite the bounded model-checking problem for $\comsymb = \cosymb$ as
    \[\begin{array}{rl}
    &\coL{\System} \cap \stwMSCs{k} \subseteq L(\phi)\\[1ex]
    \Longleftrightarrow &\ppL{\System} \cap L(\coformula) \cap \stwMSCs{k} \subseteq L(\phi)\\[1ex]
    \Longleftrightarrow &\ppL{\System} \cap \stwMSCs{k} \subseteq L(\phi) \cup L(\neg \coformula)\\[1ex]
    \Longleftrightarrow &\ppL{\System} \cap \stwMSCs{k} \subseteq L(\phi \vee \neg \coformula)\,.
    \end{array}\]

    The latter is an instance of the bounded model-checking problem for the $\oneone$ communication model, which was already shown to be decidable. The technique used to prove the decidability for the causally ordered communication can also be used for $\comsymb = \{\onen, \nn\}$.
\end{proof}

In this last section we prove a series of statements to conclude that, when we have a STW-bounded class $\Class$, the synchronizability problem can be reduced to bounded model-checking, which we showed to be decidable in Theorem~\ref{thm:bounded_model_checking}.

\begin{proposition}\label{prop:pref_stw_k+2}
	Let $k \in \N$ and $\Class \subseteq \stwMSCs{k}$. For all
	$M \in \MSCs \setminus \Class$, we have
	$(\Pref{\msc} \cap \stwMSCs{(k+2)}) \setminus \Class \neq \emptyset$.
\end{proposition}
\begin{proof}
    Already proved in \cite{BolligFG21}, but we provide more details for a clearer understanding.
    Let $k$ and $\Class$ be fixed, and let
    $\msc\in \MSCs\setminus \Class$ be fixed. If the empty MSC is not in $\Class$, then we are done, since it is a valid prefix of $\msc$ and it is in $\stwMSCs{(k+2)} \setminus \Class$.
    Otherwise, let $\msc'\in \Pref{\msc} \setminus \Class$ such that, for all $\le$-maximal events $e$ of $\msc'$, removing $e$ (along with its adjacent edges) gives an MSC in $\Class$. In other words, $\msc'$ is the "shortest" prefix of $\msc$ that is not in $\Class$. We obtain such an MSC by successively removing $\le$-maximal events. Let $e$ be $\le_{\msc'}$-maximal and let $\msc''=\msc' \setminus \{e\}$. Since $\msc'$ was taken minimal in terms of number of events,	$\msc''\in \Class$.
    So Eve has a winning strategy with $k+1$ colours for $\msc''$.
    Let us design a winning strategy with $k+3$ colours for Eve for $\msc'$, which will show the claim.

    Observe that the event $e$ occurs at the end of the timeline of a process (say $p$), and it is part of at most two edges:
    \begin{itemize}
        \item one with the previous $p$-event (if any)
        \item one with the corresponding send event (if $e$ is a receive event)
    \end{itemize}
    Let $e_1,e_2$ be the two neighbours of $e$.
    The strategy of Eve is the following: in the first round, mark $e,e_1,e_2$,
    then erase the edges $(e_1,e)$ and $(e_2,e)$, then split the remaining graph
    in two parts: $\msc''$ on the one side, and the single node graph $\{e\}$ on
    the other side. Then Eve applies its winning strategy for $\msc''$, except
    that initially the two events $e_1,e_2$ are marked (so she may need up to $k+3$
    colours).
\end{proof}

We have similar results also for the $\onen$ and $\nn$ communication models.

\begin{proposition}\label{prop:onen_pref_stw_k+2}
	Let $k \in \N$ and $\Class \subseteq \stwMSCs{k}$. For all
	$M \in \onenMSCs \setminus \Class$, we have
	$(\Prefonen{\msc} \cap \stwMSCs{(k+2)}) \setminus \Class \neq \emptyset$.
\end{proposition}
\begin{proof}
	Let $k$ and $\Class$ be fixed, and let
	$\msc\in \onenMSCs \setminus \Class$ be fixed. If the empty MSC is not in $\Class$, then we are done, since it is a valid $\onen$ prefix of $\msc$ and it is in $\stwMSCs{(k+2)} \setminus \Class$.
	Otherwise, let $\msc'\in \Prefonen{\msc} \setminus \Class$ such that, for all $\onenpartial$-maximal events $e$ of $\msc'$, removing $e$ (along with its adjacent edges) gives an MSC in $\Class$. In other words, $\msc'$ is the "shortest" prefix of $\msc$ that is not in $\Class$. We obtain such an MSC by successively removing $\onenpartial$-maximal events. Let $e$ be $\onenpartial_{\msc'}$-maximal and let $\msc''=\msc' \setminus \{e\}$. Since $\msc'$ was taken minimal in terms of number of events,	$\msc''\in \Class$.
	The proof proceeds exactly as the proof of Proposition~\ref{prop:pref_stw_k+2}. 
\end{proof}

\begin{proposition}\label{prop:nn_pref_stw_k+2}
	Let $k \in \N$ and $\Class \subseteq \stwMSCs{k}$. For all
	$M \in \nnMSCs \setminus \Class$, we have
	$(\Prefnn{\msc} \cap \stwMSCs{(k+2)}) \setminus \Class \neq \emptyset$.
\end{proposition}
\begin{proof}
	Let $k$ and $\Class$ be fixed, and let
	$\msc\in \nnMSCs \setminus \Class$ be fixed. If the empty MSC is not in $\Class$, then we are done, since it is a valid $\nn$ prefix of $\msc$ and it is in $\stwMSCs{(k+2)} \setminus \Class$.
	Otherwise, let $\msc'\in \Prefnn{\msc} \setminus \Class$ such that, for all $\bowtie_\msc$-maximal events $e$ of $\msc'$, removing $e$ (along with its adjacent edges) gives an MSC in $\Class$. In other words, $\msc'$ is the "shortest" prefix of $\msc$ that is not in $\Class$. We obtain such an MSC by successively removing $\bowtie_\msc$-maximal events. Let $e$ be $\bowtie_{\msc'}$-maximal and let $\msc''=\msc' \setminus \{e\}$. Since $\msc'$ was taken minimal in terms of number of events,	$\msc''\in \Class$.
	The proof proceeds exactly as the proof of Proposition~\ref{prop:pref_stw_k+2}. 
\end{proof}

The following proposition is the last ingredient that we need to prove Theorem~\ref{thm:sync}.

\begin{proposition}\label{prop:continuous}
	Let $\System$ be a communicating system, $\comsymb \in \{$$\asy, $ $\oneone, $ $\co, $ $\none, $ $\onen, $ $\nn, $ $\rsc\}$,
	$k \in \N$, and $\Class \subseteq \stwMSCs{k}$.
	Then, $\cL{\System} \subseteq \Class$ iff
	$\cL{\System} \cap \stwMSCs{(k+2)} \subseteq \Class$.
\end{proposition}
\begin{proof}
For $\comsymb \in \{$$\asy, $ $\oneone, $ $\co, $ $\none\}$, it follows from Proposition~\ref{prop:pref_stw_k+2}. For $\comsymb \in \{\onen, \nn\}$, it follows from Proposition~\ref{prop:onen_pref_stw_k+2} and Proposition~\ref{prop:nn_pref_stw_k+2}, respectively.
\end{proof}

\thmsync*
\begin{proof}
    According to Proposition~\ref{prop:continuous}, we have $\cL{\System} \subseteq \Class$ iff
	$\cL{\System} \cap \stwMSCs{(k+2)} \subseteq \Class$. The latter is decidable according to Theorem~\ref{thm:bounded_model_checking}.
\end{proof}




\subsection{Proof of Theorem~\ref{thm:co-weak-sync}}
\label{apx:thm-co-weak-sync}
\thmCoWeakSync

\etienne{Develop (for instance adding more precise references, or copy/paste), or drop this section?}

The proof is essentially identical to that given in \cite{DBLP:conf/concur/BolligGFLLS21} for the $\oneone$ case. 
We do the same reduction from the Post correspondence problem. 
The original proof considered a $\oneone$ system $\System$ with four machines (P1, P2, V1, V2), where we have 
unidirectional communication channels from provers (P1 and P2) to verifiers (V1 and V2). In particular notice 
that all the possible behaviours of $\System$ are causally ordered, i.e. $\ppL{\System} \subseteq \coMSCs$; 
according to how we built our system $\System$, it is impossible to have a pair of causally-related send 
events of P1 and P2\footnote{There is no channel between P1 and P2, and we only have unidirectional communication 
channels from provers to verifiers; it is impossible to have a causal path between two send events of P1 and P2.}, which implies that causal ordering is 
already ensured by any possible $\oneone$ behaviour of $\System$. The rest of the proof is identical to the 
$\oneone$ case.
