\begin{proposition} \label{prop:co_is_pp}
	Every $\co$ MSC is a $\oneone$ MSC.
\end{proposition}
\begin{proof}
	According to Definition~\ref{def:co_msc}, and MSC is $\co$ if, for any two send events $s$ and $s'$, such that $\lambda(s)=\pqsAct{\plh}{q}$, $\lambda(s')=\pqsAct{\plh}{q}$, and $s \le_\msc s'$, we have either
	\begin{enumerate*}[label={(\roman*)}]
		\item $s,s' \in \Matched{\msc}$ and $r \procrel^* r'$, where $r$ and $r'$ are two receive events such that $s \lhd r$ and $s' \lhd r'$, or
		\item $s' \in \Unm{\msc}$.
	\end{enumerate*}
	The conditions imposed by the Definition~\ref{def:pp_msc} of $\oneone$ are clearly satified by any causally ordered MSC; in particular, note that $s \procrel^+ s'$ implies $s \le_\msc s'$.
\end{proof}

\mbonennounmatched*
\begin{proof}
	We show that the contrapositive is true, i.e. if an MSC is not $\onen$ (and it does not have unmatched messages), it is also not mailbox. Suppose $\msc$ is an asynchronous MSC, but not $\onen$. There must be a cycle $\xi$ such that  $e \lessdot e$, for some event $e$. Recall that ${\lessdot} = ({\procrel} \,\cup\, {\onenrel} \,\cup\, {\mbrel})^\ast$ and ${\le} = ({\procrel} \cup {\onenrel})^\ast$. We can always explicitely write a cycle $e \lessdot e$ only using $\onenrel$ and $\le$. For instance, there might be a cycle $e \lessdot e$ because we have that $e \onenrel f \le g \onenrel h \onenrel i \le e$. Consider any two adiacent events $r_1$ and $r_2$ in the cycle $\xi$, where $\xi$ has been written using only $\onenrel$ and $\le$, and we never have two consecutive $\le$. We have two cases:
	\begin{enumerate}
		\item $r_1 \onenrel r_2$. By definition of $\onenrel$, $r_1$ and $r_2$ must be two receive events, since we are not considering unmatched send events, and $s_1 \procrel^+ s_2$, where $s_1$ and $s_2$ are the send events that match with $r_1$ and $r_2$, respectively.
		\item $r_1 \le r_2$. Since $\msc$ is asynchronous by hyphotesis, $\xi$ has to contain at least one $\onenrel$; recall that we also wrote $\xi$ in such a way that we do not have two consecutive $\le$. It is not difficult to see that $r_1$ and $r_2$ have to be receive events, since they belong to $\xi$. Let $s_1$ and $s_2$ be the two send events such that $s_1 \lhd r_1$ and $s_2 \lhd r_2$. We have two cases:
		\begin{enumerate}
			\item $s_2$ is in the causal path between $r_1$ and $r_2$, i.e. $s_1 \lhd r_1 \le s_2 \lhd r_2$. In particular, note that $s_1 \le s_2$.
			\item $s_2$ is not in the causal path between $r_1$ and $r_2$, hence there must be a message $m_k$ received by the same process that executes $r_2$, such that $r_1 \le s_k \lhd r_k \procrel^+ r_2$, where $r_k$ is the send event of $m_k$. Since messages $m_k$ and $m_2$ are received by the same process and $r_k \procrel^+ r_2$, we should have $s_k \mbrel s_2$, according to the mailbox semantics. In particular, note the we have $s_1 \le s_k \mbrel s_2$.
		\end{enumerate}
		In both case (a) and (b), we conclude that $s_1 \preceq s_2$. Recall that ${\preceq} = ({\procrel} \,\cup\, {\lhd} \,\cup\, {\mbrel_\msc})^\ast$.
	\end{enumerate}
	Notice that, for either cases, a relation between two receive events $r_1$ and $r_2$ implies a relation between the respective send events $s_1$ and $s_2$, according to the mailbox semantics. It follows that $\xi$, which is a cycle for the $\lessdot$ relation, always implies a cycle for the $\preceq$ relation.
	\end{proof}

	\begin{proposition} \label{prop:nn_is_onen}
		Every $\nn$ MSC is a $\onen$ MSC.
	\end{proposition}
	\begin{proof}
		Consider Definition~\ref{def:n_n} and Definition~\ref{def:one_n}. They are identical, except for the fact that in the $\nn$ case we consider any two send events, and not just those that are sent by a same process. This is enough to show that each $\nn$ linearization is also a $\onen$ linearization and, therefore, each $\nn$ MSC is a $\onen$ MSC.
	\end{proof}
	
	\begin{proposition} \label{prop:rsc_is_nn}
		Every $\rsc$ MSC is a $\nn$ MSC.
	\end{proposition}
	\begin{proof}
		Consider Definition~\ref{def:rsc} and Definition~\ref{def:n_n}. Let us pick an $\rsc$ linearization $\linrel$. If every send event is immediately followed by its matching receive event, and we do not have unmatched messages, then $\linrel$ is also a $\nn$ linearization; note that, for any two send events $s$ and $s'$ such that $s \linrel s'$, we also have $r \linrel r'$, where $s \lhd r$ and $s' \lhd r'$. It follows that each $\rsc$ MSC is a $\nn$ MSC.
	\end{proof}
	