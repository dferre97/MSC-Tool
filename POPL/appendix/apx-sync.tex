In order to prove Theorem \ref{thm:sync}, we first need to introduce some concepts and give preliminary proofs. 

\begin{definition}[Prefix]
	Let $\msc = (\Events,\procrel,\lhd,\lambda) \in \MSCs$ and consider
	$E \subseteq \Events$ such that $E$ is ${\happensbefore}$-\emph{downward-closed}, i.e,
	for all $(e,f) \in {\happensbefore}$ such that $f \in E$, we also have $e \in E$.
	Then, the MSC $M' = (E,{\procrel} \cap (E \times E),{\lhd} \cap (E \times E),\lambda')$,
	where $\lambda'$ is the restriction of $\Events$ to $E$, is called a \emph{prefix}
	of $\msc$. 	
\end{definition}

If we consider a set $E$ that is ${\onenpartial}$-\emph{downward-closed}, we call $M'$ a \emph{$\onen$ prefix}.
If the set $E$ is ${\nnbowtie}$-\emph{downward-closed}, we call $M'$ a \emph{$\nn$ prefix}. Note that every $\onen$ or $\nn$ prefix is also a prefix, since $\happensbefore \subseteq {\onenpartial}$ and $\happensbefore \subseteq {\nnbowtie}$.

Note that the empty MSC is a prefix of $\msc$.
We denote the set of prefixes of $\msc$ by $\Pref{\msc}$, whereas $\Prefonen{\msc}$ and $\Prefnn{\msc}$ are used for the $\onen$ and the $\nn$ variants, respectively.
This is extended to sets $L \subseteq \MSCs$ as expected, letting
$\Pref{L} = \bigcup_{\msc \in L} \Pref{\msc}$.

\begin{proposition}
	\label{prop:prefixes}
	For $\comsymb \in \{\asy, \oneone, \co, \none\}$, every prefix of a $\comsymb$ MSC is a $\comsymb$ MSC.
\end{proposition}
\begin{proof}
    For $\comsymb = \asy$ it is true by definition. For $\comsymb = \{\oneone, \none\}$ it was already shown to be true in \cite{BolligGFLLS21}, so we just consider $\comsymb = \co$. Let $\msc = (\Events, \procrel, \lhd, \lambda) \in \coMSCs$ and let $\msc_0 =
    (\Events_0, \procrel_0, \lhd_0, \lambda_0)$ be a prefix of $\msc$. By contradiction, suppose that $\msc_0$ is not a	causally ordered MSC. There must be two distinct $s,s' \in \Events_0$ such that $\lambda(s)=\pqsAct{\plh}{q}$, $\lambda(s')=\pqsAct{\plh}{q}$, $s \happensbefore^{(\msc_0)} s'$ and either
    \begin{enumerate*}[label={(\roman*)}]
        \item $r' \procrel^+ r$, where $r$ and $r'$ are two receive events such that $s \lhd r$ and $s' \lhd r'$, or
        \item $s \in \Unm{\msc_0}$ and $s' \in \Matched{\msc_0}$.
    \end{enumerate*}
    In both cases, $\msc$ would also not be a causally ordered $\MSCs$, since $\Events_0 \subseteq \Events$, ${\rightarrow_0} \subseteq {\rightarrow}$, and ${\lhd_0} \subseteq {\lhd}$. This is a contradiction, thus $\msc_0$ has to be causally ordered.
\end{proof}

Note that this proposition is not true for the $\onen$ and the $\nn$ communication models. Fig.\ref{fig:onen-prefix} shows an example of $\nn$ MSC with a prefix that is neither a $\nn$ MSC nor a $\onen$ MSC.

\begin{figure}[t]
	\captionsetup[subfigure]{justification=centering}
% \centering
\begin{subfigure}[t]{0.45\textwidth}\centering
	\begin{tikzpicture}[scale=0.7, every node/.style={transform shape}]
		\newproc{0}{p}{-1.5};
		\newproc{1}{q}{-1.5};
		\newproc{2}{r}{-1.5};

		\newmsgm{0}{1}{-0.5}{-0.5}{1}{0.3}{black};
		\newmsgm{0}{2}{-1.0}{-1.0}{2}{0.7}{black};

	\end{tikzpicture}
	\caption{A $\nn$ MSC $\msc$.}
\end{subfigure}
% \hfill
\begin{subfigure}[t]{0.45\textwidth}\centering
	\begin{tikzpicture}[scale=0.7, every node/.style={transform shape}]
		\newproc{0}{p}{-1.5};
		\newproc{1}{q}{-1.5};
		\newproc{2}{r}{-1.5};

		\newmsgum{0}{1}{-0.5}{1}{0.3}{black};
		\newmsgm{0}{2}{-1.0}{-1.0}{2}{0.7}{black};

	\end{tikzpicture}
	\caption{A prefix of $\msc$.}
\end{subfigure}
	\caption{A $\nn$ MSC with a prefix that is neither $\onen$ nor $\nn$.}
	\label{fig:onen-prefix}
\end{figure}

\begin{proposition}
	\label{prop:prefixes-onen}
	Every $\onen$ prefix of a $\onen$ MSC is a $\onen$ MSC.
\end{proposition}
\begin{proof}
    Let $\msc = (\Events, \procrel, \lhd, \lambda) \in \onenMSCs$ and let $\msc_0 =
    (\Events_0, \procrel_0, \lhd_0, \lambda_0)$ be a $\onen$ prefix of $\msc$, where $\Events_0 \subseteq \Events$. Firstly, the $\onenpartial$-downward-closeness of $\Events_0$ guarantees that ${\msc_0}$ is still an MSC. We need to prove that it is a $\onen$ MSC. By contradiction, suppose that $\msc_0$ is not a $\onen$ MSC. Then, there are distinct $e,f \in \Events_0$ such that $e \onenpartial^{(\msc_0)} f \onenpartial^{(\msc_0)} e$, where $\onenpartial^{(\msc_0)} = (\procrel_0 \cup \lhd_0 \cup \onenrel^{(\msc_0)})^\ast$. As $\Events_0 \subseteq \Events$, we have that ${\rightarrow_0} \subseteq {\rightarrow}$, ${\lhd_0} \subseteq {\lhd}$, ${\onenrel^{(\msc_0)}} \subseteq {\onenrel}$. Clearly, $\onenpartial^{(\msc_0)} \subseteq \onenpartial$, so $e \onenpartial f \onenpartial e$. This implies that $\msc$ is not a $\onen$ MSC, because $\onenpartial$ is cyclic, which is a contradiction. Hence $\msc_0$ is a $\onen$ MSC.
\end{proof}

\begin{proposition}
	\label{prop:prefixes-nn}
	Every $\nn$ prefix of a $\nn$ MSC is a $\nn$ MSC.
\end{proposition}
\begin{proof}
	Let $\msc = (\Events, \procrel, \lhd, \lambda) \in \nnMSCs$ and let $\msc_0 =
	(\Events_0, \procrel_0, \lhd_0, \lambda_0)$ be a $\nn$ prefix of $\msc$, where $\Events_0 \subseteq \Events$. Firstly, the $\nnbowtieofmsc\msc$-downward-closeness of $\Events_0$ guarantees that ${\msc_0}$ is still an MSC. We need to prove that it is a $\nn$ MSC. By contradiction, suppose that $\msc_0$ is not a $\nn$ MSC. Then, there are distinct $e,f \in \Events_0$ such that $e \nnbowtieofmsc{\msc_0} f \nnbowtieofmsc{\msc_0} e$. As $\Events_0 \subseteq \Events$, we have that ${\rightarrow_0} \subseteq {\rightarrow}$, ${\lhd_0} \subseteq {\lhd}$, ${\nnrel} \subseteq {\nnrel}$. Clearly, $\nnbowtieofmsc{\msc_0} \subseteq\; \nnbowtieofmsc\msc$, so $e \nnbowtieofmsc\msc f \nnbowtieofmsc\msc e$. This implies that $\msc$ is not a $\nn$ MSC, because $\nnbowtieofmsc\msc$ is cyclic, which is a contradiction. Hence $\msc_0$ is a $\nn$ MSC.
\end{proof}

The next lemma is about the prefix closure of a communicating system and it follows from Proposition \ref{prop:prefixes}.

\begin{proposition}\label{lem:prefix-closed}
	For all $\comsymb \in \{\asy, \ppsymb, \mbsymb, \cosymb\}$, $\cL{\Sys}$ is prefix-closed:
	$\Pref{\cL{\Sys}} \subseteq \cL{\Sys}$.
\end{proposition}

Similar results also hold for the $\onen$ and $\nn$ communication models.

\begin{proposition}\label{lem:onen-prefix-closed}
	$\onenL{\Sys}$ is $\onen$ prefix-closed:
	$\Prefonen{\onenL{\Sys}} \subseteq \onenL{\Sys}$.
\end{proposition}
\begin{proof}
	Given a system $\System$, we have that $\onenL{\System} = \ppL{\System} \cap \onenMSCs$. Note that, because of how we defined a $\onen$ prefix, we have that $\Prefonen{\onenL{\Sys}} = \Pref{\onenL{\Sys}} \cap \onenMSCs$. Moreover, $\Pref{\onenL{\Sys}} \subseteq \Pref{\ppL{\Sys}}$, and $\Pref{\onenL{\Sys}} \subseteq \ppL{\Sys}$ for Proposition~\ref{lem:prefix-closed}. Putting everything together, $\Prefonen{\onenL{\Sys}} \subseteq \ppL{\Sys} \cap \onenMSCs = \onenL{\System}$.
\end{proof}

\begin{proposition}\label{lem:nn-prefix-closed}
	$\nnL{\Sys}$ is $\nn$ prefix-closed:
	$\Prefnn{\nnL{\Sys}} \subseteq \nnL{\Sys}$.
\end{proposition}
\begin{proof}
	Given a system $\System$, we have that $\nnL{\System} = \ppL{\System} \cap \nnMSCs$. Note that, because of how we defined a $\nn$ prefix, we have that $\Prefnn{\nnL{\Sys}} = \Pref{\nnL{\Sys}} \cap \nnMSCs$. Moreover, $\Pref{\nnL{\Sys}} \subseteq \Pref{\ppL{\Sys}}$, and $\Pref{\nnL{\Sys}} \subseteq \ppL{\Sys}$ for Proposition~\ref{lem:prefix-closed}. Putting everything together, $\Prefnn{\nnL{\Sys}} \subseteq \ppL{\Sys} \cap \nnMSCs = \nnL{\System}$.
\end{proof}

In this last section we prove a series of statements to conclude that, when we have a STW-bounded class $\Class$, the synchronizability problem can be reduced to bounded model-checking, which we showed to be decidable in Theorem~\ref{thm:bounded-model-checking}.

\begin{proposition}\label{prop:pref_stw_k+2}
	Let $k \in \N$ and $\Class \subseteq \stwMSCs{k}$. For all
	$M \in \MSCs \setminus \Class$, we have
	$(\Pref{\msc} \cap \stwMSCs{(k+2)}) \setminus \Class \neq \emptyset$.
\end{proposition}
\begin{proof}
    Already proved in \cite{BolligGFLLS21}, but we provide more details for a clearer understanding.
    Let $k$ and $\Class$ be fixed, and let
    $\msc\in \MSCs\setminus \Class$ be fixed. If the empty MSC is not in $\Class$, then we are done, since it is a valid prefix of $\msc$ and it is in $\stwMSCs{(k+2)} \setminus \Class$.
    Otherwise, let $\msc'\in \Pref{\msc} \setminus \Class$ such that, for all $\happensbefore$-maximal events $e$ of $\msc'$, removing $e$ (along with its adjacent edges) gives an MSC in $\Class$. In other words, $\msc'$ is the "shortest" prefix of $\msc$ that is not in $\Class$. We obtain such an MSC by successively removing $\happensbefore$-maximal events. Let $e$ be a $\happensbefore$-maximal event of $\msc'$, and let $\msc''=\msc' \setminus \{e\}$. Since $\msc'$ was taken minimal in terms of number of events,	$\msc''\in \Class$.
    So Eve has a winning strategy with $k+1$ colors for $\msc''$.
    Let us design a winning strategy with $k+3$ colors for Eve for $\msc'$, which will show the claim.

    Observe that the event $e$ occurs at the end of the timeline of a process (say $p$), and it is part of at most two edges:
    \begin{itemize}
        \item one with the previous $p$-event (if any)
        \item one with the corresponding send event (if $e$ is a receive event)
    \end{itemize}
    Let $e_1,e_2$ be the two neighbours of $e$.
    The strategy of Eve is the following: in the first round, mark $e,e_1,e_2$,
    then erase the edges $(e_1,e)$ and $(e_2,e)$, then split the remaining graph
    in two parts: $\msc''$ on the one side, and the single node graph $\{e\}$ on
    the other side. Then Eve applies its winning strategy for $\msc''$, except
    that initially the two events $e_1,e_2$ are marked (so she may need up to $k+3$
    colors).
\end{proof}

We have similar results also for the $\onen$ and $\nn$ communication models.

\begin{proposition}\label{prop:onen_pref_stw_k+2}
	Let $k \in \N$ and $\Class \subseteq \stwMSCs{k}$. For all
	$M \in \onenMSCs \setminus \Class$, we have
	$(\Prefonen{\msc} \cap \stwMSCs{(k+2)}) \setminus \Class \neq \emptyset$.
\end{proposition}
\begin{proof}
	Let $k$ and $\Class$ be fixed, and let
	$\msc\in \onenMSCs \setminus \Class$ be fixed. If the empty MSC is not in $\Class$, then we are done, since it is a valid $\onen$ prefix of $\msc$ and it is in $\stwMSCs{(k+2)} \setminus \Class$.
	Otherwise, let $\msc'\in \Prefonen{\msc} \setminus \Class$ such that, for all $\onenpartial$-maximal events $e$ of $\msc'$, removing $e$ (along with its adjacent edges) gives an MSC in $\Class$. In other words, $\msc'$ is the "shortest" prefix of $\msc$ that is not in $\Class$. We obtain such an MSC by successively removing $\onenpartial$-maximal events. Let $e$ be $\onenpartial^{(\msc')}$-maximal and let $\msc''=\msc' \setminus \{e\}$. Since $\msc'$ was taken minimal in terms of number of events,	$\msc''\in \Class$.
	The proof proceeds exactly as the proof of Proposition~\ref{prop:pref_stw_k+2}. 
\end{proof}

\begin{proposition}\label{prop:nn_pref_stw_k+2}
	Let $k \in \N$ and $\Class \subseteq \stwMSCs{k}$. For all
	$M \in \nnMSCs \setminus \Class$, we have
	$(\Prefnn{\msc} \cap \stwMSCs{(k+2)}) \setminus \Class \neq \emptyset$.
\end{proposition}
\begin{proof}
	Let $k$ and $\Class$ be fixed, and let
	$\msc\in \nnMSCs \setminus \Class$ be fixed. If the empty MSC is not in $\Class$, then we are done, since it is a valid $\nn$ prefix of $\msc$ and it is in $\stwMSCs{(k+2)} \setminus \Class$.
	Otherwise, let $\msc'\in \Prefnn{\msc} \setminus \Class$ such that, for all $\nnbowtieofmsc\msc$-maximal events $e$ of $\msc'$, removing $e$ (along with its adjacent edges) gives an MSC in $\Class$. In other words, $\msc'$ is the "shortest" prefix of $\msc$ that is not in $\Class$. We obtain such an MSC by successively removing $\nnbowtieofmsc\msc$-maximal events. Let $e$ be $\nnbowtieofmsc{\msc'}$-maximal and let $\msc''=\msc' \setminus \{e\}$. Since $\msc'$ was taken minimal in terms of number of events,	$\msc''\in \Class$.
	The proof proceeds exactly as the proof of Proposition~\ref{prop:pref_stw_k+2}. 
\end{proof}

The following proposition is the last ingredient that we need to prove Theorem~\ref{thm:sync}.

\begin{proposition}\label{prop:continuous}
	Let $\System$ be a communicating system, $\comsymb \in \{$$\asy, $ $\oneone, $ $\co, $ $\none, $ $\onen, $ $\nn, $ $\rsc\}$,
	$k \in \N$, and $\Class \subseteq \stwMSCs{k}$.
	Then, $\cL{\System} \subseteq \Class$ iff
	$\cL{\System} \cap \stwMSCs{(k+2)} \subseteq \Class$.
\end{proposition}
\begin{proof}
For $\comsymb \in \{$$\asy, $ $\oneone, $ $\co, $ $\none\}$, it follows from Proposition~\ref{prop:pref_stw_k+2}. For $\comsymb \in \{\onen, \nn\}$, it follows from Proposition~\ref{prop:onen_pref_stw_k+2} and Proposition~\ref{prop:nn_pref_stw_k+2}, respectively.
\end{proof}

\thmsync*
\begin{proof}
    According to Proposition~\ref{prop:continuous}, we have $\cL{\System} \subseteq \Class$ iff
	$\cL{\System} \cap \stwMSCs{(k+2)} \subseteq \Class$. The latter is decidable according to Theorem~\ref{thm:bounded-model-checking}.
\end{proof}

