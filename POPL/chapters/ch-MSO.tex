% !TEX root = ../popl-paper.tex

\davidequestion{Don't we need a citation for MSO?}
In Section~\ref{sec:MSC} we introduced 7 different communication models and the corresponding classes of MSCs. Here, we show that all of these classes are MSO-definable, i.e. there is a Monadic Second Order Logic formula that defines them. Sometimes it is straightforward, but for some classes of MSCs we have to introduce alternative definitions (equivalent to those given in Section~\ref{sec:MSC}) that are closer to the logic, in order to prove MSO-definability. These alternative definitions will also be heavily used in the following sections for proofs. We first recall the formal definition of MSO logic over MSCs.

\begin{definition}[MSO logic]
The set of MSO formulas over MSCs (over $\Procs$ and $\Msg$) is given by the grammar
$
\phi ::= true \mid x \procrel y \mid x \lhd y \mid \lambda(x) = a \mid x = y \mid x \in X \mid \exists x.\phi \mid \exists X.\phi \mid \phi \vee \phi \mid \neg \phi
$,
where $a \in \Act$, $x$ and $y$ are first-order variables (taken from an infinite set of variables), interpreted as
events of an MSC, and $X$ is a second-order variable, interpreted
as a set of events.
\end{definition}

We use common abbreviations such as $\wedge$, $\Rightarrow$, $\forall$, etc. The satisfaction relation is defined in the standard way and self-explanatory.
For instance, the formula $\neg\exists x.(\bigvee_{a \in \sAct} \lambda(x) = a \;\wedge\; \neg \mathit{matched}(x))$,
with $\mathit{matched}(x) = \exists y.x \lhd y$,
says that there are no unmatched send events. All MSCs of Figure \ref{fig:exmscs} satisfy the formula. Given a sentence $\phi$, i.e., a formula without free variables,
we let $L(\phi)$ denote the set of MSCs that satisfy $\phi$. Since we have defined the set of MSO formulas over asynchronous MSCs, the formula $\asformula = true$ describes the whole set of asynchronous MSCs, i.e. $L(\asformula) = \asMSCs$. It is worth mentioning that the (reflexive) transitive closure of a binary relation defined by an MSO formula with free variables $x$ and $y$, such as $x \procrel y$, is MSO-definable (more details are given in Appendix~\ref{apx:MSO}). The logic can then freely use formulas of the form $x \procrel^+ y$, $x \procrel^* y$ or $x \happensbefore y$.

\paragraph{\bf Peer-to-peer MSCs}
The MSO formula that defines $\ppMSCs$ (i.e. the set of $\pp$-MSCs) directly follows from Definition~\ref{def:pp_msc}:
\[
	\ppformula = \neg \exists s.\exists s'. \left(
	\bigvee_{\substack{p \in \Procs, q \in \Procs}}\;
	\bigvee_{\substack{a,b \in \pqsAct{p}{q}}}\hspace{-1em}
	(\lambda(s) = a \;\wedge\; \lambda(s') = b) \;\wedge\; s \procrel^+ s' \;\wedge\;
	(\psi_1 \vee \psi_2 )
	\right)
\]
where $\psi_1$ and $\psi_2$ are
\[
	\psi_1 = \exists r.\exists r'.\left(
	\begin{array}{ll}
		s \lhd r & \wedge\\
		s' \lhd r' & \wedge\\
		r' \procrel^+ r &
	\end{array}
	\right) \quad \quad
	\psi_2 = (\neg \mathit{matched}(s) \wedge \mathit{matched}(s'))
	\]
	\[
	matched(x) = \exists y. x \lhd y
\]

The property $\ppformula$ says that there cannot be two matched send events $s$ and $s'$, with the same sender and receiver, such that either
\begin{enumerate*}[label={(\roman*)}]
	\item $s \procrel^+ s'$ and their receipts happen in the reverse order, or
	\item $s$ is unmatched and $s'$ is matched.
\end{enumerate*}
% In other words, it ensures that channels operate in FIFO mode, where an unmatched messages blocks the receipt of all the subsequent messages on that channel.
%The set $\ppMSCs$ is therefore MSO-definable as $\ppMSCs=L(\ppformula)$.


\paragraph{\bf Causally ordered MSCs}
As with $\pp$-MSCs, the MSO-definability of $\coMSCs$ follows from Definition~\ref{def:co_msc}, given in Section~\ref{sec:MSC}:
\[
	\coformula = \neg \exists s.\exists s'. \left(
	\bigvee_{\substack{q \in \Procs}}\;
	\bigvee_{\substack{a,b \in \pqsAct{\plh}{q}}}\hspace{-1em}
	(\lambda(s) = a \;\wedge\; \lambda(s') = b) \;\wedge\; s \happensbefore s' \;\wedge\;
	(\psi_1 \vee \psi_2 )
	\right)
\]
where $\psi_1$ and $\psi_2$ have been defined above for the \pp case. The property $\coformula$ says that there cannot be two send events $s$ and $s'$, with the same recipient, such that $s \happensbefore s'$ and either
\begin{enumerate*}[label={(\roman*)}]
	\item their corresponding receive events $r$ and $r'$ happen in the opposite order, i.e. $r' \procrel^+ r$, or
	\item $s$ is unmatched and $s'$ is matched.
\end{enumerate*}
%The set $\coMSCs$ of causally ordered MSCs is therefore MSO-definable as $\coMSCs=L(\coformula)$.


\paragraph{\bf Mailbox MSCs}
For the mailbox communication model, Definition~\ref{def:mb_msc} cannot be easily translated into an MSO formula. Thus, we introduce an alternative definition of $\mb$-MSC that is closer to MSO logic; in particular, we define an additional binary relation that represents a constraint under the $\none$ semantics, which ensures that messages received by a process are sent in the same order as they are received. This definition is shown to be equivalent to Definition~\ref{def:mb_msc} in Appendix~\ref{apx:MSO}.

\begin{definition} [$\none$ alternative]\label{def:n_one_alt}
	For an MSC $\msc = (\Events,\procrel,\lhd,\lambda)$.	Let ${\mbrel} \subseteq \Events \times \Events$
	be defined as $s \mbrel s'$ if there is $q \in \Procs$
	such that $\lambda(s) \in \qsAct{q}$,
	$\lambda(s') \in \qsAct{q}$, and either:
	\begin{itemize}%\itemsep=0.5ex
		\item $s \in \Matched{\msc}$ and $s' \in \Unm{\msc}$, or
		\item $s \lhd f_1$ and $s' \lhd f_2$ for some $f_1,f_2 \in \Events_q$ such that $f_1 \procrel^+ f_2$.
	\end{itemize}

	We let ${\mbpartial} = ({\procrel} \,\cup\, {\lhd} \,\cup\, {\mbrel})^\ast$.
	$\msc $ is a \emph{$\none$-MSC}
	if ${\mbpartial}$ is a partial order.
\end{definition}
The ${\mbrel}$ relation ensures that send events addressed to the same process are executed in an order that is suitable for the $\none$ communication model. If ${\mbpartial}$ is a partial order, it means that it is possible to find a linearization $\linrel$, such that $\linrel \;\subseteq\; \mbpartial$. It is easy to see that such a linearization is exactly what we called a $\none$-linearization in Definition~\ref{def:mb_msc}.
%Finally notice  that ${\le} \subseteq {\mbpartial}$. Hence entailing that a \mb-MSC is also a \pp-MSC.
The MSO-definability of $\mbMSCs$ follows from Definition~\ref{def:n_one_alt}; in particular, note that 
$\mbpartial$ is reflexive and transitive by definition, 
thus we just have to check antisymmetry:
\[
	\mbformula = \ppformula \;\wedge\; \neg \exists x.\exists y.(\neg (x = y) \wedge x \mbpartial y \wedge y \mbpartial x)
\]
%This formula closely follows Definition~\ref{def:n_one_alt}. The set $\mbMSCs$ of $\none$ MSCs is therefore MSO-definable as $\mbMSCs=L(\mbformula)$.

where $x \mbpartial y$ is obtained as the MSO-definable reflexive transitive closure of
the union of the MSO-definable relations $\procrel$, $\lhd$, and $\mbrel$; 
we may define $x \mbrel y$ as:
\[
x \mbrel y =
\displaystyle
\hspace{-1em}\bigvee_{\substack{q \in \Procs\\a,b \in \qsAct{q}}}\hspace{-1em}
(\lambda(x) = a \;\wedge\; \lambda(y) = b)
\wedge
\left(
\begin{array}{rl}
& \mathit{matched}(x) \wedge \neg \mathit{matched}(y)\\[1ex]
\vee & \exists x'.\exists y'. (x \lhd x' \;\wedge\; y \lhd y' \;\wedge\; x' \procrel^+ y')
\end{array}
\right).
\]



\paragraph{\bf \onen MSCs}

As with the mailbox communication model, we give an alternative definition of $\onen$ MSC; the equivalence with Definition~\ref{def:one_n} is shown in Appendix~\ref{apx:MSO}.

\begin{definition} [$\onen$ alternative]\label{def:one_n_alt}
	For an MSC $\msc = (\Events,\procrel,\lhd,\lambda)$, let ${\onenrel} \subseteq \Events \times \Events$ be defined as $e_1 \onenrel e_2$ if there are two events $e_1$ and $e_2$, and $p \in \Procs$ such that either:
	\begin{itemize}%\itemsep=0.5ex
		\item $\lambda(e_1) \in \psAct{p}$, $\lambda(e_2) \in \psAct{p}$, $e_1 \in \Matched{\msc}$, and $e_2 \in \Unm{\msc}$, or
		\item $\lambda(e_1) \in \prAct{p}$, $\lambda(e_2) \in \prAct{p}$, $s_1 \lhd e_1$ and $s_2 \lhd e_2$ for some $s_1,s_2 \in \Events_p$, and $s_1 \procrel^+ s_2$.
	\end{itemize}

	We let ${\onenpartial} = ({\procrel} \,\cup\, {\lhd} \,\cup\, {\onenrel})^\ast$.
		%\begin{definition}\label{def:mailbox-msc}
	$\msc$ is a \emph{$\onen$ MSC}
	if ${\onenpartial}$ is a partial order.
\end{definition}

The ${\onenrel}$ relation ensures that messages sent by a process are sent and received in an order that is suitable for the $\onen$ communication. Since ${\onenpartial}$ is a partial order, it is possible to find a linearization $\linrel$ such that $\linrel \;\subseteq\; \onenpartial$. It is not difficult to see that such a linearization is exactly what we called a $\onen$ linearization in Definition~\ref{def:one_n}.
The MSO formula that defines $\onenMSCs$ follows from  Definition~\ref{def:one_n_alt}:
\[
	\onenformula = \neg \exists x.\exists y.(\neg (x = y) \wedge x \onenpartial y \wedge y \onenpartial x)
\]
Recall that $\onenpartial$ is the union of the MSO-definable relations $\procrel$, $\lhd$, and $\onenrel$. In particular, we can define $x \onenrel y$ as
\[
x \onenrel y =
\begin{array}{rl}
& \left(
	\bigvee_{\substack{p \in \Procs\\a,b \in \psAct{p}}}\hspace{-1em}
	(\lambda(x) = a \;\wedge\; \lambda(y) = b)
	\;\wedge\; \mathit{matched}(x) \;\wedge\; \neg \mathit{matched}(y)
\right) \;\vee\\
& \left(
	\bigvee_{\substack{p \in \Procs\\a,b \in \prAct{p}}}\hspace{-1em}
	(\lambda(x) = a \;\wedge\; \lambda(y) = b)
	\;\wedge\;
	\exists x'.\exists y'. (x' \lhd x \;\wedge\; y' \lhd y \;\wedge\; x' \procrel^+ y')
\right)\\
\end{array}
\]
%The MSO formula for $x \onenrel y$ closely follows Definition~\ref{def:one_n_alt}. The set $\onenMSCs$ of $\onen$ MSCs is therefore MSO-definable as $\onenMSCs=L(\onenformula)$.

%Finally it is worth noting that ${\le} \subseteq {\onenpartial}$.
%\cinzia{maybe move previous sentence to next section}

\paragraph{\bf \nn MSCs}

In order to show the MSO-definability of \nn MSCs we give an alternative definition and prove that it is equivalent to Definition~\ref{def:n_n}. Unlike mailbox and $\onen$ communication models, the equivalence is not trivial.

\begin{definition} [$\nn$ alternative]\label{def:n_n_alt}
	For an MSC $\msc = (\Events,\procrel,\lhd,\lambda)$, let ${\nnrel} = ({\procrel} \,\cup\, {\lhd} \,\cup\, {\mbrel} \,\cup\, {\onenrel})^\ast$. We define  $\bowtie \subseteq \Events \times \Events$,  such that $e_1 \bowtie e_2$ if one of the following holds:
	\begin{enumerate}%\itemsep=0.5ex
		\item $e_1 \nnrel e_2$
		\item $\lambda(e_1) \in \prAct{\plh}$, $\lambda(e_2) \in \prAct{\plh}$, $s_1 \lhd e_1$ and $s_2 \lhd e_2$ for some $s_1,s_2 \in \Events$, $s_1 \nnrel s_2$ and $e_1 \slashed{\nnrel} e_2$.
		\item $\lambda(e_1) \in \psAct{\plh}$, $\lambda(e_2) \in \psAct{\plh}$, $e_1 \lhd r_1$ and $e_2 \lhd r_2$ for some $r_1,r_2 \in \Events$, $r_1 \nnrel r_2$ and $e_1 \slashed{\nnrel} e_2$.
		\item $e_1 \in \Matched{\msc}$, $e_2 \in \Unm{\msc}$, $e_1 \slashed{\nnrel} e_2$.
	\end{enumerate}

	%Note that $\mbpartial \subseteq \nnrel$, $\onenpartial \subseteq \nnrel$, and $\nnrel \subseteq\; \bowtie$.
	$\msc $ is a \emph{$\nn$ MSC}
	if ${\bowtie}$ is acyclic.
\end{definition}

To prove that Definition~\ref{def:n_n} and Definition~\ref{def:n_n_alt} are equivalent, we need some preliminary results and definitions. Please note that missing proofs can be found in Appendix~\ref{apx:MSO}.

\begin{restatable}{proposition}{nnfirstprop}
	Let $\msc$ be an MSC. Given two matched send events $s_1$ and $s_2$, and their respective receive events $r_1$ and $r_2$, $r_1 \bowtie r_2 \implies s_1 \bowtie s_2$.
\end{restatable}

\begin{restatable}{proposition}{nnsecondprop}
\label{prop:n_n_cycl}
	Let $\msc$ be an MSC. If $\bowtie$ is cyclic, then $\msc$ is not $\nn$.
\end{restatable}

Let the \emph{Event Dependency Graph} (EDG) of a $\nn$ MSC $\msc$ be a graph that has events as nodes and an edge between two events $e_1$ and $e_2$ if $e_1 \bowtie e_2$. We now present an algorithm that, given the EDG of an $\nn$ MSC $\msc$, computes a $\nn$ linearization of $\msc$. We then show that, if $\bowtie$ is acyclic, this algorithm always terminates correctly. This, along with Proposition~\ref{prop:n_n_cycl}, effectively shows that Definition~\ref{def:n_n} and Definition~\ref{def:n_n_alt} are equivalent.

\paragraph*{Algorithm for finding a $\nn$ linearization}
The input of this algorithm is the EDG of an MSC $\msc$, and it outputs a valid $\nn$ linearization for $\msc$, if $\msc$ is $\nn$. The algorithm works as follows:
\begin{enumerate}
	\item If there is a matched send event $s$ with in-degree 0 in the EDG, add $s$ to the linearization and remove it from the EDG, along with its outgoing edges, then jump to step 5. Otherwise, proceed to step 2.
	\item If there are no matched send events in the EDG and there is an unmatched send event $s$ with in-degree 0 in the EDG, add $s$ to the linearization and remove it from the EDG, along with its outgoing edges, then jump to step 5. Otherwise, proceed to step 3.
		\item If there is a receive event $r$ with in-degree 0 in the EDG, such that $r$ is the receive event of the first message whose sent event was already added to the linearization, add $r$ to the linearization and remove it from the EDG, along with its outgoing edges, then jump to step 5. Otherwise, proceed to step 4.
		\item Throw an error and terminate.
		\item If all the events of $\msc$ were added to the linearization, return the linearization and terminate. Otherwise, go back to step 1.
\end{enumerate} 

We now need to show that 
\begin{enumerate*}[label={(\roman*)}]
	\item if this algorithm terminates correctly (i.e. step 4 is never executed), it returns a $\nn$ linearization, and 
	\item if $\bowtie$ is acyclic, the algorithm always terminates correctly.
\end{enumerate*}

\begin{proposition}
	If the above algorithm returns a linearization for an MSC $\msc$, it is a $\nn$ linearization.
\end{proposition}
\begin{proof}
	Step 2 ensures that the order (in the linearization) in which matched messages are sent is the same as the order in which they are received. Moreover, according to step 3, an unmatched send events is added to the linearization only if all the matched send events were already added.
\end{proof}

\davide{Insert here an example of the algorithm!}

The proof for the next proposition is quite intricate, and it can be found in Appendix~\ref{apx:MSO}.

\begin{restatable}{proposition}{nnalgotermination}
\label{prop:nn_algo_term}
	Given an MSC $\msc$, the above algorithm always terminates correctly if $\bowtie$ is acyclic.
\end{restatable}

Proposition~\ref{prop:n_n_cycl} and Proposition~\ref{prop:nn_algo_term} are sufficient to prove that Definition~\ref{def:n_n_alt} and Definition~\ref{def:n_n} are equivalent.
Based on Definition~\ref{def:n_n_alt}, we can finally write an MSO formula for $\nn$ MSCs:
\[
	\nnformula = \neg \exists x.\exists y.(\neg (x = y) \wedge x \bowtie y \wedge y \bowtie x)
\]
where we can define $x \bowtie y$ as
\[
	x \bowtie y =
	\begin{array}{rl}
	& \left(
		\bigvee_{\substack{a,b \in \psAct{\plh}}}
		(\lambda(x) = a \;\wedge\; \lambda(y) = b)
		\;\wedge\; \mathit{matched}(x) \;\wedge\; \neg \mathit{matched}(y)
	\right) \;\vee\\
	& (x \nnrel y) \quad \vee \quad \psi_3 \quad \vee \quad \psi_4\\
	\end{array}
\]

\noindent where $\psi_3$ and $\psi_4$ are defined as
\[
	\psi_3 =
	\begin{array}{rl}
		& \bigvee_{\substack{a,b \in \prAct{\plh}}}
		  (\lambda(x) = a \;\wedge\; \lambda(y) = b)
		  \;\wedge\; \\
		& \exists x'.\exists y'.(x' \lhd x \;\wedge\; y' \lhd y) \;\wedge\; (x' \nnrel y') \;\wedge\; \neg(x \nnrel y)\\
	\end{array}
\]
\[
	\psi_4 =
	\begin{array}{rl}
		& \bigvee_{\substack{a,b \in \psAct{\plh}}}
		  (\lambda(x) = a \;\wedge\; \lambda(y) = b)
		  \;\wedge\; \\
		& \exists x'.\exists y'.(x \lhd x' \;\wedge\; y \lhd y') \;\wedge\; (x' \nnrel y') \;\wedge\; \neg(x \nnrel y)\\
	\end{array}
\]

Note that $\nnrel$ is MSO-definable, since it is defined as the reflexive transitive closure of MSO-definable relations.

\paragraph{\bf Realizable with Synchronous Communication MSCs} 

Following the characterization given in \cite[Theorem 4.4]{DBLP:journals/dc/Charron-BostMT96}, we also give an alternative definition of $\rsc$-MSC (the equivalence with Definition~\ref{def:rsc} is shown in Appendix~\ref{apx:MSO}). We first need to introduce the concept of \emph{crown}.

\begin{definition} [crown]
	Let $\msc$ be an MSC. A \emph{crown} of size $k$ in $\msc$ is a sequence $\langle(s_i,r_i),\, i \in \{1,\dots,k\}\rangle$ of pairs of corresponding send and receive events such that
	\[
		s_1 \happensbeforestrict r_2, s_2 \happensbeforestrict r_3, \dots, s_{k-1} \happensbeforestrict r_k, s_k \happensbeforestrict r_1.
	\]
\end{definition}
\davide{Should I specify somewhere that $\happensbeforestrict = (\procrel \cup \lhd)^+$ or is it clear?}

\begin{definition} [$\rsc$ alternative]\label{def:rsc_alt}
	An MSC $\msc = (\Events,\procrel,\lhd,\lambda)$ is a \emph{RSC MSC} if and only if it does not contain any crown.
\end{definition}

%Such a linearization will be referred to as a \emph{$\rsc$ linearization}.

The following MSO formula derives directly from previous  definition:
\[\Phi_{\rsc} = \neg \exists s_1.\exists s_2. s_1 \varpropto s_2 \;\wedge\; s_2 \varpropto^\ast s_1
\]
\noindent where $\varpropto$ is defined as
\[
s_1 \varpropto s_2 =
\bigvee_{\substack{e \in \sAct}}(\lambda(s_1) = e) \;\wedge\;
s_1 \neq s_2 \;\wedge\;
\exists r_2. (s_1 < r_2 \;\wedge\; s_2 \lhd r_2)
\]


% THIS IS WRONG!
% We show that the opposite direction is also true, which implies that the class of $\nn$ MSCs is equivalent to the class of $onen$ MSCs. The following example gives an intuition of the formal proof, which will be given right after.

% \begin{proposition}
% 	Every $\onen$ MSC without unmatched messages is an $\nn$ MSC.
% \end{proposition}
% \begin{proof}
% Let $\msc$ be a $\onen$ MSC without unmatched messages, and let $L$ be a $\onen$ linearization. We will show that, by reordering some of the events in $L$, we are always able to obtain a $\nn$ linearization for $\msc$. The algorithm works as follow:
% \begin{enumerate}
% 	\item Find a pair $(m_1,m_2)$ of distinct messages such that their send order in $L$ is the inverse of the receive order\footnote{Note that $L$ is already a $\nn$ linearization if such a pair does not exist}. This can only happen if there is not a causal path between $s_1$ and $s_2$, i.e $s_1 \le \ge s_2$, where $s_i$ is the send event of message $m_i$. To see why, suppose w.l.o.g. that $s_1 \le s_2$. As seen in the proof of Proposition~\ref{prop:onen_mb_no_unmatched}, there must be a message $m_k$ sent by the same process that sent $s_1$, such that we have $r_1 \onenrel r_k \le r_2$, and in particular $r_1 \onenpartial r_2$. Therefore, if $s_1 \le s_2$, the receive order has to match the send order in $L$ and it will never be the opposite.
% 	\item Suppose, w.l.o.g. that $s_2 \linrel s_1$ and $r_1 \linrel r_2$, and we saw that $s_1 \le \ge s_2$. We would like to invert the order of $s_1$ and $s_2$ in the linearization, so that it matches the receive order. If we simply swap $s_2$ and $s_1$ in $L$, the new linearization could be invalid for $\msc$; for instance, there might be some events between $s_2$ and $s_1$ (in the linearization $L$) that need to happen before $s_1$. However, we show that it is still possible to invert the order of $s_1$ and $s_2$ without invalidating the linearization. Suppose that $s_2$ and $s_1$ are the $i$-th and the $j$-th events of $L$, respectively, where $i<j$. The idea is to move $s_2$ right before $s_1$, along with all the events on which $s_1$ depends, that are between $s_2$ and $s_1$ in $L$; we say that $s_1$ depends on an event $e$ if $e \onenpartial s_1$.
% \end{enumerate}
% \end{proof}





% \davidequestion{The following formula should be wrong... I cannot use $n$ in an MSO formula}
% \[
% 	\rscformula =
% 	\begin{array}{rl}
% 		& \forall x.\left(\bigvee_{a \in \psAct{\plh}} \lambda(x) = a \;\implies\; \mathit{matched}(x)\right) \;\wedge\; \\
% 		& \neg \left(
% 			\bigvee_{k=1}^n \left(
% 			\exists s_1 \cdots s_k. \exists r_1 \cdots r_k.
% 			\bigwedge_{i=1}^k (s_i \lhd r_i \;\wedge\; s_i \le r_{(i+1)\%k})
% 		\right)
% 		\right)\\
% 	\end{array}
% \]
% where $n$ is the total number of messages. The formula checks that there are no unmatched send events and that in the conflict graph there is no cycle, of any length, whose edges are all SR.
