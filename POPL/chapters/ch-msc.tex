% !TEX root = ../popl-paper.tex

In this section, we give both informal descriptions and formal definitions of the communication models that will be considered in the paper. All of them impose different constraints on the order in which messages can be received.
% Implementations, or realizations, of these communication models are discussed in the following section. For convenience, we will refer to a system implementing a given communication model $\comsymb$ as a $\comsymb$-system.
% Similarly, an MSC that represents a valid computation for a communication model $\comsymb$ will be called a $\comsymb$-MSC.

 %In this section, we will present  seven different asynchronous communication models.
%We model a distributed system as a set of concurrent Finite-State Machines (FSMs) that exchange messages asynchronously through channels.
%Each FSM models a single machine/process of the system and transitions are labeled with "send" and "receive" operations, which specify the sender and the receiver of a message. In our work
%The role of the communication model is to impose an order on the reception of messages, according to its specification. For instance, the delivery of a message could be delayed or even prevented by a communication model $CM$, so as to ensure that messages are received in an order that is valid for $CM$. The seven communication models that we address all impose different constraints on the order in which messages can be received.
%

We will use the following customary conventions:  $\abinrel^+$ denotes the transitive closure of a binary relation $\abinrel$, while $\abinrel^*$ denotes the transitive and reflexive closure. When $R^*$ is denoted by a symbol suggesting 
a partial order, like $\leq$, we write e.g. 
$<$ for $R^+$.  The cardinality of a set $A$ is  $\cardinalof{A}$.
%
%\paragraph*{Processes, messages, and actions}
We assume a finite set of \emph{processes} $\Procs=\{p,q,\ldots\}$ and a finite set of messages $\Msg=\{\msg,\ldots\}$.
Each process may either (asynchronously) send a message to another one, or wait until it receives a message.
We therefore consider two kinds of actions. A \emph{send action} is of the form $\sact{p}{q}{\msg}$;
it is executed by process $p$ and sends message $\msg$ to process $q$.
The corresponding \emph{receive action} executed by $q$ is $\ract{p}{q}{\msg}$.
%
We write $\pqsAct{p}{q}$ to denote the set $\{\sact{p}{q}{\msg} \mid \msg \in \Msg\}$, and
$\pqrAct{p}{q}$ for the set $\{\ract{p}{q}{\msg} \mid \msg \in \Msg\}$.
Similarly, for $p \in \Procs$, we set
$\psAct{p} = \{\sact{p}{q}{\msg} \mid q \in \Procs
% Etienne: shall we forbid processes to send messages to themselves?
%\setminus \{p\}
\}$ and $\msg \in \Msg\}$, etc.
Moreover, $\pAct{p} = \psAct{p} \cup \qrAct{p}$ denotes the set of all actions that are
executed by $p$.
Finally, $\Act = \bigcup_{p \in \Procs} \pAct{p}$
is the set of all the actions.
\davide{Should we specify somewhere how the notations $"send(,,)"$ and $!i$ (same for receive) are related? I suppose they could confuse the reader, since we basically use them interchangeably in the paper. Also, the difference between action and event is not very clear IMO.}


\paragraph{\bf Fully asynchronous communication}
In the fully asynchronous communication model (\asy), messages can be received at any time once they have been sent, and send events are non-blocking.
%, i.e., the sender of a message does not have to wait for it to be delivered to the recipient, in order to resume normal operations.
It can be modeled as a bag where all messages are stored and retrieved by processes when necessary (as described in \cite{DBLP:journals/fac/ChevrouHQ16} and \cite{DBLP:journals/tcs/BasuB16}).
It is also referred to as NON-FIFO (cfr.  \cite{DBLP:journals/dc/Charron-BostMT96}).
An MSC that shows a valid computation for the fully asynchronous communication model will be called a fully asynchronous MSC (or simply MSC). An example of such an MSC can be found in Fig.~\ref{fig:fully_asy_ex}; indeed, even if message $m_1$ is sent before $m_2$, process $q$ does not have to receive $m_1$ first. Below, we give the formal definition of MSC.

\begin{definition}[MSC]\label{def:msc}
	An {MSC}  over $\Procs$ and $\Msg$ is a tuple $\msc = (\Events,\procrel,\lhd,\lambda)$, where 
	$\Events$ is a finite (possibly empty) set of \emph{events}, $\lambda: \Events \to \Act$ is a labelling 
	function that associates an action to each event,
	and $\procrel,\lhd$ are binary relations on $\Events$ that satisfy the following three conditions.
	For $p \in \Procs$, let $\Events_p = \{e \in \Events \mid \lambda(e) \in \pAct{p}\}$ be 
	the set of events that are executed by $p$. 
	\begin{enumerate}
		\item The \emph{process relation} $\procrel\subseteq \Events \times \Events$ 
		relates an event to its immediate successor on
		the same process:
		$\procrel=\bigcup_{p \in \Procs} \procrel_p$ for some 
		relations ${\procrel_p} \subseteq \Events_p \times \Events_p$ such that $\procrel_p$ is 
		the direct successor relation of a total order on $\Events_p$.  
		\item The \emph{message relation} ${\lhd} \subseteq \Events \times \Events$ 
		relates pairs of matching send/receive events: 	
		\begin{enumerate}%\itemsep=0.5ex
			\item[(2a)] for every pair $(e,f) \in {\lhd}$, there are processes two $p,q$ and a message $m$ such that $\lambda(e) = \sact{p}{q}{\msg}$ and $\lambda(f) = \ract{p}{q}{\msg}$.
			\item[(2b)] for all $f \in \Events$ such that $\lambda(f) = \ract{p}{q}{\msg}$, %is a receive action, 
			there is exactly one $e \in \Events$ such that $e \lhd f$.
		\end{enumerate}
		\item The \emph{happens-before} relation\footnote{This relation was introduced in~\cite{Lamport78}, and is also referred to as the \emph{happened before} relation,
		or sometimes \emph{causal relation} or \emph{causality relation}, e.g. in~\cite{DBLP:journals/dc/Charron-BostMT96,DBLP:conf/cav/BouajjaniEJQ18} .} ${\happensbefore}$, defined by $({\procrel} \cup {\lhd})^\ast$,
		is a partial order on $\Events$.
	\end{enumerate}
\end{definition}

 If, for two events $e$ and $f$, we have that $e \happensbefore f$, we   say that there is a \emph{causal path} between $e$ and $f$.
%For an event $e \in \Events$, a set of actions $A \subseteq \Act$, and a relation $\rel \subseteq \Events \times \Events$,
%let $\sametype{e}{A}{\rel} = \cardinalof{\{f \in \Events \mid (f,e) \in \rel$ and $\lambda(f) \in A\}}$.
%\davide{I think that this notation of $\sametype{e}{A}{\rel}$ should be removed here and just introduced in the last chapter (if I recall correctly we only use it for the existentially k-bounded MSCs).}
Definition~\ref{def:msc} of (fully asynchronous) MSC will serve as a basis on which the other communication models will build on, adding some additional constraints.

According to Condition (2), every receive event must have a matching send event. However, note that, there may be unmatched send events. An unmatched send event represents the scenario in which the recipient is not ready to receive a specific message. This is the case of message $m_1 $ in  Fig. \ref{fig:asy_um_ex}.
We will always depict unmatched messages with dashed arrows pointing to the time line of the
destination process (note that the vertical position of the arrow on this timeline does not correspond to any event).
We let
$\SendEv{\msc} = \{e \in \Events \mid \lambda(e)$ is a send
action$\}$,
$\RecEv{\msc} = \{e \in \Events \mid \lambda(e)$ is a receive
action$\}$,
$\Matched{\msc} = \{e \in \Events \mid$ there is $f \in \Events$
such that $e \lhd f\}$, and
$\Unm{\msc} = \{e \in \Events \mid \lambda(e)$ is a send
action and there is no $f \in \Events$ such that $e \lhd f\}$.
%
%\etienne{Add an example of a MSC with unmatched sends and illustrate the definition}

\begin{figure}[t]
		\captionsetup[subfigure]{justification=centering}
	% \centering
	\begin{subfigure}[t]{0.3\textwidth}\centering

		\begin{tikzpicture}[scale=0.7, every node/.style={transform shape}]
			\newproc{0}{p}{-2.2};
			\newproc{2}{q}{-2.2};

			\newmsgm{0}{2}{-0.5}{-1.7}{1}{0.1}{black};
			\newmsgm{0}{2}{-1.7}{-0.5}{2}{0.25}{black};

			\end{tikzpicture}
		\caption{\asy.}	\label{fig:fully_asy_ex}

		\end{subfigure}
%
%		\begin{subfigure}[t]{0.25\textwidth}
%	\begin{center}
%		\begin{tikzpicture}[scale=0.7, every node/.style={transform shape}]
%			\newproc{0}{p}{-2.2};
%			\newproc{1}{q}{-2.2};
%			\newproc{2}{r}{-2.2};
%
%			\newmsgm{0}{1}{-0.3}{-1.7}{1}{0.1}{black};
%			\newmsgm{0}{2}{-0.7}{-0.7}{2}{0.7}{black};
%			\newmsgm{2}{1}{-1.3}{-1.3}{3}{0.3}{black};
%			\newmsgm{2}{1}{-1.9}{-1.9}{4}{0.3}{black};
%
%			\end{tikzpicture}
%		\caption{A \pp MSC.}
%		\label{fig:pp_ex}
%	\end{center}
%\end{subfigure}
%
	% \centering
	\begin{subfigure}[t]{0.3\textwidth}\centering
		\begin{tikzpicture}[scale=0.7, every node/.style={transform shape}]
			\newproc{0}{p}{-2.2};
			\newproc{1}{q}{-2.2};
			\newproc{2}{r}{-2.2};

			\newmsgm{0}{1}{-0.3}{-1.7}{1}{0.1}{black};
			\newmsgm{0}{2}{-0.9}{-0.9}{2}{0.7}{black};
			\newmsgm{2}{1}{-1.5}{-1.5}{3}{0.3}{black};
			\newmsgm{2}{1}{-2}{-2}{4}{0.3}{black};

			% \newflechevert{Purple}{0}{-0.3}{-0.9};
			% \newflechehor{Purple}{-0.9}{0}{2};
			% \newflechevert{Purple}{2}{-0.9}{-1.5};
		\end{tikzpicture}
		\caption{\asy, $\pp$.} \label{fig:pp_ex}
	\end{subfigure}
	% \hfill
	\begin{subfigure}[t]{0.3\textwidth}\centering
		\begin{tikzpicture}[scale=0.7, every node/.style={transform shape}]
			\newproc{0}{p}{-2.2};
			\newproc{1}{q}{-2.2};
			\newproc{2}{r}{-2.2};

			\newmsgm{0}{2}{-0.3}{-2}{1}{0.1}{black};
			\newmsgm{0}{1}{-1.3}{-1.3}{2}{0.3}{black};
			\newmsgm{2}{1}{-1.5}{-1.5}{3}{0.3}{black};

		\end{tikzpicture}
		\caption{\asy, \pp, \co, \mb, $\onen$, $\nn$.}	    
		\label{fig:co_ex}
	\end{subfigure}
\begin{subfigure}[t]{0.3\textwidth}\centering
	\begin{center}
		\begin{tikzpicture}[scale=0.7, every node/.style={transform shape}]
			\newproc{0}{p}{-2.2};
			\newproc{1}{q}{-2.2};
			\newproc{2}{r}{-2.2};

			\newmsgm{0}{1}{-0.5}{-0.5}{1}{0.3}{black};
			\newmsgm{1}{2}{-1}{-1}{2}{0.3}{black};
			\newmsgm{1}{0}{-1.6}{-1.6}{3}{0.3}{black};

		\end{tikzpicture}
		\caption{$\rsc$.}
		\label{fig:rsc_ex}
	\end{center}
\end{subfigure}
\begin{subfigure}[t]{0.3\textwidth}\centering

	\begin{tikzpicture}[scale=0.7, every node/.style={transform shape}]
		\newproc{0}{p}{-2.2};
		\newproc{2}{q}{-2.2};

		\newmsgum{0}{2}{-0.8}{1}{0.2}{black};
		\newmsgm{0}{2}{-1.6}{-1.6}{2}{0.2}{black};

		\end{tikzpicture}
	\caption{\asy.}	
	\label{fig:asy_um_ex}
\end{subfigure}
% \centering
\begin{subfigure}[t]{0.3\textwidth}\centering
	\begin{tikzpicture}[scale=0.7, every node/.style={transform shape}]
		\newproc{0}{p}{-2.2};
		\newproc{1}{q}{-2.2};
		\newproc{2}{r}{-2.2};

		\newmsgum{0}{1}{-0.8}{1}{0.3}{black};
		\newmsgm{0}{2}{-1.6}{-1.6}{2}{0.15}{black};
	\end{tikzpicture}
	\caption{\asy, $\pp$, $\co$, $\none$.} 
	\label{fig:pp_um_ex}
\end{subfigure}
		\caption{Examples of MSCs for various communication models.}\label{fig:exmscs}
\end{figure}

%\davide{I think it would be a good idea to add another set of MSC examples that contain unmatched messages.}

%\paragraph*{Linearizations.}
Intuitively, a linearization represents the order in which events are executed by the distributed system according to \emph{absolute time}, i.e., as they are seen by an external viewer that has a global view of all the processes. More formally, let $\msc = (\Events,\procrel,\lhd,\lambda)$ be an MSC.
A \emph{linearization} of $\msc$ is a (reflexive) total order ${\linrel} \subseteq \Events \times \Events$ such that ${\happensbefore} \subseteq {\linrel}$. In other words, a linearization of $\msc$ represents a possible way to schedule its events. For convenience, we will omit the relation $\linrel$ when writing a linearization, e.g., $!1\;!3\;!2\;?2\;?3\;?1$ is a possible linearization of the MSC in Fig. \ref{fig:co_ex}.
%\davide{Provide example of linearization.}

%\etienne{I just moved here the notion of concatenation of MSC}
%\paragraph*{Concatenation}
MSCs form a monoid for the following notion of vertical concatenation.
Let $\msc_1 = (\Events_1,\procrel_1,\lhd_1,\lambda_1)$ and
$\msc_2 = (\Events_2,\procrel_2,\lhd_2,\lambda_2)$ be two MSCs.
The \emph{concatenation} $\msc_1 \cdot \msc_2$ of $\msc_1$ and $\msc_2$ is the MSC 
$(\Events,\procrel,\lhd,\lambda)$ where $\Events$ is the disjoint 
union of $\Events_1$ and $\Events_2$,
${\lhd}  = {\lhd_1} \cup {\lhd_2}$, $\lambda(e)=\lambda_i(e)$ for all $e\in \Events_i$ ($i=1,2$); 
moreover, ${\procrel} = {\procrel_1} \cup {\procrel_2} \cup R$,
were $R$ is the set of event pairs $(e_1,e_2)$ 
such that there is a process $p \in \Procs$ with $e_1$ the maximal event of 
$(\Events_1)_p$ and $e_2$ the minimal event of
$(\Events_2)_p$.
Note that $\msc_1 \cdot \msc_2$ is indeed an MSC and that
concatenation is associative.


\paragraph{\bf  Peer-to-peer communication}
In the peer-to-peer ($\pp$) communication model, any two messages sent from one process to another  are always received in the same order as they are sent. A straightforward implementation would be connecting processes pairwise with FIFO channels. %i.e. messages are delivered by channels in the order in which they were sent\footnote{Please note that our definition of Communicating Finite-State Machine is different from the classical one. FIFO channels are replaced by bag channels, which do not ensure any specific order on the delivery of messages.}. This definition of Communicating Finite-State Machines clearly uses the $\oneone$ communication model, since we have FIFO channels between processes that take care of delivering messages in the correct order. The $\oneone$ communication model is referred to as \pp in \cite{BolligGFLLS21}.
For this reason, alternative names for this communication model are FIFO $1\mathsf{-}1$ \cite{DBLP:journals/fac/ChevrouHQ16} or simply FIFO \cite{babaoglu1993consistent, DBLP:journals/dc/Charron-BostMT96, tel2000introduction}.
MSCs that show valid computations for the \pp communication model will be called \pp-MSCs.
The MSC shown in Fig.~\ref{fig:fully_asy_ex} is not a $\pp$-MSC, as $m_1$ cannot be received after $m_2$.
Fig.~\ref{fig:pp_ex} shows an example of \pp-MSC; the only two messages sent by and to the same process are $m_3$ and $m_4$, which are received in the same order as they are sent. %Below the formal definition of $\oneone$-MSC.

\begin{definition}[$\oneone$-MSCs]\label{def:pp_msc}
	A $\oneone$-MSC is an MSC $\msc = (\Events,\procrel,\lhd,\lambda)$ where, for any two send events $s$ and $s'$ such that $\lambda(s)=\pqsAct{p}{q}$, $\lambda(s')=\pqsAct{p}{q}$, and $s \procrel^+ s'$, one of the following holds
	\begin{itemize}%\itemsep=0.5ex
		\item either $s,s' \in \Matched{\msc}$ with $s \lhd r$ and $s' \lhd r'$ and $r \procrel^+ r'$,  %where $r$ and $r'$ are two receive events executed by $q$ such that .
		\item or $s' \in \Unm{\msc}$.
	\end{itemize}
	
\end{definition}

Note that, according to this definition, we cannot have two messages $m_1$ and $m_2$, both sent by  $p$ to $q$ in that order, such that $m_1$ is unmatched and $m_2$ is matched;  unmatched message $m_1$ excludes the reception of any later message. For this reason, the MSC shown in Fig.~\ref{fig:asy_um_ex} is not $\pp$. On the other hand, the MSC in Fig.~\ref{fig:pp_um_ex} is $\pp$ because the two messages are not sent by and addressed to the same process.

\paragraph{\bf  Causally ordered communication}
In the causally ordered (\co) communication model, messages are delivered to a process according to the causality of their emissions. In other words, if there are two messages $m_1$ and $m_2$ with the same recipient, such that $m_1$ is causally sent before $m_2$ (i.e., there exists a causal path from the first send to the second one), then $m_1$ must be received before $m_2$.
This type of partial order was introduced by Lamport in \cite{Lamport78} with the "happened before" order. Later on, some implementations were proposed in \cite{peterson1989preserving, DBLP:conf/wdag/SchiperES89, kshemkalyani1998necessary}. %and in \cite{coulouris2005distributed}, where the causal order is called FIFO ordering.
%\davidequestion{I don't understand the last citation about FIFO ordering. I looked into the book and it talks both about FIFO ordering and causal ordering, and FIFO ordering doesn't seem to match Lamport's definition of causal order...}
Fig.~\ref{fig:pp_ex}, shows an example of non-causally ordered MSC; there is a causal path between the sending of $m_1$ and $m_3$, hence $m_1$ should be received before $m_3$, which is not the case here. On the other hand, Fig.~\ref{fig:co_ex} is \co; note that the only two messages with the same recipient are $m_2$ and $m_3$, but there is no causal path between their respective send events. Below the formal definition of causally ordered MSC (\co-MSC).

\begin{definition}[\co-MSC]\label{def:co_msc}
	An MSC $\msc = (\Events,\procrel,\lhd,\lambda)$ is \emph{causally ordered} if, for any two send events $s$ and $s'$, such that $\lambda(s)=\pqsAct{\plh}{q}$, $\lambda(s')=\pqsAct{\plh}{q}$, and $s \happensbefore s'$, we have either:
	\begin{itemize}%\itemsep=0.5ex
		\item $s,s' \in \Matched{\msc}$ with  $s \lhd r$ and $s' \lhd r'$, and $r \procrel^* r'$, or %where $r$ and $r'$ are two receive events such that.
		\item $s' \in \Unm{\msc}$.
	\end{itemize}
\end{definition}

Note that in a \co-MSC we cannot have two send events $s$ and $s'$ addressed to the same process, such that $s$ is unmatched, $s'$ is matched, and $s \happensbefore s'$. 
%The MSC shown in Fig.~\ref{fig:pp_um_ex} has two send events that are causally related, i.e. $!1 \happensbefore !2$, and the first message $m_1$ is unmatched; however, it is still a $\co$-MSC because the two messages are not addressed to the same process.

\paragraph{\bf Mailbox communication}
In the mailbox ($\mb$) communicating model, any two messages sent to a process  must be received in the same order as they are sent (according to absolute time). These two messages might be sent by different processes and the two send events might be concurrent (i.e., there is no causal path between them). In other words, if a process  receives $m_1$ before $m_2$, then $m_1$ must have been sent before $m_2$. Essentially, $\mb$ coordinates all the senders of a single receiver. For this reason the model is also called FIFO $n\mathsf{-}1$ \cite{DBLP:journals/fac/ChevrouHQ16}.   A high-level implementation of the mailbox communication model could consist in a single incoming FIFO channel for each process $p$, in which all processes enqueue their messages to $p$. 
A low-level implementation can be obtained thanks to a shared real-time clock~\cite{cristian1999timed} or a global agreement on the order of events~\cite{defago2004total, raynal2010communication}.
The MSC shown in Fig.~\ref{fig:pp_ex} is not a mailbox MSC; $m_1$ and $m_3$ have the same recipient, but they are not received in the same order as they are sent. The MSC in Fig.~\ref{fig:co_ex} is mailbox; indeed, we are able to find a linearization that respects the mailbox constraints, such as $!1\;!2\;!3\;?2\;?3\;?1$ (note that $m_2$ is both sent and received before $m_3$). Below the definition of $\none$-MSC.

\begin{definition}[$\none$-MSC]\label{def:mb_msc}
	An MSC $\msc = (\Events,\procrel,\lhd,\lambda)$ is a \emph{$\none$-MSC} if it has a linearization $\linrel$ where, for any two send events $s$ and $s'$, such that $\lambda(s)=\pqsAct{\plh}{q}$, $\lambda(s')=\pqsAct{\plh}{q}$, and $s \linrel s'$
	\begin{itemize}%\itemsep=0.5ex
		\item either $s,s' \in \Matched{\msc}$. Note that $r \linrel r'$, since we have that $r \procrel^+ r'$,
		\item or $s' \in \Unm{\msc}$.
	\end{itemize}
\end{definition}

Such a linearization will be referred to as a \emph{$\none$-linearization}. Note that the definition of $\none$-MSC is based on the \emph{existence} of a linearization. The same kind of "existential" definition will be used for the remaining communication models. In practice, to claim that an MSC is $\none$, we just need to find a single valid $\none$-linearization, regardless of all the others. As with $\co$-MSCs, a $\none$-MSC cannot have two ordered send events $s$ and $s'$ addressed to the same process, such that $s$ is unmatched, $s'$ is matched. The message related to $s$ would indeed block the buffer and prevent all subsequent receptions included the receive event matching $s'$. 
%
%
%and $s \happensbefore s'$; indeed, it would not be possible to find a $\mb$-linearization, since $s \happensbefore s'$ implies $s \linrel s'$ for any linearization, but $s$ is unmatched and $s'$ is matched, which does not fall into either case of Definition~\ref{def:mb_msc}. 
At this stage, the difference between $\co$-MSCs and $\mb$-MSCs might be unclear. Section~\ref{sec:hierarchy} will clarify how all the classes of MSCs that we introduce are related to each other.


\paragraph{\bf  $\onen$ communication}
The $\onen$ communicating model is the dual of $\none$, it coordinates a sender with all the receivers. Any two messages sent by a process  must be received in the same order (in absolute time) as they are sent. These two messages might be received by different processes and the two receive events might be concurrent.
% In other words, if a process $p$ sends $m_1$ before $m_2$, then $m_1$ must be received before $m_2$ in absolute time.
A high-level implementation of the $\onen$ communication model could consist in a single outgoing FIFO channel for each process, which is shared by all the other processes. A send event would then push a message on the outgoing FIFO channel.
%As this type of communication is the dual of the $\none$ one, the implementation would require similar tools as above.
The MSC shown in Fig.~\ref{fig:pp_ex} is not a $\onen$ MSC; $m_1$ and $m_2$ are sent in this order by the same process, but they are received in the opposite order (note that there is a causal path between the reception of $m_2$ and the reception of $m_1$, so $?2$ happens before $?1$ in every linearization of this MSC). Fig.~\ref{fig:co_ex} shows an example of $\onen$ MSC; $m_1$ is sent before $m_2$ by the same process, and we are able to find a linearization where $m_1$ is received before $m_2$, such as $!1\;!2\;!3\;?1\;?2\;?3$. 

\begin{definition}[$\onen$ MSC]\label{def:one_n}
An MSC $\msc = (\Events,\procrel,\lhd,\lambda)$ is a \emph{$\onen$ MSC} if it has a linearization $\linrel$ where, for any two send events $s$ and $s'$, such that $\lambda(s)=\pqsAct{p}{\plh}$, $\lambda(s')=\pqsAct{p}{\plh}$, and $s \procrel^+ s'$ (which implies $s \linrel s'$)
\begin{itemize}%\itemsep=0.5ex
	\item either $s,s' \in \Matched{\msc}$ and $r \linrel r'$, with  $r$ and $r'$  two receive events such that $s \lhd r$ and $s' \lhd r'$,
	\item or $s' \in \Unm{\msc}$.
\end{itemize}
\end{definition}

Such a linearization will be referred to as a \emph{$\onen$ linearization}. Note that a $\onen$ MSC cannot have two send events $s$ and $s'$, executed by the same process, such that $s$ is unmatched, $s'$ is matched, and $s \procrel^+ s'$; indeed, it would not be possible to find a $\onen$ linearization, according to Definition~\ref{def:one_n}. The MSCs shown in Fig.~\ref{fig:asy_um_ex} and Fig.~\ref{fig:pp_um_ex} are clearly not $\onen$.


\paragraph{\bf  $\nn$ communication}
In the $\nn$ communicating model, messages are globally ordered and delivered according to  their emission order. Any two messages must be received in the same order as they are sent, in absolute time. These two messages might be sent or received by any process and the two send or receive events might be concurrent.
%In other words, if a message $m_1$ is sent before $m_2$ in absolute time, then $m_1$ must be received before $m_2$ in absolute time.
The $\nn$ coordinates all the senders with all the receivers. A high-level implementation of the $\nn$ communication model could consist in a single FIFO channel shared by all processes. It is considered also in \cite{DBLP:journals/tcs/BasuB16} where it is called  many-to-many (denoted $^\ast$-$^\ast$). However, as underlined in \cite{DBLP:journals/fac/ChevrouHQ16}, such an implementation would be inefficient and unrealistic.
The MSC shown in Fig.~\ref{fig:pp_ex} is clearly not a $\nn$ MSC; if we consider messages $m_1$ and $m_2$ we have that, in every linearization, $!1 \happensbefore !2$ and $?2 \happensbefore ?1$. This violates the constraints imposed by the $\nn$ communication model. The MSC in Fig.~\ref{fig:co_ex} is $\nn$ because we are able to find a linearization that satisfies the $\nn$ constraint, e.g. $!1\;!2\;!3\;?1\;?2\;?3$.

\begin{definition}[$\nn$ MSC]\label{def:n_n}
	An MSC $\msc = (\Events,\procrel,\lhd,\lambda)$ is a \emph{$\nn$ MSC} if it has a linearization $\linrel$ where, for any two send events $s$ and $s'$, such that $s \linrel s'$
	\begin{itemize}%\itemsep=0.5ex
		\item either $s,s' \in \Matched{\msc}$ and $r \linrel r'$, with $r$ and $r'$  two receive events such that $s \lhd r$ and $s' \lhd r'$,
		\item or $s' \in \Unm{\msc}$.
	\end{itemize}
\end{definition}

Such a linearization will be referred to as a \emph{$\nn$ linearization}. Note that, in a $\nn$ linearization, unmatched messages can be sent only after all matched messages have been sent.
As a consequence, a $\nn$ MSC cannot have an unmatched send event $s$ and a matched send event $s'$, such that $s \happensbefore s'$; indeed, $s$ would appear before $s'$ in every linearization, and we would not be able to find a $\nn$ linearization. The MSCs shown in Fig.~\ref{fig:asy_um_ex} and Fig.~\ref{fig:pp_um_ex} are both not $\nn$, since we have unmatched messages that are sent before matched messages.


\paragraph{\bf RSC communication}
The Realizable with Synchronous Communication ($\rsc$) communication model imposes the existence of a scheduling such that any send event is  immediately followed by its corresponding receive event. It was introduced in \cite{DBLP:journals/dc/Charron-BostMT96}, and it is the asynchronous model that comes closest to synchronous communication. % An asynchronous distributed system that implements the $\rsc$ communication model effectively behaves as a synchronous system. 
%The authors of \cite{kshemkalyani2011distributed} propose a strategy to implement RSC executions from a synchronous system.  
The MSC  in Fig.~\ref{fig:rsc_ex} is the only example of $\rsc$-MSC: for instance linearization $!1\;?1\;!2\;?2\;!3\;?3$ respects the constraints of the $\rsc$ communication model. 
%Such a linearization will be referred to as an \emph{$\rsc$ linearization}.
% EL : repeated below.
%Let $\rscMSCs$ be the set of $\rsc$ MSCs.

\begin{definition}[$\rsc$-MSC]\label{def:rsc}
	An MSC $\msc = (\Events,\procrel,\lhd,\lambda)$ is an \emph{\rsc-MSC} if it has no unmatched send events and there is a linearization $\linrel$ where any matched send event is immediately followed by its respective receive event.
\end{definition}

Such a linearization will be referred to as an \emph{$\rsc$ linearization}.


\paragraph*{Classes of MSCs} 
We denote by $\asMSCs$ (resp. $\ppMSCs$, $\coMSCs$, $\mbMSCs$, $\onenMSCs$, $\nnMSCs$, $\rscMSCs$) the sets of all MSCs (resp. $\pp$-MSCs, $\co$-MSCs, $\mb$-MSCs, $\onen$ MSCs, $\nn$ MSCs, $\rsc$ MSCs) over the given sets $\Procs$ and $\Msg$. Note that we do not differentiate between isomorphic MSCs.
