% !TEX root = ../popl-paper.tex


In Section~\ref{sec:com_models_overview} we gave a high-level description of 7 communication models. Here, we formally define those classes and  show that they are all MSO-definable, i.e. there is a Monadic Second Order Logic formula that defines each of them. 
We start by  recalling the formal definition of asynchronous MSC and MSO logic.

\paragraph{\bf Preliminaries.}
Asynchronous MSCs (simply MSC in the following) represent valid computations for asynchronous systems. This is the most general definition of MSC, and it will serve as a basis on which the other communication models will build on,  adding some additional constraints. We assume a finite set of processes $\Procs$ and a finite set of messages $\Msg$. A send action is of the form $\sact{p}{q}{\msg}$.
It is executed by $p$ and sends message $\msg$ to $q$.
The corresponding receive action, executed by $q$, is
$\ract{p}{q}{\msg}$.
%
$\pqsAct{p}{q} = \{\sact{p}{q}{\msg} \mid \msg \in \Msg\}$ and
$\pqrAct{p}{q} = \{\ract{p}{q}{\msg} \mid \msg \in \Msg\}$.
For $p \in \Procs$, we set
$\psAct{p} = \{\sact{p}{q}{\msg} \mid q \in \Procs \setminus \{p\}$ and $\msg \in \Msg\}$, etc.
Moreover, $\pAct{p} = \psAct{p} \cup \qrAct{p}$ will denote the set of all actions that are
executed by $p$.
Finally, $\Act = \bigcup_{p \in \Procs} \pAct{p}$
is the set of all the actions.


\begin{definition}[Asynchronous MSC]
An \emph{asynchronous MSC}  over $\Procs$ and $\Msg$ is a tuple $\msc = (\Events,\procrel,\lhd,\lambda)$, where $\Events$ is a finite (possibly empty) set of \emph{events} and $\lambda: \Events \to \Act$ is a labelling function that associates an action to each event. For $p \in \Procs$, let $\Events_p = \{e \in \Events \mid \lambda(e) \in \pAct{p}\}$ be the set of events that are executed by $p$. We require that $\procrel$ (the \emph{process relation}) is the disjoint union $\bigcup_{p \in \Procs} \procrel_p$ of relations ${\procrel_p} \subseteq \Events_p \times \Events_p$ such that $\procrel_p$ is the direct successor relation of a total order on $\Events_p$.  We require that ${\lhd} \subseteq \Events \times \Events$ (the \emph{message relation}) satisfies the following:
\begin{itemize}\itemsep=0.5ex
\item[(1)] for every pair $(e,f) \in {\lhd}$, there is a send action $\sact{p}{q}{\msg} \in \Act$ such that $\lambda(e) = \sact{p}{q}{\msg}$, $\lambda(f) = \ract{p}{q}{\msg}$.
\item[(2)] for all $f \in \Events$ such that $\lambda(f)$ is a receive action, there is exactly one $e \in \Events$ such that $e \lhd f$.
\end{itemize}
Finally, letting ${\le}_\msc = ({\procrel} \cup {\lhd})^\ast$,
we require that $\le_\msc$ is a partial order. For convenience, we simply write $\le$ when $M$ is clear from the context. We will refer to $\le$ as the \emph{happens-before} relation\footnote{In the literature, the relation is also referred to as the \emph{causal ordering} relation}. If, for two events $e$ and $f$, we have that $e \le f$, we   say that there is a \emph{causal path} between $e$ and $f$.
For an event $e \in \Events$, a set of actions $A \subseteq \Act$, and a relation $\rel \subseteq \Events \times \Events$,
let $\sametype{e}{A}{\rel} = |\{f \in \Events \mid (f,e) \in \rel$ and $\lambda(f) \in A\}|$.
\end{definition}

According to Condition (2), every receive event must have a matching send event. Note that, however, there may be unmatched send events. An unmatched send event represents the scenario in which the recipient is not ready to receive a specific message.
We let
$\SendEv{\msc} = \{e \in \Events \mid \lambda(e)$ is a send
action$\}$,
$\RecEv{\msc} = \{e \in \Events \mid \lambda(e)$ is a receive
action$\}$,
$\Matched{\msc} = \{e \in \Events \mid$ there is $f \in \Events$
such that $e \lhd f\}$, and
$\Unm{\msc} = \{e \in \Events \mid \lambda(e)$ is a send
action and there is no $f \in \Events$ such that $e \lhd f\}$.
%
We do not distinguish isomorphic MSCs and
let $\asMSCs$ be the set of all the asynchronous MSCs over the given sets $\Procs$ and $\Msg$. 
%
% \paragraph*{Linearizations.}
%
% Consider $\msc = (\Events,\procrel,\lhd,\lambda) \in \asMSCs$.
% An \emph{asynchronous linearization} of $\msc$ is a (reflexive) total order ${\linrel} \subseteq \Events \times \Events$ such that ${\le_\msc} \subseteq {\linrel}$. Intuitively, an asynchronous linearization of $\msc$ is a possible ordering of its events.
% \davide{Example of asynchronous linearization.}
%
A \emph{linearization} of $\msc$ is a total order on the events that respects the happens-before relation $\le_\msc$ defined over $\msc$, more formally is a (reflexive) total order ${\linrel} \subseteq \Events \times \Events$ such that ${\le_\msc} \subseteq {\linrel}$. 

Next we briefly recall some notion of MSO logic.

\begin{definition}[MSO logic]
The set of MSO formulas over MSCs (over $\Procs$ and $\Msg$) is given by the grammar
$
\phi ::= true \mid x \procrel y \mid x \lhd y \mid \lambda(x) = a \mid x = y \mid x \in X \mid \exists x.\phi \mid \exists X.\phi \mid \phi \vee \phi \mid \neg \phi
$,
where $a \in \Act$, $x$ and $y$ are first-order variables (taken from an infinite set of variables), interpreted as
events of an MSC, and $X$ is a second-order variable, interpreted
as a set of events. 
\end{definition}

We use common abbreviations such as $\wedge$, $\Rightarrow$, $\forall$, etc. The satisfaction relation is defined in the standard way and self-explanatory.
For instance, the formula $\neg\exists x.(\bigvee_{a \in \sAct} \lambda(x) = a \;\wedge\; \neg \mathit{matched}(x))$, 
with $\mathit{matched}(x) = \exists y.x \lhd y$, 
says that there are no unmatched send events. All MSCs of Figure \ref{fig:exmscs} satisfy the formula. Given a sentence $\phi$, i.e., a formula without free variables,
we let $L(\phi)$ denote the set of MSCs that satisfy $\phi$. Since we have defined the set of MSO formulas over asynchronous MSCs, the formula $\asformula = true$ describes the whole set of asynchronous MSCs, i.e. $L(\asformula) = \asMSCs$. It is worth mentioning that the (reflexive) transitive closure of a binary relation defined by an MSO formula (more detail are given in Appendix~\ref{apx:MSO}) with free variables $x$ and $y$, such as $x \procrel y$, is MSO-definable. The logic can then freely use formulas of the form $x \procrel^+ y$, $x \procrel^* y$ or $x \le y$ (where $\le$ is interpreted as $\le_\msc$ for the given MSC $\msc$).

As mentioned above, the class of asynchronous MSCs is the biggest and most general one out of the seven. All the others classes of MSCs will be obtained by adding some constraints. We consider the possibility of having unmatched messages and, generally speaking, given a communication model $\comsymb$ a $\comsymb$-MSC means that there is a $\comsymb$-systems that produces the behaviour shown by $M$.


\paragraph{\bf Peer-to-peer MSCs}
The additional constraint satisfied by $\oneone$-MSCs ensures that messages sent from  process $p$ to  $q$ are always received in the same order as they are sent, i.e., $q$ observes a FIFO policy with respect to the messages recieved by $p$.  Hence, for each pair $(p,q)$ of processes, we cannot have an unmatched message $m_1$ (sent by $p$) followed by a matched message $m_2$ (sent by $p$).  By definition, every $\oneone$-MSC is an asynchronous MSC. Indeed, it is always possible to find an asynchronous system that can realize a computation described by a $\oneone$ MSC. 
%In other word, the possible behaviours, i.e. MSCs, generated by a system $S$ that uses $\oneone$ communication are a subset of all the behaviours that $S$ would be able to generate using asynchronous communication.
\begin{definition}[$\oneone$-MSCs]
A $\oneone$-MSC is an asynchronous MSC $\msc = (\Events,\procrel,\lhd,\lambda)$ where we require that, for every pair $(e,f) \in {\lhd}$, such that $\lambda(e) = \sact{p}{q}{\msg}$, $\lambda(f) = \ract{p}{q}{\msg}$, we have $\sametype{e}{\pqsAct{p}{q}}{\procrel^+} = \sametype{f}{\pqrAct{p}{q}}{\procrel^+}$.
\end{definition}

The set of $\oneone$ MSCs is MSO-definable as
	\[
		\ppformula = \neg \exists s.\exists s'. \left(
		\bigvee_{\substack{p \in \Procs, q \in \Procs}}\;
		\bigvee_{\substack{a,b \in \pqsAct{p}{q}}}\hspace{-1em}
		(\lambda(s) = a \;\wedge\; \lambda(s') = b) \;\wedge\; s \procrel^+ s' \;\wedge\;
		(\psi_1 \vee \psi_2 ) 	
		\right)
	\]
	where $\psi_1$ and $\psi_2$ are
	\[
		\psi_1 = \exists r.\exists r'.\left(
		\begin{array}{ll}
			s \lhd r & \wedge\\
			s' \lhd r' & \wedge\\
			r' \procrel^+ r &
		\end{array} 
		\right) \quad \quad
		\psi_2 = (\neg \mathit{matched}(s) \wedge \mathit{matched}(s'))
		\]
		\[
		matched(x) = \exists y. x \lhd y
	\]

The property $\ppformula$ says that there cannot be two matched send events $s$ and $s'$, with the same sender and receiver, such that either
\begin{enumerate*}[label={(\roman*)}]
	\item $s \procrel^+ s'$ and their receipts happen in the reverse order, or
	\item $s$ is unmatched and $s'$ is matched.
\end{enumerate*}
In other words, it ensures that channels operate in FIFO mode, where an unmatched messages blocks the receipt of all the subsequent messages on that channel.
%The set $\ppMSCs$ is therefore MSO-definable as $\ppMSCs=L(\ppformula)$.


\paragraph{\bf Causally ordered MSCs}
We will now consider the class of \co-MSCs. Recall that an MSC is causally ordered if all the messages sent to the same process are received in an order which is consistent with the causal ordering of the corresponding send events. 

\begin{definition}[\co-MSC]
An MSC $\msc = (\Events,\procrel,\lhd,\lambda)$ is \emph{causally ordered} if, for any two send events $s$ and $s'$, such that $\lambda(s)=\pqsAct{\plh}{q}$, $\lambda(s')=\pqsAct{\plh}{q}$, and $s \le_\msc s'$, we have either:
\begin{itemize}\itemsep=0.5ex
	\item $s,s' \in \Matched{\msc}$ and $r \procrel^* r'$, where $r$ and $r'$ are two receive events such that $s \lhd r$ and $s' \lhd r'$.
	\item $s' \in \Unm{\msc}$.
\end{itemize}
\end{definition}

This definition immediately translates in the following MSO formula:
\[
	\coformula = \neg \exists s.\exists s'. \left(
	\bigvee_{\substack{q \in \Procs}}\;
	\bigvee_{\substack{a,b \in \pqsAct{\plh}{q}}}\hspace{-1em}
	(\lambda(s) = a \;\wedge\; \lambda(s') = b) \;\wedge\; s \le_\msc s' \;\wedge\;
	(\psi_1 \vee \psi_2 ) 	
	\right)
\]
where $\psi_1$ and $\psi_2$ have been defined above for \pp. The property $\coformula$ says that there cannot be two send events $s$ and $s'$, with the same recipient, such that $s \le_\msc s'$ and either
\begin{enumerate*}[label={(\roman*)}]
	\item their corresponding receive events $r$ and $r'$ happen in the opposite order, i.e. $r' \procrel^+ r$, or
	\item $s$ is unmatched and $s'$ is matched.
\end{enumerate*}
%The set $\coMSCs$ of causally ordered MSCs is therefore MSO-definable as $\coMSCs=L(\coformula)$.


\paragraph{\bf Mailbox MSCs}
Moving on to $\none$ communication:

\begin{definition}[$\none$-MSC]\label{def:mb_msc}
An MSC $\msc = (\Events,\procrel,\lhd,\lambda)$ is a \emph{$\none$-MSC} if it has a linearization $\linrel$ where, for any two send events $s$ and $s'$, such that $\lambda(s)=\pqsAct{\plh}{q}$, $\lambda(s')=\pqsAct{\plh}{q}$, and $s \linrel s'$, we have either:
\begin{itemize}\itemsep=0.5ex
	\item $s,s' \in \Matched{\msc}$. Note that $r \linrel r'$, since we have that $r \procrel^+ r'$.
	\item $s' \in \Unm{\msc}$.
\end{itemize}
\end{definition}


Such a linearization will be referred to as a \emph{$\none$-linearization}. Note that the definition of $\none$ MSC is based on the existence of a linearization that satisfies some properties. The same kind of "existential" definition will be used for the remaining communication models. In practice, to claim that an MSC is $\none$ we just need to find a single valid $\none$ linearization, regardless of all the others.% As for unmatched messages, note that we cannot have two messages $m_1$ and $m_2$, addressed to the same process but possibly sent by different processes, such that $m_1$ is unmatched and $m_2$ is matched; $m_2$ can only be received after $m_1$, and this is consistent with the high-level definition of the $\none$ communication model that we gave in Section~\ref{sec:com_models_overview}.



In order to find a proper MSO formula we consider an equivalent definition (the proof of equivalence can be found in Appendix \ref{apx:MSO}) where  we define
	an additional binary relation that represents a constraint
	under the $\none$ semantics, which ensures that messages received by a process are sent in the same order as they are received.



\begin{definition} [$\none$ alternative]\label{def:n_one_alt}
	For an MSC $\msc = (\Events,\procrel,\lhd,\lambda)$.	Let ${\mbrel}_\msc \subseteq \Events \times \Events$
	be defined as $s \mbrel_\msc s'$ if there is $q \in \Procs$
	such that $\lambda(s) \in \qsAct{q}$,
	$\lambda(s') \in \qsAct{q}$, and one of the following holds:
	\begin{itemize}\itemsep=0.5ex
		\item $s \in \Matched{\msc}$ and $s' \in \Unm{\msc}$, or
		\item $s \lhd f_1$ and $s' \lhd f_2$ for some $f_1,f_2 \in \Events_q$ such that $f_1 \procrel^+ f_2$.
	\end{itemize}
	
	We let ${\preceq_\msc} = ({\procrel} \,\cup\, {\lhd} \,\cup\, {\mbrel_\msc})^\ast$. 
	$\msc $ is a \emph{$\none$- MSC}
	if ${\preceq_\msc}$ is a partial order. 
\end{definition}
The ${\mbrel}_\msc$ relation ensures that send events addressed to the same process are executed in an order that is suitable for the $\none$ communication. If ${\preceq_\msc}$ is a partial order, it means that it is possible to find a linearization $\linrel$, such that $\linrel \subseteq \preceq_\msc$. It is easy to see that such a linearization is exactly what we called a $\none$ linearization in Definition~\ref{def:mb_msc}. 
Finally notice  that ${\le_\msc} \subseteq {\preceq_\msc}$. Hence entailing that a \mb-MSC is also a \pp-MSC.


Immediately from previous definition one can derive the corresponding  MSO formula:
\[
	\mbformula = \ppformula \;\wedge\; \neg \exists x.\exists y.(\neg (x = y) \wedge x \preceq_\msc y \wedge y \preceq_\msc x)
\]
%This formula closely follows Definition~\ref{def:n_one_alt}. The set $\mbMSCs$ of $\none$ MSCs is therefore MSO-definable as $\mbMSCs=L(\mbformula)$.




\paragraph{\bf \onen MSCs}
Moving on to $\onen$ communication, we say that $\msc$ is a $\onen$ MSC if there is a $\onen$ system that can exhibit the behaviour described by $\msc$. 

\begin{definition}[$\onen$ MSC]\label{def:one_n}
An MSC $\msc = (\Events,\procrel,\lhd,\lambda)$ is a \emph{$\onen$ MSC} if it has a linearization $\linrel$ where, for any two send events $s$ and $s'$, such that $\lambda(s)=\pqsAct{p}{\plh}$, $\lambda(s')=\pqsAct{p}{\plh}$, and $s \procrel^+ s'$ (which implies $s \linrel s'$), we have either:
\begin{itemize}\itemsep=0.5ex
	\item $s,s' \in \Matched{\msc}$ and $r \linrel r'$, where $r$ and $r'$ are two receive events such that $s \lhd r$ and $s' \lhd r'$.
	\item $s' \in \Unm{\msc}$.
\end{itemize}
\end{definition}

Such a linearization will be referred to as a \emph{$\onen$ linearization}, and we will sometimes use the symbol $\onenlinrel$ to denote one. Note that the definition is very similar to the $\none$ case, but here $s$ and $s'$ are two send events executed by the same process, and not addressed to the same process. In a $\onen$ MSC we cannot have two messages $m_1$ and $m_2$, sent by the same process, such that $m_1$ is unmatched and $m_2$ is matched; indeed, according to the $\onen$ communication model, $m_1$ must be received before $m_2$.

\davide{Provide example of $\onen$ and non-$\onen$ MSCs.}

\begin{definition}[$\nn$ MSC]\label{def:n_n}
	An MSC $\msc = (\Events,\procrel,\lhd,\lambda)$ is a \emph{$\nn$ MSC} if it has a linearization $\linrel$ where, for any two send events $s$ and $s'$, such that $s \linrel s'$, we have either:
	\begin{itemize}\itemsep=0.5ex
		\item $s,s' \in \Matched{\msc}$ and $r \linrel r'$, where $r$ and $r'$ are two receive events such that $s \lhd r$ and $s' \lhd r'$.
		\item $s' \in \Unm{\msc}$.
	\end{itemize}
\end{definition}

Such a linearization will be referred to as a \emph{$\nn$ linearization}. Intuitively, with an $\nn$ MSC we are always able to schedule events in such a way that messages are received in the same order as they were sent, and unmatched messages are sent only after all matched messages are sent. By definition, every $\nn$ MSC is a $\onen$ MSC. 

\begin{definition}[$\rsc$ MSC]\label{def:rsc}
	An MSC $\msc = (\Events,\procrel,\lhd,\lambda)$ is a \emph{RSC MSC} if it has no unmatched send events and there is a linearization $\linrel$ where any matched send event is immediately followed by its respective receive event.
\end{definition}

Such a linearization will be referred to as a \emph{$\rsc$ linearization}.

% THIS IS WRONG!
% We show that the opposite direction is also true, which implies that the class of $\nn$ MSCs is equivalent to the class of $onen$ MSCs. The following example gives an intuition of the formal proof, which will be given right after.

% \begin{proposition}
% 	Every $\onen$ MSC without unmatched messages is an $\nn$ MSC.
% \end{proposition}
% \begin{proof}
% Let $\msc$ be a $\onen$ MSC without unmatched messages, and let $L$ be a $\onen$ linearization. We will show that, by reordering some of the events in $L$, we are always able to obtain a $\nn$ linearization for $\msc$. The algorithm works as follow:
% \begin{enumerate}
% 	\item Find a pair $(m_1,m_2)$ of distinct messages such that their send order in $L$ is the inverse of the receive order\footnote{Note that $L$ is already a $\nn$ linearization if such a pair does not exist}. This can only happen if there is not a causal path between $s_1$ and $s_2$, i.e $s_1 \le \ge s_2$, where $s_i$ is the send event of message $m_i$. To see why, suppose w.l.o.g. that $s_1 \le s_2$. As seen in the proof of Proposition~\ref{prop:onen_mb_no_unmatched}, there must be a message $m_k$ sent by the same process that sent $s_1$, such that we have $r_1 \onenrel r_k \le r_2$, and in particular $r_1 \lessdot r_2$. Therefore, if $s_1 \le s_2$, the receive order has to match the send order in $L$ and it will never be the opposite.
% 	\item Suppose, w.l.o.g. that $s_2 \linrel s_1$ and $r_1 \linrel r_2$, and we saw that $s_1 \le \ge s_2$. We would like to invert the order of $s_1$ and $s_2$ in the linearization, so that it matches the receive order. If we simply swap $s_2$ and $s_1$ in $L$, the new linearization could be invalid for $\msc$; for instance, there might be some events between $s_2$ and $s_1$ (in the linearization $L$) that need to happen before $s_1$. However, we show that it is still possible to invert the order of $s_1$ and $s_2$ without invalidating the linearization. Suppose that $s_2$ and $s_1$ are the $i$-th and the $j$-th events of $L$, respectively, where $i<j$. The idea is to move $s_2$ right before $s_1$, along with all the events on which $s_1$ depends, that are between $s_2$ and $s_1$ in $L$; we say that $s_1$ depends on an event $e$ if $e \lessdot s_1$.
% \end{enumerate}
% \end{proof}


\subsection{Alternative definitions}



\begin{definition} [$\onen$ alternative]\label{def:one_n_alt}
	For an MSC $\msc = (\Events,\procrel,\lhd,\lambda)$, we define
	an additional binary relation that represents a constraint
	under the $\onen$ semantics, which ensures that messages sent from the same process are received in the same order. Let ${\onenrel}_\msc \subseteq \Events \times \Events$ be defined as $e_1 \onenrel_\msc e_2$ if there are two events $e_1$ and $e_2$, and $p \in \Procs$ such that either:
	\begin{itemize}\itemsep=0.5ex
		\item $\lambda(e_1) \in \psAct{p}$, $\lambda(e_2) \in \psAct{p}$, $e_1 \in \Matched{\msc}$, and $e_2 \in \Unm{\msc}$, or
		\item $\lambda(e_1) \in \prAct{p}$, $\lambda(e_2) \in \prAct{p}$, $s_1 \lhd e_1$ and $s_2 \lhd e_2$ for some $s_1,s_2 \in \Events_p$, and $s_1 \procrel^+ s_2$.
	\end{itemize}
	
	We let ${\lessdot_\msc} = ({\procrel} \,\cup\, {\lhd} \,\cup\, {\onenrel_\msc})^\ast$.
	Note that ${\le_\msc} \subseteq {\lessdot_\msc}$.
	%
	%\begin{definition}\label{def:mailbox-msc}
	We call $\msc \in \asMSCs$ a \emph{$\onen$ MSC}
	if ${\lessdot_\msc}$ is a partial order. The ${\onenrel}_\msc$ relation ensures that messages sent by a process are sent and received in an order that is suitable for the $\onen$ communication. Note that if ${\lessdot_\msc}$ is a partial order, it is possible to find a linearization $\linrel$, such that $\linrel \subseteq \lessdot_\msc$. It is not difficult to see that such a linearization is exactly what we called a $\onen$ linearization in Definition~\ref{def:one_n}. The two definition of $\onen$ MSC that we gave are equivalent.
\end{definition}

\davide{Proof of equivalence of 2 definitions?}

\begin{definition} [$\nn$ alternative]\label{def:n_n_alt}
	For an MSC $\msc = (\Events,\procrel,\lhd,\lambda)$, let ${\nnrel}_\msc = ({\procrel} \,\cup\, {\lhd} \,\cup\, {\mbrel_\msc} \,\cup\, {\onenrel_\msc})^\ast$. We define an additional binary relation $\bowtie_\msc \subseteq \Events \times \Events$, such that for two events $e_1$ and $e_2$ we have $e_1 \bowtie_\msc e_2$ if one of the following holds:
	\begin{enumerate}\itemsep=0.5ex
		\item $e_1 \nnrel_\msc e_2$
		\item $\lambda(e_1) \in \prAct{\plh}$, $\lambda(e_2) \in \prAct{\plh}$, $s_1 \lhd e_1$ and $s_2 \lhd e_2$ for some $s_1,s_2 \in \Events$, $s_1 \nnrel_\msc s_2$ and $e_1 \slashed{\nnrel}_\msc e_2$.
		\item $\lambda(e_1) \in \psAct{\plh}$, $\lambda(e_2) \in \psAct{\plh}$, $e_1 \lhd r_1$ and $e_2 \lhd r_2$ for some $r_1,r_2 \in \Events$, $r_1 \nnrel_\msc r_2$ and $e_1 \slashed{\nnrel}_\msc e_2$.
		\item $e_1 \in \Matched{\msc}$, $e_2 \in \Unm{\msc}$, $e_1 \slashed{\nnrel}_\msc e_2$.
	\end{enumerate}
	
	Note that $\preceq_\msc \subseteq \nnrel_\msc$, $\lessdot_\msc \subseteq \nnrel_\msc$, and $\nnrel_\msc \subseteq\; \bowtie_\msc$. We call $\msc \in \asMSCs$ a \emph{$\nn$ MSC}
	if ${\bowtie_\msc}$ is acyclic.
	\davidequestion{Initially I said "We call $\msc \in \asMSCs$ a \emph{$\nn$ MSC} if ${\bowtie_\msc}$ is a partial order", but I don't think it is true, since $\bowtie_\msc$ does not have to be transitive... correct?}
\end{definition}

It is not trivial to see that Definition~\ref{def:n_n_alt} and Definition~\ref{def:n_n} are equivalent. To show that, we need some preliminary results and definitions. 

\begin{proposition}
	Let $\msc$ be an MSC. Given two matched send events $s_1$ and $s_2$, and their respective receive events $r_1$ and $r_2$, $r_1 \bowtie_\msc r_2 \implies s_1 \bowtie_\msc s_2$.
\end{proposition}
\begin{proof}
Follows from the definition of $\bowtie_\msc$. We have $r_1 \bowtie_\msc r_2$ if either:
\begin{itemize}\itemsep=0.5ex
	\item $r_1 \nnrel_\msc r_2$. Two cases: either \begin{enumerate*}[label={(\roman*)}]
		\item $s_1 \nnrel_\msc s_2$, or 
		\item $s_1 \slashed{\nnrel}_\msc s_2$.
	\end{enumerate*}
	The first case clearly implies $s_1 \bowtie_\msc s_2$, for rule 1 in the definition of $\bowtie_\msc$. The second too, because of rule 3.
	\item  $r_1 \slashed{\nnrel}_\msc r_2$, but $r_1 \bowtie_\msc r_2$. This is only possible if rule 2 in the definition of $\bowtie_\msc$ was used, which implies $s_1 \nnrel_\msc s_2$ and, for rule 1, $s_1 \bowtie_\msc s_2$.
\end{itemize}
\end{proof}

\begin{proposition}\label{prop:n_n_cycl}
	Let $\msc$ be an MSC. If $\bowtie_\msc$ is cyclic, then $\msc$ is not $\nn$.
\end{proposition}
\begin{proof}
According to Definition~\ref{def:n_n}, an MSC is $\nn$ if it has at least one $\nn$ linearization. Note that, because of how it is defined, any $\nn$ linearization is always both a $\none$ and a $\onen$ linearization. It follows that the cyclicity of $\nnrel_\msc$ (not $\bowtie_\msc$) implies that $\msc$ is not $\nn$, because it means that we are not even able to find a linearization that is both $\none$ and $\onen$. Moreover, since in a $\nn$ linearization the order in which messages are sent matches the order in which they are received, and unmatched send events can be executed only after matched send events, a $\nn$ MSC always has to satisfy the constraints imposed by the $\bowtie_\msc$ relation. If $\bowtie_\msc$ is cyclic, then for sure there is no $\nn$ linearization for $\msc$.
\davide{This proof does not convince me... maybe rewrite it better}
\end{proof}

Let the \emph{Event Dependency Graph} (EDG) of a $\nn$ MSC $\msc$ be a graph that has events as nodes and an edge between two events $e_1$ and $e_2$ if $e_1 \bowtie_\msc e_2$. We now present an algorithm that, given the EDG of an $\nn$ MSC $\msc$, computes a $\nn$ linearization of $\msc$. We then show that, if $\bowtie_\msc$ is acyclic (i.e. it is a partial order), this algorithm always terminates correctly. This, along with Proposition~\ref{prop:n_n_cycl}, effectively shows that Definition~\ref{def:n_n} and Definition~\ref{def:n_n_alt} are equivalent.

\paragraph*{Algorithm for finding a $\nn$ linearization}
The input of this algorithm is the EDG of an MSC $\msc$, and it outputs a valid $\nn$ linearization for $\msc$, if $\msc$ is $\nn$. The algorithm works as follows:
\begin{enumerate}
	\item If there is a matched send event $s$ with in-degree 0 in the EDG, add $s$ to the linearization and remove it from the EDG, along with its outgoing edges, then jump to step 5. Otherwise, proceed to step 2.
	\item If there are no matched send events in the EDG and there is an unmatched send event $s$ with in-degree 0 in the EDG, add $s$ to the linearization and remove it from the EDG, along with its outgoing edges, then jump to step 5. Otherwise, proceed to step 3.
		\item If there is a receive event $r$ with in-degree 0 in the EDG, such that $r$ is the receive event of the first message whose sent event was already added to the linearization, add $r$ to the linearization and remove it from the EDG, along with its outgoing edges, then jump to step 5. Otherwise, proceed to step 4.
		\item Throw an error and terminate.
		\item If all the events of $\msc$ were added to the linearization, return the linearization and terminate. Otherwise, go back to step 1.
\end{enumerate} 

We now need to show that 
\begin{enumerate*}[label={(\roman*)}]
	\item if this algorithm terminates correctly (i.e. step 4 is never executed), it returns a $\nn$ linearization, and 
	\item if $\bowtie_\msc$ is acyclic, the algorithm always terminates correctly.
\end{enumerate*}
\begin{proposition}
	If the above algorithm returns a linearization for an MSC $\msc$, it is a $\nn$ linearization.
\end{proposition}
\begin{proof}
	Step 2 ensures that the order (in the linearization) in which matched messages are sent is the same as the order in which they are received. Moreover, according to step 3, an unmatched send events is added to the linearization only if all the matched send events were already added.
\end{proof}

\begin{proposition}
	Given an MSC $\msc$, the above algorithm always terminates correctly if $\bowtie_\msc$ is acyclic.
\end{proposition}
\begin{proof}
We want to prove that, if $\bowtie_\msc$ is acyclic, step 4 of the algorithm is never executed, i.e. it terminates correctly. Note that the acyclicity of $\bowtie_\msc$ implies that the EDG of $\msc$ is a DAG. Moreover, at every step of the algorithm we remove nodes and edges from the EDG, so it still remains a DAG. The proof goes by induction on the number of events added to the linearization.\newline
Base case: no event has been added to the linearization yet. Since the EDG is a DAG, there must be an event with in-degree 0. In particular, this has to be a send event (a receive event depends on its respective send event, so it cannot have in-degree 0). If it is a matched send event, step 1 is applied. If there are no matched send events, step 2 is applied on an unmatched send. We show that it is impossible to have an unmatched send event of in-degree 0 if there are still matched send events in the EDG, so either step 1 or 2 are applied in the base case. Let $s$ be one of those matched send events and let $u$ be an unmatched send. Because of rule 4 in the definition of $\bowtie_\msc$, we have that $s \bowtie_\msc u$, which implies that $u$ cannot have in-degree 0 if $s$ is still in the EDG.\newline
Inductive step: we want to show that we are never going to execute step 4. In particular, Step 4 is executed when none of the first three steps can be applied. This happens when there are no matched send events with in-degree 0 and one of the following holds:
\begin{itemize}\itemsep=0.5ex
	\item \emph{There are still matched send events in the EDG with in-degree $>0$, there are no unmatched messages with in-degree 0, and there is no receive event $r$ with in-degree 0 in the EDG, such that $r$ is the receive event of the first message whose sent event was already added to the linearization}. Since the EDG is a DAG, there must be at least one receive event with in-degree 0. We want to show that, between these receive events with in-degree 0, there is also the receive event $r$ of the first message whose send event was added to the linearization, so that we can apply step 3 and step 4 is not executed. Suppose, by contradiction, that $r$ has in-degree $>0$, so it depends on other events. For any maximal chain in the EDG that contains one of these events, consider the first event $e$, which clearly has in-degree 0. In particular, $e$ cannot be a send event, because we would have applied step 1 or step 2. Hence, $e$ can only be a receive event for a send event that was not the first added to the linearization (and whose respective receive still has not been added). However, this is also impossible, since $r_e \bowtie_\msc r$ implies $s_e \bowtie_\msc s$, and we could not have added $s$ to the linearization before $s_e$. Because we got to a contradiction, the hypothesis that $r$ has in-degree $>0$ must be false, and we can indeed apply step 3.
	\item \emph{There are still matched send events in the EDG with in-degree $>0$, there is at least one unmatched message with in-degree 0, and there is no receive event $r$ with in-degree 0 in the EDG, such that $r$ is the receive event of the first message whose sent event was already added to the linearization}. We show that it is impossible to have an unmatched send event of in-degree 0 if there are still matched send events in the EDG. Let $s$ be one of those matched send events and let $u$ be an unmatched send. Because of rule 4 in the definition of $\bowtie_\msc$, we have that $s \bowtie_\msc u$, which implies that $u$ cannot have in-degree 0 if $s$ is still in the EDG.
	\item \emph{There are no more matched send events in the EDG, there are no unmatched messages with in-degree 0, and there is no receive event $r$ with in-degree 0 in the EDG, such that $r$ is the receive event of the first message whose sent event was already added to the linearization}. Very similar to the first case. Since the EDG is a DAG, there must be at least one receive event with in-degree 0. We want to show that, between these receive events with in-degree 0, there is also the receive event $r$ of the first message whose send event was added to the linearization, so that we can apply step 3 and step 4 is not executed. Suppose, by contradiction, that $r$ has in-degree $>0$, so it depends on other events. For any maximal chain in the EDG that contains one of these events, consider the first event $e$, which clearly has in-degree 0. In particular, $e$ cannot be a send event, because by hypothesis there are no more send events with in-degree 0 in the EDG. Hence, $e$ can only be a receive event for a send event that was not the first added to the linearization (and whose respective receive still has not been added). However, this is also impossible, since $r_e \bowtie_\msc r$ implies $s_e \bowtie_\msc s$, and we could not have added $s$ to the linearization before $s_e$. Because we got to a contradiction, the hypothesis that $r$ has in-degree $>0$ must be false, and we can indeed apply step 3.
\end{itemize}
We showed that, if $\bowtie_\msc$ is acyclic, the algorithm always terminates correctly and computes a valid $\nn$ linearization.
\end{proof}

\noindent We have now effectively proved that Definition~\ref{def:n_n_alt} of $\nn$ MSC is equivalent to Definition~\ref{def:n_n}.

\medskip

Following the characterization given in \cite[Theorem 4.4]{DBLP:journals/dc/Charron-BostMT96}, we also give an alternative but equivalent definition of $\rsc$ MSC.

\begin{definition}
	Let $\msc$ be an MSC. A crown of size $k$ in $\msc$ is a sequence $\langle(s_i,r_i),\, i \in \{1,\dots,k\}\rangle$ of pairs of corresponding send and receive events such that
	\[
		s_1 <_\msc r_2, s_2 <_\msc r_3, \dots, s_{k-1} <_\msc r_k, s_k <_\msc r_1.
	\]
\end{definition}
\davide{Specify somewhere that $<_\msc = (\procrel \cup \lhd)^+$.}

\begin{definition} [$\rsc$ alternative]\label{def:rsc_alt}
	An MSC $\msc = (\Events,\procrel,\lhd,\lambda)$ is a \emph{RSC MSC} if and only if it does not contain any crown.
\end{definition}



\paragraph*{$\oneone$ MSCs}
	

\paragraph*{Causally ordered MSCs}


\paragraph*{$\none$ MSCs}


\paragraph*{$\onen$ MSCs}

Following Definition~\ref{def:one_n_alt}, an MSC $\msc$ is a $\onen$ MSC if and only if it satisfies the MSO formula
\[
	\onenformula = \neg \exists x.\exists y.(\neg (x = y) \wedge x \lessdot_\msc y \wedge y \lessdot_\msc x)
\]
Recall that $\lessdot_\msc$ is the union of the MSO-definable relations $\procrel$, $\lhd$, and $\onenrel_\msc$. In particular, we can define $x \onenrel_\msc y$ as 
\[
x \onenrel_\msc y =
\begin{array}{rl}
& \left(
	\bigvee_{\substack{p \in \Procs\\a,b \in \psAct{p}}}\hspace{-1em}
	(\lambda(x) = a \;\wedge\; \lambda(y) = b)
	\;\wedge\; \mathit{matched}(x) \;\wedge\; \neg \mathit{matched}(y)
\right) \;\vee\\
& \left(
	\bigvee_{\substack{p \in \Procs\\a,b \in \prAct{p}}}\hspace{-1em}
	(\lambda(x) = a \;\wedge\; \lambda(y) = b)
	\;\wedge\; 
	\exists x'.\exists y'. (x' \lhd x \;\wedge\; y' \lhd y \;\wedge\; x' \procrel^+ y')
\right)\\
\end{array}
\]
The MSO formula for $x \onenrel_\msc y$ closely follows Definition~\ref{def:one_n_alt}. The set $\onenMSCs$ of $\onen$ MSCs is therefore MSO-definable as $\onenMSCs=L(\onenformula)$.

\paragraph*{$\nn$ MSCs}

Following Definition~\ref{def:n_n_alt}, an MSC $\msc$ is a $\nn$ MSC if and only if it satisfies the MSO formula
\[
	\nnformula = \neg \exists x.\exists y.(\neg (x = y) \wedge x \bowtie_\msc y \wedge y \bowtie_\msc x)
\]
In particular, we can define $x \bowtie_\msc y$ as
\[
	x \bowtie_\msc y =
	\begin{array}{rl}
	& \left(
		\bigvee_{\substack{a,b \in \psAct{\plh}}}
		(\lambda(x) = a \;\wedge\; \lambda(y) = b)
		\;\wedge\; \mathit{matched}(x) \;\wedge\; \neg \mathit{matched}(y)
	\right) \;\vee\\
	& (x \nnrel_\msc y) \quad \vee \quad \psi_3 \quad \vee \quad \psi_4\\
	\end{array}
\]

\noindent where $\psi_3$ and $\psi_4$ are defined as 
\[
	\psi_3 =
	\begin{array}{rl}
		& \bigvee_{\substack{a,b \in \prAct{\plh}}}
		  (\lambda(x) = a \;\wedge\; \lambda(y) = b)
		  \;\wedge\; \\
		& \exists x'.\exists y'.(x' \lhd x \;\wedge\; y' \lhd y) \;\wedge\; (x' \nnrel_\msc y') \;\wedge\; \neg(x \nnrel_\msc y)\\
	\end{array}
\]
\[
	\psi_4 =
	\begin{array}{rl}
		& \bigvee_{\substack{a,b \in \psAct{\plh}}}
		  (\lambda(x) = a \;\wedge\; \lambda(y) = b)
		  \;\wedge\; \\
		& \exists x'.\exists y'.(x \lhd x' \;\wedge\; y \lhd y') \;\wedge\; (x' \nnrel_\msc y') \;\wedge\; \neg(x \nnrel_\msc y)\\
	\end{array}
\]

The MSO formula for $x \nnrel_\msc y$ closely follows Definition~\ref{def:n_n_alt}. The set $\nnMSCs$ of $\nn$ MSCs is therefore MSO-definable as $\nnMSCs=L(\nnformula)$.

\paragraph*{$\rsc$ MSCs}

Following Definition~\ref{def:rsc_alt}, an MSC $\msc$ is a $\rsc$ MSC if and only if it satisfies the MSO formula
\[
\Phi_{\rsc} = \neg \exists s_1.\exists s_2. s_1 \varpropto s_2 \;\wedge\; s_2 \varpropto^\ast s_1
\]
\noindent where $\varpropto$ is defined as
\[
s_1 \varpropto s_2 = 
\bigvee_{\substack{e \in \sAct}}(\lambda(s_1) = e) \;\wedge\;
s_1 \neq s_2 \;\wedge\; 
\exists r_2. (s_1 < r_2 \;\wedge\; s_2 \lhd r_2)
\]
% \davidequestion{The following formula should be wrong... I cannot use $n$ in an MSO formula}
% \[
% 	\rscformula = 
% 	\begin{array}{rl}
% 		& \forall x.\left(\bigvee_{a \in \psAct{\plh}} \lambda(x) = a \;\implies\; \mathit{matched}(x)\right) \;\wedge\; \\
% 		& \neg \left( 
% 			\bigvee_{k=1}^n \left(
% 			\exists s_1 \cdots s_k. \exists r_1 \cdots r_k. 
% 			\bigwedge_{i=1}^k (s_i \lhd r_i \;\wedge\; s_i \le_\msc r_{(i+1)\%k})
% 		\right)
% 		\right)\\
% 	\end{array}
% \]
% where $n$ is the total number of messages. The formula checks that there are no unmatched send events and that in the conflict graph there is no cycle, of any length, whose edges are all SR.
