We presented seven different communication models and formally defined the corresponding classes of MSCs, which represent valid computations for a given model. We then characterised each of these classes with MSO logic, going through some alternative definitions. Next, we showed how these classes form a non-sequential hierarchy of communication models. These results were then applied to deal with (un)decidability of some verification problems based on MSO. 

For future work, other communication models could be studied, such as the lightweight implementation of the causally ordered communication model proposed in \cite{DBLP:conf/dagstuhl/MatternF94}, which should sit somewhere between mailbox and causally ordered within the hierarchy that we presented. Moreover, as shown by Fig.~\ref{fig:stw-bound}, the decidability of the synchronizability problem for weakly synchronous MSCs and fully asynchronous communication is still an open problem, which we plan to investigate further.