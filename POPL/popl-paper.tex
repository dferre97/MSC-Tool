%% For double-blind review submission, w/o CCS and ACM Reference (max submission space)
\documentclass[acmsmall,review,anonymous]{acmart}\settopmatter{printfolios=true,printccs=false,printacmref=true}
%% For double-blind review submission, w/ CCS and ACM Reference
%\documentclass[acmsmall,review,anonymous]{acmart}\settopmatter{printfolios=true}
%% For single-blind review submission, w/o CCS and ACM Reference (max submission space)
%\documentclass[acmsmall,review]{acmart}\settopmatter{printfolios=true,printccs=false,printacmref=false}
%% For single-blind review submission, w/ CCS and ACM Reference
%\documentclass[acmsmall,review]{acmart}\settopmatter{printfolios=true}
%% For final camera-ready submission, w/ required CCS and ACM Reference
%\documentclass[acmsmall]{acmart}\settopmatter{}


%% Journal information
%% Supplied to authors by publisher for camera-ready submission;
%% use defaults for review submission.
\acmJournal{PACMPL}
\acmVolume{1}
\acmNumber{CONF} % CONF = POPL or ICFP or OOPSLA
\acmArticle{1}
\acmYear{2018}
\acmMonth{1}
\acmDOI{} % \acmDOI{10.1145/nnnnnnn.nnnnnnn}
\startPage{1}

%% Copyright information
%% Supplied to authors (based on authors' rights management selection;
%% see authors.acm.org) by publisher for camera-ready submission;
%% use 'none' for review submission.
\setcopyright{none}
%\setcopyright{acmcopyright}
%\setcopyright{acmlicensed}
%\setcopyright{rightsretained}
%\copyrightyear{2018}           %% If different from \acmYear

%% Bibliography style
\bibliographystyle{ACM-Reference-Format}
%% Citation style
%% Note: author/year citations are required for papers published as an
%% issue of PACMPL.
\citestyle{acmauthoryear}   %% For author/year citations


%%%%%%%%%%%%%%%%%%%%%%%%%%%%%%%%%%%%%%%%%%%%%%%%%%%%%%%%%%%%%%%%%%%%%%
%% Note: Authors migrating a paper from PACMPL format to traditional
%% SIGPLAN proceedings format must update the '\documentclass' and
%% topmatter commands above; see 'acmart-sigplanproc-template.tex'.
%%%%%%%%%%%%%%%%%%%%%%%%%%%%%%%%%%%%%%%%%%%%%%%%%%%%%%%%%%%%%%%%%%%%%%


%% Some recommended packages.
\usepackage{booktabs}   %% For formal tables:
                        %% http://ctan.org/pkg/booktabs
\usepackage{subcaption} %% For complex figures with subfigures/subcaptions
                        %% http://ctan.org/pkg/subcaption


%% MY STUFF
\usepackage{preamble}
% \addbibresource{my_biblio.bib} % do not put this in preamble or it breakes automatic suggestions for citations in vscode

% % macros
\usepackage{my_macro}
\usepackage{laetitia_macro}

\begin{document}

%% Title information
\title[Short Title]{Demystifying asynchronous communication and its variants}         %% [Short Title] is optional;
                                        %% when present, will be used in
                                        %% header instead of Full Title.
\titlenote{with title note}             %% \titlenote is optional;
                                        %% can be repeated if necessary;
                                        %% contents suppressed with 'anonymous'
\subtitle{Subtitle}                     %% \subtitle is optional
\subtitlenote{with subtitle note}       %% \subtitlenote is optional;
                                        %% can be repeated if necessary;
                                        %% contents suppressed with 'anonymous'


%% Author information
%% Contents and number of authors suppressed with 'anonymous'.
%% Each author should be introduced by \author, followed by
%% \authornote (optional), \orcid (optional), \affiliation, and
%% \email.
%% An author may have multiple affiliations and/or emails; repeat the
%% appropriate command.
%% Many elements are not rendered, but should be provided for metadata
%% extraction tools.

%% Author with single affiliation.
\author{First1 Last1}
\authornote{with author1 note}          %% \authornote is optional;
                                        %% can be repeated if necessary
\orcid{nnnn-nnnn-nnnn-nnnn}             %% \orcid is optional
\affiliation{
  \position{Position1}
  \department{Department1}              %% \department is recommended
  \institution{Institution1}            %% \institution is required
  \streetaddress{Street1 Address1}
  \city{City1}
  \state{State1}
  \postcode{Post-Code1}
  \country{Country1}                    %% \country is recommended
}
\email{first1.last1@inst1.edu}          %% \email is recommended

%% Author with two affiliations and emails.
\author{First2 Last2}
\authornote{with author2 note}          %% \authornote is optional;
                                        %% can be repeated if necessary
\orcid{nnnn-nnnn-nnnn-nnnn}             %% \orcid is optional
\affiliation{
  \position{Position2a}
  \department{Department2a}             %% \department is recommended
  \institution{Institution2a}           %% \institution is required
  \streetaddress{Street2a Address2a}
  \city{City2a}
  \state{State2a}
  \postcode{Post-Code2a}
  \country{Country2a}                   %% \country is recommended
}
\email{first2.last2@inst2a.com}         %% \email is recommended
\affiliation{
  \position{Position2b}
  \department{Department2b}             %% \department is recommended
  \institution{Institution2b}           %% \institution is required
  \streetaddress{Street3b Address2b}
  \city{City2b}
  \state{State2b}
  \postcode{Post-Code2b}
  \country{Country2b}                   %% \country is recommended
}
\email{first2.last2@inst2b.org}         %% \email is recommended


%% Abstract
%% Note: \begin{abstract}...\end{abstract} environment must come
%% before \maketitle command
\begin{abstract}
Text of abstract \ldots.
\end{abstract}


%% 2012 ACM Computing Classification System (CSS) concepts
%% Generate at 'http://dl.acm.org/ccs/ccs.cfm'.
\begin{CCSXML}
<ccs2012>
<concept>
<concept_id>10011007.10011006.10011008</concept_id>
<concept_desc>Software and its engineering~General programming languages</concept_desc>
<concept_significance>500</concept_significance>
</concept>
<concept>
<concept_id>10003456.10003457.10003521.10003525</concept_id>
<concept_desc>Social and professional topics~History of programming languages</concept_desc>
<concept_significance>300</concept_significance>
</concept>
</ccs2012>
\end{CCSXML}

\ccsdesc[500]{Software and its engineering~General programming languages}
\ccsdesc[300]{Social and professional topics~History of programming languages}
%% End of generated code


%% Keywords
%% comma separated list
\keywords{keyword1, keyword2, keyword3}  %% \keywords are mandatory in final camera-ready submission


%% \maketitle
%% Note: \maketitle command must come after title commands, author
%% commands, abstract environment, Computing Classification System
%% environment and commands, and keywords command.
\maketitle



\section{Introduction}
\begin{itemize}
  \item Interleaving based semantics VS partial order/graph based semantics
  \item Synchronous and asynchronous communication
  \item The problem of synchronizability
\end{itemize}

\section{Preliminaries/Basics}
\begin{itemize}
  \item Communicating systems (communicating finite-state automata with bag channels)
  \item MSCs and conflict graph
  \item Monadic Second-Order logic on MSCs
  \item (Language of a system as a set of MSCs)
  \item (Model checking and synchronizability)
\end{itemize}

\section{Asynchronous communication models overview}
\begin{itemize}
  \item Overview of asynchronous variants
  \item High-level description of each variant along with references to implementations (if existing)
  \item (Definitions based on linearization, intuitive)
  \item (Language of a system with a given communication model as a set of MSCs)
  \item Hint of hierarchy result
\end{itemize}

\section{Asynchronous communication models operational semantics}
\begin{itemize}
  \item TODO...
\end{itemize}

\section{Asynchronous communication models as classes of MSCs, MSO-definability}
\begin{itemize}
  \item Definition of MSC class for each communication model (alternative definitions)
  \item MSO-definability of each class
\end{itemize}

\section{Equivalence of the two definitions}
\begin{itemize}
  \item TODO...
\end{itemize}

\section{Hierarchy of asynchronous classes of MSCs}
\ldots

\section{An application: special treewidth and decidability of the synchronizability problem}
\begin{itemize}
  \item The synchronizability problem
  \item Special treewidth and how the results regarding the hierarchy are useful for detecting STW-boundness of certain classes
  \item MSO-decidability and STW-boundess tables
\end{itemize}

\section{Conclusion}

%% Acknowledgments
\begin{acks}                            %% acks environment is optional
                                        %% contents suppressed with 'anonymous'
  %% Commands \grantsponsor{<sponsorID>}{<name>}{<url>} and
  %% \grantnum[<url>]{<sponsorID>}{<number>} should be used to
  %% acknowledge financial support and will be used by metadata
  %% extraction tools.
  This material is based upon work supported by the
  \grantsponsor{GS100000001}{National Science
    Foundation}{http://dx.doi.org/10.13039/100000001} under Grant
  No.~\grantnum{GS100000001}{nnnnnnn} and Grant
  No.~\grantnum{GS100000001}{mmmmmmm}.  Any opinions, findings, and
  conclusions or recommendations expressed in this material are those
  of the author and do not necessarily reflect the views of the
  National Science Foundation.
\end{acks}




%% Appendix
\appendix
\section{Appendix}

Text of appendix \ldots



% !TEX root = popl-paper.tex

\subsection{Communicating Systems}

We now recall the definition of communicating systems (aka communicating finite-state
machines or message-passing automata), which consist of finite-state machines $A_p$
(one for every process $p \in \Procs$) that can communicate through channels from $\Ch$. In our case, we consider the most general definition in which channels behave as bags, which means that no specific order is enforced on the delivery of messages.

\begin{definition}\label{def:cs}
A \emph{communicating system} over $\Procs$ and $\Msg$ is a tuple
   $ \Sys = (A_p)_{p\in\procSet}$. For each
   $p \in \Procs$, $A_p = (Loc_p, \delta_p, \ell^0_p)$ is a finite transition system where
   $\Loc_p$ is a finite set of local (control) states, $\delta_p
   \subseteq \Loc_p \times \pAct{p} \times \Loc_p$ is the
   transition relation, and $\ell^0_p \in Loc_p$ is the initial state.
\end{definition}

Given $p \in \Procs$ and a transition $t = (\ell,a,\ell') \in \delta_p$, we let
$\tsource(t) = \ell$, $\ttarget(t) = \ell'$, $\tlabel(t) = a$, and
$\tmessage(t) = \msg$ if $a \in \msAct{\msg} \cup \mrAct{\msg}$.

\smallskip

There are in general two ways to define the semantics of a communicating system.
Most often it is defined as a global infinite transition system that keeps track
of the various local control states and all (unbounded) channel contents.
As, in this paper, our arguments are based on a graph view of MSCs, we will define
the language of $\Sys$ directly as a set of MSCs. These two semantic views are essentially
equivalent, but they have different advantages depending on the context.
We refer to \cite{CyriacG14} for a thorough discussion.

\davidequestion{MSCs are formally defined in chapter 5... maybe move here in the preliminaries just the asynchronous definition?}
Let $\msc = (\Events,\procrel,\lhd,\lambda)$ be an MSC.
A \emph{run} of $\Sys$ on $\msc$ is a mapping
$\rho: \Events \to \bigcup_{p \in \Procs} \delta_p$
that assigns to every event $e$ the transition $\rho(e)$
that is executed at $e$. Thus, we require that
\begin{enumerate*}[label={(\roman*)}]
\item for all $e \in \Events$, we have $\tlabel(\rho(e)) = \lambda(e)$,
\item for all $(e,f) \in {\procrel}$, $\ttarget(\rho(e)) = \tsource(\rho(f))$,
\item for all $(e,f) \in {\lhd}$, $\tmessage(\rho(e)) = \tmessage(\rho(f))$,
and
\item for all $p \in \Procs$ and $e \in \Events_p$ such that there is no $f \in \Events$ with $f \procrel e$, we have $\tsource(\rho(e)) = \ell_p^0$.
\end{enumerate*}

Letting run $\Sys$ directly on MSCs is actually very convenient.
This allows us to associate with $\Sys$ its p2p language and mailbox language
in one go. The \emph{\pp language} of $\Sys$ is $\ppL{\Sys} = \{\msc \in \ppMSCs \mid$ there is a run of $\Sys$ on $\msc\}$.
The \emph{mailbox language} of $\Sys$ is $\mbL{\Sys} = \{\msc \in \mbMSCs \mid$ there is a run of $\Sys$ on $\msc\}$.

Note that, following \cite{DBLP:conf/cav/BouajjaniEJQ18,DBLP:conf/fossacs/GiustoLL20},
we do not consider final states or final configurations, as our purpose is to
reason about all possible
traces that can be \emph{generated} by $\Sys$.
%We will discuss this issue in more detail later in the paper. \todo{Do we still discuss this in the conference version or can we omit this sentence?}
The next lemma is obvious for the p2p semantics and follows from Lemma~\ref{lem:mb-prefix} for
	the mailbox semantics.

\begin{lemma}\label{lem:prefix-closed}
For all $\comsymb \in \{\ppsymb, \mbsymb\}$, $\cL{\Sys}$ is prefix-closed:
$\Pref{\cL{\Sys}} \subseteq \cL{\Sys}$.
\end{lemma}

\begin{example}
	Fig.~\ref{fig:system_weak_univer} depicts $\systemweakuniver = (A_p, A_q, A_r)$ such that MSC $\mscweakuniver$ in Fig.~\ref{fig:msc_weak_univer} belongs to $\ppL{\systemweakuniver}$ and to $\mbL{\systemweakuniver}$.
	There is a unique run $\rho$ of $\systemweakuniver$ on $\mscweakuniver$.
	We can see that $(e_3',e_4) \in {\rightarrow}$ and $\ttarget(\rho(e_3')) = \tsource(\rho(e_4)) = \ell_r^{1}$, $(e_2, e_2') \in \lhd_{\mscweakuniver}$, and $\tmessage(\rho(e_2)) = \tmessage(\rho(e_2')) = \msg_2$.
	\end{example}
	\begin{figure}[t]
	\begin{center}
	  \begin{tikzpicture}[>=stealth,node distance=3.2cm,shorten >=1pt,
		every state/.style={text=black, scale =0.75}, semithick,
		font={\fontsize{8pt}{12}\selectfont},
		scale = 0.9
		]
	  \begin{scope}[->]
		  \node[state,initial,initial text={}] (q0)  {$\ell_p^{0}$};
		  \node[state, right of=q0] (q1)  {$\ell_p^{1}$};
				\node[state, right of=q1] (q2) {$\ell_p^{2}$};
	
			\path (q0) edge node [above] {$\send{p}{q}{\msg_1}$} (q1);
				\path (q1) edge node [above] {$\rec{q}{p}{\msg_2}$}(q2);
			\node[thick] at (-1.1,0) {$A_p$};
	  \end{scope}
	
	  \begin{scope}[->, shift={(7.5,0)}]
		  \node[state,initial,initial text={}] (q0)  {$\ell_q^{0}$};
				\node[state, right of=q0] (q1)  {$\ell_q^{1}$};
				\node[state, below of=q1, node distance = 1.5cm] (q2) {$\ell_q^{2}$};
				\node[state, left of=q2] (q3)  {$\ell_q^{3}$};
	
				\path (q0) edge node [above] {$\send{q}{p}{\msg_2}$} (q1);
				\path (q1) edge node [right] {$\send{q}{r}{\msg_3}$}(q2);
				\path (q2) edge node [above] {$\rec{r}{q}{\msg_4}$} (q3);
				\node[thick] at (-1.1,0) {$A_q$};
	  \end{scope}
	
		\begin{scope}[->, shift={(0,-1.25)} ]
		  \node[state,initial,initial text={}] (q0)  {$\ell_r^{0}$};
				\node[state, right of=q0] (q1)  {$\ell_r^{1}$};
				\node[state, right of=q1] (q2) {$\ell_r^{2}$};
	
				\path (q0) edge node [above] {$\rec{q}{r}{\msg_3}$} (q1);
				\path (q1) edge node [above] {$\send{r}{q}{\msg_4}$}(q2);
				\node[thick] at (-1.1,0) {$A_r$};
	  \end{scope}
	\end{tikzpicture}
	\captionof{figure}{System $\systemweakuniver$}
	\label{fig:system_weak_univer}
	\end{center}
	\end{figure}	

\subsection{Conflict Graph}

\davidequestion{The conflict graph is needed only for some proofs on the cuncur paper. I think it makes sense not to include it here and eventually introduce it later in the MSO-treewidth chapter if we really need it.}

We now recall the notion of a conflict graph associated to an MSC defined in \cite{DBLP:conf/cav/BouajjaniEJQ18}. This graph is used to depict the causal dependencies between message exchanges.  Intuitively, we have a dependency whenever
two messages have a process in common. For instance, an $\xrightarrow{SS}$
dependency between message exchanges $v$ and $v'$ expresses the fact that
$v'$ has been sent after $v$, by the same process. This notion is of interest because it was seen in \cite{DBLP:conf/cav/BouajjaniEJQ18} that the notion of synchronizability in MSCs (which is studied in this paper) can be graphically characterized by the nature of the associated conflict graph.
It is defined in terms of linearizations
in \cite{DBLP:conf/fossacs/GiustoLL20}, but we equivalently express it
directly in terms of MSCs.

\newcommand{\type}{\tau}
\newcommand{\stype}{S}
\newcommand{\rtype}{R}
\newcommand{\mexch}{\mu}
\newcommand{\Edges}{\mathit{Edges}}
\newcommand{\Nodes}{\mathit{Nodes}}

For an MSC $\msc = (\Events, \rightarrow, \lhd, \lambda)$ and
$e \in \Events$, we define the type $\type(e) \in \{\stype,\rtype\}$ of $e$ by $\type(e) = \stype$ if $e \in \SendEv{\msc}$
and $\type(e) = \rtype$ if $e \in \RecEv{\msc}$.
Moreover, for $e \in \Unm{\msc}$, we let $\mexch(e) = e$,
and for $(e,e') \in \lhd$, we let $\mexch(e) = \mexch(e') = (e,e')$.


\begin{definition}[Conflict graph]
	The \emph{conflict graph} $\cgraph{\msc}$ of an MSC $\msc = (\Events, \rightarrow, \lhd, \lambda)$ is the labeled graph $(\Nodes, \Edges)$, with $\Edges \subseteq \Nodes \times \{\stype,\rtype\}^2 \times \Nodes$, defined by
	$\Nodes = {\lhd} \cup \Unm{\msc}$ and $\Edges = \{(\mu(e),\type(e)\type(f),\mu(f)) \mid (e,f) \in {\to^+}\}$.
In particular, a node of $\cgraph{\msc}$ is either a single unmatched send event or a message pair $(e,e') \in {\lhd}$.
\end{definition}	

\subsection{Logic and Special treewidth}

\davidequestion{Special treewidth is used only in the final chapter, to me it makes more sense to introduce it there}

\paragraph*{Monadic Second-Order Logic.}
The set of MSO formulas over MSCs (over $\Procs$ and $\Msg$) is given by the grammar
$
\phi ::= x \procrel y \mid x \lhd y \mid \lambda(x) = a \mid x = y \mid x \in X \mid \exists x.\phi \mid \exists X.\phi \mid \phi \vee \phi \mid \neg \phi
$,
where $a \in \Act$, $x$ and $y$ are first-order variables, interpreted as
events of an MSC, and $X$ is a second-order variable, interpreted
as a set of events. We assume that we have an infinite supply of variables,
and we use common abbreviations such as $\wedge$, $\forall$, etc.
The satisfaction relation is defined in the standard way and self-explanatory.
For example, the formula $\neg\exists x.(\bigvee_{a \in \sAct} \lambda(x) = a \;\wedge\; \neg \mathit{matched}(x))$
with $\mathit{matched}(x) = \exists y.x \lhd y$
says that there are no unmatched send events.
It is not satisfied by  MSC $\mscweakuniver$
of Fig.~\ref{fig:msc_weak_univer},
as message $\msg_1$ is not received,
but by $\mscstrongexist$ from Fig.~\ref{fig:msc_strong_exist}.

Given a sentence $\phi$, i.e., a formula without free variables,
we let $L(\phi)$ denote the set of (p2p) MSCs that satisfy $\phi$.
%
It is worth mentioning that the (reflexive) transitive closure of
a binary relation defined by an MSO formula with free variables $x$ and $y$,
such as $x \procrel y$, is MSO-definable so that the logic can freely
use formulas of the form $x \procrel^+ y$ or $x \le y$ (where $\le$
is interpreted as $\le$ for the given MSC $\msc$).
Therefore, the definition of a mailbox MSC can be readily translated into
the formula $\mbformula = \neg \exists x.\exists y.(\neg (x = y) \wedge x \mbpartial y \wedge y \mbpartial x)$ so that we have $L(\mbformula) = \mbMSCs$.
Here, $x \mbpartial y$ is obtained as the MSO-definable reflexive transitive closure of
the union of the MSO-definable relations $\procrel$, $\lhd$, and $\sqsubset$.
In particular, we may define $x \sqsubset y$ by :
\[
x \sqsubset y =
\displaystyle
\hspace{-1em}\bigvee_{\substack{q \in \Procs\\a,b \in \qsAct{q}}}\hspace{-1em}
\lambda(x) = a \;\wedge\; \lambda(y) = b
\wedge
\left(
\begin{array}{rl}
& \mathit{matched}(x) \wedge \neg \mathit{matched}(y)\\[1ex]
\vee & \exists x'.\exists y'. (x \lhd x' \;\wedge\; y \lhd y' \;\wedge\; x' \procrel^+ y')
\end{array}
\right)
\]

\paragraph*{Special treewidth.}

\emph{Special treewidth} \cite{Courcelle10},
is a graph measure that indicates how close
a graph is to a tree (we may also use classical
	\emph{treewidth} instead).
This or similar measures are commonly employed in verification. For instance, treewidth and split-width have been used in \cite{MadhusudanP11} and, respectively, \cite{DBLP:conf/concur/CyriacGK12,AiswaryaGK14} to reason about graph behaviors generated by pushdown and queue systems.
%Here we apply it to reason about MSCs.
There are several ways to define the special treewidth of an MSC.
We adopt the following game-based definition from \cite{DBLP:journals/corr/abs-1904-06942}.

Adam and Eve play a two-player turn based "decomposition game"
whose positions
are MSCs with some pebbles placed on some events.
More precisely, Eve's positions are
\emph{marked MSC fragments} $(M, U)$, where
$\msc = (\Events, \procrel, \lhd, \lambda)$
is an \emph{MSC fragment} (an MSC with possibly some edges from
$\lhd$ or $\to$ removed)
 and $U \subseteq \Events$ is the subset of marked events.
Adam's positions are pairs of marked MSC fragments.
A move by Eve consists in the following steps:
\begin{enumerate}
	\item marking some events of the MSC resulting in $(M, U')$ with $U \subseteq U' \subseteq \Events$,
	\item removing (process and/or message) edges whose endpoints are marked,
	\item dividing $(M, U)$ in $(M_1, U_1)$ and $(M_2, U_2)$ such that $M$ is the disjoint (unconnected) union of $M_1$ and $M_2$
	and marked nodes are inherited.
\end{enumerate}
When it is Adam's turn, he simply chooses one of the two marked MSC fragments.
The initial position is $(\msc,\emptyset)$ where $M$ is the (complete) MSC at hand. A terminal position is any position belonging to Eve such that all events are marked.
%
For $k \in \N$, we say that the game is $k$-winning for Eve if she has a (positional) strategy that allows her,
starting in the initial position and independently of Adam's moves, to reach a terminal position such that, in every single position visited along the play, there are at most $k+1$ marked events.

\newcommand{\CS}[2]{\mathsf{CS}_{(#1,#2)}}
\newcommand{\MSO}[2]{\mathsf{MSO}_{(#1,#2)}}
\newcommand{\LCPDL}[2]{\mathsf{LCPDL}_{(#1,#2)}}
\newcommand{\MSCpm}[2]{\mathsf{MSC}_{(#1,#2)}}
\newcommand{\mbMSCpm}[2]{\mathsf{MSC}_{(#1,#2)}^{\mathsf{mb}}}


\begin{fact}[\cite{DBLP:journals/corr/abs-1904-06942}]
	The special treewidth of an MSC is the least $k$ such that
	the associated game is $k$-winning for Eve.
\end{fact}

The set of MSCs whose special treewidth is at most $k$ is denoted by $\stwMSCs{k}$.
% !TEX root = ../popl-paper.tex

Next, we informally present the communication models we consider. All of them impose different constraints on the order in which messages can be received.
For convenience, we will refer to the system implementing a given communication model with the name of the model. Implementation, or realizations of a communication model are discussed in the following section.
In a similar way, an MSC will be referred with the name of the communication model, since it represents a computation that is valid for that model.

 %In this section, we will present  7 different asynchronous communication models.
%We model a distributed system as a set of concurrent Finite-State Machines (FSMs) that exchange messages asynchronously through channels.
%Each FSM models a single machine/process of the system and transitions are labeled with "send" and "receive" operations, which specify the sender and the receiver of a message. In our work
%The role of the communication model is to impose an order on the reception of messages, according to its specification. For instance, the delivery of a message could be delayed or even prevented by a communication model $CM$, so as to ensure that messages are received in an order that is valid for $CM$. The 7 communication models that we address all impose different constraints on the order in which messages can be received.
%
\paragraph{\bf Fully asynchronous}
In the fully asychronous communication model (\asy) messages can be received at any time once they have been sent, and send events are non-blocking.
%, i.e., the sender of a message does not have to wait for it to be delivered to the recipient, in order to resume normal operations.
\asy systems can be realized (as described in \cite{DBLP:journals/fac/ChevrouHQ16} and \cite{DBLP:journals/tcs/BasuB16})  by a bag where all messages are stored and retrieved when necessary.
This communication model is also referred to as NON-FIFO (cfr.  \cite{DBLP:journals/dc/Charron-BostMT96}).
Fig.~\ref{fig:fully_asy_ex} shows a computation that can be executed by an \asy  system; indeed, even if message $m_1$ is sent before $m_2$, process $q$ does not have to receive $m_1$ first.   We will call $\asMSCs$ the set of all asynchronous MSCs.

\begin{figure}[t]
		\captionsetup[subfigure]{justification=centering}
	% \centering
	\begin{subfigure}[t]{0.2\textwidth}\centering

		\begin{tikzpicture}[scale=0.7, every node/.style={transform shape}]
			\newproc{0}{p}{-2.2};
			\newproc{2}{q}{-2.2};

			\newmsgm{0}{2}{-0.5}{-1.7}{1}{0.1}{black};
			\newmsgm{0}{2}{-1.7}{-0.5}{2}{0.25}{black};

			\end{tikzpicture}
		\caption{Only \asy.}	\label{fig:fully_asy_ex}

		\end{subfigure}
%
%		\begin{subfigure}[t]{0.25\textwidth}
%	\begin{center}
%		\begin{tikzpicture}[scale=0.7, every node/.style={transform shape}]
%			\newproc{0}{p}{-2.2};
%			\newproc{1}{q}{-2.2};
%			\newproc{2}{r}{-2.2};
%
%			\newmsgm{0}{1}{-0.3}{-1.7}{1}{0.1}{black};
%			\newmsgm{0}{2}{-0.7}{-0.7}{2}{0.7}{black};
%			\newmsgm{2}{1}{-1.3}{-1.3}{3}{0.3}{black};
%			\newmsgm{2}{1}{-1.9}{-1.9}{4}{0.3}{black};
%
%			\end{tikzpicture}
%		\caption{A \pp MSC.}
%		\label{fig:pp_ex}
%	\end{center}
%\end{subfigure}
%
	% \centering
	\begin{subfigure}[t]{0.25\textwidth}\centering
		\begin{tikzpicture}[scale=0.7, every node/.style={transform shape}]
			\newproc{0}{p}{-2.2};
			\newproc{1}{q}{-2.2};
			\newproc{2}{r}{-2.2};

			\newmsgm{0}{1}{-0.3}{-1.7}{1}{0.1}{black};
			\newmsgm{0}{2}{-0.9}{-0.9}{2}{0.7}{black};
			\newmsgm{2}{1}{-1.5}{-1.5}{3}{0.3}{black};
			\newmsgm{2}{1}{-2}{-2}{4}{0.3}{black};

			\newflechevert{Purple}{0}{-0.3}{-0.9};
			\newflechehor{Purple}{-0.9}{0}{2};
			\newflechevert{Purple}{2}{-0.9}{-1.5};
		\end{tikzpicture}
		\caption{\asy, \pp, not \co, \\not \mb, not $\onen$, \\not $\nn$, not $\rsc$.} \label{fig:pp_ex}
	\end{subfigure}
	% \hfill
	\begin{subfigure}[t]{0.2\textwidth}\centering
		\begin{tikzpicture}[scale=0.7, every node/.style={transform shape}]
			\newproc{0}{p}{-2.2};
			\newproc{1}{q}{-2.2};
			\newproc{2}{r}{-2.2};

			\newmsgm{0}{2}{-0.3}{-2}{1}{0.1}{black};
			\newmsgm{0}{1}{-1.3}{-1.3}{2}{0.3}{black};
			\newmsgm{2}{1}{-1.5}{-1.5}{3}{0.3}{black};

		\end{tikzpicture}
		\caption{\asy, \pp, \co, \mb, $\onen$, $\nn$, not $\rsc$.}	    \label{fig:co_ex}
	\end{subfigure}
\begin{subfigure}[t]{0.2\textwidth}\centering
	\begin{center}
		\begin{tikzpicture}[scale=0.7, every node/.style={transform shape}]
			\newproc{0}{p}{-2};
			\newproc{1}{q}{-2};
			\newproc{2}{r}{-2};

			\newmsgm{0}{1}{-0.5}{-0.5}{1}{0.3}{black};
			\newmsgm{1}{2}{-1}{-1}{2}{0.3}{black};
			\newmsgm{1}{0}{-1.6}{-1.6}{3}{0.3}{black};

		\end{tikzpicture}
		\caption{Only $\rsc$.}
		\label{fig:rsc_ex}
	\end{center}
\end{subfigure}

		\caption{Examples of MSCs for various communication models.}\label{fig:exmscs}

\end{figure}

\paragraph{\bf Peer-to-peer}
In the peer-to-peer ($\pp$ for short) communication model, any two messages sent from one process to another  are always received in the same order as they were sent. Systems are realized by processes that are  pairwise connected by FIFO channels. %i.e. messages are delivered by channels in the order in which they were sent\footnote{Please note that our definition of Communicating Finite-State Machine is different from the classical one. FIFO channels are replaced by bag channels, which do not ensure any specific order on the delivery of messages.}. This definition of Communicating Finite-State Machines clearly uses the $\oneone$ communication model, since we have FIFO channels between processes that take care of delivering messages in the correct order. The $\oneone$ communication model is referred to as \pp in \cite{DBLP:conf/concur/BolligGFLLS21}.
 \pp systems are also referred to as FIFO $1\mathsf{-}1$ systems as in  \cite{DBLP:journals/fac/ChevrouHQ16} or simply FIFO as in \cite{babaoglu1993consistent, DBLP:journals/dc/Charron-BostMT96, tel2000introduction}).
The MSC shown in Fig.~\ref{fig:fully_asy_ex} is not a $\pp$ MSC; as the receive order of messages $m_1$ and $m_2$  must match the send order.
Fig.~\ref{fig:pp_ex} shows an example of \pp MSC; the only two messages sent by and to the same process are $m_3$ and $m_4$, which are received in the same order  they have been sent. An example of linearization that can be executed by a $\pp$ system is $!1\;!2\;?2\;!3\;!4\;?3\;?1\;?4$. We denote by  $\ppMSCs$  the set of \pp MSCs.




\paragraph{\bf Causally ordered}
In the causally ordered (\co) communication model, messages are delivered to a process according to the causality of their emissions. In other words, if there are two messages $m_1$ and $m_2$ with the same recipient, such that $m_1$ is causally sent before $m_2$ (i.e., there exists a causal path from the first send to the second one), then $m_1$ must be received before $m_2$.
This type of partial order was introduced by Lamport in \cite{lamport2019time} with the "happened before" order. Later, some implementations was proposed in   \cite{kshemkalyani1998necessary, peterson1989preserving} and in \cite{coulouris2005distributed}, where the causal order is called FIFO ordering.
Fig.~\ref{fig:pp_ex}, shows an example of non-causally ordered MSC; there is a causal path between the sending of $m_1$ and $m_3$ (highlighted with red arrows), hence $m_1$ should be received before $m_3$, which is not the case here. On the other hand, Fig.~\ref{fig:co_ex} is \co; note that the only two messages with the same recipient are $m_2$ and $m_3$, but there is no causal path between their respective send events
%(i.e. the causally ordered communication model does not introduce any new constraint that must be satisfied).
 $\coMSCs$ is the set of causally ordered MSCs.


\paragraph{\bf Mailbox}
In the mailbox ($\mb$) communicating model, any two messages sent to a process  must be received in the same order as they have been sent (according to absolute time). Note that these two messages might be sent by different processes and the two send events might be concurrent (i.e., there is no causal path between them). In other words, if a process  receives $m_1$ before $m_2$, then $m_1$ must have been sent before $m_2$. Essentially, $\mb$ coordinates all the senders of a single receiver. For this reason the model is also called FIFO $n\mathsf{-}1$ \cite{DBLP:journals/fac/ChevrouHQ16}.   A high-level implementation of the mailbox communication model could consist in a single incoming FIFO channel for each process, which is shared by all the other processes. A send event would consist in pushing the message on the shared FIFO channel.
A low-level implementation needs a shared real-time clock \cite{cristian1999timed} or a global agreement on event order \cite{defago2004total, raynal2010communication}.
The MSC shown in Fig.~\ref{fig:pp_ex} is not a mailbox MSC; $m_1$ and $m_3$ have the same recipient, but they are not received in the same order as they are sent. The MSC in Fig.~\ref{fig:co_ex} is mailbox; indeed, we are able to find a linearization that respects the mailbox constraints, such as $!1\;!2\;!3\;?2\;?3\;?1$ (note that $m_2$ is both sent and received before $m_3$). Such a linearization will be referred to as a \emph{mailbox linearization}. At this stage, the difference between the class of causally ordered MSCs and the class of mailbox MSCs might not be clear. We will clarify later how all these classes of MSCs are related to each other. Let $\mbMSCs$ be the set of mailbox MSCs.

\paragraph{\bf $\onen$}
The $\onen$ communicating model is the dual of $\none$, it coordinates a sender with all the receivers. Any two messages sent by a process  must be received in the same order (in absolute time) as they are sent. Note that these two messages might be received by different processes and the two receive events might be concurrent.
% In other words, if a process $p$ sends $m_1$ before $m_2$, then $m_1$ must be received before $m_2$ in absolute time.
A high-level implementation of the $\onen$ communication model could consist in a single outgoing FIFO channel for each process, which is shared by all the other processes. A send event would consist in pushing the message on the outgoing FIFO channel.
%As this type of communication is the dual of the $\none$ one, the implementation would require similar tools as above.
The MSC shown in Fig.~\ref{fig:pp_ex} is not a $\onen$ MSC; $m_1$ and $m_2$ are sent in this order by the same process, but they are received in the opposite order (note that there is a causal path between the reception of $m_2$ and the reception of $m_1$, so $?2$ happens before $?1$ in every linearization of this MSC). Fig.~\ref{fig:co_ex} shows an example of $\onen$ MSC; $m_1$ is sent before $m_2$ by the same process, and we are able to find a linearization where $m_1$ is received before $m_2$, such as $!1\;!2\;!3\;?1\;?2\;?3$. Such a linearization will be referred to as a \emph{$\onen$ linearization}. Let $\onenMSCs$ be the set of $\onen$ MSCs.

\paragraph{\bf $\nn$}
In  $\nn$ communicating model, messages are globally ordered and delivered according to  their emission order. Any two messages must be received in the same order as they are sent, in absolute time. Note that these two messages might be received by different processes and the two receive events might be concurrent.
%In other words, if a message $m_1$ is sent before $m_2$ in absolute time, then $m_1$ must be received before $m_2$ in absolute time.
The $\nn$ coordinates all the senders with all the receivers. A high-level implementation of the $\nn$ communication model could consist in a single FIFO channel shared by all processes. It is considered also in \cite{DBLP:journals/tcs/BasuB16} where it is called  many-to-many (denoted $^\ast$-$^\ast$). But, as underlined in \cite{DBLP:journals/fac/ChevrouHQ16}, such an implementation would be inefficient and unrealistic.
 The MSC shown in Fig.~\ref{fig:pp_ex} is clearly not a $\nn$ MSC; if we consider messages $m_1$ and $m_2$ we have that, in every linearization, $!1$ happens before $!2$ and $?2$ happens before $?1$. This violates the constraints imposed by the $\nn$ communication model. The MSC in Fig.~\ref{fig:co_ex} is $\nn$ because we are able to find a linearization that satisfies the $\nn$ constraint, e.g. $!1\;!2\;!3\;?1\;?2\;?3$. Such a linearization will be referred to as an \emph{$\nn$ linearization}.  $\nnMSCs$ is the set of $\nn$ MSCs.

\paragraph{\bf Realizable with Synchronous Communication}
The Realizable with Synchronous Communication ($\rsc$) communication model imposes that a send event is  immediately followed by its corresponding receive event. It was introduced in \cite{DBLP:journals/dc/Charron-BostMT96}. It is the asynchronous model which is the  closest to the synchronous one. % An asynchronous distributed system that implements the $\rsc$ communication model effectively behaves as a synchronous system. 
%The authors of \cite{kshemkalyani2011distributed} propose a strategy to implement RSC executions from a synchronous system.  
Only the MSC shown in Fig.~\ref{fig:rsc_ex} is an example of $\rsc$ MSC; we can easily find a linearization that respects the constraints of the $\rsc$ communication model, such as $!1\;?1\;!2\;?2\;!3\;?3$. Such a linearization will be referred to as an \emph{$\rsc$ linearization}. Let $\rscMSCs$ be the set of $\rsc$ MSCs.

% !TEX root = ../popl-paper.tex


In Section~\ref{sec:com_models_overview} we gave a high-level description of 7 communication models and we talked about the corresponding classes of MSCs. Here, we formally define those classes and  show that they are all MSO-definable, i.e. there is a Monadic Second Order Logic formula that defines each of them. 

\subsection{Definitions}

We start with asynchronous MSCs, which represent valid computations for asynchronous systems. This is the most general definition of MSC, and it will serve as a basis on which the other communication models will build on, by adding some additional constraints.

\begin{definition}[Asynchronous MSC]
An \emph{asynchronous MSC} (or simply MSC) over $\Procs$ and $\Msg$ is a tuple $\msc = (\Events,\procrel,\lhd,\lambda)$, where $\Events$ is a finite (possibly empty) set of \emph{events} and $\lambda: \Events \to \Act$ is a labeling function that associates an action to each event. For $p \in \Procs$, let $\Events_p = \{e \in \Events \mid \lambda(e) \in \pAct{p}\}$ be the set of events that are executed by $p$. We require that $\procrel$ (the \emph{process relation}) is the disjoint union $\bigcup_{p \in \Procs} \procrel_p$ of relations ${\procrel_p} \subseteq \Events_p \times \Events_p$ such that $\procrel_p$ is the direct successor relation of a total order on $\Events_p$. For an event $e \in \Events$, a set of actions $A \subseteq \Act$, and a relation $\rel \subseteq \Events \times \Events$,
let $\sametype{e}{A}{\rel} = |\{f \in \Events \mid (f,e) \in \rel$ and $\lambda(f) \in A\}|$. We require that ${\lhd} \subseteq \Events \times \Events$ (the \emph{message relation}) satisfies the following:
\begin{itemize}\itemsep=0.5ex
\item[(1)] for every pair $(e,f) \in {\lhd}$, there is a send action $\sact{p}{q}{\msg} \in \Act$ such that $\lambda(e) = \sact{p}{q}{\msg}$, $\lambda(f) = \ract{p}{q}{\msg}$.
\item[(2)] for all $f \in \Events$ such that $\lambda(f)$ is a receive action, there is exactly one $e \in \Events$ such that $e \lhd f$.
\end{itemize}
Finally, letting ${\le}_\msc = ({\procrel} \cup {\lhd})^\ast$,
we require that $\le_\msc$ is a partial order. For convenience, we simply write $\le$ when $M$ is clear from the context. We will refer to $\le$ as the \emph{causal ordering} or \emph{happens-before} relation. If, for two events $e$ and $f$, we have that $e \le f$, we will equivalently say that there is a \emph{causal path} between $e$ and $f$.
\davide{Show example of causal path.}
\end{definition}

According to Condition (2), every receive event must have a matching send event. Note that, however, there may be unmatched send events.
We let
$\SendEv{\msc} = \{e \in \Events \mid \lambda(e)$ is a send
action$\}$,
$\RecEv{\msc} = \{e \in \Events \mid \lambda(e)$ is a receive
action$\}$,
$\Matched{\msc} = \{e \in \Events \mid$ there is $f \in \Events$
such that $e \lhd f\}$, and
$\Unm{\msc} = \{e \in \Events \mid \lambda(e)$ is a send
action and there is no $f \in \Events$ such that $e \lhd f\}$.
%
We do not distinguish isomorphic MSCs and
let $\asMSCs$ be the set of all the asynchronous MSCs over the given sets $\Procs$ and $\Msg$.

% \paragraph*{Linearizations.}

% Consider $\msc = (\Events,\procrel,\lhd,\lambda) \in \asMSCs$.
% An \emph{asynchronous linearization} of $\msc$ is a (reflexive) total order ${\linrel} \subseteq \Events \times \Events$ such that ${\le_\msc} \subseteq {\linrel}$. Intuitively, an asynchronous linearization of $\msc$ is a possible ordering of its events.
% \davide{Example of asynchronous linearization.}

\paragraph*{Linearizations.}

Consider $\msc = (\Events,\procrel,\lhd,\lambda) \in \asMSCs$.
A \emph{linearization} of $\msc$ is a (reflexive) total order ${\linrel} \subseteq \Events \times \Events$ such that ${\le_\msc} \subseteq {\linrel}$. In other words, a linearization of $\msc$ is a total order on the events that respects the happens-before relation $\le_\msc$ defined over $\msc$.
\davide{Provide example of linearization.}

\medskip

The class of asynchronous MSCs is the biggest and most general one out of the seven. All the others classes of MSCs will be obtained by adding some constraints to asynchronous MSCs, according to the specific communication model. As in the asynchronous case, we say that $\msc$ is a $\oneone$ MSC if there is a $\oneone$ system that can produce the behaviour shown by $\msc$. We give here the formal definition of $\oneone$ MSC, which also considers unmatched messages.

\begin{definition}[$\oneone$ MSCs]
A $\oneone$ MSC is an asynchronous MSC where we require that, for every pair $(e,f) \in {\lhd}$, such that $\lambda(e) = \sact{p}{q}{\msg}$, $\lambda(f) = \ract{p}{q}{\msg}$, we have $\sametype{e}{\pqsAct{p}{q}}{\procrel^+} = \sametype{f}{\pqrAct{p}{q}}{\procrel^+}$.
\end{definition}

The additional constraint satisfied by $\oneone$ MSCs ensures that messages sent from any fixed process $p$ to another fixed process $q$ are always received in the same order as they are sent, i.e. when $q$ receives a message from $p$, it must have already received all the messages that were previously sent to him by $p$. Note that, for each pair $(p,q)$ of processes, we cannot have an unmatched message $m_1$ (sent by $p$) followed by a matched message $m_2$ (sent by $p$). In order to receive $m_2$, $q$ must have already received $m_1$, according to the definition of $\oneone$ MSC; because of the FIFO policy, $m_1$ is blocking the reception of $m_2$. By definition, every $\oneone$ MSC is an asynchronous MSC. Indeed, it is always possible to find an asynchronous system that can realize a computation described by a $\oneone$ MSC. In other word, the possible behaviours, i.e. MSCs, generated by a system $S$ that uses $\oneone$ communication are a subset of all the behaviours that $S$ would be able to generate using asynchronous communication.

\medskip

We will now consider the class of causally ordered MSCs. Recall that an MSC is causally ordered if all the messages sent to the same process are received in an order which is consistent with the causal ordering of the corresponding send events. Below the formal definition, which also considers unmatched messages.
\davide{Provide example of $\oneone$ and non-$\oneone$ MSCs.}

\begin{definition}[Causally ordered MSC]
An MSC $\msc = (\Events,\procrel,\lhd,\lambda)$ is \emph{causally ordered} if, for any two send events $s$ and $s'$, such that $\lambda(s)=\pqsAct{\plh}{q}$, $\lambda(s')=\pqsAct{\plh}{q}$, and $s \le_\msc s'$, we have either:
\begin{itemize}\itemsep=0.5ex
	\item $s,s' \in \Matched{\msc}$ and $r \procrel^* r'$, where $r$ and $r'$ are two receive events such that $s \lhd r$ and $s' \lhd r'$.
	\item $s' \in \Unm{\msc}$.
\end{itemize}
\end{definition}

\davide{Provide example of causally ordered and non-causally ordered MSCs.}

Moving on to $\none$ communication, we say that $\msc$ is a $\none$ MSC if there is a $\none$ system that can realize the computation described by $\msc$.

\begin{definition}[$\none$ MSC]\label{def:mb_msc}
An MSC $\msc = (\Events,\procrel,\lhd,\lambda)$ is a \emph{$\none$ MSC} if it has a linearization $\linrel$ where, for any two send events $s$ and $s'$, such that $\lambda(s)=\pqsAct{\plh}{q}$, $\lambda(s')=\pqsAct{\plh}{q}$, and $s \linrel s'$, we have either:
\begin{itemize}\itemsep=0.5ex
	\item $s,s' \in \Matched{\msc}$. Note that $r \linrel r'$, since we have that $r \procrel^+ r'$.
	\item $s' \in \Unm{\msc}$.
\end{itemize}
\end{definition}


Such a linearization will be referred to as a \emph{$\none$ linearization}, and we will sometimes use the symbol $\mblinrel$ to denote one. Note that the definition of $\none$ MSC is based on the existence of a linearization that satisfies some properties. The same kind of "existential" definition will be used for the remaining communication models. In practice, to claim that an MSC is $\none$ we just need to find a single valid $\none$ linearization, regardless of all the others; that linearization is a total order on the events that can be executed by a $\none$ system. As for unmatched messages, note that we cannot have two messages $m_1$ and $m_2$, addressed to the same process but possibly sent by different processes, such that $m_1$ is unmatched and $m_2$ is matched; $m_2$ can only be received after $m_1$, and this is consistent with the high-level definition of the $\none$ communication model that we gave in Section~\ref{sec:com_models_overview}.

\davide{Provide example of $\none$ and non-$\none$ MSCs.}

Moving on to $\onen$ communication, we say that $\msc$ is a $\onen$ MSC if there is a $\onen$ system that can exhibit the behaviour described by $\msc$. 

\begin{definition}[$\onen$ MSC]\label{def:one_n}
An MSC $\msc = (\Events,\procrel,\lhd,\lambda)$ is a \emph{$\onen$ MSC} if it has a linearization $\linrel$ where, for any two send events $s$ and $s'$, such that $\lambda(s)=\pqsAct{p}{\plh}$, $\lambda(s')=\pqsAct{p}{\plh}$, and $s \procrel^+ s'$ (which implies $s \linrel s'$), we have either:
\begin{itemize}\itemsep=0.5ex
	\item $s,s' \in \Matched{\msc}$ and $r \linrel r'$, where $r$ and $r'$ are two receive events such that $s \lhd r$ and $s' \lhd r'$.
	\item $s' \in \Unm{\msc}$.
\end{itemize}
\end{definition}

Such a linearization will be referred to as a \emph{$\onen$ linearization}, and we will sometimes use the symbol $\onenlinrel$ to denote one. Note that the definition is very similar to the $\none$ case, but here $s$ and $s'$ are two send events executed by the same process, and not addressed to the same process. In a $\onen$ MSC we cannot have two messages $m_1$ and $m_2$, sent by the same process, such that $m_1$ is unmatched and $m_2$ is matched; indeed, according to the $\onen$ communication model, $m_1$ must be received before $m_2$.

\davide{Provide example of $\onen$ and non-$\onen$ MSCs.}

\begin{definition}[$\nn$ MSC]\label{def:n_n}
	An MSC $\msc = (\Events,\procrel,\lhd,\lambda)$ is a \emph{$\nn$ MSC} if it has a linearization $\linrel$ where, for any two send events $s$ and $s'$, such that $s \linrel s'$, we have either:
	\begin{itemize}\itemsep=0.5ex
		\item $s,s' \in \Matched{\msc}$ and $r \linrel r'$, where $r$ and $r'$ are two receive events such that $s \lhd r$ and $s' \lhd r'$.
		\item $s' \in \Unm{\msc}$.
	\end{itemize}
\end{definition}

Such a linearization will be referred to as a \emph{$\nn$ linearization}. Intuitively, with an $\nn$ MSC we are always able to schedule events in such a way that messages are received in the same order as they were sent, and unmatched messages are sent only after all matched messages are sent. By definition, every $\nn$ MSC is a $\onen$ MSC. 

\begin{definition}[$\rsc$ MSC]\label{def:rsc}
	An MSC $\msc = (\Events,\procrel,\lhd,\lambda)$ is a \emph{RSC MSC} if it has no unmatched send events and there is a linearization $\linrel$ where any matched send event is immediately followed by its respective receive event.
\end{definition}

Such a linearization will be referred to as a \emph{$\rsc$ linearization}.

% THIS IS WRONG!
% We show that the opposite direction is also true, which implies that the class of $\nn$ MSCs is equivalent to the class of $onen$ MSCs. The following example gives an intuition of the formal proof, which will be given right after.

% \begin{proposition}
% 	Every $\onen$ MSC without unmatched messages is an $\nn$ MSC.
% \end{proposition}
% \begin{proof}
% Let $\msc$ be a $\onen$ MSC without unmatched messages, and let $L$ be a $\onen$ linearization. We will show that, by reordering some of the events in $L$, we are always able to obtain a $\nn$ linearization for $\msc$. The algorithm works as follow:
% \begin{enumerate}
% 	\item Find a pair $(m_1,m_2)$ of distinct messages such that their send order in $L$ is the inverse of the receive order\footnote{Note that $L$ is already a $\nn$ linearization if such a pair does not exist}. This can only happen if there is not a causal path between $s_1$ and $s_2$, i.e $s_1 \le \ge s_2$, where $s_i$ is the send event of message $m_i$. To see why, suppose w.l.o.g. that $s_1 \le s_2$. As seen in the proof of Proposition~\ref{prop:onen_mb_no_unmatched}, there must be a message $m_k$ sent by the same process that sent $s_1$, such that we have $r_1 \onenrel r_k \le r_2$, and in particular $r_1 \lessdot r_2$. Therefore, if $s_1 \le s_2$, the receive order has to match the send order in $L$ and it will never be the opposite.
% 	\item Suppose, w.l.o.g. that $s_2 \linrel s_1$ and $r_1 \linrel r_2$, and we saw that $s_1 \le \ge s_2$. We would like to invert the order of $s_1$ and $s_2$ in the linearization, so that it matches the receive order. If we simply swap $s_2$ and $s_1$ in $L$, the new linearization could be invalid for $\msc$; for instance, there might be some events between $s_2$ and $s_1$ (in the linearization $L$) that need to happen before $s_1$. However, we show that it is still possible to invert the order of $s_1$ and $s_2$ without invalidating the linearization. Suppose that $s_2$ and $s_1$ are the $i$-th and the $j$-th events of $L$, respectively, where $i<j$. The idea is to move $s_2$ right before $s_1$, along with all the events on which $s_1$ depends, that are between $s_2$ and $s_1$ in $L$; we say that $s_1$ depends on an event $e$ if $e \lessdot s_1$.
% \end{enumerate}
% \end{proof}


\subsection{Alternative definitions}

In this section, we will provide some alternative equivalent definitions of $\none$ MSC, $\onen$ MSC, $\nn$ MSC, and $\rsc$ MSC. These definitions will be useful to prove the MSO-definability of these classes of MSCs.

\begin{definition} [$\none$ alternative]\label{def:n_one_alt}
	For an MSC $\msc = (\Events,\procrel,\lhd,\lambda)$, we define
	an additional binary relation that represents a constraint
	under the $\none$ semantics, which ensures that messages received by a process are sent in the same order as they are received.
	Let ${\mbrel}_\msc \subseteq \Events \times \Events$
	be defined as $s \mbrel_\msc s'$ if there is $q \in \Procs$
	such that $\lambda(s) \in \qsAct{q}$,
	$\lambda(s') \in \qsAct{q}$, and one of the following holds:
	\begin{itemize}\itemsep=0.5ex
		\item $s \in \Matched{\msc}$ and $s' \in \Unm{\msc}$, or
		\item $s \lhd f_1$ and $s' \lhd f_2$ for some $f_1,f_2 \in \Events_q$ such that $f_1 \procrel^+ f_2$.
	\end{itemize}
	
	We let ${\preceq_\msc} = ({\procrel} \,\cup\, {\lhd} \,\cup\, {\mbrel_\msc})^\ast$.
	Note that ${\le_\msc} \subseteq {\preceq_\msc}$.
	We call $\msc \in \asMSCs$ a \emph{$\none$ MSC}
	if ${\preceq_\msc}$ is a partial order. The ${\mbrel}_\msc$ relation ensures that send events addressed to the same process are executed in an order that is suitable for the $\none$ communication. Note that if ${\preceq_\msc}$ is a partial order, it means that it is possible to find a linearization $\linrel$, i.e. a total order on the events, such that $\linrel \subseteq \preceq_\msc$. It is not difficult to see that such a linearization is exactly what we called a $\none$ linearization in Definition~\ref{def:mb_msc}. The two definition of $\none$ MSC that we gave are equivalent.
\end{definition}
	
\davide{Proof of equivalence of 2 definitions?}

\begin{definition} [$\onen$ alternative]\label{def:one_n_alt}
	For an MSC $\msc = (\Events,\procrel,\lhd,\lambda)$, we define
	an additional binary relation that represents a constraint
	under the $\onen$ semantics, which ensures that messages sent from the same process are received in the same order. Let ${\onenrel}_\msc \subseteq \Events \times \Events$ be defined as $e_1 \onenrel_\msc e_2$ if there are two events $e_1$ and $e_2$, and $p \in \Procs$ such that either:
	\begin{itemize}\itemsep=0.5ex
		\item $\lambda(e_1) \in \psAct{p}$, $\lambda(e_2) \in \psAct{p}$, $e_1 \in \Matched{\msc}$, and $e_2 \in \Unm{\msc}$, or
		\item $\lambda(e_1) \in \prAct{p}$, $\lambda(e_2) \in \prAct{p}$, $s_1 \lhd e_1$ and $s_2 \lhd e_2$ for some $s_1,s_2 \in \Events_p$, and $s_1 \procrel^+ s_2$.
	\end{itemize}
	
	We let ${\lessdot_\msc} = ({\procrel} \,\cup\, {\lhd} \,\cup\, {\onenrel_\msc})^\ast$.
	Note that ${\le_\msc} \subseteq {\lessdot_\msc}$.
	%
	%\begin{definition}\label{def:mailbox-msc}
	We call $\msc \in \asMSCs$ a \emph{$\onen$ MSC}
	if ${\lessdot_\msc}$ is a partial order. The ${\onenrel}_\msc$ relation ensures that messages sent by a process are sent and received in an order that is suitable for the $\onen$ communication. Note that if ${\lessdot_\msc}$ is a partial order, it is possible to find a linearization $\linrel$, such that $\linrel \subseteq \lessdot_\msc$. It is not difficult to see that such a linearization is exactly what we called a $\onen$ linearization in Definition~\ref{def:one_n}. The two definition of $\onen$ MSC that we gave are equivalent.
\end{definition}

\davide{Proof of equivalence of 2 definitions?}

\begin{definition} [$\nn$ alternative]\label{def:n_n_alt}
	For an MSC $\msc = (\Events,\procrel,\lhd,\lambda)$, let ${\nnrel}_\msc = ({\procrel} \,\cup\, {\lhd} \,\cup\, {\mbrel_\msc} \,\cup\, {\onenrel_\msc})^\ast$. We define an additional binary relation $\bowtie_\msc \subseteq \Events \times \Events$, such that for two events $e_1$ and $e_2$ we have $e_1 \bowtie_\msc e_2$ if one of the following holds:
	\begin{enumerate}\itemsep=0.5ex
		\item $e_1 \nnrel_\msc e_2$
		\item $\lambda(e_1) \in \prAct{\plh}$, $\lambda(e_2) \in \prAct{\plh}$, $s_1 \lhd e_1$ and $s_2 \lhd e_2$ for some $s_1,s_2 \in \Events$, $s_1 \nnrel_\msc s_2$ and $e_1 \slashed{\nnrel}_\msc e_2$.
		\item $\lambda(e_1) \in \psAct{\plh}$, $\lambda(e_2) \in \psAct{\plh}$, $e_1 \lhd r_1$ and $e_2 \lhd r_2$ for some $r_1,r_2 \in \Events$, $r_1 \nnrel_\msc r_2$ and $e_1 \slashed{\nnrel}_\msc e_2$.
		\item $e_1 \in \Matched{\msc}$, $e_2 \in \Unm{\msc}$, $e_1 \slashed{\nnrel}_\msc e_2$.
	\end{enumerate}
	
	Note that $\preceq_\msc \subseteq \nnrel_\msc$, $\lessdot_\msc \subseteq \nnrel_\msc$, and $\nnrel_\msc \subseteq\; \bowtie_\msc$. We call $\msc \in \asMSCs$ a \emph{$\nn$ MSC}
	if ${\bowtie_\msc}$ is acyclic.
	\davidequestion{Initially I said "We call $\msc \in \asMSCs$ a \emph{$\nn$ MSC} if ${\bowtie_\msc}$ is a partial order", but I don't think it is true, since $\bowtie_\msc$ does not have to be transitive... correct?}
\end{definition}

It is not trivial to see that Definition~\ref{def:n_n_alt} and Definition~\ref{def:n_n} are equivalent. To show that, we need some preliminary results and definitions. 

\begin{proposition}
	Let $\msc$ be an MSC. Given two matched send events $s_1$ and $s_2$, and their respective receive events $r_1$ and $r_2$, $r_1 \bowtie_\msc r_2 \implies s_1 \bowtie_\msc s_2$.
\end{proposition}
\begin{proof}
Follows from the definition of $\bowtie_\msc$. We have $r_1 \bowtie_\msc r_2$ if either:
\begin{itemize}\itemsep=0.5ex
	\item $r_1 \nnrel_\msc r_2$. Two cases: either \begin{enumerate*}[label={(\roman*)}]
		\item $s_1 \nnrel_\msc s_2$, or 
		\item $s_1 \slashed{\nnrel}_\msc s_2$.
	\end{enumerate*}
	The first case clearly implies $s_1 \bowtie_\msc s_2$, for rule 1 in the definition of $\bowtie_\msc$. The second too, because of rule 3.
	\item  $r_1 \slashed{\nnrel}_\msc r_2$, but $r_1 \bowtie_\msc r_2$. This is only possible if rule 2 in the definition of $\bowtie_\msc$ was used, which implies $s_1 \nnrel_\msc s_2$ and, for rule 1, $s_1 \bowtie_\msc s_2$.
\end{itemize}
\end{proof}

\begin{proposition}\label{prop:n_n_cycl}
	Let $\msc$ be an MSC. If $\bowtie_\msc$ is cyclic, then $\msc$ is not $\nn$.
\end{proposition}
\begin{proof}
According to Definition~\ref{def:n_n}, an MSC is $\nn$ if it has at least one $\nn$ linearization. Note that, because of how it is defined, any $\nn$ linearization is always both a $\none$ and a $\onen$ linearization. It follows that the cyclicity of $\nnrel_\msc$ (not $\bowtie_\msc$) implies that $\msc$ is not $\nn$, because it means that we are not even able to find a linearization that is both $\none$ and $\onen$. Moreover, since in a $\nn$ linearization the order in which messages are sent matches the order in which they are received, and unmatched send events can be executed only after matched send events, a $\nn$ MSC always has to satisfy the constraints imposed by the $\bowtie_\msc$ relation. If $\bowtie_\msc$ is cyclic, then for sure there is no $\nn$ linearization for $\msc$.
\davide{This proof does not convince me... maybe rewrite it better}
\end{proof}

Let the \emph{Event Dependency Graph} (EDG) of a $\nn$ MSC $\msc$ be a graph that has events as nodes and an edge between two events $e_1$ and $e_2$ if $e_1 \bowtie_\msc e_2$. We now present an algorithm that, given the EDG of an $\nn$ MSC $\msc$, computes a $\nn$ linearization of $\msc$. We then show that, if $\bowtie_\msc$ is acyclic (i.e. it is a partial order), this algorithm always terminates correctly. This, along with Proposition~\ref{prop:n_n_cycl}, effectively shows that Definition~\ref{def:n_n} and Definition~\ref{def:n_n_alt} are equivalent.

\paragraph*{Algorithm for finding a $\nn$ linearization}
The input of this algorithm is the EDG of an MSC $\msc$, and it outputs a valid $\nn$ linearization for $\msc$, if $\msc$ is $\nn$. The algorithm works as follows:
\begin{enumerate}
	\item If there is a matched send event $s$ with in-degree 0 in the EDG, add $s$ to the linearization and remove it from the EDG, along with its outgoing edges, then jump to step 5. Otherwise, proceed to step 2.
	\item If there are no matched send events in the EDG and there is an unmatched send event $s$ with in-degree 0 in the EDG, add $s$ to the linearization and remove it from the EDG, along with its outgoing edges, then jump to step 5. Otherwise, proceed to step 3.
		\item If there is a receive event $r$ with in-degree 0 in the EDG, such that $r$ is the receive event of the first message whose sent event was already added to the linearization, add $r$ to the linearization and remove it from the EDG, along with its outgoing edges, then jump to step 5. Otherwise, proceed to step 4.
		\item Throw an error and terminate.
		\item If all the events of $\msc$ were added to the linearization, return the linearization and terminate. Otherwise, go back to step 1.
\end{enumerate} 

We now need to show that 
\begin{enumerate*}[label={(\roman*)}]
	\item if this algorithm terminates correctly (i.e. step 4 is never executed), it returns a $\nn$ linearization, and 
	\item if $\bowtie_\msc$ is acyclic, the algorithm always terminates correctly.
\end{enumerate*}
\begin{proposition}
	If the above algorithm returns a linearization for an MSC $\msc$, it is a $\nn$ linearization.
\end{proposition}
\begin{proof}
	Step 2 ensures that the order (in the linearization) in which matched messages are sent is the same as the order in which they are received. Moreover, according to step 3, an unmatched send events is added to the linearization only if all the matched send events were already added.
\end{proof}

\begin{proposition}
	Given an MSC $\msc$, the above algorithm always terminates correctly if $\bowtie_\msc$ is acyclic.
\end{proposition}
\begin{proof}
We want to prove that, if $\bowtie_\msc$ is acyclic, step 4 of the algorithm is never executed, i.e. it terminates correctly. Note that the acyclicity of $\bowtie_\msc$ implies that the EDG of $\msc$ is a DAG. Moreover, at every step of the algorithm we remove nodes and edges from the EDG, so it still remains a DAG. The proof goes by induction on the number of events added to the linearization.\newline
Base case: no event has been added to the linearization yet. Since the EDG is a DAG, there must be an event with in-degree 0. In particular, this has to be a send event (a receive event depends on its respective send event, so it cannot have in-degree 0). If it is a matched send event, step 1 is applied. If there are no matched send events, step 2 is applied on an unmatched send. We show that it is impossible to have an unmatched send event of in-degree 0 if there are still matched send events in the EDG, so either step 1 or 2 are applied in the base case. Let $s$ be one of those matched send events and let $u$ be an unmatched send. Because of rule 4 in the definition of $\bowtie_\msc$, we have that $s \bowtie_\msc u$, which implies that $u$ cannot have in-degree 0 if $s$ is still in the EDG.\newline
Inductive step: we want to show that we are never going to execute step 4. In particular, Step 4 is executed when none of the first three steps can be applied. This happens when there are no matched send events with in-degree 0 and one of the following holds:
\begin{itemize}\itemsep=0.5ex
	\item \emph{There are still matched send events in the EDG with in-degree $>0$, there are no unmatched messages with in-degree 0, and there is no receive event $r$ with in-degree 0 in the EDG, such that $r$ is the receive event of the first message whose sent event was already added to the linearization}. Since the EDG is a DAG, there must be at least one receive event with in-degree 0. We want to show that, between these receive events with in-degree 0, there is also the receive event $r$ of the first message whose send event was added to the linearization, so that we can apply step 3 and step 4 is not executed. Suppose, by contradiction, that $r$ has in-degree $>0$, so it depends on other events. For any maximal chain in the EDG that contains one of these events, consider the first event $e$, which clearly has in-degree 0. In particular, $e$ cannot be a send event, because we would have applied step 1 or step 2. Hence, $e$ can only be a receive event for a send event that was not the first added to the linearization (and whose respective receive still has not been added). However, this is also impossible, since $r_e \bowtie_\msc r$ implies $s_e \bowtie_\msc s$, and we could not have added $s$ to the linearization before $s_e$. Because we got to a contradiction, the hypothesis that $r$ has in-degree $>0$ must be false, and we can indeed apply step 3.
	\item \emph{There are still matched send events in the EDG with in-degree $>0$, there is at least one unmatched message with in-degree 0, and there is no receive event $r$ with in-degree 0 in the EDG, such that $r$ is the receive event of the first message whose sent event was already added to the linearization}. We show that it is impossible to have an unmatched send event of in-degree 0 if there are still matched send events in the EDG. Let $s$ be one of those matched send events and let $u$ be an unmatched send. Because of rule 4 in the definition of $\bowtie_\msc$, we have that $s \bowtie_\msc u$, which implies that $u$ cannot have in-degree 0 if $s$ is still in the EDG.
	\item \emph{There are no more matched send events in the EDG, there are no unmatched messages with in-degree 0, and there is no receive event $r$ with in-degree 0 in the EDG, such that $r$ is the receive event of the first message whose sent event was already added to the linearization}. Very similar to the first case. Since the EDG is a DAG, there must be at least one receive event with in-degree 0. We want to show that, between these receive events with in-degree 0, there is also the receive event $r$ of the first message whose send event was added to the linearization, so that we can apply step 3 and step 4 is not executed. Suppose, by contradiction, that $r$ has in-degree $>0$, so it depends on other events. For any maximal chain in the EDG that contains one of these events, consider the first event $e$, which clearly has in-degree 0. In particular, $e$ cannot be a send event, because by hypothesis there are no more send events with in-degree 0 in the EDG. Hence, $e$ can only be a receive event for a send event that was not the first added to the linearization (and whose respective receive still has not been added). However, this is also impossible, since $r_e \bowtie_\msc r$ implies $s_e \bowtie_\msc s$, and we could not have added $s$ to the linearization before $s_e$. Because we got to a contradiction, the hypothesis that $r$ has in-degree $>0$ must be false, and we can indeed apply step 3.
\end{itemize}
We showed that, if $\bowtie_\msc$ is acyclic, the algorithm always terminates correctly and computes a valid $\nn$ linearization.
\end{proof}

\noindent We have now effectively proved that Definition~\ref{def:n_n_alt} of $\nn$ MSC is equivalent to Definition~\ref{def:n_n}.

\medskip

Following the characterization given in \cite[Theorem 4.4]{DBLP:journals/dc/Charron-BostMT96}, we also give an alternative but equivalent definition of $\rsc$ MSC.

\begin{definition}
	Let $\msc$ be an MSC. A crown of size $k$ in $\msc$ is a sequence $\langle(s_i,r_i),\, i \in \{1,\dots,k\}\rangle$ of pairs of corresponding send and receive events such that
	\[
		s_1 <_\msc r_2, s_2 <_\msc r_3, \dots, s_{k-1} <_\msc r_k, s_k <_\msc r_1.
	\]
\end{definition}
\davide{Specify somewhere that $<_\msc = (\procrel \cup \lhd)^+$.}

\begin{definition} [$\rsc$ alternative]\label{def:rsc_alt}
	An MSC $\msc = (\Events,\procrel,\lhd,\lambda)$ is a \emph{RSC MSC} if and only if it does not contain any crown.
\end{definition}


\subsection{Monadic Second-Order Logic}

The set of MSO formulas over (asynchronous) MSCs (over $\Procs$ and $\Msg$) is given by the grammar
$
\phi ::= true \mid x \procrel y \mid x \lhd y \mid \lambda(x) = a \mid x = y \mid x \in X \mid \exists x.\phi \mid \exists X.\phi \mid \phi \vee \phi \mid \neg \phi
$,
where $a \in \Act$, $x$ and $y$ are first-order variables, interpreted as
events of an MSC, and $X$ is a second-order variable, interpreted
as a set of events. We assume that we have an infinite supply of variables,
and we use common abbreviations such as $\wedge$, $\Rightarrow$, $\forall$, etc.
The satisfaction relation is defined in the standard way and self-explanatory.
For example, the formula $\neg\exists x.(\bigvee_{a \in \sAct} \lambda(x) = a \;\wedge\; \neg \mathit{matched}(x))$
with $\mathit{matched}(x) = \exists y.x \lhd y$
says that there are no unmatched send events.
It is not satisfied by  MSC $\mscweakuniver$
of Fig.~\ref{fig:msc_weak_univer},
as message $\msg_1$ is not received,
but by $\mscstrongexist$ from Fig.~\ref{fig:msc_strong_exist}.

Given a sentence $\phi$, i.e., a formula without free variables,
we let $L(\phi)$ denote the set of MSCs that satisfy $\phi$. Since we have defined the set of MSO formulas over asynchronous MSCs, the formula $\asformula = true$ clearly describes the set of asynchronous MSCs, i.e. $L(\asformula) = \asMSCs$. It is worth mentioning that the (reflexive) transitive closure of a binary relation defined by an MSO formula\footnote{See Section~\ref{sec:mso_extra} for details.} with free variables $x$ and $y$, such as $x \procrel y$, is MSO-definable so that the logic can freely use formulas of the form $x \procrel^+ y$, $x \procrel^* y$ or $x \le y$ (where $\le$ is interpreted as $\le_\msc$ for the given MSC $\msc$).

\paragraph*{$\oneone$ MSCs}
	The set of $\oneone$ MSCs is MSO-definable as
	\[
		\ppformula = \neg \exists s.\exists s'. \left(
		\bigvee_{\substack{p \in \Procs, q \in \Procs}}\;
		\bigvee_{\substack{a,b \in \pqsAct{p}{q}}}\hspace{-1em}
		(\lambda(s) = a \;\wedge\; \lambda(s') = b) \;\wedge\; s \procrel^+ s' \;\wedge\;
		(\psi_1 \vee \psi_2 ) 	
		\right)
	\]
	where $\psi_1$ and $\psi_2$ are
	\[
		\psi_1 = \exists r.\exists r'.\left(
		\begin{array}{ll}
			s \lhd r & \wedge\\
			s' \lhd r' & \wedge\\
			r' \procrel^+ r &
		\end{array} 
		\right) \quad \quad
		\psi_2 = (\neg \mathit{matched}(s) \wedge \mathit{matched}(s'))
		\]
		\[
		matched(x) = \exists y. x \lhd y
	\]

The property $\ppformula$ says that there cannot be two matched send events $s$ and $s'$, with the same sender and receiver, such that either
\begin{enumerate*}[label={(\roman*)}]
	\item $s \procrel^+ s'$ and their receipts happen in the reverse order, or
	\item $s$ is unmatched and $s'$ is matched.
\end{enumerate*}
In other words, it ensures that channels operate in FIFO mode, where an unmatched messages blocks the receipt of all the subsequent messages on that channel.
The set $\ppMSCs$ is therefore MSO-definable as $\ppMSCs=L(\ppformula)$.

\paragraph*{Causally ordered MSCs}
Given an MSC $\msc$, it is causally ordered if and only if it satisfies the MSO formula
\[
	\coformula = \neg \exists s.\exists s'. \left(
	\bigvee_{\substack{q \in \Procs}}\;
	\bigvee_{\substack{a,b \in \pqsAct{\plh}{q}}}\hspace{-1em}
	(\lambda(s) = a \;\wedge\; \lambda(s') = b) \;\wedge\; s \le_\msc s' \;\wedge\;
	(\psi_1 \vee \psi_2 ) 	
	\right)
\]
where $\psi_1$ and $\psi_2$ are the same formulas used for \pp.

The property $\coformula$ says that there cannot be two send events $s$ and $s'$, with the same recipient, such that $s \le_\msc s'$ and either
\begin{enumerate*}[label={(\roman*)}]
	\item their corresponding receive events $r$ and $r'$ happen in the opposite order, i.e. $r' \procrel^+ r$, or
	\item $s$ is unmatched and $s'$ is matched.
\end{enumerate*}
The set $\coMSCs$ of causally ordered MSCs is therefore MSO-definable as $\coMSCs=L(\coformula)$.

\paragraph*{$\none$ MSCs}

Given an MSC $\msc$, it is a $\none$ MSC if and only if it satisfies the MSO formula
\[
	\mbformula = \ppformula \;\wedge\; \neg \exists x.\exists y.(\neg (x = y) \wedge x \preceq_\msc y \wedge y \preceq_\msc x)
\]
This formula closely follows Definition~\ref{def:n_one_alt}. The set $\mbMSCs$ of $\none$ MSCs is therefore MSO-definable as $\mbMSCs=L(\mbformula)$.

\paragraph*{$\onen$ MSCs}

Following Definition~\ref{def:one_n_alt}, an MSC $\msc$ is a $\onen$ MSC if and only if it satisfies the MSO formula
\[
	\onenformula = \neg \exists x.\exists y.(\neg (x = y) \wedge x \lessdot_\msc y \wedge y \lessdot_\msc x)
\]
Recall that $\lessdot_\msc$ is the union of the MSO-definable relations $\procrel$, $\lhd$, and $\onenrel_\msc$. In particular, we can define $x \onenrel_\msc y$ as 
\[
x \onenrel_\msc y =
\begin{array}{rl}
& \left(
	\bigvee_{\substack{p \in \Procs\\a,b \in \psAct{p}}}\hspace{-1em}
	(\lambda(x) = a \;\wedge\; \lambda(y) = b)
	\;\wedge\; \mathit{matched}(x) \;\wedge\; \neg \mathit{matched}(y)
\right) \;\vee\\
& \left(
	\bigvee_{\substack{p \in \Procs\\a,b \in \prAct{p}}}\hspace{-1em}
	(\lambda(x) = a \;\wedge\; \lambda(y) = b)
	\;\wedge\; 
	\exists x'.\exists y'. (x' \lhd x \;\wedge\; y' \lhd y \;\wedge\; x' \procrel^+ y')
\right)\\
\end{array}
\]
The MSO formula for $x \onenrel_\msc y$ closely follows Definition~\ref{def:one_n_alt}. The set $\onenMSCs$ of $\onen$ MSCs is therefore MSO-definable as $\onenMSCs=L(\onenformula)$.

\paragraph*{$\nn$ MSCs}

Following Definition~\ref{def:n_n_alt}, an MSC $\msc$ is a $\nn$ MSC if and only if it satisfies the MSO formula
\[
	\nnformula = \neg \exists x.\exists y.(\neg (x = y) \wedge x \bowtie_\msc y \wedge y \bowtie_\msc x)
\]
In particular, we can define $x \bowtie_\msc y$ as
\[
	x \bowtie_\msc y =
	\begin{array}{rl}
	& \left(
		\bigvee_{\substack{a,b \in \psAct{\plh}}}
		(\lambda(x) = a \;\wedge\; \lambda(y) = b)
		\;\wedge\; \mathit{matched}(x) \;\wedge\; \neg \mathit{matched}(y)
	\right) \;\vee\\
	& (x \nnrel_\msc y) \quad \vee \quad \psi_3 \quad \vee \quad \psi_4\\
	\end{array}
\]

\noindent where $\psi_3$ and $\psi_4$ are defined as 
\[
	\psi_3 =
	\begin{array}{rl}
		& \bigvee_{\substack{a,b \in \prAct{\plh}}}
		  (\lambda(x) = a \;\wedge\; \lambda(y) = b)
		  \;\wedge\; \\
		& \exists x'.\exists y'.(x' \lhd x \;\wedge\; y' \lhd y) \;\wedge\; (x' \nnrel_\msc y') \;\wedge\; \neg(x \nnrel_\msc y)\\
	\end{array}
\]
\[
	\psi_4 =
	\begin{array}{rl}
		& \bigvee_{\substack{a,b \in \psAct{\plh}}}
		  (\lambda(x) = a \;\wedge\; \lambda(y) = b)
		  \;\wedge\; \\
		& \exists x'.\exists y'.(x \lhd x' \;\wedge\; y \lhd y') \;\wedge\; (x' \nnrel_\msc y') \;\wedge\; \neg(x \nnrel_\msc y)\\
	\end{array}
\]

The MSO formula for $x \nnrel_\msc y$ closely follows Definition~\ref{def:n_n_alt}. The set $\nnMSCs$ of $\nn$ MSCs is therefore MSO-definable as $\nnMSCs=L(\nnformula)$.

\paragraph*{$\rsc$ MSCs}

Following Definition~\ref{def:rsc_alt}, an MSC $\msc$ is a $\rsc$ MSC if and only if it satisfies the MSO formula
\[
\Phi_{\rsc} = \neg \exists s_1.\exists s_2. s_1 \varpropto s_2 \;\wedge\; s_2 \varpropto^\ast s_1
\]
\noindent where $\varpropto$ is defined as
\[
s_1 \varpropto s_2 = 
\bigvee_{\substack{e \in \sAct}}(\lambda(s_1) = e) \;\wedge\;
s_1 \neq s_2 \;\wedge\; 
\exists r_2. (s_1 < r_2 \;\wedge\; s_2 \lhd r_2)
\]
% \davidequestion{The following formula should be wrong... I cannot use $n$ in an MSO formula}
% \[
% 	\rscformula = 
% 	\begin{array}{rl}
% 		& \forall x.\left(\bigvee_{a \in \psAct{\plh}} \lambda(x) = a \;\implies\; \mathit{matched}(x)\right) \;\wedge\; \\
% 		& \neg \left( 
% 			\bigvee_{k=1}^n \left(
% 			\exists s_1 \cdots s_k. \exists r_1 \cdots r_k. 
% 			\bigwedge_{i=1}^k (s_i \lhd r_i \;\wedge\; s_i \le_\msc r_{(i+1)\%k})
% 		\right)
% 		\right)\\
% 	\end{array}
% \]
% where $n$ is the total number of messages. The formula checks that there are no unmatched send events and that in the conflict graph there is no cycle, of any length, whose edges are all SR.



% Bibliography
\bibliography{bibfile}

\end{document}
