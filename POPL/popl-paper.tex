%% For double-blind review submission, w/o CCS and ACM Reference (max submission space)
\documentclass[acmsmall,review,anonymous]{acmart}\settopmatter{printfolios=true,printccs=false,printacmref=true}
%% For double-blind review submission, w/ CCS and ACM Reference
%\documentclass[acmsmall,review,anonymous]{acmart}\settopmatter{printfolios=true}
%% For single-blind review submission, w/o CCS and ACM Reference (max submission space)
%\documentclass[acmsmall,review]{acmart}\settopmatter{printfolios=true,printccs=false,printacmref=false}
%% For single-blind review submission, w/ CCS and ACM Reference
%\documentclass[acmsmall,review]{acmart}\settopmatter{printfolios=true}
%% For final camera-ready submission, w/ required CCS and ACM Reference
%\documentclass[acmsmall]{acmart}\settopmatter{}


%% Journal information
%% Supplied to authors by publisher for camera-ready submission;
%% use defaults for review submission.
\acmJournal{PACMPL}
\acmVolume{1}
\acmNumber{CONF} % CONF = POPL or ICFP or OOPSLA
\acmArticle{1}
\acmYear{2018}
\acmMonth{1}
\acmDOI{} % \acmDOI{10.1145/nnnnnnn.nnnnnnn}
\startPage{1}

%% Copyright information
%% Supplied to authors (based on authors' rights management selection;
%% see authors.acm.org) by publisher for camera-ready submission;
%% use 'none' for review submission.
\setcopyright{none}
%\setcopyright{acmcopyright}
%\setcopyright{acmlicensed}
%\setcopyright{rightsretained}
%\copyrightyear{2018}           %% If different from \acmYear

%% Bibliography style
\bibliographystyle{ACM-Reference-Format}
%% Citation style
%% Note: author/year citations are required for papers published as an
%% issue of PACMPL.
\citestyle{acmauthoryear}   %% For author/year citations


%%%%%%%%%%%%%%%%%%%%%%%%%%%%%%%%%%%%%%%%%%%%%%%%%%%%%%%%%%%%%%%%%%%%%%
%% Note: Authors migrating a paper from PACMPL format to traditional
%% SIGPLAN proceedings format must update the '\documentclass' and
%% topmatter commands above; see 'acmart-sigplanproc-template.tex'.
%%%%%%%%%%%%%%%%%%%%%%%%%%%%%%%%%%%%%%%%%%%%%%%%%%%%%%%%%%%%%%%%%%%%%%


%% Some recommended packages.
\usepackage{booktabs}   %% For formal tables:
                        %% http://ctan.org/pkg/booktabs
\usepackage{subcaption} %% For complex figures with subfigures/subcaptions
                        %% http://ctan.org/pkg/subcaption


%% MY STUFF
\usepackage{preamble}
% \addbibresource{my_biblio.bib} % do not put this in preamble or it breakes automatic suggestions for citations in vscode

% % macros
\usepackage{my_macro}
% \usepackage{laetitia_macro}

\begin{document}

%% Title information
\title[Short Title]{Full Title}         %% [Short Title] is optional;
                                        %% when present, will be used in
                                        %% header instead of Full Title.
\titlenote{with title note}             %% \titlenote is optional;
                                        %% can be repeated if necessary;
                                        %% contents suppressed with 'anonymous'
\subtitle{Subtitle}                     %% \subtitle is optional
\subtitlenote{with subtitle note}       %% \subtitlenote is optional;
                                        %% can be repeated if necessary;
                                        %% contents suppressed with 'anonymous'


%% Author information
%% Contents and number of authors suppressed with 'anonymous'.
%% Each author should be introduced by \author, followed by
%% \authornote (optional), \orcid (optional), \affiliation, and
%% \email.
%% An author may have multiple affiliations and/or emails; repeat the
%% appropriate command.
%% Many elements are not rendered, but should be provided for metadata
%% extraction tools.

%% Author with single affiliation.
\author{First1 Last1}
\authornote{with author1 note}          %% \authornote is optional;
                                        %% can be repeated if necessary
\orcid{nnnn-nnnn-nnnn-nnnn}             %% \orcid is optional
\affiliation{
  \position{Position1}
  \department{Department1}              %% \department is recommended
  \institution{Institution1}            %% \institution is required
  \streetaddress{Street1 Address1}
  \city{City1}
  \state{State1}
  \postcode{Post-Code1}
  \country{Country1}                    %% \country is recommended
}
\email{first1.last1@inst1.edu}          %% \email is recommended

%% Author with two affiliations and emails.
\author{First2 Last2}
\authornote{with author2 note}          %% \authornote is optional;
                                        %% can be repeated if necessary
\orcid{nnnn-nnnn-nnnn-nnnn}             %% \orcid is optional
\affiliation{
  \position{Position2a}
  \department{Department2a}             %% \department is recommended
  \institution{Institution2a}           %% \institution is required
  \streetaddress{Street2a Address2a}
  \city{City2a}
  \state{State2a}
  \postcode{Post-Code2a}
  \country{Country2a}                   %% \country is recommended
}
\email{first2.last2@inst2a.com}         %% \email is recommended
\affiliation{
  \position{Position2b}
  \department{Department2b}             %% \department is recommended
  \institution{Institution2b}           %% \institution is required
  \streetaddress{Street3b Address2b}
  \city{City2b}
  \state{State2b}
  \postcode{Post-Code2b}
  \country{Country2b}                   %% \country is recommended
}
\email{first2.last2@inst2b.org}         %% \email is recommended


%% Abstract
%% Note: \begin{abstract}...\end{abstract} environment must come
%% before \maketitle command
\begin{abstract}
Text of abstract \ldots.
\end{abstract}


%% 2012 ACM Computing Classification System (CSS) concepts
%% Generate at 'http://dl.acm.org/ccs/ccs.cfm'.
\begin{CCSXML}
<ccs2012>
<concept>
<concept_id>10011007.10011006.10011008</concept_id>
<concept_desc>Software and its engineering~General programming languages</concept_desc>
<concept_significance>500</concept_significance>
</concept>
<concept>
<concept_id>10003456.10003457.10003521.10003525</concept_id>
<concept_desc>Social and professional topics~History of programming languages</concept_desc>
<concept_significance>300</concept_significance>
</concept>
</ccs2012>
\end{CCSXML}

\ccsdesc[500]{Software and its engineering~General programming languages}
\ccsdesc[300]{Social and professional topics~History of programming languages}
%% End of generated code


%% Keywords
%% comma separated list
\keywords{keyword1, keyword2, keyword3}  %% \keywords are mandatory in final camera-ready submission


%% \maketitle
%% Note: \maketitle command must come after title commands, author
%% commands, abstract environment, Computing Classification System
%% environment and commands, and keywords command.
\maketitle



\section{Introduction}
\begin{itemize}
  \item Interleaving based semantics VS partial order/graph based semantics
  \item Synchronous and asynchronous communication
  \item The problem of synchronizability
\end{itemize}

\section{Preliminaries/Basics}
\begin{itemize}
  \item Communicating systems (communicating finite-state automata with bag channels)
  \item MSCs and conflict graph
  \item Monadic Second-Order logic on MSCs
  \item (Language of a system as a set of MSCs)
  \item (Model checking and synchronizability)
\end{itemize}

\section{Asynchronous communication models overview}
\begin{itemize}
  \item Overview of asynchronous variants
  \item High-level description of each variant along with references to implementations (if existing)
  \item (Definitions based on linearization, intuitive)
  \item (Language of a system with a given communication model as a set of MSCs)
  \item Hint of hierarchy result
\end{itemize}

\section{Asynchronous communication models operational semantics}
\begin{itemize}
  \item TODO...
\end{itemize}

\section{Asynchronous communication models as classes of MSCs, MSO-definability}
\begin{itemize}
  \item Definition of MSC class for each communication model (alternative definitions)
  \item MSO-definability of each class
\end{itemize}

\section{Equivalence of the two definitions}
\begin{itemize}
  \item TODO...
\end{itemize}

\section{Hierarchy of asynchronous classes of MSCs}
\ldots

\section{An application: special treewidth and decidability of the synchronizability problem}
\begin{itemize}
  \item The synchronizability problem
  \item Special treewidth and how the results regarding the hierarchy are useful for detecting STW-boundness of certain classes
  \item MSO-decidability and STW-boundess tables
\end{itemize}

\section{Conclusion}

%% Acknowledgments
\begin{acks}                            %% acks environment is optional
                                        %% contents suppressed with 'anonymous'
  %% Commands \grantsponsor{<sponsorID>}{<name>}{<url>} and
  %% \grantnum[<url>]{<sponsorID>}{<number>} should be used to
  %% acknowledge financial support and will be used by metadata
  %% extraction tools.
  This material is based upon work supported by the
  \grantsponsor{GS100000001}{National Science
    Foundation}{http://dx.doi.org/10.13039/100000001} under Grant
  No.~\grantnum{GS100000001}{nnnnnnn} and Grant
  No.~\grantnum{GS100000001}{mmmmmmm}.  Any opinions, findings, and
  conclusions or recommendations expressed in this material are those
  of the author and do not necessarily reflect the views of the
  National Science Foundation.
\end{acks}




%% Appendix
\appendix
\section{Appendix}

Text of appendix \ldots

\section{(3) Asynchronous communication models overview}

In synchronous communication, send and receive elements are essentially viewed as a single entity, i.e. a receive event is always executed immediately after the respective send event. The whole idea behind (fully) asynchronous communication is to decouple send and receive events, so that a receive event can happen indefinitely after the corresponding send event. However, by introducing some additional contraints on asynchronous communication, we can obtain new communication models that sit somewhere between synchronous and fully asynchronous communication. In this section we will present seven different asynchronous communication models, which were already described in \cite{DBLP:journals/fac/ChevrouHQ16}.
\begin{itemize}
	\item \emph{Fully asynchronous}: a fully asychronous architecture (or simply asynchronous from now on) can be modeled as a collection of channels between each pair $(M_1,M_2)$ of machines, such that $M_1$ can send messages to $M_2$. Following this description, given two machines $M_1$ and $M_2$ that can exchange messages in both directions, we have two channels: one for the messages sent by $M_1$ to $M_2$, and the other for the messages sent by $M_2$ to $M_1$. The channels behave as \emph{bags}, which means that they do not guarantee any specific order on the delivery of messages. This architecture is equivalent to the "Fully asynchronous" communication model in \cite{DBLP:journals/fac/ChevrouHQ16}.
	\davideanswer{Is it true? For the set of MSCs yes, it should be.}
	\item \emph{FIFO $\oneone$ (\pp)}: this is a variant of the asynchronous architecture in which channels operate in FIFO mode (i.e. as queues). This means that messages between a couple of peers are delivered in their send order, whereas messages from/to different peers are delivered independently. We will use the term \emph{peer-to-peer} (\pp) as a synonym of FIFO $\oneone$. This architecture is equivalent to the "FIFO $\oneone$" communication model in \cite{DBLP:journals/fac/ChevrouHQ16}.
	\item \emph{Causally orderered}: this can be described as a more specific variant of the \pp architecture. In a causally orderered architecture, the communicating system ensures that messages are delivered according to the causality of their emissions. In other words, if a message $m_1$ is causally sent before a message $m_2$ (i.e. there exists a causal path from the first emission to the second one), then a peer cannot receive $m_2$ before $m_1$. Channels still operate in FIFO mode, but the delivery of some messages might be delayed to enforce causal ordering. This architecture is equivalent to the "Causally ordered" communication model in \cite{DBLP:journals/fac/ChevrouHQ16}. In literature, several implementations of causal ordering have been proposed. For instance, the algorithm described in \cite{DBLP:conf/wdag/SchiperES89} makes use of the logical vector clocks introduced by Mattern-Fidge \cite{Fidge88timestampsin, Mattern89virtualtime} to enforce causal ordering.
	\item \emph{FIFO $\none$ (mailbox)}: in a mailbox architecture, each machine merges all of its incoming messages (from any source) into a unique queue. In other words, we can model it as each machine $P_i$ having a single incoming FIFO channel, which is shared by the other machines that can send messages to $P_i$. A send event consists in adding the message at the end of the queue of the destination peer. This architecture is equivalent to the "FIFO $\none$" communication model in \cite{DBLP:journals/fac/ChevrouHQ16}.
	\item \emph{FIFO $\onen$ (mailbox)}: in a $\onen$ architecture, the messages sent from a single machine are always delivered in their send order, independently of the recipients. Its implementation is
	not expensive: each machine has a unique queue to store sent messages. The recipients fetch messages from this queue and acknowledge their reception. After the acknowledgement, the next message in the queue can be fetched. This architecture is equivalent to the "FIFO $\onen$" communication model in \cite{DBLP:journals/fac/ChevrouHQ16}.
	\item \emph{FIFO $\nn$}: this architecture can be modeled as a unique shared FIFO channel. Messages are globally ordered and delivered according to the their emission order. As noted in \cite{DBLP:journals/fac/ChevrouHQ16}, this model is generally unrealistic and its implementations are inefficient.

\end{itemize}
For convenience, we will refer to a system that uses the \pp architecture simply as a \pp system. The same shorthand will be used for the other communication architectures. Note that, for each of these communicating architectures, there may be other equivalent ways of describing/modeling them.

%% Appendix
\appendix
\section{Appendix}

Text of appendix \ldots

% Bibliography
\bibliography{bibfile}

\end{document}
