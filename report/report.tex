\documentclass{article}

% page size and margins
\usepackage[a4paper,top=2cm,bottom=2cm,left=3cm,right=3cm,marginparwidth=1.75cm]{geometry}

% packages
\usepackage{preamble}
\addbibresource{my_biblio.bib} % do not put this in preamble or it breakes automatic suggestions for citations in vscode

% theorems
\newtheorem{theorem}{Theorem}[section]
\newtheorem{corollary}{Corollary}[theorem]
\newtheorem{lemma}[theorem]{Lemma}
\newtheorem{fact}[theorem]{Fact}
\newtheorem{conjecture}[theorem]{Conjecture}
\newtheorem*{theorem*}{Theorem}
\newtheorem*{corollary*}{Corollary}
\newtheorem*{lemma*}{Lemma}
\newtheorem*{fact*}{Fact}
\newtheorem*{conjecture*}{Conjecture}

\theoremstyle{remark}
\newtheorem{remark}{Remark}[section]
\newtheorem*{remark*}{Remark}

\theoremstyle{definition}
\newtheorem{definition}{Definition}[section]
\newtheorem*{definition*}{Definition}

\theoremstyle{definition}
\newtheorem{example}{Example}[section]
\newtheorem*{example*}{Example}

\theoremstyle{definition}
\newtheorem{proposition}{Proposition}[section]
\newtheorem*{proposition*}{Proposition}

% macros
\usepackage{my_macro}

\title{Stage M2}
\author{Davide Ferre'}

\begin{document}

\maketitle

\section{Useful stuff}
\subsection{Message Sequence Charts}

Assume a finite set of processes $\Procs$ and a finite set of messages $\Msg$.
The set of (\pp) channels is $\Ch = \{(p,q) \in \Procs \times \Procs \mid p \neq q\}$.
%
A send action is of the form $\sact{p}{q}{\msg}$
where $(p,q) \in \Ch$ and $\msg \in \Msg$.
It is executed by $p$ and sends message $\msg$ to $q$.
The corresponding receive action, executed by $q$, is
$\ract{p}{q}{\msg}$.
%
For $(p,q) \in \Ch$, let
$\pqsAct{p}{q} = \{\sact{p}{q}{\msg} \mid \msg \in \Msg\}$ and
$\pqrAct{p}{q} = \{\ract{p}{q}{\msg} \mid \msg \in \Msg\}$.
For $p \in \Procs$, we set
$\psAct{p} = \{\sact{p}{q}{\msg} \mid q \in \Procs \setminus \{p\}$ and $\msg \in \Msg\}$, etc.
Moreover, $\pAct{p} = \psAct{p} \cup \qrAct{p}$ will denote the set of all actions that are
executed by $p$.
Finally, $\Act = \bigcup_{p \in \Procs} \pAct{p}$
is the set of all the actions.

\paragraph*{Peer-to-peer MSCs.}
%\alain{add an example of MSC and of system and illustrate notions with it}
A \emph{\pp MSC} (or simply \emph{MSC}) over $\Procs$ and $\Msg$ is a tuple $\msc = (\Events,\procrel,\lhd,\lambda)$
where $\Events$ is a finite (possibly empty) set of \emph{events}
and $\lambda: \Events \to \Act$ is a labeling function.
For $p \in \Procs$, let $\Events_p = \{e \in \Events \mid \lambda(e) \in \pAct{p}\}$ be the set of events
that are executed by $p$.
We require that $\procrel$ (the \emph{process relation}) is the disjoint union $\bigcup_{p \in \Procs} \procrel_p$
of relations ${\procrel_p} \subseteq \Events_p \times \Events_p$ such that
$\procrel_p$ is the direct successor relation of a total order on $\Events_p$.
For an event $e \in \Events$, a set of actions $A \subseteq \Act$, and a relation $\rel \subseteq \Events \times \Events$,
let $\sametype{e}{A}{\rel} = |\{f \in \Events \mid (f,e) \in \rel$ and $\lambda(f) \in A\}|$.
We require that ${\lhd} \subseteq \Events \times \Events$ (the \emph{message relation}) satisfies the following:
\begin{itemize}\itemsep=0.5ex
\item[(1)] for every pair $(e,f) \in {\lhd}$, there is a send action $\sact{p}{q}{\msg} \in \Act$ such that
$\lambda(e) = \sact{p}{q}{\msg}$, $\lambda(f) = \ract{p}{q}{\msg}$, and
$\sametype{e}{\pqsAct{p}{q}}{\procrel^+} = \sametype{f}{\pqrAct{p}{q}}{\procrel^+}$,
\item[(2)] for all $f \in \Events$ such that $\lambda(f)$ is a receive action, there is $e \in \Events$ such that $e \lhd f$.
\end{itemize}
Finally, letting ${\le}_\msc = ({\procrel} \cup {\lhd})^\ast$,
we require that $\le_\msc$ is a partial order. For convenience, we will simply write $\le$ when $M$ is clear from the context. 
%We may also simply write ${\le}$ instead of $\le_{\msc}$ if
%the MSC is clear from the context.

\medskip

Condition (1) above ensures that every (p2p) channel $(p,q)$ behaves in a FIFO manner.
By Condition (2), every receive event has a matching send event.
Note that, however, there may be unmatched send events in an MSC.
We let
$\SendEv{\msc} = \{e \in \Events \mid \lambda(e)$ is a send
action$\}$,
$\RecEv{\msc} = \{e \in \Events \mid \lambda(e)$ is a receive
action$\}$,
$\Matched{\msc} = \{e \in \Events \mid$ there is $f \in \Events$
such that $e \lhd f\}$, and
$\Unm{\msc} = \{e \in \Events \mid \lambda(e)$ is a send
action and there is no $f \in \Events$ such that $e \lhd f\}$.
%
We do not distinguish isomorphic MSCs and
let $\ppMSCs$ be the set of all MSCs over the given sets $\Procs$ and $\Msg$.
% For readability, and if there is no ambiguity,
% $\lambda^{-1}(\send{p}{q}{\msg})$, resp. $\lambda^{-1}(\rec{p}{q}{\msg})$,
% will be written $\ssymb(p,q,\msg)$ and, resp., $\rsymb(p,q,\msg)$ in the examples.

\bigskip

\noindent \begin{minipage}[c]{10.5cm}
  \begin{example}\label{ex:msc}
    For a set of processes $\procSet = \{p,q,r\}$ and a set of messages $\paylodSet = \{\msg_1, \msg_2, \msg_3, \msg_4 \}$,
    $\mscweakuniver = (\Events, \rightarrow, \lhd, \lambda)$
    %in Fig.~\ref{fig:msc_weak_univer}
    is an MSC where, for example,
  $e_2 \lhd e_2'$ and $e_3' \rightarrow e_4$.
  The dashed arrow means that the send event $e_1$ does not have
  a matching receive, so $e_1 \in Unm(\mscweakuniver)$.
  Moreover, $e_2 \le_{\mscweakuniver} e_4$, but
  $e_1 \not\le_{\mscweakuniver} e_4$.
We can find a total order ${\pplin} \supseteq {\le}_{\mscweakuniver}$
  such that $e_1 \pplin e_2 \pplin e_2' \pplin e_3
  \pplin e_3' \pplin e_4 \pplin e_4'$. We call $\pplin$ a linearization,
  which is formally defined  below.
  \end{example}
\end{minipage}
\hfill
\begin{minipage}[c]{3cm}

%\begin{figure}
  \begin{center}
    \begin{tikzpicture}[>=stealth,node distance=3.2cm,shorten >=1pt,
      every state/.style={text=black, scale =0.8}, semithick,
      font={\fontsize{8pt}{12}\selectfont}]
	\begin{scope}[xshift = 9cm, scale = 0.8]
		\node at (-0.3, -0.65)  (e)    {$e_1$};
		\node at (1.3, -1.2)  (e)    {$e_2$};
		\node at (-0.3, -1.2)  (e)    {$e_2'$};
		\node at (0.7, -1.85)  (e)    {$e_3$};
		\node at (2.3, -1.85)  (e)    {$e_3'$};
		\node at (2.3, -2.4)  (e)    {$e_4$};
		\node at (0.7, -2.4)  (e)    {$e_4'$};
		%MACHINES
		\draw (0,0) node{$p$} ;
		\draw (1,0) node{$q$} ;
		\draw (2,0) node{$r$} ;
		\draw (0,-0.2) -- (0,-2.8) ;
		\draw (1,-0.2) -- (1,-2.8);
		\draw (2, -0.2) -- (2, -2.8) ;
		%MESSAGES
		\draw[>=latex,->, dashed] (0,-0.65) -- (1, -0.65) node[midway,above]{$\amessage_1$};

		\draw[>=latex,->] (1, -1.2) -- (0, -1.2) node[midway, above] {$\amessage_2$};

		\draw[>=latex,->] (1,-1.85) -- (2,-1.85) node[midway, above] {$\amessage_3$};

		\draw[>=latex,->] (2,-2.4) -- (1,-2.4) node[midway,above] {$\amessage_4$};
	\end{scope}

\end{tikzpicture}
\captionof{figure}{MSC $\mscweakuniver$}
\label{fig:msc_weak_univer}
\end{center}
%\alain{NIce}
%\end{figure}

\end{minipage}

\paragraph*{Mailbox MSCs.}

For an MSC $\msc = (\Events,\procrel,\lhd,\lambda)$, we define
an additional binary relation that represents a constraint
under the mailbox semantics, where each process has only one incoming channel.
Let ${\mbrel}_\msc \subseteq \Events \times \Events$
be defined by: $e_1 \mbrel_\msc e_2$ if there is $q \in \Procs$
such that $\lambda(e_1) \in \qsAct{q}$,
$\lambda(e_2) \in \qsAct{q}$, and one of the following holds:
\begin{itemize}\itemsep=0.5ex
\item $e_1 \in \Matched{\msc}$ and $e_2 \in \Unm{\msc}$, or
\item $e_1 \lhd f_1$ and $e_2 \lhd f_2$ for some $f_1,f_2 \in \Events_q$ such that $f_1 \procrel^+ f_2$.
\end{itemize}

We let ${\preceq_\msc} = ({\procrel} \,\cup\, {\lhd} \,\cup\, {\mbrel_\msc})^\ast$.
Note that ${\le_\msc} \subseteq {\preceq_\msc}$.
%
%\begin{definition}\label{def:mailbox-msc}
We call $\msc \in \ppMSCs$ a \emph{mailbox MSC}
if ${\preceq_\msc}$ is a partial order.
%\end{definition}
Intuitively, this means that events can be scheduled in a way that corresponds
to the mailbox semantics, i.e., with one incoming channel per process.
Following the terminology in \cite{DBLP:conf/cav/BouajjaniEJQ18}, we also say that
a mailbox MSC satisfies \emph{causal delivery}.
The set of mailbox MSCs $\msc \in \ppMSCs$ is denoted by $\cdMSCs$.

\begin{example}\label{ex:mailbox-msc}
    MSC $\mscweakuniver$ is a mailbox MSC. Indeed, even though the order $\linrel$ defined in Example~\ref{ex:msc} does not respect all mailbox constraints, particularly the fact that $e_4 \mbrel_{\mscweakuniver} e_1$, there is a total order $ {\mblin} \supseteq {\preceq_{\mscweakuniver}}$ such that $
    e_2 \mblin e_3 \mblin e_3' \mblin e_4 \mblin e_1 \mblin e_2' \mblin e_4'$. We call $\mblin$ a mailbox linearization, which is formally defined below.
\end{example}

\paragraph*{Linearizations, Prefixes, and Concatenation.}

Consider $\msc = (\Events,\procrel,\lhd,\lambda) \in \MSCs$.
A \emph{\pp linearization} (or simply \emph{linearization}) of $\msc$ is a (reflexive) total order
${\linrel} \subseteq \Events \times \Events$ such that ${\le_\msc} \subseteq
{\linrel}$. Similarly,
a \emph{mailbox linearization} of $\msc$ is a total order
${\linrel} \subseteq \Events \times \Events$ such that ${\preceq_\msc} \subseteq
{\linrel}$. That is, every mailbox linearization is a \pp linearization,
but the converse is not necessarily true (Example~\ref{ex:mailbox-msc}).
Note that an MSC is a mailbox MSC iff it has at least one mailbox linearization.

\medskip

Let $\msc = (\Events,\procrel,\lhd,\lambda) \in \MSCs$ and consider
$E \subseteq \Events$ such that $E$ is ${\le_\msc}$-\emph{downward-closed}, i.e,
for all $(e,f) \in {\le_\msc}$ such that $f \in E$, we also have $e \in E$.
Then, the MSC $(E,{\procrel} \cap (E \times E),{\lhd} \cap (E \times E),\lambda')$,
where $\lambda'$ is the restriction of $\Events$ to $E$, is called a \emph{prefix}
of $\msc$. In particular, the empty MSC is a prefix of $\msc$.
We denote the set of prefixes of $\msc$ by $\Pref{\msc}$.
This is extended to sets $L \subseteq \MSCs$ as expected, letting
$\Pref{L} = \bigcup_{\msc \in L} \Pref{\msc}$.

\begin{lemma}
\label{lem:mb-prefix}
Every prefix of a mailbox MSC is a mailbox MSC.
\end{lemma}
\begin{proof}
Let $\msc = (\Events, \procrel, \lhd, \lambda) \in \mbMSCs$ and $\msc_0 =
(\Events_0, \procrel_0, \lhd_0, \lambda_0)$ be a prefix of $\msc$, i.e.,
$\Events_0 \subseteq \Events$. By contradiction, suppose that $\msc_0$ is not a
mailbox MSC. Then, there are distinct $e,f \in \Events_0$ such that $e \preceq_{M_0} f \preceq_{M_0}
e$ with ${\preceq_{\msc_0}} = ({\rightarrow_0} \cup {\lhd_0} \cup {\mbrel_{\msc_0}})^*$.
As $\Events_0 \subseteq \Events$, we have that ${\rightarrow_0} \subseteq {\rightarrow}$, ${\lhd_0} \subseteq {\lhd}$, and ${\mbrel_{\msc_0}} \subseteq {\mbrel_{\msc}}$. Finally, ${\preceq_{\msc_0}} \subseteq {\preceq_{\msc}}$ and $\msc$ is not a mailbox MSC, which is a contradiction.
\end{proof}

Let $\msc_1 = (\Events_1,\procrel_1,\lhd_1,\lambda_1)$ and
$\msc_2 = (\Events_2,\procrel_2,\lhd_2,\lambda_2)$ be two MSCs.
Their \emph{concatenation} $\msc_1 \cdot \msc_2 = (\Events,\procrel,\lhd,\lambda)$ is defined if, for all $(p,q) \in \Ch$,
$e_1 \in \Unm{\msc_1}$, and
$e_2 \in \Events_2$ such that $\lambda(e_1) \in \pqsAct{p}{q}$
and $\lambda(e_2) \in \pqsAct{p}{q}$,
we have $e_2 \in \Unm{\msc_2}$.
As expected, $\Events$ is the disjoint union of $\Events_1$ and $\Events_2$,
${\lhd}  = {\lhd_1} \cup {\lhd_2}$, $\lambda$ is the ``union'' of $\lambda_1$
and $\lambda_2$, and ${\procrel} = {\procrel_1} \cup {\procrel_2} \cup R$.
Here, $R$ contains, for all $p \in \Procs$ such that $(\Events_1)_p$ and
$(\Events_2)_p$ are non-empty, the pair $(e_1,e_2)$ where $e_1$ is the
maximal $p$-event in $M_1$ and $e_2$ is the minimal $p$-event in $M_2$.
Note that $\msc_1 \cdot \msc_2$ is indeed an MSC and that
concatenation is associative.

\subsection{Communicating Systems}

We now recall the definition of communicating systems (aka communicating finite-state
machines or message-passing automata), which consist of finite-state machines $A_p$
(one for every process $p \in \Procs$) that can communicate through the FIFO channels
from $\Ch$.

\begin{definition}\label{def:cs}
A \emph{communicating system} over $\Procs$ and $\Msg$ is a tuple
   $ \Sys = (A_p)_{p\in\procSet}$. For each
   $p \in \Procs$, $A_p = (Loc_p, \delta_p, \ell^0_p)$ is a finite transition system where
   $\Loc_p$ is a finite set of local (control) states, $\delta_p
   \subseteq \Loc_p \times \pAct{p} \times \Loc_p$ is the
   transition relation, and $\ell^0_p \in Loc_p$ is the initial state.
\end{definition}

Given $p \in \Procs$ and a transition $t = (\ell,a,\ell') \in \delta_p$, we let
$\tsource(t) = \ell$, $\ttarget(t) = \ell'$, $\tlabel(t) = a$, and
$\tmessage(t) = \msg$ if $a \in \msAct{\msg} \cup \mrAct{\msg}$.

\smallskip

There are in general two ways to define the semantics of a communicating system.
Most often it is defined as a global infinite transition system that keeps track
of the various local control states and all (unbounded) channel contents.
As, in this paper, our arguments are based on a graph view of MSCs, we will define
the language of $\Sys$ directly as a set of MSCs. These two semantic views are essentially
equivalent, but they have different advantages depending on the context.
We refer to \cite{CyriacG14} for a thorough discussion.

Let $\msc = (\Events,\procrel,\lhd,\lambda)$ be an MSC.
A \emph{run} of $\Sys$ on $\msc$ is a mapping
$\rho: \Events \to \bigcup_{p \in \Procs} \delta_p$
that assigns to every event $e$ the transition $\rho(e)$
that is executed at $e$. Thus, we require that
\begin{enumerate*}[label={(\roman*)}]
\item for all $e \in \Events$, we have $\tlabel(\rho(e)) = \lambda(e)$,
\item for all $(e,f) \in {\procrel}$, $\ttarget(\rho(e)) = \tsource(\rho(f))$,
\item for all $(e,f) \in {\lhd}$, $\tmessage(\rho(e)) = \tmessage(\rho(f))$,
and
\item for all $p \in \Procs$ and $e \in \Events_p$ such that there is no $f \in \Events$ with $f \procrel e$, we have $\tsource(\rho(e)) = \ell_p^0$.
\end{enumerate*}

Letting run $\Sys$ directly on MSCs is actually very convenient.
This allows us to associate with $\Sys$ its p2p language and mailbox language
in one go. The \emph{\pp language} of $\Sys$ is $\ppL{\Sys} = \{\msc \in \ppMSCs \mid$ there is a run of $\Sys$ on $\msc\}$.
The \emph{mailbox language} of $\Sys$ is $\mbL{\Sys} = \{\msc \in \mbMSCs \mid$ there is a run of $\Sys$ on $\msc\}$.

Note that, following \cite{DBLP:conf/cav/BouajjaniEJQ18,DBLP:conf/fossacs/GiustoLL20},
we do not consider final states or final configurations, as our purpose is to
reason about all possible
traces that can be \emph{generated} by $\Sys$.
%We will discuss this issue in more detail later in the paper. \todo{Do we still discuss this in the conference version or can we omit this sentence?}
The next lemma is obvious for the p2p semantics and follows from Lemma~\ref{lem:mb-prefix} for
	the mailbox semantics.

\begin{lemma}\label{lem:prefix-closed}
For all $\comsymb \in \{\ppsymb, \mbsymb\}$, $\cL{\Sys}$ is prefix-closed:
$\Pref{\cL{\Sys}} \subseteq \cL{\Sys}$.
\end{lemma}

\begin{example}
	Fig.~\ref{fig:system_weak_univer} depicts $\systemweakuniver = (A_p, A_q, A_r)$ such that MSC $\mscweakuniver$ in Fig.~\ref{fig:msc_weak_univer} belongs to $\ppL{\systemweakuniver}$ and to $\mbL{\systemweakuniver}$.
	There is a unique run $\rho$ of $\systemweakuniver$ on $\mscweakuniver$.
	We can see that $(e_3',e_4) \in {\rightarrow}$ and $\ttarget(\rho(e_3')) = \tsource(\rho(e_4)) = \ell_r^{1}$, $(e_2, e_2') \in \lhd_{\mscweakuniver}$, and $\tmessage(\rho(e_2)) = \tmessage(\rho(e_2')) = \msg_2$.
	\end{example}
	\begin{figure}[t]
	\begin{center}
	  \begin{tikzpicture}[>=stealth,node distance=3.2cm,shorten >=1pt,
		every state/.style={text=black, scale =0.75}, semithick,
		font={\fontsize{8pt}{12}\selectfont},
		scale = 0.9
		]
	  \begin{scope}[->]
		  \node[state,initial,initial text={}] (q0)  {$\ell_p^{0}$};
		  \node[state, right of=q0] (q1)  {$\ell_p^{1}$};
				\node[state, right of=q1] (q2) {$\ell_p^{2}$};
	
			\path (q0) edge node [above] {$\send{p}{q}{\msg_1}$} (q1);
				\path (q1) edge node [above] {$\rec{q}{p}{\msg_2}$}(q2);
			\node[thick] at (-1.1,0) {$A_p$};
	  \end{scope}
	
	  \begin{scope}[->, shift={(7.5,0)}]
		  \node[state,initial,initial text={}] (q0)  {$\ell_q^{0}$};
				\node[state, right of=q0] (q1)  {$\ell_q^{1}$};
				\node[state, below of=q1, node distance = 1.5cm] (q2) {$\ell_q^{2}$};
				\node[state, left of=q2] (q3)  {$\ell_q^{3}$};
	
				\path (q0) edge node [above] {$\send{q}{p}{\msg_2}$} (q1);
				\path (q1) edge node [right] {$\send{q}{r}{\msg_3}$}(q2);
				\path (q2) edge node [above] {$\rec{r}{q}{\msg_4}$} (q3);
				\node[thick] at (-1.1,0) {$A_q$};
	  \end{scope}
	
		\begin{scope}[->, shift={(0,-1.25)} ]
		  \node[state,initial,initial text={}] (q0)  {$\ell_r^{0}$};
				\node[state, right of=q0] (q1)  {$\ell_r^{1}$};
				\node[state, right of=q1] (q2) {$\ell_r^{2}$};
	
				\path (q0) edge node [above] {$\rec{q}{r}{\msg_3}$} (q1);
				\path (q1) edge node [above] {$\send{r}{q}{\msg_4}$}(q2);
				\node[thick] at (-1.1,0) {$A_r$};
	  \end{scope}
	\end{tikzpicture}
	\captionof{figure}{System $\systemweakuniver$}
	\label{fig:system_weak_univer}
	\end{center}
	\end{figure}	

\subsection{Conflict Graph}

We now recall the notion of a conflict graph associated to an MSC defined in \cite{DBLP:conf/cav/BouajjaniEJQ18}. This graph is used to depict the causal dependencies between message exchanges.  Intuitively, we have a dependency whenever
two messages have a process in common. For instance, an $\xrightarrow{SS}$
dependency between message exchanges $v$ and $v'$ expresses the fact that
$v'$ has been sent after $v$, by the same process. This notion is of interest because it was seen in \cite{DBLP:conf/cav/BouajjaniEJQ18} that the notion of synchronizability in MSCs (which is studied in this paper) can be graphically characterized by the nature of the associated conflict graph.
It is defined in terms of linearizations
in \cite{DBLP:conf/fossacs/GiustoLL20}, but we equivalently express it
directly in terms of MSCs.

\newcommand{\type}{\tau}
\newcommand{\stype}{S}
\newcommand{\rtype}{R}
\newcommand{\mexch}{\mu}
\newcommand{\Edges}{\mathit{Edges}}
\newcommand{\Nodes}{\mathit{Nodes}}

For an MSC $\msc = (\Events, \rightarrow, \lhd, \lambda)$ and
$e \in \Events$, we define the type $\type(e) \in \{\stype,\rtype\}$ of $e$ by $\type(e) = \stype$ if $e \in \SendEv{\msc}$
and $\type(e) = \rtype$ if $e \in \RecEv{\msc}$.
Moreover, for $e \in \Unm{\msc}$, we let $\mexch(e) = e$,
and for $(e,e') \in \lhd$, we let $\mexch(e) = \mexch(e') = (e,e')$.


\begin{definition}[Conflict graph]
	The \emph{conflict graph} $\cgraph{\msc}$ of an MSC $\msc = (\Events, \rightarrow, \lhd, \lambda)$ is the labeled graph $(\Nodes, \Edges)$, with $\Edges \subseteq \Nodes \times \{\stype,\rtype\}^2 \times \Nodes$, defined by
	$\Nodes = {\lhd} \cup \Unm{\msc}$ and $\Edges = \{(\mu(e),\type(e)\type(f),\mu(f)) \mid (e,f) \in {\to^+}\}$.
In particular, a node of $\cgraph{\msc}$ is either a single unmatched send event or a message pair $(e,e') \in {\lhd}$.
\end{definition}	

\subsection{Logic and Special Tree-Width}

\paragraph*{Monadic Second-Order Logic.}
The set of MSO formulas over MSCs (over $\Procs$ and $\Msg$) is given by the grammar
$
\phi ::= x \procrel y \mid x \lhd y \mid \lambda(x) = a \mid x = y \mid x \in X \mid \exists x.\phi \mid \exists X.\phi \mid \phi \vee \phi \mid \neg \phi
$,
where $a \in \Act$, $x$ and $y$ are first-order variables, interpreted as
events of an MSC, and $X$ is a second-order variable, interpreted
as a set of events. We assume that we have an infinite supply of variables,
and we use common abbreviations such as $\wedge$, $\forall$, etc.
The satisfaction relation is defined in the standard way and self-explanatory.
For example, the formula $\neg\exists x.(\bigvee_{a \in \sAct} \lambda(x) = a \;\wedge\; \neg \mathit{matched}(x))$
with $\mathit{matched}(x) = \exists y.x \lhd y$
says that there are no unmatched send events.
It is not satisfied by  MSC $\mscweakuniver$
of Fig.~\ref{fig:msc_weak_univer},
as message $\msg_1$ is not received,
but by $\mscstrongexist$ from Fig.~\ref{fig:msc_strong_exist}.

Given a sentence $\phi$, i.e., a formula without free variables,
we let $L(\phi)$ denote the set of (p2p) MSCs that satisfy $\phi$.
%
It is worth mentioning that the (reflexive) transitive closure of
a binary relation defined by an MSO formula with free variables $x$ and $y$,
such as $x \procrel y$, is MSO-definable so that the logic can freely
use formulas of the form $x \procrel^+ y$ or $x \le y$ (where $\le$
is interpreted as $\le_\msc$ for the given MSC $\msc$).
Therefore, the definition of a mailbox MSC can be readily translated into
the formula $\mbformula = \neg \exists x.\exists y.(\neg (x = y) \wedge x \preceq y \wedge y \preceq x)$ so that we have $L(\mbformula) = \mbMSCs$.
Here, $x \preceq y$ is obtained as the MSO-definable reflexive transitive closure of
the union of the MSO-definable relations $\procrel$, $\lhd$, and $\sqsubset$.
In particular, we may define $x \sqsubset y$ by :
\[
x \sqsubset y =
\displaystyle
\hspace{-1em}\bigvee_{\substack{q \in \Procs\\a,b \in \qsAct{q}}}\hspace{-1em}
\lambda(x) = a \;\wedge\; \lambda(y) = b
\wedge
\left(
\begin{array}{rl}
& \mathit{matched}(x) \wedge \neg \mathit{matched}(y)\\[1ex]
\vee & \exists x'.\exists y'. (x \lhd x' \;\wedge\; y \lhd y' \;\wedge\; x' \procrel^+ y')
\end{array}
\right)
\]

\paragraph*{Special Tree-Width.}

\emph{Special tree-width} \cite{Courcelle10},
is a graph measure that indicates how close
a graph is to a tree (we may also use classical
	\emph{tree-width} instead).
This or similar measures are commonly employed in verification. For instance, tree-width and split-width have been used in \cite{MadhusudanP11} and, respectively, \cite{DBLP:conf/concur/CyriacGK12,AiswaryaGK14} to reason about graph behaviors generated by pushdown and queue systems.
%Here we apply it to reason about MSCs.
There are several ways to define the special tree-width of an MSC.
We adopt the following game-based definition from \cite{DBLP:journals/corr/abs-1904-06942}.

Adam and Eve play a two-player turn based ``decomposition game''
whose positions
are MSCs with some pebbles placed on some events.
More precisely, Eve's positions are
\emph{marked MSC fragments} $(M, U)$, where
$\msc = (\Events, \procrel, \lhd, \lambda)$
is an \emph{MSC fragment} (an MSC with possibly some edges from
$\lhd$ or $\to$ removed)
 and $U \subseteq \Events$ is the subset of marked events.
Adam's positions are pairs of marked MSC fragments.
A move by Eve consists in the following steps:
\begin{enumerate}
	\item marking some events of the MSC resulting in $(M, U')$ with $U \subseteq U' \subseteq \Events$,
	\item removing (process and/or message) edges whose endpoints are marked,
	\item dividing $(M, U)$ in $(M_1, U_1)$ and $(M_2, U_2)$ such that $M$ is the disjoint (unconnected) union of $M_1$ and $M_2$
	and marked nodes are inherited.
\end{enumerate}
When it is Adam's turn, he simply chooses one of the two marked MSC fragments.
The initial position is $(\msc,\emptyset)$ where $M$ is the (complete) MSC at hand. A terminal position is any position belonging to Eve such that all events are marked.
%
For $k \in \N$, we say that the game is $k$-winning for Eve if she has a (positional) strategy that allows her,
starting in the initial position and independently of Adam's moves, to reach a terminal position such that, in every single position visited along the play, there are at most $k+1$ marked events.

\newcommand{\CS}[2]{\mathsf{CS}_{(#1,#2)}}
\newcommand{\MSO}[2]{\mathsf{MSO}_{(#1,#2)}}
\newcommand{\LCPDL}[2]{\mathsf{LCPDL}_{(#1,#2)}}
\newcommand{\MSCpm}[2]{\mathsf{MSC}_{(#1,#2)}}
\newcommand{\mbMSCpm}[2]{\mathsf{MSC}_{(#1,#2)}^{\mathsf{mb}}}


\begin{fact}[\cite{DBLP:journals/corr/abs-1904-06942}]
	The special tree-width of an MSC is the least $k$ such that
	the associated game is $k$-winning for Eve.
\end{fact}

The set of MSCs whose special tree-width is at most $k$ is denoted by $\stwMSCs{k}$.

\subsection{Model Checking}

In general, even simple verification problems, such
as control-state reachability, are undecidable for
communicating systems \cite{DBLP:journals/jacm/BrandZ83}.
However, they are decidable when we restrict to behaviors of
bounded special tree-width, which motivates the following
definition of a generic {\bf bounded model-checking problem} for $\comsymb \in \{\ppsymb, \mbsymb\}$:\\
%
{\bf Input:} Two finite sets $\Procs$ and $\Msg$, a communicating system $\System$, an MSO sentence $\phi$, and $k \in \N$ (given in unary).\\
%
{\bf Question:} Do we have $\cL{\Sys} \cap \stwMSCs{k} \subseteq L(\phi)$?


\begin{fact}[\cite{DBLP:journals/corr/abs-1904-06942}]\label{p2p}
The bounded model-checking problem for $\comsymb = \ppsymb$ is decidable.
When the formulas $\phi$ are from LCPDL, then the problem is solvable
in exponential time.
\end{fact}

Note that \cite{DBLP:journals/corr/abs-1904-06942} does not employ
the LCPDL modality $\jump$, but it can be integrated easily.
Using $\mbformula$ or $\mbFormula$, we obtain the corresponding result
for mailbox systems as a corollary:
% (cf.\ Appendix~\ref{app:mailbox} for the proof):

\begin{theorem}
\label{thm:mailbox_bounded_model_checking}
The bounded model-checking problem for $\comsymb =  \mbsymb$ is decidable.
When the formulas $\phi$ are from LCPDL, then the problem is solvable
in exponential time.
\end{theorem}

\subsection{Synchronizability}

The above model-checking approach is incomplete in the sense that
a positive answer does not imply correctness of the whole
system. The system may still produce behaviors of special tree-width greater than $k$
that violate the given property.
However, if we know that a system only generates
behaviors from a class whose special tree-width is bounded by $k$,
we can still conclude that the system is correct.

This motivates the \emph{synchronizability problem}.
Several notions of synchronizability have been introduced in the literature.
However, they all amount to asking whether all behaviors generated by
a given communicating system have a particular shape,
i.e., whether they are all included in a fixed (or given) set of MSCs $\Class$.
Thus, the synchronizability problem is essentially an inclusion problem,
namely $\ppL{\Sys} \subseteq \Class$ or $\mbL{\Sys} \subseteq \Class$.
%
We show that, for decidability, it is enough to have that $\Class$
is MSO-definable and special-tree-width-bounded (STW-bounded):
%
We call $\Class \subseteq \MSCs$
\begin{enumerate*}[label={(\roman*)}]
\item \emph{MSO-definable} if there is
an MSO-formula $\phi$ such that $L(\phi) = \Class$,
\item \emph{LCPDL-definable} if there is an
an LCPDL-formula $\Phi$ such that $L(\Phi) = \Class$,
\item \emph{STW-bounded} if there is $k \in \N$
such that $\Class \subseteq \stwMSCs{k}$.
%\item \emph{\nameclass} if it is MSO-definable and STW-bounded.
\end{enumerate*}

An important component of the decidability proof is the following lemma,
which shows that we can reduce synchronizability
wrt.\ an STW-bounded class to bounded model-checking.

\begin{lemma}\label{lem:continuous}
Let $\System$ be a communicating system, $\comsymb \in \{\ppsymb, \mbsymb\}$,
$k \in \N$, and $\Class \subseteq \stwMSCs{k}$.
Then, $\cL{\System} \subseteq \Class$ iff
$\cL{\System} \cap \stwMSCs{(k+2)} \subseteq \Class$.
\end{lemma}

The result follows from the following lemma.
%, whose proof is in Appendix~\ref{app:continuous2}.
Note that a similar property
was shown in \cite[Proposition~5.4]{GKM07} for the specific class of existentially
$k$-bounded MSCs.

\begin{lemma}\label{lem:continuous2}
	Let $k \in \N$ and $\Class \subseteq \stwMSCs{k}$. For all
	$M \in \MSCs \setminus \Class$, we have
	$(\Pref{M} \cap \stwMSCs{(k+2)}) \setminus \Class \neq \emptyset$.
\end{lemma}

We now have all ingredients to state a generic decidability result
for synchronizability:

\begin{theorem}\label{thm:sync}
Fix finite sets $\Procs$ and $\Msg$.
Suppose $\comsymb \in \{\ppsymb, \mbsymb\}$ and let $\Class \subseteq \MSCs$ be an MSO-definable and STW-bounded class (over $\Procs$ and $\Msg$).
The following problem is decidable:
Given a communicating system $\System$, do we have $\cL{\System} \subseteq \Class$?
\end{theorem}

\begin{proof}
Consider the MSO-formula $\phi$ such that $L(\phi) = \Class$, and
let $k \in \N$ such that $\Class \subseteq \stwMSCs{k}$.
We have
$\cL{\System} \subseteq \Class
 \stackrel{\textup{Lemma~\ref{lem:continuous}}}{\Longleftrightarrow} \cL{\System} \cap \stwMSCs{(k+2)} \subseteq \Class
 \Longleftrightarrow \cL{\System} \cap \stwMSCs{(k+2)} \subseteq L(\phi)$.
The latter can be solved thanks to Fact~\ref{p2p} and Theorem~\ref{thm:mailbox_bounded_model_checking}.
\end{proof}

\subsection{Application to Concrete Classes of Synchronizability}

In this section, we instantiate our general framework by specific classes. Table~\ref{table:summary} gives a summary of the results.

\begin{table}
	\centering
	\caption{Summary of the decidability of the synchronizability problem in various classes\label{table:summary}}
	\begin{tabular}{ |p{4.8cm}||p{3.5cm}|p{3.5cm} | }
		\hline
		& \hfil\textsc{Peer-to-Peer}& \hfil\textsc{Mailbox}\\
		\hline
		Weakly synchronous   & Undecidable [Thm.~\ref{thm:p2p-weak-sync}]   & EXPTIME [Thm.~\ref{thm:mailbox-weak-sync}] \\
		\hline
		Weakly $k$-synchronous &  \multicolumn{2}{c|} {Decidable \cite{DBLP:conf/cav/BouajjaniEJQ18,DBLP:conf/fossacs/GiustoLL20} and [Thm.~\ref{thm:weak-sync}]}  \\
		\hline
		Strongly $k$-synchronous &  \hfil---   &  Decidable [Thm.~\ref{thm:strong-sync}]  \\
		\hline
		Existentially $k$-p2p-bounded & \multicolumn{2}{c|} {Decidable \cite[Prop.~5.5]{GKM07}}\\
		\hline
		Existentially $k$-mailbox-bounded &  \hfil---   & Decidable [Prop.~\ref{prop:exist-k-mailbox-bounded}]\\
		\hline
	\end{tabular}
\end{table}

\subsection{A New General Class: Weakly Synchronous MSCs} \label{sec:weakly-sync}

We first introduce the class of weakly synchronous MSCs. This is a generalization of synchronous MSCs studied earlier, in \cite{DBLP:conf/cav/BouajjaniEJQ18, DBLP:conf/fossacs/GiustoLL20}, which we shall discuss later. We say an MSC is weakly synchronous if it is breakable into \emph{exchanges} where an exchange is an MSC that allows one to schedule all sends before all receives. Let us define this formally:

\begin{definition}[exchange]\label{def:weak-synchr}
Let $\msc = (\Events,\procrel,\lhd,\lambda)$ be an MSC.
We say that $\msc$ is an \emph{exchange} if
$\SendEv{\msc}$ is
a ${\le_\msc}$-downward-closed set.
\end{definition}

\begin{definition}[weakly synchronous]\label{def:weaksync-new}
We say that $\msc \in \MSCs$ is
\emph{weakly synchronous} if it is of the form
$\msc = \msc_1 \cdot \ldots \cdot \msc_n$
such that every $\msc_i$ is an exchange.
\end{definition}


\noindent
\begin{minipage}[c]{11.cm}
We use the term \emph{weakly} to distinguish
from variants introduced later.

\begin{example}\label{example:msc_W}
	Consider the MSC $\mscW$ in Fig.~\ref{fig:msc_W}. It is is weakly synchronous. Indeed, $\amessage_1$, $\amessage_2$, and $\amessage_5$ are independent and can be put alone in an exchange. Repetitions of $\amessage_3$ and $\amessage_4$ are interlaced, but they constitute an exchange, as we can do all sends and then all receptions.
\end{example}

An easy adaptation of a characterization from \cite{DBLP:conf/fossacs/GiustoLL20} yields
the following result for weakly synchronous MSCs:
\end{minipage}
\begin{minipage}[c]{3cm}
	\hspace*{1cm}
\begin{center}
  \begin{tikzpicture}[>=stealth,node distance=3.4cm,shorten >=1pt,
  every state/.style={text=black, scale =0.6}, semithick,
    font={\fontsize{8pt}{12}\selectfont}]

\begin{scope}[shift = {(8,0.75)}, scale = 0.7]
  %MACHINES
  \draw (0,1.25) node{$q$} ;
  \draw (1.6,1.25) node{$r$} ;
  \draw (-1.25,1.25) node{$p$} ;
  \draw (0,1) -- (0,-3.1) ;
  \draw (1.6,1) -- (1.6,-3.1);
  \draw (-1.25,1) -- (-1.25,-3.1);

  %MESSAGES
  \draw[>=latex,->, dashed] (-1.25, 0.5) -- (0, 0.5) node[ above, midway] {$\amessage_1$};
  \draw[>=latex,->] (0, 0) -- (-1.25, 0) node[ above, midway] {$\amessage_2$};


  \draw[>=latex,->] (0, -0.4) -- (1.6, -1.75) node[pos=0.1, sloped, above] {$\amessage_3$};
  \draw[>=latex,->] (0, -1.1) -- (1.6, -2.45) node[pos=0.05, sloped, above] {$\amessage_3$}; %{$\amessage_1'$};
  %\draw[>=latex,->, dashed] (0, -2.5) -- (1.25, -3.25) node[pos=0.55, sloped, above] {$\amessage_1''$};

  \draw[>=latex,->] (1.6, -0.4) -- (0, -1.75) node[pos=0.1, sloped, above] {$\amessage_4$};
  \draw[>=latex,->] (1.6, -1.1) -- (0, -2.45) node[pos=0.05, sloped, above] {$\amessage_4$}; %{$\amessage_2'$};
  %\node[rotate = 90, left]at (1.13, -0.65) {$\cdots$};
  %\node[rotate = -90, right]at (0.1, -0.65) {$\cdots$};

  \draw[>=latex,->] (1.6, -2.6) -- (0, -2.6) node[ below, midway] {$\amessage_5$};


\end{scope}

\end{tikzpicture}
\captionof{figure}{MSC $\mscW$}
\label{fig:msc_W}
\end{center}
\end{minipage}

\begin{proposition}\label{prop:newweaklogiccg}
Let $\msc$ be an MSC. Then, $\msc$ is weakly synchronous iff no RS edge occurs on any cyclic path in the conflict graph $\cgraph{\msc}$.
\end{proposition}

It is easily seen that the characterization from
Proposition~\ref{prop:newweaklogiccg} is LCPDL-definable:

\begin{corollary}\label{cor:weak-sync-lcpdl}
The sets of weakly synchronous MSCs and weakly synchronous \emph{mailbox} MSCs are LCPDL-definable.
Both formulas have polynomial size.
\end{corollary}

Moreover, under the mailbox semantics, we can show:

\begin{proposition}\label{prop:new-weak-logic-bounded}
The set of weakly synchronous mailbox MSCs is
STW-bounded (in fact, it is included in $\stwMSCs{4|\Procs|}$).
\end{proposition}


\begin{proof}
Let $\msc$ be fixed, and let us sketch Eve's winning strategy.
Let $n = |\Procs|$.

The first step for Eve is to split $\msc$ in exchanges. She first disconnects
the first exchange from the rest of the graph ($2n$ pebbles are needed),
then she disconnects the second exchange from the rest of the graph ($2n$ pebbles needed, plus $n$ pebbles remaining from the first round), and so on for
each exchange.

So we are left with designing a winning strategy for Eve with $4n+1$ pebbles
on the graph of an exchange $\msc_0$,
where initially there are (at most) $n$ pebbles
placed on the first event of each process and also (at most) $n$ pebbles placed
on the last event of each process. Eve also places (at most) $n$ pebbles on the last
send event of each process and also (at most) $n$ pebbles on the first receive event of
each process. Eve erases the (at most) $n$ $\procrel$-edges between the last send event and the first receive event.

We are now in a configuration that will be our invariant.

Let us fix a mailbox linearization of $\msc_0$
and let $e$ be the first send event in this linearization.
\begin{itemize}
\item if $e$ is an unmatched send of process $p$,
Eve places her last pebble on the next
send event of $p$ (if it exists), let us call it $e'$. Then Eve erases the
$\procrel$-edge $(e,e')$, and now $e$ is completely disconnected,
so it can be removed and the pebble can be taken back.
\item if $e\lhd e'$, with $e'$ a receive event of process $q$,
then due to the mailbox semantics $e'$ is the first receive event of $q$,
so it has a pebble placed on it. Eve removes the $\lhd$-edge between
$e$ and $e'$, then using the extra pebble she disconnects $e$ and places a
pebble on the $\procrel$-successor of $e$, then she also
disconnects $e'$ and places a pebble on the $\procrel$-successor of $e'$.
\end{itemize}
After that, we are back to our invariant, so we can repeat the same
strategy with the second send event of the linearization, and so on until
all edges have been erased.
\end{proof}

We obtain the following result as a corollary.
Note that it assumes the mailbox semantics.

\begin{theorem}\label{thm:mailbox-weak-sync}
The following problem is decidable in exponential time:
Given $\Procs$, $\Msg$, and a communicating system $\System$ (over $\Procs$ and $\Msg$), is every MSC in $\mbL{\System}$ weakly synchronous?
\end{theorem}

\begin{proof}
According to Corollary~\ref{cor:weak-sync-lcpdl},
we determine the LCPDL formula $\Phi_\mathsf{wsmb}$ such
that $L(\Phi_\mathsf{wsmb})$ is the set of weakly synchronous mailbox MSCs. Moreover,
recall from Proposition~\ref{prop:new-weak-logic-bounded} that
the special tree-width of all weakly synchronous mailbox MSCs is bounded by
$4|\Procs|$.
%
By Lemma~\ref{lem:continuous},
$\mbL{\System} \subseteq L(\Phi_\mathsf{wsmb})$ iff
$\mbL{\System} \cap \stwMSCs{(4|\Procs|+2)} \subseteq L(\Phi_\mathsf{wsmb})$.
The latter is an instance of the bounded model-checking problem.
As the length of $\Phi_\mathsf{wsmb}$ is polynomial in
$|\Procs|$, we obtain that the original problem is decidable
in exponential time by Theorem~\ref{thm:mailbox_bounded_model_checking}.
\end{proof}

For the same reasons, the model-checking problem for ``weakly
synchronous'' systems is decidable.
%
Interestingly, a reduction from Post's correspondence problem
shows that decidability fails when adopting the p2p semantics:

\begin{theorem}\label{thm:p2p-weak-sync}
The following problem is undecidable:
Given finite sets $\Procs$ and $\Msg$ as well as a communicating system $\System$,
is every MSC in $\ppL{\System}$ weakly synchronous?
\end{theorem}
\newcommand{\pcpSigma}{A}
\begin{proof}
	We show that the control state reachability problem for p2p weakly
	synchronizable systems is not decidable. This immediately shows
	that the model-checking problem for p2p weak synchronizable systems
	is not decidable. \davide{Clarify why undecidability of control state reachability implies undecidability of model checking.} With some extra coding, it also shows that the
	membership problem (decide whether a given system is p2p weakly synchronizable)
	also is undecidable: indeed, it is enough to add a non weak
	synchronizable behavior after the control states for which reachability is
	undecidable: the system will be not weakly synchronizable iff the control
	states are reached.
	
	We reduce from Post correspondence problem (PCP).
	Let us recall that a PCP instance consists of $N$ pairs $(u_i,v_i)$ of
	finite words over an alphabet $\pcpSigma$, and that PCP undecidability holds already for
	$N=7$ and $\pcpSigma=\{0,1\}$. We let the set of messages be
	$\{1,\dots,N\}\uplus\pcpSigma\uplus\{\sharp\}$, and we consider a system with
	four machines: Prover1, Prover2, Verifier1, and Verifier2. We have
	unidirectional communication channels from provers to verifiers,
	so the system is weakly synchronous by construction.
	
	Informally, the system works as follows:
	\begin{itemize}
	\item Prover1 guesses a solution $u_{i_1}\dots u_{i_m}$ of the PCP instance,
	and Prover2 also guesses the same solution $v_{i_1}...v_{i_m}$.
	\item Prover1 sends $u_{i_1}\dots u_{i_n}$ to Verifier1 and sends
	simultaneously $i_1\dots i_m$ to Verifier2
	\item Prover2 sends $v_{i_1}\dots v_{i_m}$ to Verifier1 and sends
	simultaneously $i_1\dots i_m$ to Verifier 2
	\item Verifier1 checks that the two words are equal and Verifier2 checks that the sequences of indices are equal.
	\end{itemize}
	
	Let us now formally define these machines.
	We describe them with regular expressions. For $w=a_1\cdots a_n$,
	we write $\mathit{send}^*(p,q,w)$ (resp $\mathit{rec}^*(p,q,w)$)
	for $\send{p}{q}{a_1}\cdots \send{p}{q}{a_n}$ (resp $\rec{p}{q}{a_1}\cdots\rec{p}{q}{a_n}$). We abbreviate Prover1 as P1, Prover2 as P2, Verifier1 as V1, and Verifier2 as V2
	
	\begin{itemize}
	\item Prover1 is $$\big(\sum_{i=1}^N\send{P_1}{V_1}{i}\mathit{send}^*(P_1,V_2,u_i)\big)^+\send{P_1}{V_1}{\sharp}\send{P_1}{V_2}{\sharp}$$
	
	\item Prover2 is $$\big(\sum_{i=1}^N\send{P_2}{V_1}{i}\mathit{send}^*(P_2,V_2,v_i)\big)^+\send{P_2}{V_1}{\sharp}\send{P_2}{V_2}{\sharp}$$
	
	\item Verifier1 is $$\big(\sum_{i=1}^N\rec{P_1}{V_1}{i}\rec{P_2}{V_1}{i}\big)^*\rec{P_1}{V_1}{\sharp}\rec{P_2}{V_1}{\sharp}$$
	
	\item Verifier2 is $$\big(\sum_{a\in\Sigma}\rec{P_1}{V_2}{a}\rec{P_2}{V_2}{a}\big)^*\rec{P_1}{V_2}{\sharp}\rec{P_2}{V_2}{\sharp}$$
	
	\end{itemize}
	
	It can be checked that all machines reach their own final
	state if and only if the PCP instance has a solution.
\end{proof}	

\subsection{Weakly \emph{k}-Synchronous MSCs}\label{sec:weakly-k}

This negative result for the p2p semantics motivates the study of other classes.
In fact, our framework captures several classes introduced in the literature.

\begin{definition}[$k$-exchange]\label{def:weak-k-synchr}
Let $\msc = (\Events,\procrel,\lhd,\lambda)$ be an MSC
and $k \in \N$.
We call $\msc$ a $k$-\emph{exchange} if
$\msc$ is an exchange and $|\SendEv{\msc}| \le k$.
\end{definition}


Let us now recall the definition
from \cite{DBLP:conf/cav/BouajjaniEJQ18,DBLP:conf/fossacs/GiustoLL20}, but (equivalently)
expressed directly in terms of MSCs rather than via \emph{executions}. It differs from the weakly synchronous MSCs in that here, we insist on constraining the number of messages sent per exchange to be at most $k$.

\begin{definition}[weakly $k$-synchronous]\label{def:weaksync}
Let $k \in \N$.
We say that $\msc \in \MSCs$ is
weakly $k$-synchronous if it is of the form
$\msc = \msc_1 \cdot \ldots \cdot \msc_n$
such that every $\msc_i$ is a $k$-exchange.
\end{definition}

\noindent \begin{minipage}[c]{10.5cm}
\begin{example}
MSC $\mscweakSexist$ in Fig.~\ref{fig:msc_weak_S_exist} is weakly $1$-synchronous, as it can be
decomposed  into three \kE{1}s (the decomposition is depicted by the
horizontal dashed lines). We remark that $\mscweakSexist \in
\mbMSCs$. Note that there is a p2p linearization that respects the decomposition.
On the other hand, a mailbox linearization needs to reorganize actions from different MSCs: the sending of
$\msg_3$ needs to be done before the sending of $\msg_1$. Note that $\mscweakuniver$ in
Fig.~\ref{fig:msc_weak_univer} is also weakly $1$-synchronous.
\end{example}
\end{minipage}
\hfill
\begin{minipage}[c]{3cm}
\begin{center}

\begin{tikzpicture}[>=stealth,node distance=3.4cm,shorten >=1pt,
    every state/.style={text=black, scale =0.7}, semithick,
    font={\fontsize{8pt}{12}\selectfont}, scale = 0.8]

  %MACHINES
  \draw (0,0) node{$p$} ;
  \draw (1,0) node{$q$} ;
  \draw (2,0) node{$r$} ;
  \draw (0,-0.25) -- (0,-3.1) ;
  \draw (1,-0.25) -- (1,-3.1);
  \draw (2, -0.25) -- (2, -3.1) ;
  %MESSAGES
  \draw[>=latex,->, dashed] (0,-0.75) -- (1, -0.75) node[midway,above]{$\amessage_1$};

  \draw[>=latex,->] (1, -1.75) -- (0, -1.75) node[midway, above] {$\amessage_2$};
  %\draw (0.5,-1.7) node{$\cdots$};
  %\draw[>=latex,->] (1, -2.25) -- (0, -2.25) node[midway, above] {$\amessage_2$};

  \draw[>=latex,->] (2,-2.75) -- (1,-2.75) node[midway, above] {$\amessage_3$};
%\end{scope}
  \draw[dashed] (-0.5,-1.25) -- (2.5,-1.25) ;
  \draw[dashed] (-0.5,-2.25) -- (2.5,-2.25) ;


\end{tikzpicture}
\captionof{figure}{MSC $\mscweakSexist$}
\label{fig:msc_weak_S_exist}

\end{center}
\end{minipage}

\medskip


\begin{proposition}
	\label{prop:weak-logic-bounded}
	Let $k \in \N$. The set of weakly $k$-synchronous p2p (mailbox, respectively) MSCs
	is effectively MSO-definable.
\end{proposition}


In fact, MSO-definability essentially follows from the following known theorem:

\begin{theorem}[\cite{DBLP:conf/fossacs/GiustoLL20}] \label{th:scccharactweak}
	Let $M$ be an MSC. Then, $M$ is weakly $k$-synchronous iff every SCC in its conflict graph
	$\cgraph{\msc}$ is of size at most $k$ and no RS edge occurs on any cyclic path.
\end{theorem}

This property is similar to the graphical characterization of weakly synchronous MSCs, except  for the condition that every SCC in the conflict graph is of size at most $k$.
Furthermore, it is easy to establish a bound on the special tree-width:

\begin{proposition}
\label{prop:kweakstw}
	Let $k \in \N$. The set of MSCs that are weakly $k$-synchronous have special tree-width bounded by $2k+|\Procs|$.
\end{proposition}

Hence, we can conclude that the class of weakly $k$-synchronous MSCs is MSO-definable and STW-bounded.
As a corollary, we get the following (known) decidability result, but via an alternative proof:

\begin{theorem}[\cite{DBLP:conf/cav/BouajjaniEJQ18,DBLP:conf/fossacs/GiustoLL20}]\label{thm:weak-sync}
For $\comsymb \in \{\ppsymb, \mbsymb\}$, the following problem is decidable:
Given finite sets $\Procs$ and $\Msg$, a communicating system $\System$, and $k \in \N$,
is every MSC in $\cL{\System}$ weakly $k$-synchronous?
\end{theorem}

\begin{proof}
We proceed similarly to the proof of Theorem~\ref{thm:mailbox-weak-sync}.
%
For the given $\Procs$, $\Msg$, and $k$, we first determine,
using Proposition~\ref{prop:weak-logic-bounded}, the MSO formula $\phi_k$ such that $L(\phi_k)$ is the set of weakly $k$-synchronous p2p/mailbox MSCs. From Proposition~\ref{prop:kweakstw}, we know that the special tree-width of all weakly $k$-synchronous MSCs is bounded by
$2k + |\Procs|$.
%
By Lemma~\ref{lem:continuous}, we have
$\cL{\System} \subseteq L(\phi_k)$ iff
$\cL{\System} \cap \stwMSCs{(2k + |\Procs| + 2)} \subseteq L(\phi_k)$.
The latter is an instance of the bounded model-checking problem.
By Fact~\ref{p2p} and Theorem~\ref{thm:mailbox_bounded_model_checking}, we obtain decidability.
\end{proof}

\begin{remark}
The set of weakly $k$-synchronous MSCs is not directly expressible in LCPDL
(the reason is that LCPDL does not have a built-in counting mechanism).
However, its \emph{complement} is expressible in the extension of LCPDL with
existentially quantified propositions
(we need $k+1$ of them). The model-checking problem for
this kind of property is still in EXPTIME and, therefore, so is the problem from
Theorem~\ref{thm:weak-sync} when $k$ is given in unary. It is very likely that our approach can also be used to
infer the PSPACE upper bound from \cite{DBLP:conf/cav/BouajjaniEJQ18}
by showing bounded \emph{path width} and using finite word automata instead of tree automata.
Finally, note that
the problem to decide whether there exists an integer $k \in \N$ such that all MSCs in $\cL{\System}$
are weakly $k$-synchronous
has recently been studied in \cite{DLL2021} and requires different techniques.
\end{remark}

Observe also that we can remove the constraint of all the sends preceding all the receives in a $k$-exchange, and still have decidability. We then have the following definition.

\begin{definition}[modified $k$-exchange]\label{def:mod-weak-synchr}
	Let $\msc = (\Events,\procrel,\lhd,\lambda)$ be an MSC
	and $k \in \N$.
	We call $\msc$ a \emph{modified} $k$-\emph{exchange} if $|\SendEv{\msc}| \le k$.
\end{definition}

We extend this notion to consider modified weakly $k$-synchronous executions as before, and the graphical characterization of this property is that there are at most $k$ nodes in every SCC of the conflict graph. Hence, this class is also MSO-definable, and since each modified $k$-exchange has at most $2k$ events, it also has bounded special tree-width. 

\section{My stuff}

\davide{Consider also unmatched messages in causal order communication (causal delivery using the terminology from Laetitia's thesis).}

\subsection{Semantics of causal ordering}

\newcommand{\buffers}{\vv{\text{Buf}}}
\newcommand{\clocks}{\vv{\text{Vec}}}
\begin{definition}
	Given a system $\System = (Loc_p, \delta_p, \ell^0_p)_{p\in\procSet}$ with $n$ processes, a \emph{configuration} is a tuple $(\vv{\ell},\buffers,\clocks)$, where $\vv{\ell}=(\ell_p)_{p \in \procSet}$ represents the global state of the system, $\buffers=(b_p)_{p \in \procSet}$ is a vector of buffers, with each $b_p \in \Msg^*$ representing the content of the buffer of process $p$, and $\clocks=(v_p)_{p \in \procSet}$ is a vector of Mattern-Fidge logical clocks, where each $v_p = (time_i)_{i \in \procSet}$ represents the content of the logical vector clock of process $p$.
\end{definition}

\subsection{Message Sequence Charts}

% \davide{Add definition of causally ordered MSC}

% \begin{definition}[{Causally ordered linearization without unmatched messages}]
% Given an MSC $\msc = (\Events,\procrel,\lhd,\lambda)$ with ${\le}_\msc = ({\procrel} \cup {\lhd})^\ast$, a \emph{causally ordered} ($\cosymb$) \emph{linearization} of $\msc$ is a total order $\linrel_\cosymb \subseteq \Events \times \Events$ such that ${\le_\msc} \subseteq {\linrel_\cosymb}$ and, for any pair of send events $(x,y)$, we have that
% \[
% \begin{cases} 
% x \lhd x' \\
% y \lhd y' \\
% x \le y  
% \end{cases}
% \implies \quad  x' \linrel_\cosymb y'
% \]
% \end{definition}

% \begin{definition}[{Causally ordered MSC without unmatched messages}]
% An MSC is a \emph{causally ordered MSC} iff it has at least one causally ordered linearization.
% \end{definition}

\paragraph*{Causally ordered MSCs}

An MSC is causally ordered if all the messages sent to the same process are received in an order which is consistent with the causal ordering of the corresponding send events. More formally, for an MSC $\msc = (\Events,\procrel,\lhd,\lambda)$ we define an additional binary relation ${\corel}_\msc \subseteq \Events \times \Events$ that represents a constraint
under the causal ordering semantics.  In particular, given two receive events $f_1$ and $f_2$, we have that $f_1 \corel_\msc f_2$ if both the following hold:
\begin{itemize}\itemsep=0.5ex
	\item $\lambda(f_1) \in \qrAct{q}$,
	$\lambda(f_2) \in \qrAct{q}$
	\item $e_1 \lhd f_1$ and $e_2 \lhd f_2$ for some $e_1,e_2 \in \Events$, such that $e_1 \le_\msc e_2$.
\end{itemize}

We let ${\lessdot_\msc} = ({\procrel} \,\cup\, {\lhd} \,\cup\, {\corel_\msc})^\ast$.
Note that ${\le_\msc} \subseteq {\lessdot_\msc}$.
We call $\msc \in \ppMSCs$ a \emph{causally ordered (CO) MSC}
if ${\lessdot_\msc}$ is a partial order. The set of causally ordered MSCs $\msc \in \ppMSCs$ is denoted by $\coMSCs$.

\medskip

In literature, several possible implementations of causal ordering can be found. For instance, the algorithm described in \cite{DBLP:conf/wdag/SchiperES89} makes use of the logical vector clocks introduced by Mattern-Fidge \cite{Fidge88timestampsin, Mattern89virtualtime} to enforce causal ordering.

\subsection{Communicating Systems}

\begin{lemma}
\label{lem:co-prefix}
Every prefix of a causally ordered MSC is a causally ordered MSC.
\end{lemma}
\begin{proof}
Let $\msc = (\Events, \procrel, \lhd, \lambda) \in \coMSCs$ and let $\msc_0 =
(\Events_0, \procrel_0, \lhd_0, \lambda_0)$ be a prefix of $\msc$. By contradiction, suppose that $\msc_0$ is not a	causally ordered MSC. Then, there are distinct $e,f \in \Events_0$ such that $e \lessdot_{\msc_0}f$ and $f \lessdot_{\msc_0}e$, with ${\lessdot_{\msc_0}} = ({\rightarrow_0} \cup {\lhd_0} \cup {\corel_{\msc_0}})^*$.
As $\Events_0 \subseteq \Events$, we have that ${\rightarrow_0} \subseteq {\rightarrow}$, ${\lhd_0} \subseteq {\lhd}$, and ${\corel_{\msc_0}} \subseteq {\corel_{\msc}}$. Finally, we have that ${\lessdot_{\msc_0}} \subseteq {\lessdot_{\msc}}$ and $\msc$ cannot be a causally ordered MSC, which is a contradiction.
\end{proof}

Lemma~\ref{lem:prefix-closed} can be easily extendend to $\comsymb = \cosymb$.

\begin{lemma}\label{lem:co-prefix-closed}
	For all $\comsymb \in \{\ppsymb, \mbsymb, \cosymb\}$, $\cL{\Sys}$ is prefix-closed:
	$\Pref{\cL{\Sys}} \subseteq \cL{\Sys}$.
\end{lemma}
\begin{proof}
	Follows from Lemma~\ref{lem:co-prefix}.
\end{proof}

\subsection{Model Checking}

\begin{proposition}\label{prop:co_mso}
	The set $\coMSCs$ of causally odered MSCs is MSO-definable.
\end{proposition}
\begin{proof}
	The set of causally ordered MSCs can be defined using the MSO formula
	\[
	\coformula =
	\neg \exists x.\exists y. \left(
	\bigvee_{\substack{q \in \Procs\\a,b \in \qsAct{q}}}\hspace{-1em}
	\lambda(x) = a \;\wedge\; \lambda(y) = b
	\;\wedge\; x \le y \;\wedge\;
	\exists x'.\exists y'. \left(
	\begin{array}{ll}
		x \lhd x' & \wedge\\
		y \lhd y' & \wedge\\
		y' \procrel^+ x' &
	\end{array} \right)
	\right)
	\]
	The property $\coformula$ says that there cannot be two send events $x$ and $y$, with the same recipient, such that $x \le y$ and their corresponding receive events $x'$ and $y'$ happen in the opposite order, i.e. $y' \procrel^+ x'$. The set $\coMSCs$ of causally ordered MSCs is therefore MSO-definable as $\coMSCs=L(\coformula)$.
\end{proof}

Knowing that $\coMSCs$ is MSO-definable, Theorem~\ref{thm:mailbox_bounded_model_checking} can be restated for $\comsymb = \cosymb$.

\begin{theorem}
	\label{thm:co_bounded_model_checking}
	The bounded model-checking problem for $\comsymb =  \cosymb$ is decidable.
\end{theorem}
\begin{proof}
By Proposition~\ref{prop:co_mso}, $\coMSCs=L(\coformula)$.Given a system $\System$, we have that $\coL{\System} = \ppL{\System} \cap L(\coformula)$. Therefore, we can rewrite the bounded model checking problem for $\comsymb = \cosymb$ as

\[\begin{array}{rl}
&\coL{\System} \cap \stwMSCs{k} \subseteq L(\phi)\\[1ex]
\Longleftrightarrow &\ppL{\System} \cap L(\coformula) \cap \stwMSCs{k} \subseteq L(\phi)\\[1ex]
\Longleftrightarrow &\ppL{\System} \cap \stwMSCs{k} \subseteq L(\phi) \cup L(\neg \coformula)\\[1ex]
\Longleftrightarrow &\ppL{\System} \cap \stwMSCs{k} \subseteq L(\phi \vee \neg \coformula)\,.
\end{array}\]
The latter is decidable due to Fact~\ref{p2p}.
\end{proof}

\subsection{Synchronizability}

\begin{table}
	\centering
	\caption{Summary of the decidability of the synchronizability problem in various classes\label{table:my_summary}}
	\begin{adjustbox}{max width=\textwidth}
	\begin{tabular}{ |c||c|c|c| }
		\hline
		& \textsc{P2P}& \textsc{Causal ordering}& \textsc{Mailbox} \\
		\hline
		Weakly synchronous   & Undecidable [Thm.~\ref{thm:p2p-weak-sync}] & \textcolor{red}{Undecidable} [Thm.~\ref{thm:co-weak-sync}]   & EXPTIME [Thm.~\ref{thm:mailbox-weak-sync}] \\
		\hline
		Weakly $k$-synchronous &  \multicolumn{3}{c|} {\textcolor{red}{Decidable} [Thm.~\ref{thm:co-weak-k-sync}]}  \\
		\hline
	\end{tabular}
	\end{adjustbox}
\end{table}

Note that Lemma~\ref{lem:continuous} can be extended to $\comsymb = \cosymb$, since Lemma~\ref{lem:continuous2} does not depend on the kind of communication used by the system. 

\begin{lemma}\label{lem:continuous_co}
Let $\System$ be a communicating system, $\comsymb \in \{\ppsymb, \mbsymb, \cosymb\}$,
$k \in \N$, and $\Class \subseteq \stwMSCs{k}$.
Then, $\cL{\System} \subseteq \Class$ iff
$\cL{\System} \cap \stwMSCs{(k+2)} \subseteq \Class$.
\end{lemma}

Theorem~\ref{thm:sync} can also be extended to $\comsymb = \cosymb$.

\begin{theorem}\label{thm:sync_co}
Fix finite sets $\Procs$ and $\Msg$.
Suppose $\comsymb \in \{\ppsymb, \mbsymb,\cosymb\}$ and let $\Class \subseteq \MSCs$ be an MSO-definable and STW-bounded class (over $\Procs$ and $\Msg$).
The following problem is decidable:
Given a communicating system $\System$, do we have $\cL{\System} \subseteq \Class$?
\end{theorem}
\begin{proof}
Same as the proof for Theorem~\ref{thm:sync}, but using Lemma~\ref{lem:continuous_co} in place of Lemma~\ref{lem:continuous}, and Theorem~\ref{thm:co_bounded_model_checking} in place of Theorem~\ref{thm:mailbox_bounded_model_checking}.
\end{proof}

\subsubsection{Weakly synchronous causally ordered MSCs}

Corollary~\ref{cor:weak-sync-lcpdl} can  be extended to $\comsymb = \cosymb$. 

\begin{proposition}\label{cor:co-weak-sync-mso}
The set of weakly synchronous \emph{causally ordered} MSCs is MSO-definable.
\end{proposition}
\begin{proof}
Both the sets of weakly synchronous MSCs and of causally ordered MSCs are MSO-definable, as shown by Corollary~\ref{cor:weak-sync-lcpdl} and Proposition~\ref{prop:co_mso}. Recall that any LCPDL-definable property is also MSO-definable. It suffices to take the conjuction of the two respective MSO formulas.
\end{proof}


\begin{theorem}\label{thm:co-weak-sync}
The following problem is undecidable:
Given finite sets $\Procs$ and $\Msg$ as well as a communicating system $\System$,
is every MSC in $\coL{\System}$ weakly synchronous?
\end{theorem}
\begin{proof}
The proof is essentially identical to the \pp case. We do the same reduction from the Post correspondence problem. Recall from the proof of Theorem~\ref{thm:p2p-weak-sync} that we consider a system $\System$ with four machines (P1, P2, V1, V2), where we have unidirectional communication channels from provers to verifiers. In particular notice that all the possible behaviours of $\System$ are causally ordered, i.e. $\ppL{\System} \subseteq \coMSCs$; according to how we built our system $\System$, it is impossible to have a pair of causally-related send events of P1 and P2\footnote{Notice that there is no channel between P1 and P2, and we only have unidirectional communication channels from provers to verifiers.}, hence the causal ordering binary relation $\corel_\msc$ will be empty for any $\msc \in \ppL{\System}$ (i.e. causal ordering is already ensured by any possible \pp behaviour of $\System$). The rest of the proof is identical to the \pp case.
\end{proof}

\begin{proposition}
The set of weakly synchronous causally ordered MSCs has unbounded special tree-width.
\end{proposition}
\begin{proof}
Suppose that the set of weakly synchronous causally ordered MSCs is STW-bounded. By Proposition \ref{cor:co-weak-sync-mso} and Theorem~\ref{thm:sync_co}, we have that the syncronicity problem for the class of weakly synchronous causally ordered MSCs would be decidable. This is a contradiction, since Theorem~\ref{thm:co-weak-sync} states that this problem is undecidable.
\end{proof}

\subsubsection{Weakly \emph{k}-synchronous causally ordered MSCs} 

\begin{proposition}\label{prop:co-weak-k-sync-mso}
The set of weakly \emph{k}-synchronous causally ordered MSCs is MSO-definable.
\end{proposition}
\begin{proof}
Both the sets of weakly \emph{k}-synchronous MSCs and of causally ordered MSCs are MSO-definable, as shown by Proposition~\ref{prop:weak-logic-bounded} and Proposition~\ref{prop:co_mso}. It suffices to take the conjuction of the two respective MSO formulas.
\end{proof}

% Note that every causally ordered \emph{k}-synchronous MSC is also a  $\ppsymb$ \emph{k}-synchronous MSC. 
Theorem~\ref{thm:sync} can be easily extended to $\comsymb = \cosymb$.

\begin{theorem}\label{thm:co-weak-k-sync}
For $\comsymb \in \{\ppsymb, \mbsymb,\cosymb\}$, the following problem is decidable:
Given finite sets $\Procs$ and $\Msg$, a communicating system $\System$, and $k \in \N$,
is every MSC in $\cL{\System}$ weakly $k$-synchronous?
\end{theorem}
\begin{proof}
By Proposition~\ref{prop:co-weak-k-sync-mso} and Proposition~\ref{prop:kweakstw} we have that the class of causally ordered \emph{k}-synchronous MSCs is MSO-definable and STW-bounded\footnote{Note that Proposition~\ref{prop:kweakstw} is independent from the type of communication.}. Theorem~\ref{thm:sync_co} ends the proof.
\end{proof}


\printbibliography

\end{document}