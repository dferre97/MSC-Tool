\documentclass[a4paper,UKenglish,cleveref, autoref, thm-restate]{lipics-v2021}

\bibliographystyle{plainurl}% the mandatory bibstyle
%
\usepackage{graphicx}
\usepackage{booktabs}
\usepackage{array}
\usepackage{verbatim}
\usepackage{sectsty}
\allsectionsfont{\sffamily\mdseries\upshape}
\usepackage{amsmath}
\usepackage{amsbsy}
\usepackage{qtree}
\usepackage{cite}
\usepackage[inline]{enumitem}

\usepackage{thmtools}
\declaretheorem{fact}
\usepackage{thm-restate}

\usepackage[
    linecolor=blue,
    bordercolor=blue,
    backgroundcolor=white]{todonotes}

\usepackage{float}


\usepackage{xspace}

\usepackage{amsfonts}
\usepackage{mathtools}
\usepackage{bussproofs}
\usepackage{stackengine}
\usepackage{amssymb}
\usepackage{xcolor}
\usepackage{tikz}
\usepackage{stmaryrd}
\usepackage{tikz}
\usepackage{proof}
\usetikzlibrary{automata, positioning, arrows, petri}
\usepackage[titles,subfigure]{tocloft}

\usepackage{macro}
\usepackage[normalem]{ulem}

\nolinenumbers

\def\islongversion{yes}
\def\yes{yes}
\def\no{no}



\title{A Unifying Framework for Deciding Synchronizability}


\author{Benedikt Bollig}
       {Universit{\'e} Paris-Saclay, ENS Paris-Saclay, CNRS, LMF, France}
       {bollig@lsv.ens-cachan.fr}
       {https://orcid.org/0000-0003-0985-6115}{}
\author{Cinzia Di Giusto}
       {Universit{\'e} C\^{o}te d’Azur, CNRS, I3S, France}
       {cinzia.di-giusto@univ-cotedazur.fr}
       {https://orcid.org/0000-0003-1563-6581}{}
\author{Alain Finkel}
       {Universit{\'e} Paris-Saclay, ENS Paris-Saclay, CNRS, LMF, France \and
         Institut Universitaire de France}
       {finkel@lsv.fr}
       {https://orcid.org/0000-0003-0702-3232}{}
\author{Laetitia Laversa}
       {Universit{\'e} C\^{o}te d’Azur, CNRS, I3S, France}
       {laetitia.laversa@univ-cotedazur.fr}
       {https://orcid.org/0000-0003-3775-6496}{}
\author{Etienne Lozes}
       {Universit{\'e} C\^{o}te d’Azur, CNRS, I3S, France}
       {etienne.lozes@univ-cotedazur.fr}
       {https://orcid.org/0000-0001-8505-585X}{}
\author{Amrita Suresh}
       {Universit{\'e} Paris-Saclay, ENS Paris-Saclay, CNRS, LMF, France}
       {amrita.suresh@ens-paris-saclay.fr}
       {https://orcid.org/0000-0001-6819-9093}{}

\authorrunning{B. Bollig et al.}


\Copyright{Benedikt Bollig, Cinzia Di Giusto, Alain Finkel, Laetitia Laversa, Etienne Lozes, and Amrita Suresh}

\ccsdesc[500]{Theory of computation~Formal languages and automata theory}

\keywords{MSO, MSC, bounded tree-width, communicating automata, synchronisability}


%Editor-only macros:: begin (do not touch as author)%%%%%%%%%%%%%%%%%%%%%%%%%%%\
%%%%%%%
%\EventEditors{Serge Haddad and Daniele Varacca}
%\EventNoEds{2}
%\EventLongTitle{32nd International Conference on Concurrency Theory (CONCUR 2021)}
%\EventShortTitle{CONCUR 2021}
%\EventAcronym{CONCUR}
%\EventYear{2021}
%\EventDate{August 23--27, 2021}
%\EventLocation{Virtual Conference}
%\EventLogo{}
%\SeriesVolume{203}
%\ArticleNo{5}
%%%%%%%%%%%%%%%%%%%%%%%%%%%%%%%%%%%%%%%%%%%%%%%%%%%%%%




\begin{document}

\maketitle

\begin{abstract}
Several notions of synchronizability of a message-passing
system have been introduced in the literature. Roughly, a
system is called synchronizable if every execution
can be rescheduled so that it meets certain criteria, e.g.,
a channel bound. We provide a framework, based on MSO
logic and (special) tree-width, that unifies existing definitions,
explains their good properties, and allows one to easily derive other,
more general definitions and decidability results for synchronizability.

\keywords{communicating finite-state machines, message sequence charts,
synchronizability, MSO logic, special tree-width}
\end{abstract}

\section{Introduction}

\paragraph*{Communication systems.}
The model of concurrent processes communicating asynchronously through FIFO channels is used since the 1960s in  applications such as communication protocols \cite{DBLP:journals/sigops/Bochmann75}, hardware design, MPI
programs, and more recently for designing and verifying session types \cite{DBLP:journals/corr/abs-1901-09606}, web contracts, choreo\-graphies, concurrent programs, Erlang, Rust, etc.
~Since communication systems use FIFO channels, it is well known that all non-trivial properties (e.g., are all channels bounded?) are undecidable \cite{DBLP:journals/jacm/BrandZ83}, essentially because a FIFO channel may simulate the tape of Turing machines and the counters of Minsky machines.
%
However, there are many subclasses of communication systems for which the control-state reachability problem becomes decidable: e.g., synchronizable systems and existentially bounded systems (executions can be reorganized or decomposed into a finite number of sequences in which all channels are bounded), flat FIFO machines \cite{EGM2012,FP-lmcs20} (the graph of the machine does not contain nested loops), channel-recognizable systems \cite{DBLP:conf/cav/BoigelotG96}, unreliable (lossy, insertion, duplication) FIFO systems \cite{GC-AF-SPI-IC-96}, input-bounded FIFO machines \cite{BDM-concur20}, and half-duplex systems \cite{CF-icomp05}.

\paragraph*{On the boundedness problem.}
We focus on the  boundedness problem, which is known to be undecidable.  We could limit our analysis to decide whether for a given integer $k\geq 0$, known in advance, the FIFO channels are $k$-bounded, and this property is generally decidable in PSPACE. %; but we cannot know in advance the bound $k$ to test.
%	Because of this undecidability frontier and because
Unfortunately, the $k$-boundedness property is too binding since we could want to design an \emph{unbounded} system that is able, for example, to make unbounded iterations of sending  and receiving messages. Hence, to cope with this limitation, one can find
%in the literature
variants of the boundedness property that essentially reduce to say
%, for p2p systems,
that every unbounded execution of a system (i.e., channels are unbounded along the execution) is equi\-va\-lent (for instance, causally equivalent) to another \emph{bounded} execution.
%		The litterature is not unified and many similar names have been given to different properties.
%Here, we will consider the weaker $k$-synchronizable property where every $k'$-bounded behavior with $k' > k$, is the "same" than an $k$-bounded behavior.

%\alain{end of the new introduction, what follows has not been read again}

%		 let us use for now the word "Slack elasticity".

%Depending the exact definition of "Slack elasticity", $k$-synchronizability and sometimes synchronizability are decidable: for example the $k$-synchronizability in \cite{DBLP:journals/corr/abs-1804-06612,DBLP:conf/fossacs/GiustoLL20} is decidable and in fact more surprisingly, the synchronizability is also decidable \cite{DBLP:conf/fossacs/GiustoLL20}\\


%We will try to clean the vocabulary and the results and proofs

%The decidability proofs of the question "is a system $k$-synchronizable ?" and then "is reachability decidable ?" for all these models suffer frome a lack of unification.\\


%\noindent {\bf About synchronizability.}
\paragraph*{About synchronizability.}
%\alain{how to clearly and shortly present the two semantics words and MSC, the two types of communication p2p and mailbox, and the treatment of matched and unmatched messages ?}
To mention some examples, Lohrey and Mus\-choll introduced
\emph{existentially k-bounded} systems \cite{DBLP:journals/iandc/LohreyM04} (see
also \cite{DBLP:conf/fossacs/LohreyM02,DBLP:conf/dlt/GenestMK04,GKM07}) where
all accepting executions leading to a stable (with empty channels) final
configuration can be re-ordered into a $k$-bounded execution. This property is
undecidable, even for a given $k$ \cite{GKM07}. A  more general definition,
still called existentially bounded, is given in 2014 where the considered
executions are \emph{not} supposed to be final or stable
\cite{kuske2014communicating}. In  \cite{DBLP:journals/iandc/LohreyM04,HENRIKSEN20051}, the notion of
\emph{universally k-bounded} (all possible schedulings of an execution are $k$-bounded) is also
discussed and the authors show that the property is undecidable in
general.
%
%There exist many other variations of this idea in the litterature.
In 2011, Basu and Bultan introduced \emph{synchronizable} systems \cite{DBLP:conf/www/BasuB11}, for which every execution is equivalent (for the projection on sending messages)
to one of the same system but communicating by rendezvous; to avoid ambiguity, we call such systems \emph{send-synchronizable}.
%		and many papers studied it \cite{DBLP:conf/popl/BasuBO12}. undecidable in \cite{DBLP:conf/icalp/FinkelL17}
%
In 2018, Bouajjani et al., called a system $\System$ \emph{\kSable{k}} \cite{DBLP:conf/cav/BouajjaniEJQ18} (to avoid confusion %with previous and new definitions of the so-called synchronizability,
we call such systems \emph{weakly \kSable{k}}) if every MSC of $\System$
admits a linearization (which is not necessarily an execution) that can be divided into blocks of at most $k$ messages. After each block, a message is either read, or will never be read.
This constraint
%about all the MSCs of the system
seems to imply that buffers are bounded to $k$ messages.
 However,
as the linearization need not be an execution, this implies that a weakly $k$-synchronizable execution, even with the more efficient reschedule, can need unbounded channels to be run by the system.
%\alain{we may call a system "C-bounded“  if C is robust and  L(S) is included in C}

\paragraph*{Communication architecture and variants.}
A key difference between these works is that
they consider different communication architectures.
Existentially bounded systems
have been studied for p2p (with one queue per
pair of processes), whereas $k$-synchronizability has been studied
for mailbox communication, for which each process merges
all its incoming messages in a unique queue. The decidability
results for $k$-synchronizability have been extended to p2p
communications~\cite{DBLP:conf/fossacs/GiustoLL20}, but
it is unknown whether the decidability results for existentially
bounded systems extend to mailbox communication. Moreover,
variants of those definitions can be
obtained depending on if we consider %or not
messages that are sent but never read,
called unmatched messages.
Indeed the challenges that arise  in  \cite{DBLP:conf/cav/BouajjaniEJQ18} are due to mailbox communication and unmatched messages blocking a channel so that all messages sent afterwards will never be read.
%This type of complication only happens in mailbox communication, as \pp one is more permissive.
To clarify and overcome this issue, we propose \emph{strong \kSity{k}}, a new definition that is suitable for mailbox communication: an execution is called \emph{strongly $k$-synchronizable} if it can be rescheduled into another $k$-bounded execution such that there are at most $k$ messages in the channels before emptying them.

%  Indeed, if each execution can be rescheduled and divided into sections of $k$ messages, we could think that buffers are bounded to $k$ messages. However,
% % 	      in \cite{DBLP:conf/cav/BouajjaniEJQ18}, the definition of weak \kSity{k},
% the reschedule of the execution has not to be an execution. It results that a weak $k$-synchronous execution, even with the more efficient reschedule, can need unbounded channels to be run by the system.
 %Another source of ambiguity is the messages that are sent but never read, called unmatched messages. Such messages block the concerned channel, and all messages sent after will never be read too.
%		They still be stored in the channel and prevent messages sent after in the same (FIFO) channel to be read.
%So in a weakly \kSable{k} system, strongly \kSable{k} system and existentially bounded system, an infinity of unmatched messages can be stored in a channel and never consummed.
% To clarify this issue, we propose a new definition, the \emph{strong \kSity{k}}:
% % (or simply \kSity{k} when there is no ambiguity)
% %in which the reschedule of an execution has to be an execution of the system.
% %\alain{say something about p2p and mailbox}
% %			Therefore, channels are not $k$-bounded.
% %
% %		We propose to count unmatched messages and
% an execution is said \emph{strongly $k$-synchronizable} if it can be rescheduled into another $k$-bounded execution
% such that \xout{there are at most $k$ unmatched messages in each channel}.
% \comL{there are at most $k$ messages in the channels before emptying them}.

% They are two classes usually called existentially bounded systems that \cite{DBLP:journals/corr/abs-1901-09606} distinguish into existentially bounded systems and existentially \emph{stable} bounded systems when the considered executions must arrive in a final stable configuration (there exists a set of final control-states) with empty channels.


%Finally, a strongly \kSable{k} system is also \existb{k} system: each execution can be reschedule to have, at each moment, at most $k$ messages in each channel.

%\iffalse
%
%
%%There exist variants of this definition where executions are replaced by linearization of MSC (which are not necessarly executions) and some other definitions don't consider messages that are never consummed.
%\begin{center}
%	\begin{tabular}{| l | l | l| }
%		\hline
%		& $k$-weakly-synchronizable   & $k$-existentially-bounded \\ \hline
%
%		control-state reachability  & Decidable \cite{DBLP:journals/corr/abs-1804-06612,DBLP:conf/fossacs/GiustoLL20}  & D \cite{GKM07} \\  \hline
%
%		repeated control-state  reachability &  \textcolor{red}{Unknown}   &  \textcolor{red}{Unknown}  \\  \hline
%
%		configuration reachability &  \textcolor{red}{Unknown}  &  \textcolor{red}{Unknown} \\
%		\hline
%		boundedness &   \textcolor{red}{Unknown}  & Undecidable  \cite{GKM07}  \\
%		\hline
%	\end{tabular}
%\end{center}
%
%%		for later may be
%\begin{center}
%	\begin{tabular}{| l | l | l | l| }
%		\hline
%		& k-weakly-synchro  & k-strongly-synchro & existentially-bounded \\ \hline
%
%		cs   reachability & D (2018-20) &  \textcolor{red}{D here} & D (GKM 07) \\  \hline
%
%		repeated cs   reach &  \textcolor{red}{Unknown} &  \textcolor{red}{D}   &  \textcolor{red}{Unknown}  \\  \hline
%
%		conf reachability &  \textcolor{red}{Unknown} &   \textcolor{red}{here } &  \textcolor{red}{Unknown} \\
%		\hline
%		boundedness &   \textcolor{red}{Unknown} & \textcolor{red}{Unknown } & U  (GKM 07)  \\
%		\hline
%	\end{tabular}
%\end{center}
%
%
%%%%%%%%%%%%%
%\subsection{Decidability of weak and strong \kSity{k} and existentially k-bounded }\label{section:decidability_with_k}
%
%
%
%%{\bf The three existential problems} \\
%
%\begin{itemize}
%	\item {\bf The strongly k-synchronizable problem} \\
%	input : a system $S$ and $k \geq 0$ \\
%	question : is S strongly k-synchronizable (SkS) ?
%	\item {\bf The weakly k-synchronizable problem} \\
%	input : a system $S$ and $k \geq 0$ \\
%	question : is S weakly k-synchronizable (WkS) ?
%	\item {\bf The existentially k-bounded problem} \\
%	input : a system $S$ and $k \geq 0$ \\
%	question : is S existentially k-bounded (EkB)?
%\end{itemize}
%%%%%%%%%%%%
%\fi
%%%%%%%%%%%
%\comL{The following definitions can be applied to systems with different type of communication, such as \pp or mailbox, or compare sometimes executions of the system or MSCs, abstraction of executions preserving causal dependencies between messages.}

%\noindent {\bf Our contributions.}
\paragraph*{Contributions.}
Our contributions can be summarized as follows:
\begin{itemize}
\item
In order to unify the notions of synchronizability, we introduce a general
framework based on monadic second-order (MSO) logic and (special) tree-width
that captures most existing definitions of systems that may work with bounded
channels. % Our framework allows us to unify and to simplify the proofs and
%sometimes also to extend the statement.
Moreover, reachability and model
checking are shown decidable in this framework.
\item We show that existentially %(resp. universally)
%\comEtienne{what about not talking about universally bounded in this paper (currently, we are very incomplete about it)}
bounded systems can be expressed in our framework and, as a consequence, the existentially %(resp. universally)
$k$-bounded property is decidable by using the  generic proof.

\item We generalize the existing notion of (weak) $k$-synchronizability in \cite{DBLP:conf/cav/BouajjaniEJQ18}
and we introduce three new classes of synchronizable systems: weakly synchronizable (which are more general than weakly $k$-synchronizable), strongly synchronizable and strongly $k$-synchronizable (which are particular cases of weakly synchronizable). We then prove that these properties all fit in our framework and are all shown decidable using the  generic proof.
%
%\xout{that is always (for p2p and mailbox) a strict subclass of existentially bounded systems and for which the strong $k$-synchronizability property is}

\item We then deduce that reachability and model
checking are decidable for these classes
%six system classes
(only control-state reachability was shown to be decidable for weakly $k$-synchronizable  in \cite{DBLP:conf/cav/BouajjaniEJQ18} and it is clearly also decidable for existentially/universally bounded systems but reachability properties are generally not studied for these classes of systems).

\item In order to obtain better complexity results for some classes (strongly and weakly synchronizable systems), we also use the fragment of propositional dynamic logic with loop and converse (LCPDL) instead of MSO logic in our framework.
%
%\alain{say more or less: Notice that for existentially and
%universally bounded systems, we are able to retrieve decidability as we consider
%systems without final states.}\\

%\alain{too much detailled for the introduction: Our framework allows to unify and to simplify the proofs and sometimes it also allows to extend the statement as Definition \ref{def:weak-synchr-new} of a $k$-exchange is more general than the existing Definition \ref{def:weak-synchr} but it is sufficient to obtain decidability.}
%\item

%	\item
%\alain{to modify with the new classes of weakly/strongly synchronizable}
%
%\item a general framework to capture most of the variants of synchronizability discussed above and to assess the decidability of the synchronizability problem as well as model checking (only the control-state is shown decidable in \cite{DBLP:conf/cav/BouajjaniEJQ18});


\item We provide a  comparison between synchronizable classes both for p2p and mailbox semantics (see Fig.~\ref{fig:diagram_p2p} for p2p systems and Fig.~\ref{fig:diagram_mailbox_all} for mailbox systems). In particular, we clarify the link between weakly synchronizable and existentially bounded systems for both p2p and mailbox systems, which was left open in \cite{DBLP:conf/cav/BouajjaniEJQ18} and  solved only for p2p systems in \cite[Theorem 7]{DBLP:journals/corr/abs-1901-09606} where  weakly synchronizable systems are shown to be included into existentially bounded ones when considering executions (and not MSCs as in our case).
\end{itemize}

%\noindent {\bf Outline.}
\paragraph*{Outline.}
Section 2 defines some preliminary notions such as p2p/mailbox message sequence charts (MSCs), and communicating systems. Section 3 presents the unifying MSO framework and two general theorems on $k$-synchro\-nizability and model checking. %In Section 4, we apply the MSO framework to different existing definitions of synchronizability including existentially bounded systems, and we introduce a new decidable one.
In Section 4, we apply the MSO framework to different existing definitions of synchronizability, and we introduce a new decidable one. Section 5 studies the relations between the classes. In Section 6, we conclude with some final remarks. Due to space constraints, some proofs are given in the appendix.
%An Appendix with additional material and proofs is added for the reviewer convenience.






\section{Preliminaries}

%!TEX root = concur2021.tex


\subsection{Message Sequence Charts}

Assume a finite set of processes $\Procs$ and a finite set of messages $\Msg$.
The set of (\pp) channels is $\Ch = \{(p,q) \in \Procs \times \Procs \mid p \neq q\}$.
%
A send action is of the form $\sact{p}{q}{\msg}$
where $(p,q) \in \Ch$ and $\msg \in \Msg$.
It is executed by $p$ and sends message $\msg$ to $q$.
The corresponding receive action, executed by $q$, is
$\ract{p}{q}{\msg}$.
%
For $(p,q) \in \Ch$, let
$\pqsAct{p}{q} = \{\sact{p}{q}{\msg} \mid \msg \in \Msg\}$ and
$\pqrAct{p}{q} = \{\ract{p}{q}{\msg} \mid \msg \in \Msg\}$.
For $p \in \Procs$, we set
$\psAct{p} = \{\sact{p}{q}{\msg} \mid q \in \Procs \setminus \{p\}$ and $\msg \in \Msg\}$, etc.
Moreover, $\pAct{p} = \psAct{p} \cup \qrAct{p}$ will denote the set of all actions that are
executed by $p$.
Finally, $\Act = \bigcup_{p \in \Procs} \pAct{p}$
is the set of all the actions.

\paragraph*{Peer-to-peer MSCs.}
%\alain{add an example of MSC and of system and illustrate notions with it}
A \emph{\pp MSC} (or simply \emph{MSC}) over $\Procs$ and $\Msg$ is a tuple $\msc = (\Events,\procrel,\lhd,\lambda)$
where $\Events$ is a finite (possibly empty) set of \emph{events}
and $\lambda: \Events \to \Act$ is a labeling function.
For $p \in \Procs$, let $\Events_p = \{e \in \Events \mid \lambda(e) \in \pAct{p}\}$ be the set of events
that are executed by $p$.
We require that $\procrel$ (the \emph{process relation}) is the disjoint union $\bigcup_{p \in \Procs} \procrel_p$
of relations ${\procrel_p} \subseteq \Events_p \times \Events_p$ such that
$\procrel_p$ is the direct successor relation of a total order on $\Events_p$.
For an event $e \in \Events$, a set of actions $A \subseteq \Act$, and a relation $\rel \subseteq \Events \times \Events$,
let $\sametype{e}{A}{\rel} = |\{f \in \Events \mid (f,e) \in \rel$ and $\lambda(f) \in A\}|$.
We require that ${\lhd} \subseteq \Events \times \Events$ (the \emph{message relation}) satisfies the following:
\begin{itemize}\itemsep=0.5ex
\item[(1)] for every pair $(e,f) \in {\lhd}$, there is a send action $\sact{p}{q}{\msg} \in \Act$ such that
$\lambda(e) = \sact{p}{q}{\msg}$, $\lambda(f) = \ract{p}{q}{\msg}$, and
$\sametype{e}{\pqsAct{p}{q}}{\procrel^+} = \sametype{f}{\pqrAct{p}{q}}{\procrel^+}$,
\item[(2)] for all $f \in \Events$ such that $\lambda(f)$ is a receive action, there is $e \in \Events$ such that $e \lhd f$.
\end{itemize}
Finally, letting ${\le}_\msc = ({\procrel} \cup {\lhd})^\ast$,
we require that $\le_\msc$ is a partial order.
%We may also simply write ${\le}$ instead of $\le_{\msc}$ if
%the MSC is clear from the context.

\medskip

Condition (1) above ensures that every (p2p) channel $(p,q)$ behaves in a FIFO manner.
By Condition (2), every receive event has a matching send event.
Note that, however, there may be unmatched send events in an MSC.
We let
$\SendEv{\msc} = \{e \in \Events \mid \lambda(e)$ is a send
action$\}$,
$\RecEv{\msc} = \{e \in \Events \mid \lambda(e)$ is a receive
action$\}$,
$\Matched{\msc} = \{e \in \Events \mid$ there is $f \in \Events$
such that $e \lhd f\}$, and
$\Unm{\msc} = \{e \in \Events \mid \lambda(e)$ is a send
action and there is no $f \in \Events$ such that $e \lhd f\}$.
%
We do not distinguish isomorphic MSCs and
let $\ppMSCs$ be the set of all MSCs over the given sets $\Procs$ and $\Msg$.
% For readability, and if there is no ambiguity,
% $\lambda^{-1}(\send{p}{q}{\msg})$, resp. $\lambda^{-1}(\rec{p}{q}{\msg})$,
% will be written $\ssymb(p,q,\msg)$ and, resp., $\rsymb(p,q,\msg)$ in the examples.

\bigskip

\noindent \begin{minipage}[c]{10.5cm}
  \begin{example}\label{ex:msc}
    For a set of processes $\procSet = \{p,q,r\}$ and a set of messages $\paylodSet = \{\msg_1, \msg_2, \msg_3, \msg_4 \}$,
    $\mscweakuniver = (\Events, \rightarrow, \lhd, \lambda)$
    %in Fig.~\ref{fig:msc_weak_univer}
    is an MSC where, for example,
  $e_2 \lhd e_2'$ and $e_3' \rightarrow e_4$.
  The dashed arrow means that the send event $e_1$ does not have
  a matching receive, so $e_1 \in Unm(\mscweakuniver)$.
  Moreover, $e_2 \le_{\mscweakuniver} e_4$, but
  $e_1 \not\le_{\mscweakuniver} e_4$.
We can find a total order ${\pplin} \supseteq {\le}_{\mscweakuniver}$
  such that $e_1 \pplin e_2 \pplin e_2' \pplin e_3
  \pplin e_3' \pplin e_4 \pplin e_4'$. We call $\pplin$ a linearization,
  which is formally defined  below.
  \end{example}
\end{minipage}
\hfill
\begin{minipage}[c]{3cm}

%\begin{figure}
  \begin{center}
    \begin{tikzpicture}[>=stealth,node distance=3.2cm,shorten >=1pt,
      every state/.style={text=black, scale =0.8}, semithick,
      font={\fontsize{8pt}{12}\selectfont}]
	\begin{scope}[xshift = 9cm, scale = 0.8]
		\node at (-0.3, -0.65)  (e)    {$e_1$};
		\node at (1.3, -1.2)  (e)    {$e_2$};
		\node at (-0.3, -1.2)  (e)    {$e_2'$};
		\node at (0.7, -1.85)  (e)    {$e_3$};
		\node at (2.3, -1.85)  (e)    {$e_3'$};
		\node at (2.3, -2.4)  (e)    {$e_4$};
		\node at (0.7, -2.4)  (e)    {$e_4'$};
		%MACHINES
		\draw (0,0) node{$p$} ;
		\draw (1,0) node{$q$} ;
		\draw (2,0) node{$r$} ;
		\draw (0,-0.2) -- (0,-2.8) ;
		\draw (1,-0.2) -- (1,-2.8);
		\draw (2, -0.2) -- (2, -2.8) ;
		%MESSAGES
		\draw[>=latex,->, dashed] (0,-0.65) -- (1, -0.65) node[midway,above]{$\amessage_1$};

		\draw[>=latex,->] (1, -1.2) -- (0, -1.2) node[midway, above] {$\amessage_2$};

		\draw[>=latex,->] (1,-1.85) -- (2,-1.85) node[midway, above] {$\amessage_3$};

		\draw[>=latex,->] (2,-2.4) -- (1,-2.4) node[midway,above] {$\amessage_4$};
	\end{scope}

\end{tikzpicture}
\captionof{figure}{MSC $\mscweakuniver$}
\label{fig:msc_weak_univer}
\end{center}
%\alain{NIce}
%\end{figure}

\end{minipage}




\paragraph*{Mailbox MSCs.}

For an MSC $\msc = (\Events,\procrel,\lhd,\lambda)$, we define
an additional binary relation that represents a constraint
under the mailbox semantics, where each process has only one incoming channel.
Let ${\mbrel}_\msc \subseteq \Events \times \Events$
be defined by: $e_1 \mbrel_\msc e_2$ if there is $q \in \Procs$
such that $\lambda(e_1) \in \qsAct{q}$,
$\lambda(e_2) \in \qsAct{q}$, and one of the following holds:
\begin{itemize}\itemsep=0.5ex
\item $e_1 \in \Matched{\msc}$ and $e_2 \in \Unm{\msc}$, or
\item $e_1 \lhd f_1$ and $e_2 \lhd f_2$ for some $f_1,f_2 \in \Events_q$ such that $f_1 \procrel^+ f_2$.
\end{itemize}

We let ${\preceq_\msc} = ({\procrel} \,\cup\, {\lhd} \,\cup\, {\mbrel_\msc})^\ast$.
Note that ${\le_\msc} \subseteq {\preceq_\msc}$.
%
%\begin{definition}\label{def:mailbox-msc}
We call $\msc \in \ppMSCs$ a \emph{mailbox MSC}
if ${\preceq_\msc}$ is a partial order.
%\end{definition}
Intuitively, this means that events can be scheduled in a way that corresponds
to the mailbox semantics, i.e., with one incoming channel per process.
Following the terminology in \cite{DBLP:conf/cav/BouajjaniEJQ18}, we also say that
a mailbox MSC satisfies \emph{causal delivery}.
The set of mailbox MSCs $\msc \in \ppMSCs$ is denoted by $\cdMSCs$.

\begin{example}\label{ex:mailbox-msc}
  MSC $\mscweakuniver$ is a mailbox MSC. Indeed, even though the order $\linrel$ defined in Example~\ref{ex:msc} does not respect all mailbox constraints, particularly the fact that $e_4 \mbrel_{\mscweakuniver} e_1$, there is a total order $ {\mblin} \supseteq {\preceq_{\mscweakuniver}}$ such that $
  e_2 \mblin e_3 \mblin e_3' \mblin e_4 \mblin e_1 \mblin e_2' \mblin e_4'$. We call $\mblin$ a mailbox linearization, which is formally defined below.
\end{example}

\paragraph*{Linearizations, Prefixes, and Concatenation.}

Consider $\msc = (\Events,\procrel,\lhd,\lambda) \in \MSCs$.
A \emph{\pp linearization} (or simply \emph{linearization}) of $\msc$ is a (reflexive) total order
${\linrel} \subseteq \Events \times \Events$ such that ${\le_\msc} \subseteq
{\linrel}$. Similarly,
a \emph{mailbox linearization} of $\msc$ is a total order
${\linrel} \subseteq \Events \times \Events$ such that ${\preceq_\msc} \subseteq
{\linrel}$. That is, every mailbox linearization is a \pp linearization,
but the converse is not necessarily true (Example~\ref{ex:mailbox-msc}).
Note that an MSC is a mailbox MSC iff it has at least one mailbox linearization.

\medskip

Let $\msc = (\Events,\procrel,\lhd,\lambda) \in \MSCs$ and consider
$E \subseteq \Events$ such that $E$ is ${\le_\msc}$-\emph{downward-closed}, i.e,
for all $(e,f) \in {\le_\msc}$ such that $f \in E$, we also have $e \in E$.
Then, the MSC $(E,{\procrel} \cap (E \times E),{\lhd} \cap (E \times E),\lambda')$,
where $\lambda'$ is the restriction of $\Events$ to $E$, is called a \emph{prefix}
of $\msc$. In particular, the empty MSC is a prefix of $\msc$.
We denote the set of prefixes of $\msc$ by $\Pref{\msc}$.
This is extended to sets $L \subseteq \MSCs$ as expected, letting
$\Pref{L} = \bigcup_{\msc \in L} \Pref{\msc}$.

\begin{restatable}{lemma}{prefixmailbox}
\label{lem:mb-prefix}
Every prefix of a mailbox MSC is a mailbox MSC.
\end{restatable}

Let $\msc_1 = (\Events_1,\procrel_1,\lhd_1,\lambda_1)$ and
$\msc_2 = (\Events_2,\procrel_2,\lhd_2,\lambda_2)$ be two MSCs.
Their \emph{concatenation} $\msc_1 \cdot \msc_2 = (\Events,\procrel,\lhd,\lambda)$ is defined if, for all $(p,q) \in \Ch$,
$e_1 \in \Unm{\msc_1}$, and
$e_2 \in \Events_2$ such that $\lambda(e_1) \in \pqsAct{p}{q}$
and $\lambda(e_2) \in \pqsAct{p}{q}$,
we have $e_2 \in \Unm{\msc_2}$.
As expected, $\Events$ is the disjoint union of $\Events_1$ and $\Events_2$,
${\lhd}  = {\lhd_1} \cup {\lhd_2}$, $\lambda$ is the ``union'' of $\lambda_1$
and $\lambda_2$, and ${\procrel} = {\procrel_1} \cup {\procrel_2} \cup R$.
Here, $R$ contains, for all $p \in \Procs$ such that $(\Events_1)_p$ and
$(\Events_2)_p$ are non-empty, the pair $(e_1,e_2)$ where $e_1$ is the
maximal $p$-event in $M_1$ and $e_2$ is the minimal $p$-event in $M_2$.
Note that $\msc_1 \cdot \msc_2$ is indeed an MSC and that
concatenation is associative.


\subsection{Communicating Systems}

We now recall the definition of communicating systems (aka communicating finite-state
machines or message-passing automata), which consist of finite-state machines $A_p$
(one for every process $p \in \Procs$) that can communicate through the FIFO channels
from $\Ch$.

\begin{definition}\label{def:cs}
A \emph{communicating system} over $\Procs$ and $\Msg$ is a tuple
   $ \Sys = (A_p)_{p\in\procSet}$. For each
   $p \in \Procs$, $A_p = (Loc_p, \delta_p, \ell^0_p)$ is a finite transition system where
   $\Loc_p$ is a finite set of local (control) states, $\delta_p
   \subseteq \Loc_p \times \pAct{p} \times \Loc_p$ is the
   transition relation, and $\ell^0_p \in Loc_p$ is the initial state.
\end{definition}

Given $p \in \Procs$ and a transition $t = (\ell,a,\ell') \in \delta_p$, we let
$\tsource(t) = \ell$, $\ttarget(t) = \ell'$, $\tlabel(t) = a$, and
$\tmessage(t) = \msg$ if $a \in \msAct{\msg} \cup \mrAct{\msg}$.

\smallskip

There are in general two ways to define the semantics of a communicating system.
Most often it is defined as a global infinite transition system that keeps track
of the various local control states and all (unbounded) channel contents.
As, in this paper, our arguments are based on a graph view of MSCs, we will define
the language of $\Sys$ directly as a set of MSCs. These two semantic views are essentially
equivalent, but they have different advantages depending on the context.
We refer to \cite{CyriacG14} for a thorough discussion.

Let $\msc = (\Events,\procrel,\lhd,\lambda)$ be an MSC.
A \emph{run} of $\Sys$ on $\msc$ is a mapping
$\rho: \Events \to \bigcup_{p \in \Procs} \delta_p$
that assigns to every event $e$ the transition $\rho(e)$
that is executed at $e$. Thus, we require that
\begin{enumerate*}[label={(\roman*)}]
\item for all $e \in \Events$, we have $\tlabel(\rho(e)) = \lambda(e)$,
\item for all $(e,f) \in {\procrel}$, $\ttarget(\rho(e)) = \tsource(\rho(f))$,
\item for all $(e,f) \in {\lhd}$, $\tmessage(\rho(e)) = \tmessage(\rho(f))$,
and
\item for all $p \in \Procs$ and $e \in \Events_p$ such that there is no $f \in \Events$ with $f \procrel e$, we have $\tsource(\rho(e)) = \ell_p^0$.
\end{enumerate*}

Letting run $\Sys$ directly on MSCs is actually very convenient.
This allows us to associate with $\Sys$ its p2p language and mailbox language
in one go. The \emph{\pp language} of $\Sys$ is $\ppL{\Sys} = \{\msc \in \ppMSCs \mid$ there is a run of $\Sys$ on $\msc\}$.
The \emph{mailbox language} of $\Sys$ is $\mbL{\Sys} = \{\msc \in \mbMSCs \mid$ there is a run of $\Sys$ on $\msc\}$.

Note that, following \cite{DBLP:conf/cav/BouajjaniEJQ18,DBLP:conf/fossacs/GiustoLL20},
we do not consider final states or final configurations, as our purpose is to
reason about all possible
traces that can be \emph{generated} by $\Sys$.
%We will discuss this issue in more detail later in the paper. \todo{Do we still discuss this in the conference version or can we omit this sentence?}
The next lemma is obvious for the p2p semantics and follows from Lemma~\ref{lem:mb-prefix} for
	the mailbox semantics.

\begin{restatable}{lemma}{prefixclosed}
For all $\comsymb \in \{\ppsymb, \mbsymb\}$, $\cL{\Sys}$ is prefix-closed:
$\Pref{\cL{\Sys}} \subseteq \cL{\Sys}$.
\end{restatable}

\begin{example}
Fig.~\ref{fig:system_weak_univer} depicts $\systemweakuniver = (A_p, A_q, A_r)$ such that MSC $\mscweakuniver$ in Fig.~\ref{fig:msc_weak_univer} belongs to $\ppL{\systemweakuniver}$ and to $\mbL{\systemweakuniver}$.
There is a unique run $\rho$ of $\systemweakuniver$ on $\mscweakuniver$.
We can see that $(e_3',e_4) \in {\rightarrow}$ and $\ttarget(\rho(e_3')) = \tsource(\rho(e_4)) = \ell_r^{1}$, $(e_2, e_2') \in \lhd_{\mscweakuniver}$, and $\tmessage(\rho(e_2)) = \tmessage(\rho(e_2')) = \msg_2$.
\end{example}
\begin{figure}[t]
\begin{center}
  \begin{tikzpicture}[>=stealth,node distance=3.2cm,shorten >=1pt,
    every state/.style={text=black, scale =0.75}, semithick,
    font={\fontsize{8pt}{12}\selectfont},
    scale = 0.9
    ]
  \begin{scope}[->]
      \node[state,initial,initial text={}] (q0)  {$\ell_p^{0}$};
      \node[state, right of=q0] (q1)  {$\ell_p^{1}$};
			\node[state, right of=q1] (q2) {$\ell_p^{2}$};

    	\path (q0) edge node [above] {$\send{p}{q}{\msg_1}$} (q1);
			\path (q1) edge node [above] {$\rec{q}{p}{\msg_2}$}(q2);
    	\node[thick] at (-1.1,0) {$A_p$};
  \end{scope}

  \begin{scope}[->, shift={(7.5,0)}]
      \node[state,initial,initial text={}] (q0)  {$\ell_q^{0}$};
			\node[state, right of=q0] (q1)  {$\ell_q^{1}$};
			\node[state, below of=q1, node distance = 1.5cm] (q2) {$\ell_q^{2}$};
			\node[state, left of=q2] (q3)  {$\ell_q^{3}$};

			\path (q0) edge node [above] {$\send{q}{p}{\msg_2}$} (q1);
			\path (q1) edge node [right] {$\send{q}{r}{\msg_3}$}(q2);
			\path (q2) edge node [above] {$\rec{r}{q}{\msg_4}$} (q3);
			\node[thick] at (-1.1,0) {$A_q$};
  \end{scope}

	\begin{scope}[->, shift={(0,-1.25)} ]
      \node[state,initial,initial text={}] (q0)  {$\ell_r^{0}$};
			\node[state, right of=q0] (q1)  {$\ell_r^{1}$};
			\node[state, right of=q1] (q2) {$\ell_r^{2}$};

			\path (q0) edge node [above] {$\rec{q}{r}{\msg_3}$} (q1);
			\path (q1) edge node [above] {$\send{r}{q}{\msg_4}$}(q2);
			\node[thick] at (-1.1,0) {$A_r$};
  \end{scope}
\end{tikzpicture}
\captionof{figure}{System $\systemweakuniver$}
\label{fig:system_weak_univer}
\end{center}
\end{figure}


\subsection{Conflict Graph}

We now recall the notion of a conflict graph associated to an MSC defined in \cite{DBLP:conf/cav/BouajjaniEJQ18}. This graph is used to depict the causal dependencies between message exchanges.  Intuitively, we have a dependency whenever
two messages have a process in common. For instance, an $\xrightarrow{SS}$
dependency between message exchanges $v$ and $v'$ expresses the fact that
$v'$ has been sent after $v$, by the same process. This notion is of interest because it was seen in \cite{DBLP:conf/cav/BouajjaniEJQ18} that the notion of synchronizability in MSCs (which is studied in this paper) can be graphically characterized by the nature of the associated conflict graph.
It is defined in terms of linearizations
in \cite{DBLP:conf/fossacs/GiustoLL20}, but we equivalently express it
directly in terms of MSCs.

\newcommand{\type}{\tau}
\newcommand{\stype}{S}
\newcommand{\rtype}{R}
\newcommand{\mexch}{\mu}
\newcommand{\Edges}{\mathit{Edges}}
\newcommand{\Nodes}{\mathit{Nodes}}

For an MSC $\msc = (\Events, \rightarrow, \lhd, \lambda)$ and
$e \in \Events$, we define the type $\type(e) \in \{\stype,\rtype\}$ of $e$ by $\type(e) = \stype$ if $e \in \SendEv{\msc}$
and $\type(e) = \rtype$ if $e \in \RecEv{\msc}$.
Moreover, for $e \in \Unm{\msc}$, we let $\mexch(e) = e$,
and for $(e,e') \in \lhd$, we let $\mexch(e) = \mexch(e') = (e,e')$.


\begin{definition}[Conflict graph]
	The \emph{conflict graph} $\cgraph{\msc}$ of an MSC $\msc = (\Events, \rightarrow, \lhd, \lambda)$ is the labeled graph $(\Nodes, \Edges)$, with $\Edges \subseteq \Nodes \times \{\stype,\rtype\}^2 \times \Nodes$, defined by
	$\Nodes = {\lhd} \cup \Unm{\msc}$ and $\Edges = \{(\mu(e),\type(e)\type(f),\mu(f)) \mid (e,f) \in {\to^+}\}$.
In particular, a node of $\cgraph{\msc}$ is either a single unmatched send event or a message pair $(e,e') \in {\lhd}$.
\end{definition}


\section{Model Checking and Synchronizability}

In this section, we survey two classical decision problems
for communicating systems.
The first problem is the model-checking problem, in which
one checks whether a given system satisfies a given
specification. A canonical specification language for MSCs is
monadic second-order (MSO) logic.
However, model checking in full generality is undecidable.
A common approach is, therefore, to restrict the behavior of
the given system to MSCs of bounded (special) tree-width.
%
Next, we introduce MSO logic and special tree-width.

\subsection{Logic and Special Tree-Width}

\paragraph*{Monadic Second-Order Logic.}
The set of MSO formulas over MSCs (over $\Procs$ and $\Msg$) is given by the grammar
$
\phi ::= x \procrel y \mid x \lhd y \mid \lambda(x) = a \mid x = y \mid x \in X \mid \exists x.\phi \mid \exists X.\phi \mid \phi \vee \phi \mid \neg \phi
$,
where $a \in \Act$, $x$ and $y$ are first-order variables, interpreted as
events of an MSC, and $X$ is a second-order variable, interpreted
as a set of events. We assume that we have an infinite supply of variables,
and we use common abbreviations such as $\wedge$, $\forall$, etc.
The satisfaction relation is defined in the standard way and self-explanatory.
For example, the formula $\neg\exists x.(\bigvee_{a \in \sAct} \lambda(x) = a \;\wedge\; \neg \mathit{matched}(x))$
with $\mathit{matched}(x) = \exists y.x \lhd y$
says that there are no unmatched send events.
It is not satisfied by  MSC $\mscweakuniver$
of Fig.~\ref{fig:msc_weak_univer},
as message $\msg_1$ is not received,
but by $\mscstrongexist$ from Fig.~\ref{fig:msc_strong_exist}.

Given a sentence $\phi$, i.e., a formula without free variables,
we let $L(\phi)$ denote the set of (p2p) MSCs that satisfy $\phi$.
%
It is worth mentioning that the (reflexive) transitive closure of
a binary relation defined by an MSO formula with free variables $x$ and $y$,
such as $x \procrel y$, is MSO-definable so that the logic can freely
use formulas of the form $x \procrel^+ y$ or $x \le y$ (where $\le$
is interpreted as $\le_\msc$ for the given MSC $\msc$).
Therefore, the definition of a mailbox MSC can be readily translated into
the formula $\mbformula = \neg \exists x.\exists y.(\neg (x = y) \wedge x \preceq y \wedge y \preceq x)$ so that we have $L(\mbformula) = \mbMSCs$.
Here, $x \preceq y$ is obtained as the MSO-definable reflexive transitive closure of
the union of the MSO-definable relations $\procrel$, $\lhd$, and $\sqsubset$.
In particular, we may define $x \sqsubset y$ by :
\[
x \sqsubset y =
\displaystyle
\hspace{-1em}\bigvee_{\substack{q \in \Procs\\a,b \in \qsAct{q}}}\hspace{-1em}
\lambda(x) = a \;\wedge\; \lambda(y) = b
\wedge
\left(
\begin{array}{rl}
& \mathit{matched}(x) \wedge \neg \mathit{matched}(y)\\[1ex]
\vee & \exists x'.\exists y'. (x \lhd x' \;\wedge\; y \lhd y' \;\wedge\; x' \procrel^+ y')
\end{array}
\right).
\]


\paragraph*{Propositional Dynamic Logic (PDL).}

For better complexity, we also consider PDL with Loop and Converse, henceforth called LCPDL
(cf.\ \cite{DBLP:journals/corr/abs-1904-06942,BolligFG21,Streett81} for more details).
Its syntax is:
\begin{align*}
	&\Phi ::=  \Exists \sigma \mid \Phi \vee \Phi \mid \neg \Phi &\text{(sentence)} \\
	&\sigma ::= a \mid \sigma \vee \sigma \mid \neg \sigma \mid \langle \pi \rangle \sigma \mid \Loop{\pi} &\text{(event formula)} \\
	&\pi ::= {\to} \mid {\lhd} \mid \test{\sigma} \mid \jump \mid \pi + \pi \mid \pi \cdot \pi \mid \pi^* \mid \inv{\pi} &\text{(path formula)}
\end{align*}
where $a \in \Sigma$.
We use the symbol $\top $ to denote a tautology event formula (such as $a \vee \neg a$).
%, with the dual symbol $\bot = \neg \top$.
We describe the semantics for the logic in Fig.~\ref{fig:sem-lcpdl} (apart from the obvious cases).
A sentence $\Phi$ is evaluated wrt. an MSC $\msc = (\Events, \rightarrow, \lhd, \lambda)$.
An event formula $\sigma$ is
evaluated wrt.\ $\msc$ and an event $e \in \Events$ so that it defines a unary relation $\sem{\sigma}_\msc \subseteq \Events$. Finally, a path formula $\pi$ is evaluated over two events, and so it defines a binary relation $\llbracket \pi \rrbracket _\msc \subseteq \Events \times \Events$.
Finally, we let $L(\Phi) = \{\msc \in \MSCs \mid \msc \models \Phi\}$.
Note that every LCPDL-definable property is MSO-definable.
\begin{figure}[t]
\centering
\def\arraystretch{1.2}
$\begin{array}{ll}
M \models \Exists \sigma  \text{~ if ~} \sem{\sigma}_\msc \neq \emptyset &
\llbracket \to \rrbracket_\msc := {\to} \text{~~ and ~~} \llbracket \lhd \rrbracket_\msc := {\lhd}\\
\makebox[5.3em][l]{$\sem{a}_\msc$} := \{e \in \Events \mid \lambda(e) = a\} &
\makebox[5em][l]{$\sem{\test{\sigma}}_\msc$} := \{ (e,e) \mid e \in \llbracket \sigma \rrbracket _\msc\}\\
\makebox[5.3em][l]{$\sem{\langle \pi \rangle \sigma}_\msc$} := \{e \in \Events \mid \exists f \in \sem{\sigma}_\msc: (e,f) \in \sem{\pi}_\msc \}~~~~~ &
\makebox[5em][l]{$\llbracket \jump \rrbracket_\msc$} := \Events \times \Events\\
\makebox[5.3em][l]{$\sem{\Loop{\pi}}_\msc$} := \{e \in \Events \mid (e, e) \in \sem{\pi}_\msc\} &
\makebox[5em][l]{$\llbracket \pi_1 + \pi_2 \rrbracket_\msc$} := \llbracket \pi_1\rrbracket_\msc \cup \llbracket \pi_2 \rrbracket_\msc \\
\makebox[5.3em][l]{$\llbracket \inv{\pi} \rrbracket_\msc$} := \{ (e, f) \in \Events \times \Events \mid (f,e) \in \llbracket \pi \rrbracket_\msc\} &
\makebox[5em][l]{$\llbracket \pi^* \rrbracket_\msc$} := \bigcup_{n \in \mathbb{N}}\llbracket \pi \rrbracket_\msc^n\\
\multicolumn{2}{l}{\makebox[5.3em][l]{$\llbracket \pi_1 \cdot \pi_2 \rrbracket_\msc$} := \{(e,f) \in \Events \times \Events \mid \exists g \in \Events : (e,g) \in \llbracket \pi_1 \rrbracket_\msc \text{~and~} (g, f) \in \llbracket \pi_2 \rrbracket_\msc\}  }
\end{array}$
\caption{Semantics of LCPDL\label{fig:sem-lcpdl}}
\end{figure}

It can be seen below that the mailbox semantics can be readily translated into the LCPDL formula
$\mbFormula = \neg \Exists\, (\Loop{({\lhd} + {\procrel} + {\sqsubset})^+})$
such that $L(\mbFormula) = \mbMSCs$. Hereby, we let
\[{\sqsubset} = {\lhd} \cdot {\procrel^+} \cdot {\inv{\lhd}} ~~+ \sum_{\substack{q \in \Procs\\a,b \in \qsAct{q}}} \test{a} \cdot {\lhd} \cdot \jump \cdot \test{b \wedge \neg \langle \lhd \rangle \top}\,.\]

\paragraph*{Special Tree-Width.}


\emph{Special tree-width} \cite{Courcelle10},
is a graph measure that indicates how close
a graph is to a tree (we may also use classical
	\emph{tree-width} instead).
This or similar measures are commonly employed in verification. For instance, tree-width and split-width have been used in \cite{MadhusudanP11} and, respectively, \cite{DBLP:conf/concur/CyriacGK12,AiswaryaGK14} to reason about graph behaviors generated by pushdown and queue systems.
%Here we apply it to reason about MSCs.
There are several ways to define the special tree-width of an MSC.
We adopt the following game-based definition from \cite{DBLP:journals/corr/abs-1904-06942}.

Adam and Eve play a two-player turn based ``decomposition game''
whose positions
are MSCs with some pebbles placed on some events.
More precisely, Eve's positions are
\emph{marked MSC fragments} $(M, U)$, where
$\msc = (\Events, \procrel, \lhd, \lambda)$
is an \emph{MSC fragment} (an MSC with possibly some edges from
$\lhd$ or $\to$ removed)
 and $U \subseteq \Events$ is the subset of marked events.
Adam's positions are pairs of marked MSC fragments.
A move by Eve consists in the following steps:
\begin{enumerate}
	\item marking some events of the MSC resulting in $(M, U')$ with $U \subseteq U' \subseteq \Events$,
	\item removing (process and/or message) edges whose endpoints are marked,
	\item dividing $(M, U)$ in $(M_1, U_1)$ and $(M_2, U_2)$ such that $M$ is the disjoint (unconnected) union of $M_1$ and $M_2$
	and marked nodes are inherited.
\end{enumerate}
When it is Adam's turn, he simply chooses one of the two marked MSC fragments.
The initial position is $(\msc,\emptyset)$ where $M$ is the (complete) MSC at hand. A terminal position is any position belonging to Eve such that all events are marked.
%
For $k \in \N$, we say that the game is $k$-winning for Eve if she has a (positional) strategy that allows her,
starting in the initial position and independently of Adam's moves, to reach a terminal position such that, in every single position visited along the play, there are at most $k+1$ marked events.

\newcommand{\CS}[2]{\mathsf{CS}_{(#1,#2)}}
\newcommand{\MSO}[2]{\mathsf{MSO}_{(#1,#2)}}
\newcommand{\LCPDL}[2]{\mathsf{LCPDL}_{(#1,#2)}}
\newcommand{\MSCpm}[2]{\mathsf{MSC}_{(#1,#2)}}
\newcommand{\mbMSCpm}[2]{\mathsf{MSC}_{(#1,#2)}^{\mathsf{mb}}}

\begin{fact}[\!\!\cite{DBLP:journals/corr/abs-1904-06942}]
	The special tree-width of an MSC is the least $k$ such that
	the associated game is $k$-winning for Eve.
\end{fact}

The set of MSCs whose special tree-width is at most $k$ is denoted by $\stwMSCs{k}$.


\subsection{Model Checking}

In general, even simple verification problems, such
as control-state reachability, are undecidable for
communicating systems \cite{DBLP:journals/jacm/BrandZ83}.
However, they are decidable when we restrict to behaviors of
bounded special tree-width, which motivates the following
definition of a generic {\bf bounded model-checking problem} for $\comsymb \in \{\ppsymb, \mbsymb\}$:\\
%
{\bf Input:} Two finite sets $\Procs$ and $\Msg$, a communicating system $\System$, an MSO sentence $\phi$, and $k \in \N$ (given in unary).\\
%
{\bf Question:} Do we have $\cL{\Sys} \cap \stwMSCs{k} \subseteq L(\phi)$?


\begin{fact}[\!\!\cite{DBLP:journals/corr/abs-1904-06942}]\label{p2p}
The bounded model-checking problem for $\comsymb = \ppsymb$ is decidable.
When the formulas $\phi$ are from LCPDL, then the problem is solvable
in exponential time.
\end{fact}

Note that \cite{DBLP:journals/corr/abs-1904-06942} does not employ
the LCPDL modality $\jump$, but it can be integrated easily.
Using $\mbformula$ or $\mbFormula$, we obtain the corresponding result
for mailbox systems as a corollary:
% (cf.\ Appendix~\ref{app:mailbox} for the proof):

\begin{restatable}{theorem}{boundedmc}
\label{mailbox}
The bounded model-checking problem for $\comsymb =  \mbsymb$ is decidable.
When the formulas $\phi$ are from LCPDL, then the problem is solvable
in exponential time.
\end{restatable}

\subsection{Synchronizability}

The above model-checking approach is incomplete in the sense that
a positive answer does not imply correctness of the whole
system. The system may still produce behaviors of special tree-width greater than $k$
that violate the given property.
However, if we know that a system only generates
behaviors from a class whose special tree-width is bounded by $k$,
we can still conclude that the system is correct.

This motivates the \emph{synchronizability problem}.
Several notions of synchronizability have been introduced in the literature.
However, they all amount to asking whether all behaviors generated by
a given communicating system have a particular shape,
i.e., whether they are all included in a fixed (or given) set of MSCs $\Class$.
Thus, the synchronizability problem is essentially an inclusion problem,
namely $\ppL{\Sys} \subseteq \Class$ or $\mbL{\Sys} \subseteq \Class$.
%
We show that, for decidability, it is enough to have that $\Class$
is MSO-definable and special-tree-width-bounded (STW-bounded):
%
We call $\Class \subseteq \MSCs$
\begin{enumerate*}[label={(\roman*)}]
\item \emph{MSO-definable} if there is
an MSO-formula $\phi$ such that $L(\phi) = \Class$,
\item \emph{LCPDL-definable} if there is an
an LCPDL-formula $\Phi$ such that $L(\Phi) = \Class$,
\item \emph{STW-bounded} if there is $k \in \N$
such that $\Class \subseteq \stwMSCs{k}$.
%\item \emph{\nameclass} if it is MSO-definable and STW-bounded.
\end{enumerate*}

An important component of the decidability proof is the following lemma,
which shows that we can reduce synchronizability
wrt.\ an STW-bounded class to bounded model-checking.

\begin{restatable}{lemma}{lemcontinuous}\label{lem:continuous}
Let $\System$ be a communicating system, $\comsymb \in \{\ppsymb, \mbsymb\}$,
$k \in \N$, and $\Class \subseteq \stwMSCs{k}$.
Then, $\cL{\System} \subseteq \Class$ iff
$\cL{\System} \cap \stwMSCs{(k+2)} \subseteq \Class$.
\end{restatable}

The result follows from the following lemma.
%, whose proof is in Appendix~\ref{app:continuous2}.
Note that a similar property
was shown in \cite[Proposition~5.4]{GKM07} for the specific class of existentially
$k$-bounded MSCs.

\begin{restatable}{lemma}{lemcontinuoustwo}\label{lem:continuous2}
	Let $k \in \N$ and $\Class \subseteq \stwMSCs{k}$. For all
	$M \in \MSCs \setminus \Class$, we have
	$(\Pref{M} \cap \stwMSCs{(k+2)}) \setminus \Class \neq \emptyset$.
\end{restatable}


We now have all ingredients to state a generic decidability result
for synchronizability:

\begin{theorem}\label{thm:sync}
Fix finite sets $\Procs$ and $\Msg$.
Suppose $\comsymb \in \{\ppsymb, \mbsymb\}$ and let $\Class \subseteq \MSCs$ be an MSO-definable and STW-bounded class (over $\Procs$ and $\Msg$).
The following problem is decidable:
Given a communicating system $\System$, do we have $\cL{\System} \subseteq \Class$?
\end{theorem}

\begin{proof}
Consider the MSO-formula $\phi$ such that $L(\phi) = \Class$, and
let $k \in \N$ such that $\Class \subseteq \stwMSCs{k}$.
We have
$\cL{\System} \subseteq \Class
 \stackrel{\textup{Lemma~\ref{lem:continuous}}}{\Longleftrightarrow} \cL{\System} \cap \stwMSCs{(k+2)} \subseteq \Class
 \Longleftrightarrow \cL{\System} \cap \stwMSCs{(k+2)} \subseteq L(\phi)$.
The latter can be solved thanks to Fact~\ref{p2p} and Theorem~\ref{mailbox}.
\end{proof}

\begin{remark}
Note that, in some cases (cf.\ Section~\ref{sec:concrete-classes}),
$\Procs$ and $\Msg$ are part of the input
and the concrete class $\Class$ may be parameterized by a natural number
so that it is part of the input, too. Then, we need to be able to compute the
MSO formula characterizing the class as well as the bound on the special tree-width.
%This is also important for complexity-theoretic characterizations.
\end{remark}





\section{Application to Concrete Classes of Synchronizability}
\label{sec:concrete-classes}

In this section, we instantiate our general framework by specific classes. Table~\ref{table:summary} gives a summary of the results.

\begin{table}
	\centering
	\caption{Summary of the decidability of the synchronizability problem in various classes\label{table:summary}}
	\begin{tabular}{ |p{4.8cm}||p{3.5cm}|p{3.5cm} | }
		\hline
		& \hfil\textsc{Peer-to-Peer}& \hfil\textsc{Mailbox}\\
		\hline
		Weakly synchronous   & Undecidable [Thm.~\ref{thm:p2p-weak-sync}]   & EXPTIME [Thm.~\ref{thm:mailbox-weak-sync}] \\
		\hline
		Weakly $k$-synchronous &  \multicolumn{2}{c|} {Decidable \cite{DBLP:conf/cav/BouajjaniEJQ18,DBLP:conf/fossacs/GiustoLL20} and [Thm.~\ref{thm:weak-sync}]}  \\
		\hline
		Strongly $k$-synchronous &  \hfil---   &  Decidable [Thm.~\ref{thm:strong-sync}]  \\
		\hline
		Existentially $k$-p2p-bounded & \multicolumn{2}{c|} {Decidable \cite[Prop.~5.5]{GKM07}}\\
		\hline
		Existentially $k$-mailbox-bounded &  \hfil---   & Decidable [Prop.~\ref{prop:exist-k-mailbox-bounded}]\\
		\hline
	\end{tabular}
\end{table}


\subsection{A New General Class: Weakly Synchronous MSCs} \label{sec:weakly-sync}

We first introduce the class of weakly synchronous MSCs. This is a generalization of synchronous MSCs studied earlier, in \cite{DBLP:conf/cav/BouajjaniEJQ18, DBLP:conf/fossacs/GiustoLL20}, which we shall discuss later. We say an MSC is weakly synchronous if it is breakable into \emph{exchanges} where an exchange is an MSC that allows one to schedule all sends before all receives. Let us define this formally:

\begin{definition}[exchange]\label{def:weak-synchr}
Let $\msc = (\Events,\procrel,\lhd,\lambda)$ be an MSC.
We say that $\msc$ is an \emph{exchange} if
$\SendEv{\msc}$ is
a ${\le_\msc}$-downward-closed set.
\end{definition}




\begin{definition}[weakly synchronous]\label{def:weaksync-new}
We say that $\msc \in \MSCs$ is
\emph{weakly synchronous} if it is of the form
$\msc = \msc_1 \cdot \ldots \cdot \msc_n$
such that every $\msc_i$ is an exchange.
\end{definition}


\noindent
\begin{minipage}[c]{11.cm}
We use the term \emph{weakly} to distinguish
from variants introduced later.

\begin{example}\label{example:msc_W}
	Consider the MSC $\mscW$ in Fig.~\ref{fig:msc_W}. It is is weakly synchronous. Indeed, $\amessage_1$, $\amessage_2$, and $\amessage_5$ are independent and can be put alone in an exchange. Repetitions of $\amessage_3$ and $\amessage_4$ are interlaced, but they constitute an exchange, as we can do all sends and then all receptions.
\end{example}

An easy adaptation of a characterization from \cite{DBLP:conf/fossacs/GiustoLL20} yields
the following result for weakly synchronous MSCs:
\end{minipage}
\begin{minipage}[c]{3cm}
	\hspace*{1cm}
\begin{center}
  \begin{tikzpicture}[>=stealth,node distance=3.4cm,shorten >=1pt,
  every state/.style={text=black, scale =0.6}, semithick,
    font={\fontsize{8pt}{12}\selectfont}]

\begin{scope}[shift = {(8,0.75)}, scale = 0.7]
  %MACHINES
  \draw (0,1.25) node{$q$} ;
  \draw (1.6,1.25) node{$r$} ;
  \draw (-1.25,1.25) node{$p$} ;
  \draw (0,1) -- (0,-3.1) ;
  \draw (1.6,1) -- (1.6,-3.1);
  \draw (-1.25,1) -- (-1.25,-3.1);

  %MESSAGES
  \draw[>=latex,->, dashed] (-1.25, 0.5) -- (0, 0.5) node[ above, midway] {$\amessage_1$};
  \draw[>=latex,->] (0, 0) -- (-1.25, 0) node[ above, midway] {$\amessage_2$};


  \draw[>=latex,->] (0, -0.4) -- (1.6, -1.75) node[pos=0.1, sloped, above] {$\amessage_3$};
  \draw[>=latex,->] (0, -1.1) -- (1.6, -2.45) node[pos=0.05, sloped, above] {$\amessage_3$}; %{$\amessage_1'$};
  %\draw[>=latex,->, dashed] (0, -2.5) -- (1.25, -3.25) node[pos=0.55, sloped, above] {$\amessage_1''$};

  \draw[>=latex,->] (1.6, -0.4) -- (0, -1.75) node[pos=0.1, sloped, above] {$\amessage_4$};
  \draw[>=latex,->] (1.6, -1.1) -- (0, -2.45) node[pos=0.05, sloped, above] {$\amessage_4$}; %{$\amessage_2'$};
  %\node[rotate = 90, left]at (1.13, -0.65) {$\cdots$};
  %\node[rotate = -90, right]at (0.1, -0.65) {$\cdots$};

  \draw[>=latex,->] (1.6, -2.6) -- (0, -2.6) node[ below, midway] {$\amessage_5$};


\end{scope}

\end{tikzpicture}
\captionof{figure}{MSC $\mscW$}
\label{fig:msc_W}
\end{center}
\end{minipage}



\begin{restatable}{proposition}{newweaklogiccg}
\label{prop:newweaklogiccg}
Let $\msc$ be an MSC. Then, $\msc$ is weakly synchronous iff no RS edge occurs on any cyclic path in the conflict graph $\cgraph{\msc}$.
\end{restatable}


It is easily seen that the characterization from
Proposition~\ref{prop:newweaklogiccg} is LCPDL-definable:

\begin{restatable}{corollary}{weaksynclcpdl}
\label{cor:weak-sync-lcpdl}
The sets of weakly synchronous MSCs and weakly synchronous \emph{mailbox} MSCs are LCPDL-definable.
Both formulas have polynomial size.
\end{restatable}

Moreover, under the mailbox semantics, we can show:

\begin{restatable}{proposition}{newweaklogicstw}
	\label{prop:new-weak-logic-bounded}
The set of weakly synchronous mailbox MSCs is
STW-bounded (in fact, it is included in $\stwMSCs{4|\Procs|}$).
\end{restatable}


\begin{proof}
Let $\msc$ be fixed, and let us sketch Eve's winning strategy.
Let $n = |\Procs|$.

The first step for Eve is to split $\msc$ in exchanges. She first disconnects
the first exchange from the rest of the graph ($2n$ pebbles are needed),
then she disconnects the second exchange from the rest of the graph ($2n$ pebbles needed, plus $n$ pebbles remaining from the first round), and so on for
each exchange.

So we are left with designing a winning strategy for Eve with $4n+1$ pebbles
on the graph of an exchange $\msc_0$,
where initially there are (at most) $n$ pebbles
placed on the first event of each process and also (at most) $n$ pebbles placed
on the last event of each process. Eve also places (at most) $n$ pebbles on the last
send event of each process and also (at most) $n$ pebbles on the first receive event of
each process. Eve erases the (at most) $n$ $\procrel$-edges between the last send event and the first receive event.

We are now in a configuration that will be our invariant.

Let us fix a mailbox linearization of $\msc_0$
and let $e$ be the first send event in this linearization.
\begin{itemize}
\item if $e$ is an unmatched send of process $p$,
Eve places her last pebble on the next
send event of $p$ (if it exists), let us call it $e'$. Then Eve erases the
$\procrel$-edge $(e,e')$, and now $e$ is completely disconnected,
so it can be removed and the pebble can be taken back.
\item if $e\lhd e'$, with $e'$ a receive event of process $q$,
then due to the mailbox semantics $e'$ is the first receive event of $q$,
so it has a pebble placed on it. Eve removes the $\lhd$-edge between
$e$ and $e'$, then using the extra pebble she disconnects $e$ and places a
pebble on the $\procrel$-successor of $e$, then she also
disconnects $e'$ and places a pebble on the $\procrel$-successor of $e'$.
\end{itemize}
After that, we are back to our invariant, so we can repeat the same
strategy with the second send event of the linearization, and so on until
all edges have been erased.
\end{proof}

We obtain the following result as a corollary.
Note that it assumes the mailbox semantics.

\begin{theorem}\label{thm:mailbox-weak-sync}
The following problem is decidable in exponential time:
Given $\Procs$, $\Msg$, and a communicating system $\System$ (over $\Procs$ and $\Msg$), is every MSC in $\mbL{\System}$ weakly synchronous?
\end{theorem}

\begin{proof}
According to Corollary~\ref{cor:weak-sync-lcpdl},
we determine the LCPDL formula $\Phi_\mathsf{wsmb}$ such
that $L(\Phi_\mathsf{wsmb})$ is the set of weakly synchronous mailbox MSCs. Moreover,
recall from Proposition~\ref{prop:new-weak-logic-bounded} that
the special tree-width of all weakly synchronous mailbox MSCs is bounded by
$4|\Procs|$.
%
By Lemma~\ref{lem:continuous},
$\mbL{\System} \subseteq L(\Phi_\mathsf{wsmb})$ iff
$\mbL{\System} \cap \stwMSCs{(4|\Procs|+2)} \subseteq L(\Phi_\mathsf{wsmb})$.
The latter is an instance of the bounded model-checking problem.
As the length of $\Phi_\mathsf{wsmb}$ is polynomial in
$|\Procs|$, we obtain that the original problem is decidable
in exponential time by Theorem~\ref{mailbox}.
\end{proof}

For the same reasons, the model-checking problem for ``weakly
synchronous'' systems is decidable.
%
Interestingly, a reduction from Post's correspondence problem
shows that decidability fails when adopting the p2p semantics:

\begin{restatable}{theorem}{ppweaksynch}\label{thm:p2p-weak-sync}
The following problem is undecidable:
Given finite sets $\Procs$ and $\Msg$ as well as a communicating system $\System$,
is every MSC in $\ppL{\System}$ weakly synchronous?
\end{restatable}


\subsection{Weakly $k$-Synchronous MSCs}\label{sec:weakly-k}

This negative result for the p2p semantics motivates the study of other classes.
In fact, our framework captures several classes introduced in the literature.

\begin{definition}[$k$-exchange]\label{def:weak-synchr}
Let $\msc = (\Events,\procrel,\lhd,\lambda)$ be an MSC
and $k \in \N$.
We call $\msc$ a $k$-\emph{exchange} if
$\msc$ is an exchange and $|\SendEv{\msc}| \le k$.
\end{definition}


Let us now recall the definition
from \cite{DBLP:conf/cav/BouajjaniEJQ18,DBLP:conf/fossacs/GiustoLL20}, but (equivalently)
expressed directly in terms of MSCs rather than via \emph{executions}. It differs from the weakly synchronous MSCs in that here, we insist on constraining the number of messages sent per exchange to be at most $k$.

\begin{definition}[weakly $k$-synchronous]\label{def:weaksync}
Let $k \in \N$.
We say that $\msc \in \MSCs$ is
weakly $k$-synchronous if it is of the form
$\msc = \msc_1 \cdot \ldots \cdot \msc_n$
such that every $\msc_i$ is a $k$-exchange.
\end{definition}

\noindent \begin{minipage}[c]{10.5cm}
\begin{example}
MSC $\mscweakSexist$ in Fig.~\ref{fig:msc_weak_S_exist} is weakly $1$-synchronous, as it can be
decomposed  into three \kE{1}s (the decomposition is depicted by the
horizontal dashed lines). We remark that $\mscweakSexist \in
\mbMSCs$. Note that there is a p2p linearization that respects the decomposition.
On the other hand, a mailbox linearization needs to reorganize actions from different MSCs: the sending of
$\msg_3$ needs to be done before the sending of $\msg_1$. Note that $\mscweakuniver$ in
Fig.~\ref{fig:msc_weak_univer} is also weakly $1$-synchronous.
\end{example}
\end{minipage}
\hfill
\begin{minipage}[c]{3cm}
\begin{center}

\begin{tikzpicture}[>=stealth,node distance=3.4cm,shorten >=1pt,
    every state/.style={text=black, scale =0.7}, semithick,
    font={\fontsize{8pt}{12}\selectfont}, scale = 0.8]

  %MACHINES
  \draw (0,0) node{$p$} ;
  \draw (1,0) node{$q$} ;
  \draw (2,0) node{$r$} ;
  \draw (0,-0.25) -- (0,-3.1) ;
  \draw (1,-0.25) -- (1,-3.1);
  \draw (2, -0.25) -- (2, -3.1) ;
  %MESSAGES
  \draw[>=latex,->, dashed] (0,-0.75) -- (1, -0.75) node[midway,above]{$\amessage_1$};

  \draw[>=latex,->] (1, -1.75) -- (0, -1.75) node[midway, above] {$\amessage_2$};
  %\draw (0.5,-1.7) node{$\cdots$};
  %\draw[>=latex,->] (1, -2.25) -- (0, -2.25) node[midway, above] {$\amessage_2$};

  \draw[>=latex,->] (2,-2.75) -- (1,-2.75) node[midway, above] {$\amessage_3$};
%\end{scope}
  \draw[dashed] (-0.5,-1.25) -- (2.5,-1.25) ;
  \draw[dashed] (-0.5,-2.25) -- (2.5,-2.25) ;


\end{tikzpicture}
\captionof{figure}{MSC $\mscweakSexist$}
\label{fig:msc_weak_S_exist}

\end{center}
\end{minipage}

\medskip


\begin{restatable}{proposition}{weaklogicstw}
	\label{prop:weak-logic-bounded}
	Let $k \in \N$. The set of weakly $k$-synchronous p2p (mailbox, respectively) MSCs
	is effectively MSO-definable.
\end{restatable}


In fact, MSO-definability essentially follows from the following known theorem:

\begin{theorem}[\!\!\cite{DBLP:conf/fossacs/GiustoLL20}] \label{th:scccharactweak}
	Let $M$ be an MSC. Then, $M$ is weakly $k$-synchronous iff every SCC in its conflict graph
	$\cgraph{\msc}$ is of size at most $k$ and no RS edge occurs on any cyclic path.
\end{theorem}

This property is similar to the graphical characterization of weakly synchronous MSCs, except  for the condition that every SCC in the conflict graph is of size at most $k$.
Furthermore, it is easy to establish a bound on the special tree-width:

\begin{restatable}{proposition}{kweakstw}
\label{prop:kweakstw}
	Let $k \in \N$. The set of MSCs that are weakly $k$-synchronous have special tree-width bounded by $2k+|\Procs|$.
\end{restatable}

Hence, we can conclude that the class of weakly $k$-synchronous MSCs is MSO-definable and STW-bounded.
As a corollary, we get the following (known) decidability result, but via an alternative proof:

\begin{theorem}[\!\!\cite{DBLP:conf/cav/BouajjaniEJQ18,DBLP:conf/fossacs/GiustoLL20}]\label{thm:weak-sync}
For $\comsymb \in \{\ppsymb, \mbsymb\}$, the following problem is decidable:
Given finite sets $\Procs$ and $\Msg$, a communicating system $\System$, and $k \in \N$,
is every MSC in $\cL{\System}$ weakly $k$-synchronous?
\end{theorem}


\begin{proof}
We proceed similarly to the proof of Theorem~\ref{thm:mailbox-weak-sync}.
%
For the given $\Procs$, $\Msg$, and $k$, we first determine,
using Proposition~\ref{prop:weak-logic-bounded}, the MSO formula $\phi_k$ such
that $L(\phi_k)$ is the set of weakly $k$-synchronous p2p/mailbox MSCs. From
Proposition~\ref{prop:kweakstw}, we know that
the special tree-width of all weakly $k$-synchronous MSCs is bounded by
$2k + |\Procs|$.
%
By Lemma~\ref{lem:continuous}, we have
$\cL{\System} \subseteq L(\phi_k)$ iff
$\cL{\System} \cap \stwMSCs{(2k + |\Procs| + 2)} \subseteq L(\phi_k)$.
The latter is an instance of the bounded model-checking problem.
By Fact~\ref{p2p} and Theorem~\ref{mailbox}, we obtain decidability.
\end{proof}

\begin{remark}
The set of weakly $k$-synchronous MSCs is not directly expressible in LCPDL
(the reason is that LCPDL does not have a built-in counting mechanism).
However, its \emph{complement} is expressible in the extension of LCPDL with
existentially quantified propositions
(we need $k+1$ of them). The model-checking problem for
this kind of property is still in EXPTIME and, therefore, so is the problem from
Theorem~\ref{thm:weak-sync} when $k$ is given in unary. It is very likely that our approach can also be used to
infer the PSPACE upper bound from \cite{DBLP:conf/cav/BouajjaniEJQ18}
by showing bounded \emph{path width} and using finite word automata instead of tree automata.
Finally, note that
the problem to decide whether there exists an integer $k \in \N$ such that all MSCs in $\cL{\System}$
are weakly $k$-synchronous
has recently been studied in \cite{DLL2021} and requires different techniques.
\end{remark}

	Observe also that we can remove the constraint of all the sends preceding all the receives in a $k$-exchange, and still have decidability. We then have the following definition.

\begin{definition}[modified $k$-exchange]\label{def:mod-weak-synchr}
	Let $\msc = (\Events,\procrel,\lhd,\lambda)$ be an MSC
	and $k \in \N$.
	We call $\msc$ a \emph{modified} $k$-\emph{exchange} if $|\SendEv{\msc}| \le k$.
\end{definition}

We extend this notion to consider modified weakly $k$-synchronous executions as before, and the graphical characterization of this property is that there are at most $k$ nodes in every SCC of the conflict graph. Hence, this class is also MSO-definable, and since each modified $k$-exchange has at most $2k$ events, it also has bounded special tree-width. 


%\bigskip
%
%\noindent
%\begin{minipage}[c]{10cm}
%	\begin{example}
%	MSC $\msc_4$ in Fig.~\ref{fig:msc_W_S} is strongly synchronous. Indeed, repetitions of  messages $\amessage_1$ and $\amessage_2$ are interlaced and form an exchange. So, the division into exchanges is consistant with the partial order imposed by the mailbox communication. However, if we consider $\msc_2$ in Fig.~\ref{fig:msc_W}, which is weakly-synchronous, we can see that it is not strongly synchronous, as the decomposition as described in Example~\ref{example:msc_W}, does not correspond to the partial order: $\amessage_1$ has to be sent after $\amessage_5$ as this last one is matched and $\amessage_1$ is not (within mailbox communication).
%	\end{example}
%\end{minipage}
%\hfill
%\begin{minipage}[c]{3cm}
%\begin{center}
%  \begin{tikzpicture}[>=stealth,node distance=3.4cm,shorten >=1pt,
%  every state/.style={text=black, scale =0.7}, semithick,
%    font={\fontsize{8pt}{12}\selectfont}]
%
%\begin{scope}[shift = {(8,0.75)}, scale = 0.8]
%%	\draw (0.75, -4) node{\textbf{(b)}};
%  %MACHINES
%  \draw (0,0) node{$p$} ;
%  \draw (1.5,0) node{$q$} ;
%  \draw (0,-0.25) -- (0,-3.5) ;
%  \draw (1.5,-0.25) -- (1.5,-3.5);
%  %MESSAGES
%
%  \draw[>=latex,->] (0, -0.7) -- (1.5, -1.8) node[pos=0.2, sloped, above] {$\amessage_1$};
%  \draw[>=latex,->] (0, -1.4) -- (1.5, -2.5) node[pos=0.1, sloped, above] {$\amessage_1$}; %{$\amessage_1'$};
%  %\draw[>=latex,->, dashed] (0, -2.5) -- (1.25, -3.25) node[pos=0.55, sloped, above] {$\amessage_1''$};
%
%  \draw[>=latex,->] (1.5, -0.5) -- (0, -2.2) node[pos=0.1, sloped, above] {$\amessage_2$};
%  \draw[>=latex,->] (1.5, -1.2) -- (0, -2.9) node[pos=0.05, sloped, above] {$\amessage_2$};
%  % \node[rotate = 90, left]at (1.13, -0.65) {$\cdots$};
%  % \node[rotate = -90, right]at (0.1, -0.65) {$\cdots$};
%
%\end{scope}
%
%\end{tikzpicture}
%\captionof{figure}{MSC $\msc_4$}
%\label{fig:msc_W_S}
%\end{center}
%
%\end{minipage}
%
%

\subsection{Strongly $k$-Synchronous MSCs and Other Classes}\label{sec:strong-sync}
Our framework can be applied to a variety of other classes. Here we show how the decidability results can be shown for a variant of the class of weakly $k$-synchronous MSCs.

\begin{definition}\label{def:strongksync}
Let $\msc = (\Events,\procrel,\lhd,\lambda) \in \mbMSCs$.
We call $\msc$ \emph{strongly $k$-synchronous}
if it can be written as
$\msc = \msc_1 \cdot \ldots \cdot \msc_n$ %(with $\msc_i$ an MSC)
such that every MSC $\msc_i = (\Events_i,\procrel_i,\lhd_i,\lambda_i)$ is a $k$-exchange
and, for all $(e,f) \in {\sqsubset}_\msc$, there are $1 \le i \le j \le n$
such that $e \in \Events_i$ and $f \in \Events_j$.
%\benedikt{is this equivalent to the other definition?}
\end{definition}

\smallskip

\noindent\begin{minipage}[c]{10.5cm}
	\begin{example}
	MSC $\mscstrongexist \in \mbMSCs$ in Fig.~\ref{fig:msc_strong_exist}
%(from system $\System_4$ in Fig.~\ref{fig:system_strong_exist})
is
	strongly $1$-synchronous. Indeed, we can decompose it into \kE{1}s and this
	decomposition allows for a total order compatible with $\sqsubset_{\msc_4}$.
	Moreover, MSC $\mscweakSexist$ in Fig.~\ref{fig:msc_weak_S_exist}, which is
	weakly $1$-synchronous, is strongly $3$-synchronous. Indeed, we need
	to put the three messages in the same \kE{k} to regain our total order.
	Finally, for all $k$, MSC $\mscweakuniver$ in Fig.~\ref{fig:msc_weak_univer} is not
	strongly $k$-synchronous, as we cannot put all messages in the same \kE{k},
	where all sends are followed by all receptions.  Here, this is not possible as the reception of $\msg_3$ has to take place before the sending of
$\msg_4$.

\end{example}
\end{minipage}~~
\begin{minipage}[c]{3cm}
	\begin{center}

\begin{tikzpicture}[>=stealth,node distance=3.4cm,shorten >=1pt,
    every state/.style={text=black, scale =0.7}, semithick,
      font={\fontsize{8pt}{12}\selectfont}]

      \begin{scope}[shift = {(3.5,0.75)}, scale = 0.8]
           %\draw (0.5, -4) node{\textbf{(b)}};
          %MACHINES
          \draw (0,0) node{$p$} ;
          \draw (1,0) node{$q$} ;
          \draw (0,-0.25) -- (0,-3.5) ;
          \draw (1,-0.25) -- (1,-3.5);
          %MESSAGES

           \draw[>=latex,->] (0, -0.75) -- (1, -0.75) node[midway, above] {$\amessage_1$};
           \draw[>=latex,->] (0, -1.75) -- (1, -1.75) node[midway, above] {$\amessage_1$};
           \draw[>=latex,->] (0, -2.75) -- (1, -2.75) node[midway, above] {$\amessage_1$};

           %\draw (0.5, -3.25) node{$\cdots$};
      \end{scope}

\end{tikzpicture}
\captionof{figure}{MSC $\mscstrongexist$}
\label{fig:msc_strong_exist}
\end{center}

\end{minipage}
%We can see also that, even if $\msc_4$ in Fig.~\ref{fig:msc_W_S} is strongly synchronous, there is no $k$ such that all MSCs that can be done by the corresponding $\System_4$ (Fig.~\ref{fig:system_W_S}) are strongly $k$-synchronous. Indeed, the exchanges containing messages $\amessage_1$ and $\amessage_2$ have no bound, and can be as big as we want.

\smallskip

\begin{restatable}{proposition}{stronglogicstw}
	\label{prop:strong-logic-bounded}
For all $k \in \N$, the set of strongly $k$-synchronous mailbox MSCs
is MSO-definable and STW-bounded.%d, and continuous.
\end{restatable}

The proof proceeds similarly to what has been shown in the previous cases,
but MSO-definability now relies on an \emph{extended} conflict graph.
As a corollary, we thus obtain:

\begin{theorem}\label{thm:strong-sync}
The following problem is decidable:
Given finite sets $\Procs$ and $\Msg$, a communicating system $\System$, and $k \in \N$,
is every MSC in $\mbL{\System}$ strongly $k$-synchronous?
\end{theorem}

%\begin{proof}
%The proof follows exactly the same lines as that of Theorem~\ref{thm:weak-sync},
%but using Proposition~\ref{prop:strong-logic-bounded} instead of
%Proposition~\ref{prop:weak-logic-bounded}.
%\end{proof}


\begin{remark}
\label{rem:stronglykp2p}
	Only mailbox MSCs are considered for the definition of
	strongly $k$-synchronous MSCs
	for the following reason:
	A natural p2p analogue of Definition~\ref{def:strongksync}
	would require from the decomposition that, for all $(e,f) \in {\le}_\msc$, there are indices $1 \le i \le j \le n$
	such that $e \in \Events_i$ and $f \in \Events_j$. But this is always satisfied
	(cf.\ Appendix~\ref{app:stronglykp2p}). So the natural definition of ``strongly $k$-synchronous MSCs'' would coincide with weakly $k$-synchronous MSCs.
\end{remark}

Like the variant for the case of weakly synchronous MSCs, we can also generalize strongly $k$-synchronous MSCs by removing the restriction on the number of messages per exchange:
%We then have the following definition.

\begin{definition}\label{def:strongsync}
	Let $\msc = (\Events,\procrel,\lhd,\lambda) \in \mbMSCs$.
	We call $\msc$ \emph{strongly synchronous}
	if it can be written as
	$\msc = \msc_1 \cdot \ldots \cdot \msc_n$ %(with $\msc_i$ an MSC)
	such that every MSC $\msc_i = (\Events_i,\procrel_i,\lhd_i,\lambda_i)$ is an exchange
	and, for all $(e,f) \in {\sqsubset}_\msc$, there are indices $1 \le i \le j \le n$
	such that $e \in \Events_i$ and $f \in \Events_j$.
	%\benedikt{is this equivalent to the other definition?}
\end{definition}

Similarly to the constructions for strongly $k$-synchronous MSCs, we can obtain a graphical characterization where we only look for the absence of $RS$-edges in a cycle. Hence, this class is also MSO-definable (in fact, even LCPDL-definable) and STW-bounded.


%{\color{red}DEBUT AJOUT}

\subsection{Existentially $k$-Bounded MSCs}

Now, we turn to existentially $k$-bounded MSCs \cite{DBLP:conf/fossacs/LohreyM02,DBLP:conf/dlt/GenestMK04,GKM07}.
Synchronizability has been studied for the p2p case in \cite{GKM07}, so we only consider the mailbox case here.
A linearization $\linrel$ of an MSC $\msc = (\Events,\procrel,\lhd,\lambda) \in \MSCs$ is called
%$k$-\emph{\pp-bounded} if, for all $e \in \Matched{\msc}$, say with $\lambda(e) = \sact{p}{q}{\msg}$,
%\\ $\sametype{e}{\pqsAct{p}{q}}{\linrel} - \sametype{e}{\pqrAct{p}{q}}{\linrel} \le k\,,$
$k$-\emph{mailbox-bounded} if, for all $e \in \Matched{\msc}$, say with $\lambda(e) = \sact{p}{q}{\msg}$,
we have $\sametype{e}{\pqsAct{\plh}{q}}{\linrel} - \sametype{e}{\pqrAct{\plh}{q}}{\linrel} \le k\,.$


\begin{definition}%\label{def:}
Let $\msc = (\Events,\procrel,\lhd,\lambda) \in \MSCs$ and $k \in \N$.
We call $\msc$
%\item \emph{existentially $k$-\pp-bounded} if
%it has some \pp linearization that is $k$-\pp-bounded,
%\item
\emph{existentially $k$-mailbox-bounded} if
it has some mailbox linearization that is $k$-mailbox-bounded.
\end{definition}

\noindent
\begin{minipage}[c]{10.5cm}
  Note that every existentially $k$-mailbox-bounded MSC is a mailbox MSC.
	\begin{example}
MSC $\mscexist$ in Fig.~\ref{fig:msc_exist}
is %existentially $1$-p2p-bounded and
existentially $1$-mailbox-bounded, as witnessed by the (informally given)
linearization
%${\lin} \subseteq %{\le_{\msc_4}}$, ${\lin} \subseteq
%{\preceq_{\mscexist}}$ and
%$e_2 \lin e_1 \lin e_3 \lin e_3' \lin e_1' \lin e_1'' \lin e_2' \lin e_3'' \cdots$
$\ssymb(q,p,\msg_2) \lin \ssymb(p,q,\msg_1) \lin \ssymb(q,r,\msg_3) \lin \rsymb(q,r, \msg_3) \lin \rsymb(p,q, \msg_1) \lin
  \ssymb(p,q,\msg_1) \lin \rsymb(q,p, \msg_2) \lin \ssymb(q,r,\msg_3)
  %\lin \rsymb(q,r, \msg_3) \lin \ssymb(q,p,\msg_2) \lin \rsymb(p,q, \msg_1) \lin
\ldots $
Note that $\mscexist$ is neither weakly nor strongly synchronous as we cannot divide it into exchanges.
	\end{example}
\end{minipage}
\begin{minipage}[c]{3.5cm}
%  \vspace*{0.2cm}
	\begin{center}
  \begin{tikzpicture}[>=stealth,node distance=3.4cm,shorten >=1pt,
  every state/.style={text=black, scale =0.7}, semithick,
    font={\fontsize{8pt}{12}\selectfont}]
\begin{scope}[shift = {(0,0)}, scale = 0.8]
  % \draw (1.25, -4.75) node{\textbf{(a)}};
  % \draw (4.5, -4.75) node{\textbf(b)};

  %MACHINES
  \draw (0,-0.1) node{$p$} ;
  \draw (1.25,-0.1) node{$q$} ;
  \draw (2.5,-0.1) node{$r$} ;
  \draw (2.5, -0.35) -- (2.5, -3.4) ;
  \draw (0,-0.35) -- (0,-3.4) ;
  \draw (1.25,-0.35) -- (1.25,-3.4);
  %MESSAGES

  \draw[>=latex,->] (0, -0.5) -- (1.25, -1.25) node[pos=0.4, sloped, above] {$\amessage_1$};
  \draw[>=latex,->] (0, -1.5) -- (1.25, -2.25) node[pos=0.55, sloped, above] {$\amessage_1$};
  \draw[>=latex,->, dashed] (0, -2.5) -- (1.25, -3.25) node[pos=0.65, sloped, above] {$\amessage_1$};

  \draw[>=latex,->] (1.25, -0.75) -- (0, -2) node[pos=0.5, sloped, above] {$\amessage_2$};
  \draw[>=latex,->] (1.25, -1.75) -- (0, -3) node[pos=0.5, sloped, above] {$\amessage_2$};


  \draw[>=latex,->] (1.25, -1) -- (2.5, -1) node[midway, above] {$\amessage_3$};
  \draw[>=latex,->] (1.25, -1.5) -- (2.5, -1.5) node[midway, above] {$\amessage_3$};
  \draw[>=latex,->] (1.25, -2.5) -- (2.5, -2.5) node[midway, above] {$\amessage_3$};

  %\draw (0.6, -3.35) node{$\cdots$};
  %\draw (1.9, -3.35) node{$\cdots$};
\end{scope}
      % \begin{scope}[shift = {(3,0)}, scale = 0.8]
      %   % \draw (0.5, -4) node{\textbf{(b)}};
      %   %MACHINES
      %   \draw (0,0) node{$p$} ;
      %   \draw (1,0) node{$q$} ;
      %   \draw (0,-0.25) -- (0,-3.5) ;
      %   \draw (1,-0.25) -- (1,-3.5);
      %   %MESSAGES
      %
      %    \draw[>=latex,->] (0, -0.75) -- (1, -0.75) node[midway, above] {$\amessage_1$};
      %    \draw[>=latex,->] (1, -1.5) -- (0, -1.5) node[midway, above] {$\amessage_2$};
      %    \draw[>=latex,->] (0, -2.25) -- (1, -2.25) node[midway, above] {$\amessage_1$};
      %    \draw[>=latex,->] (1, -3) -- (0, -3) node[midway, above] {$\amessage_2$};
      %
      %
      %   % \draw (0.5, -3.25) node{$\cdots$};
      % \end{scope}
        \end{tikzpicture}
\captionof{figure}{MSC $\mscexist$}
\label{fig:msc_exist}
      \end{center}

\end{minipage}

%\medskip

% \begin{proposition}
% \label{prop:exists-k-p2p-bounded}
% For all $k \in \N$, the set of existentially $k$-\pp-bounded MSCs
% is MSO-definable and STW-bounded.
% \end{proposition}

% \begin{proof}
% The set of existentially $k$-\pp-bounded MSCs was shown to be MSO-definable
% (in fact, even FO-definable) in \cite{DBLP:journals/iandc/LohreyM04}. Note that there are minor differences
% in the definitions (in particular, the fact that we deal with unmatched messages),
% which, however, do not affect FO-definability.
% In \cite[Proposition 5.4, page 163]{DBLP:journals/corr/abs-1904-06942},\
% it was shown that their special tree-width is bounded by $k|\Procs|^2 + |\Procs|$.
% \end{proof}

% We obtain the following result as a corollary:
%
% \begin{theorem}\label{thm:exists-sync}
% For $\comsymb \in \{\ppsymb, \mbsymb\}$, the following problem is decidable:
% Given finite sets $\Procs$ and $\Msg$, a communicating system $\System$, and $k \in \N$,
% is every MSC in $\cL{\System}$ existentially $k$-\pp-bounded?
% \end{theorem}

% \begin{proof}
% Again, the proof follows exactly the same lines as that or Theorem~\ref{thm:weak-sync},
% now using Proposition~\ref{prop:exists-k-p2p-bounded}.
% \end{proof}
%
% Note that this is similar to the problem considered
% in \cite{GKM07,kuske2014communicating},
% though there is a subtle difference: in \cite{GKM07,kuske2014communicating},
% there are a notion of deadlock and distinguished final configurations.
%
% \ifx\islongversion\yes
% We define the following relation in order to characterize $k$-mailbox-bounded MSCs.
%
% Let $k\geq 1$, and let $\msc$ be a fixed mailbox MSC. Let $\revb$ be the binary relation among events of $\msc$ defined as follows: $r\revb s$ if
\begin{enumerate}
\item $r$ is a receive event of a process $p$;
\item let $r'$ be the $k$-th receive event of process $p$ after $r$; then
$s\lhd r'$.
\end{enumerate}

\begin{lemma}\label{lem:exists-k-mailbox-acyclicity-condition}
$\msc$ is existential k-mailbox-bounded if and only if $\preceq_M\cup\revb$ is
acyclic.
\end{lemma}

\begin{proof}
Assume that $\msc$ is existential k-mailbox-bounded. Let $\linrel$ be a mailbox
linearisation of $\msc$ such that
for all $e \in \Matched{\msc}$,
say with $\lambda(e) = \sact{p}{q}{\msg}$,
\[\sametype{e}{\pqsAct{-}{q}}{\linrel} - \sametype{e}{\pqrAct{-}{q}}{\linrel}
\leq k\,.\]
Then $\linrel$ is also a linearisation of $(\preceq_M\cup\revb)^*$. Indeed,
if it was not the case, there would be a pair of events $r,s$ such
that $r\revb s$ and $s\linrel r$. But then we would have
\[\sametype{s}{\pqsAct{-}{q}}{\linrel} - \sametype{s}{\pqrAct{-}{q}}{\linrel}
> k\,,\]
and the contradiction. So $\linrel$ is
a linearisation of $(\preceq_M\cup\revb)^*$ and $\preceq_M\cup\revb$ is
acyclic.

Conversely, assume that $\preceq_M\cup\revb$, and let
$\linrel$ be a linearisation of $(\preceq_M\cup\revb)^*$.
In particular, $\linrel$ is a mailbox linearisation of $\msc$.
Let us show that
for all $s \in \Matched{\msc}$,
say with $\lambda(s) = \sact{p}{q}{\msg}$,
\[\sametype{s}{\pqsAct{-}{q}}{\linrel} - \sametype{s}{\pqrAct{-}{q}}{\linrel}
\leq k\,.\]
Let $s\in\Matched{\msc}$ be fixed, and let $r'$ be such that
$s\lhd r'$. There are two cases:
\begin{itemize}
\item $\sametype{r'}{\pqrAct{-}{q}}{\procrel}\leq k$. Then
\[\sametype{s}{\pqsAct{-}{q}}{\linrel}\leq k\,,\] because all sends before $s$
are matched. So \[\sametype{s}{\pqsAct{-}{q}}{\linrel} - \sametype{s}{\pqrAct{-}{q}}{\linrel}
\leq k\,,\]
\item $\sametype{r'}{\pqrAct{-}{q}}{\procrel}\leq k$. Then there is
$r$ on process $q$ such that $r\revb s$. So $r\linrel s$, and
there are at most $k$ messages in the buffer of $q$ at the time of event $s$, or in other words,
\[\sametype{e}{\pqsAct{-}{q}}{\linrel} - \sametype{e}{\pqrAct{-}{q}}{\linrel}
\leq k\,.\]
\end{itemize}
So $\linrel$ is a mailbox linearisation with $k$ bounded buffers,
and $\msc$ is existential k-mailbox-bounded.
\end{proof}

% \fi
\begin{restatable}{proposition}{existkmailboxbounded}\label{prop:exist-k-mailbox-bounded}
	For all $k \in \N$, the set of existentially $k$-mailbox-bounded MSCs
	is MSO-definable and STW-bounded.
\end{restatable}
% \ifx\islongversion\yes
%
% 
\begin{proof}
Let $k\geq 1$ be fixed.
Since every existentially k-mailbox-bounded MSCs is also
existentially k-p2p-bounded, and since the class of existentially k-p2p-bounded MSCs is STW bounded (cf Proposition~\ref{prop:exists-k-p2p-bounded}),
the class of existentially k-mailbox-bounded MSCs is also STW bounded.

Let us show that it is moreover MSO definable.
%\comEtienne{we could also prove that it is FO definable with the same trick as in the p2p case. Maybe it is important complexity wise?}

By Lemma~\ref{lem:exists-k-mailbox-acyclicity-condition}, it is
enough to show that the acyclicity of $\preceq_M\cup\revb$ is
MSO definable, and since $\preceq_M$ was already shown MSO definable
and acyclicity is easily MSO definable, it is enough to show
that $\revb$ is MSO definable. It is indeed the case, as demonstrated by
this formula
$$
\phi(r,s)=\exists r_1,r_2,\dots,r_n. r\procrel r_1\procrel r_2\procrel\dots
\procrel r_n \wedge s\lhd r_n.
$$

Finally, let us show that existentially k-mailbox-bounded is also LCPDL
definable. This follows from the following formulas:

$$
\begin{array}{rl}
\prec_M= & (\lhd+\procrel)^+
\\
R = & \langle\lhd^{-1}\rangle\top
\\
\xrightarrow{\tiny{\mbox{next R}}} = & (\procrel\wedge\test{\neg R})^*\cdot (\procrel\wedge \test{R})
\\
\revb= & (\xrightarrow{\tiny{\mbox{next R}}})^k.(\lhd)^{-1}.
\\
\Phi_{\exists k~\mbox{mb-bounded}}=&\neg\Exists\Loop{(\prec_M+\revb)^+}
\end{array}
$$

\end{proof}

% \fi

This extension is also valid for the \pp definition of existentially $k$-bounded MSCs, which
were addressed in \cite{GKM07}. Finally, our framework can also be adapted to treat universally bounded systems \cite{HENRIKSEN20051,DBLP:conf/fossacs/LohreyM02}. Those extensions and the missing proofs are available in Appendix~\ref{appendix:section4}.

% \begin{definition}
% 	Let $\msc = (\Events, \rightarrow, \lhd, \lambda)$ and $k \in \mathbb{N}$. We call $\msc$ universally $k$-\pp-bounded (resp., universally $k$-mailbox-bounded) if every p2p (resp., mailbox) linearization ${\lin} \subseteq \Events \times \Events$
% is $k$-\pp-bounded (resp., $k$-mailbox-bounded).
% \end{definition}


%{\color{red}FIN AJOUT}

\section{Relations Between Classes}\label{section:comparison}



In this section we study how the classes introduced and recalled so far are related to each other. Notably, depending on the semantics (\pp or mailbox), we obtain two different classifications. The results are summed up in Figs. \ref{fig:diagram_p2p} and \ref{fig:diagram_mailbox_all}.
Systems cited as examples are available in Appendix~\ref{appendix:comparison}.
Here, we define existentially $k$-p2p-bounded MSCs and universally bounded counterparts as expected (formal definitions are available in Appendices~\ref{appendix:existentiallyptop} and \ref{appendix:universally}).


To refer to those systems we use the following terminology:
a system $\System$ is called
\wS{} (resp.\ \sS{}) if all MSCs $\msc$ in the respective language
are weakly synchronous (resp.\ strongly synchronous).
%, there is a $k$ such that $\msc$ is \wkSo{k} (resp. \skSo{k});
A system is called \wks{k} (resp. \sks{k}, \eb{} or \ub{})
if all MSCs are \wkSo{k} (resp. \skSo{k}, \ekb{k} or \ukb{k}).
A similar comparison relating existentially bounded systems, \wks{k} systems, as well as other systems that have not been described here, can also be found in \cite{DBLP:journals/corr/abs-1901-09606} for \pp systems.

%\begin{definition}
%	Let $\msc = (\Events,\procrel,\lhd,\lambda) \in \MSCs$.
%	 $\msc$ is \emph{strongly $k$-synchronous}
%	if it is of the form
%	$\msc = \msc_1 \cdot \ldots \cdot \msc_n$
%	such that every MSC $\msc_i = (\Events_i,\procrel_i,\lhd_i,\lambda_i)$ is a $k$-exchange
%	and, for all $(e,f) \in {\le}_\msc$, there are indices $1 \le i \le j \le n$
%	such that $e \in \Events_i$ and $f \in \Events_j$.
%\end{definition}

We give some results showing the inclusion of certain classes.
%As said in Remark~\ref{rem:stronglykp2p}, with a \pp semantics, all \skSo{k} MSCs are \wkSo{k}. Since strong  synchronizability is a refinement of weak synchronizability this implies that the two classes are equivalent. The same applies to  strong $k$-synchronizability and weak $k$-synchronizability (proofs are available in Appendix~\ref{appendix:comparison}).
%Thus, we do not consider \sks{k} nor strongly synchronizable systems for \pp semantics, but only \wks{k} and weakly synchronizable ones.
Recall that strong $k$-synchronizability is tailored to mailbox systems
(cf.\ also Remark~\ref{rem:stronglykp2p})
so that, for p2p systems, we only consider the case of
weak ($k$-)synchronizability.
%Thus, we do not have \sks{k} nor strongly synchronizable systems for \pp semantics,
%but only \wks{k} and weakly synchronizable ones.


%\begin{restatable}{proposition}{strongequalweakptop}
%\label{proposition:strong_equal_weak_p2p}
%	A \pp MSC is \skSo{k} iff it is \wkSo{k}.
%\end{restatable}

\begin{figure}[t]
	\begin{center}

\begin{tikzpicture}[scale = 0.5, opacity = 0.7]
	\draw[rounded corners = 5pt, gray!80, fill=gray!20]  (-2.5,3.4) rectangle (10.5,7);
	\draw[rounded corners = 5pt, gray!85, fill=gray!40]  (-1,3.4) rectangle (20,6);
		\draw[rounded corners = 5pt, gray!95, fill= gray!60]  (1,3.4) rectangle (10.5,5.5);
		\draw[rounded corners = 5pt, gray!90, fill = gray!70]  (9,3.4) rectangle (17,5);

		\draw (-2.3,6.5) node[right]{\textbf{Weakly synchronizable}};
		\draw (19.8,5.5) node[left]{\textbf{Existentially bounded}};
		\draw (16.8,4.5) node[left]{\textbf{Universally}};
		\draw (16.8,4) node[left]{\textbf{bounded}};
		\draw (1.2,5) node[right]{\textbf{Weakly }};
		\draw (1.2,4.4) node[right]{\textbf{k-synchronizable}};
	%	\draw (1.2,3.8) node[right]{\textbf{}};

		\tikzstyle{vertex}=[draw,circle,fill=black, minimum size = 3pt, inner sep = 0pt]

	%	\draw (4.2,3) node[vertex, label = left:{$\System_6$}] (v) {};
	%	\draw (7.5,2.5) node[vertex, label = above left:{$\System_7$}] (v) {};
		\draw (18.5,3.8) node[vertex, label = above :{$\systemexist$}] (v) {};%E
		\draw (12,3.8) node[vertex, label = above :{$\systemuniver$}] (v) {};%C

		\draw (9.8,3.8) node[vertex, label = above :{$\systemweakuniver$}] (v) {};%A
		\draw (8.3,3.8) node[vertex, label =  above:{$\systemweakSexist$}] (v) {};%B
		\draw (0,3.8) node[vertex, label =  above:{$\systemWSexist$}] (v) {}; %J
		\draw (-1.7,3.8) node[vertex, label =  above:{$\systemWS$}] (v) {}; %H

	%	\draw (4.7,1.5) node[vertex, label =  left:{$\System_4$}] (v) {};

\end{tikzpicture}
\caption{Hierarchy of classes for \pp systems}
\label{fig:diagram_p2p}

\end{center}
\end{figure}

\begin{figure}[t]
	\begin{center}

\begin{tikzpicture}[scale = 0.5, opacity = 0.7]

	\draw[rounded corners = 5pt, gray!75, fill=gray!20]
  (-2,0) rectangle (15,6.5);
  \draw[rounded corners = 5pt, gray!95, fill=gray!40]
  (5,0) rectangle (22,6);
  \draw[rounded corners = 5pt, gray!90, fill=gray!30]
  (-1.5,0) rectangle (15,3);
  \draw[rounded corners = 5pt, gray!90, fill=gray!50]
  (3,1) rectangle (15,4.5);
  \draw[rounded corners = 5pt, gray!95, fill=gray!60]
  (7,1) rectangle (15,2.5);
  \draw[rounded corners = 5pt, gray, fill=gray!70]
  (13,0) rectangle (21,2);


	 \draw (-1.8,6) node[right]{\textbf{Weakly synchro.}};
	 \draw (-1.8,5.4) node[right] {\textbf{}};
   \draw (21.8, 5.5) node[left] {\textbf{Existentially }} ;
  \draw (21.8, 4.9) node[left] {\textbf{ bounded}};
   \draw (-1.3, 2.5) node[right] {\textbf{Weakly }};
   \draw (-1.3, 1.9) node[right] {\textbf{k-synchro.}};
   \draw (14.8, 4) node[left] {\textbf{Strongly}};
   \draw (14.8, 3.4) node[left] {\textbf{synchro.}};
   \draw (7.2, 2) node[right] {\textbf{Strongly}};
     \draw (7.2, 1.4) node[right] {\textbf{k-synchro.}};
   \draw (20.8, 1.5) node[left] {\textbf{Universally }};
   \draw (20.8, 0.9) node[left] {\textbf{bounded}};



	\tikzstyle{vertex}=[draw,circle,fill=black, minimum size = 3pt, inner sep = 0pt]
	\draw (0,3.8) node[vertex, label = right :{$\systemW$}] (v) {};%I
	\draw (0,0.5) node[vertex, label = right :{$\systemweak$}] (v) {};%F
	\draw (3.5,3.8) node[vertex, label = right :{$\systemWS$}] (v) {};%H
	\draw (3.5,1.5) node[vertex, label = right :{$\systemweakS$}] (v) {}; %K
	\draw (5.5,1.5) node[vertex, label = right :{$\systemweakSexist$}] (v) {}; %B
	\draw (8,0.5) node[vertex, label = right :{$\systemweakexist$}] (v) {};%M
	\draw (8,5.3) node[vertex, label = right :{$\systemWexist$}] (v) {};%L
	\draw (8,3.8) node[vertex, label = right :{$\systemWSexist$}] (v) {};%J
	\draw (11.5,1.5) node[vertex, label = right :{$\systemstrongexist$}] (v) {};%D
	\draw (13.4,1.5) node[vertex, label = right :{$\systemstronguniver$}] (v) {};%G
	\draw (13.4,0.5) node[vertex, label = right :{$\systemweakuniver$}] (v) {};%A
	\draw (16,3.8) node[vertex, label = right :{$\systemexist$}] (v) {};%E
	\draw (16,0.5) node[vertex, label = right :{$\systemuniver$}] (v) {};%C

\end{tikzpicture}
 \caption{Hierarchy of classes for mailbox systems}
\label{fig:diagram_mailbox_all}

\end{center}
\end{figure}

%\vspace*{-0.2cm}
\begin{restatable}{proposition}{synchroinexists}
  \label{proposition:synchro_in_exists}
  Every \wkSo{k} MSC is existentially $k$-p2p-bounded. Moreover, every \skSo{k} mailbox MSC is existentially $k$-mailbox-bounded.
\end{restatable}

Finally, if a system is weakly synchronizable and universally $k$-bounded then, there is a $k'$ such that it is also weakly $k'$-synchronizable. The equivalent property is also valid for strong classes.
\begin{restatable}{proposition}{weakuniveruweak}
  \label{proposition:weak_univer_uweak}
  Every weakly (resp.\ strongly) synchronizable and \ukb{k} system is weakly (resp.\ strongly) $k'$-synchronizable for a $k'$.
\end{restatable}

%
%\paragraph*{Send-synchronizability.}
% In \cite{DBLP:conf/www/BasuB11,DBLP:conf/popl/BasuBO12}, Basu et al. studied the \emph{synchronizability} property defined as: a system $\System$ is \emph{synchronizable} if every execution is equivalent (in terms of the projection on sending messages) to the same system $\System$ but communicating by rendezvous. To avoid ambiguity, we call such systems \emph{send-synchronizable}.
%The class of weakly $1$-synchronizable systems is incomparable with the class of send-synchronizable systems.
%However, with an additional hypothesis about the set of reachable configurations, a weakly $1$-synchronizable system is send-synchronizable.
%We say that a system is \emph{stabilising} (called deadlockfree in \cite{DBLP:journals/iandc/LohreyM04}) if from any reachable configuration it can reach a confi\-gu\-ration with empty buffers (this is a stable configuration). Then every weakly $1$-synchronizable and stabilising system is send-synchronizable. The send-synchronizability does not fit in our framework because if send-synchronizability would be captured by our logical framework, then in particular, the question $\ppL{\Sys} \subseteq \Class_0$ would be decidable (by Theorem \ref{thm:sync}), where $\Class_0$ is the set of send-synchronizable MSCs. But this property is equivalent to checking whether the system $\System$ is send-synchronizable (since we are checking if every linearization does indeed correspond to the send projection of a rendezvous run), which was shown to be undecidable in \cite{DBLP:conf/icalp/FinkelL17}.


\section{Conclusion and Perspectives}
We have presented a unifying framework based on MSO logic and (special) tree-width, that brings together existing definitions, explains their good properties, and allows one to easily derive other, more general definitions and decidability results for synchronizability.
%
Let us notice that the send-synchronizability does not fit in our framework because the question $\ppL{\Sys} \subseteq \Class_0$ would be decidable (by Theorem \ref{thm:sync}), where $\Class_0$ is the set of send-synchronizable MSCs, but this property is equivalent to checking whether the system $\System$ is send-synchronizable and this last property is undecidable \cite{DBLP:conf/icalp/FinkelL17}.

Many other related questions could be studied in the future. For example, we could think about the hypotheses to add to our general framework to make the problem \emph{``does there exist an $k \geq 0$ such that $\ppL{\Sys} \subseteq \Class_k$?''}  decidable. From very recent work \cite{DLL2021}, one knows that the problem \emph{``does there exist an $k \geq 0$ such that the system is (weakly/strongly) $k$-synchronizable?"} is decidable; but it remains to be seen if it would be possible to obtain these results by showing that these properties can be expressed in a decidable extension of our framework. Let us remark that the decidability of the question whether there exists an $k \geq 0$ such that $\ppL{\Sys} \subseteq \Class_k$ allows us to build a bounded model checking strategy by first deciding whether there exists such an $k \geq 0$ and then by testing if $\ppL{\Sys} \subseteq \Class_k$ for $k=0,1,2 \dots$.
One may use this strategy for weakly/strongly synchronizable systems, but not for existentially bounded systems (except for deadlock-free systems) or for deterministic deadlock-free universally bounded systems. %One may also analyse the complexity of the studied problems and to get better complexity, one could use logic ICPDL instead of MSO logic.
 In \cite{DBLP:journals/corr/abs-1901-09606}, Lange and Yoshida introduced an \emph{asynchronous compatibility} property and it would also be interesting to verify whether this property could be expressed into our framework.






\bibliography{biblio}
\clearpage
\appendix

\section{Proof for Section 2}

\subsection{Proof of Lemma~\ref{lem:mb-prefix}}
\label{app:mb-prefix}

\prefixmailbox*

\begin{proof}
	Let $\msc = (\Events, \procrel, \lhd, \lambda) \in \mbMSCs$ and $\msc_0 =
	(\Events_0, \procrel_0, \lhd_0, \lambda_0)$ be a prefix of $\msc$, i.e.,
	$\Events_0 \subseteq \Events$. By contradiction, suppose that $\msc_0$ is not a
	mailbox MSC. Then, there are distinct $e,f \in \Events_0$ such that $e \preceq_{M_0} f \preceq_{M_0}
	e$ with ${\preceq_{\msc_0}} = ({\rightarrow_0} \cup {\lhd_0} \cup {\mbrel_{\msc_0}})^*$.
	As $\Events_0 \subseteq \Events$, we have that ${\rightarrow_0} \subseteq {\rightarrow}$, ${\lhd_0} \subseteq {\lhd}$, and ${\mbrel_{\msc_0}} \subseteq {\mbrel_{\msc}}$. Finally, ${\preceq_{\msc_0}} \subseteq {\preceq_{\msc}}$ and $\msc$ is not a mailbox MSC, which is a contradiction.
\end{proof}






\section{Proofs for Section 3}

\subsection{Proof of Theorem~\ref{mailbox}}
\label{app:mailbox}

\boundedmc*

\begin{proof}
Using the mailbox semantics and the MSO formula $\mbformula$, we get
\[\begin{array}{rl}
 &\mbL{\System} \cap \stwMSCs{k} \subseteq L(\phi)\\[1ex]
 \Longleftrightarrow &\ppL{\System} \cap \stwMSCs{k}\cap L(\mbformula)  \subseteq L(\phi)\\[1ex]
 \Longleftrightarrow &\ppL{\System} \cap \stwMSCs{k} \subseteq L(\phi \vee \neg \mbformula)\,.
\end{array}\]
The latter is decidable due to Fact~\ref{p2p}.
Similarly, we can use the LCPDL formula $\mbFormula$, whose size is polynomial in the number of processes and messages.
\end{proof}



\subsection{Proof of Lemma~\ref{lem:continuous2}}
\label{app:continuous2}

\lemcontinuoustwo*

\begin{proof}
	Let $k$ and $\Class$ be fixed, and let
	$\msc\in \MSCs\setminus \Class$.
	Let $\msc'\in \Pref{\msc}\setminus \Class$
	%such that $\Pref{\msc'}\subseteq \Class\cup\{\msc'\}$
	%(in other words, $\msc'$ is a shortest prefix of $\msc$ that is in $\MSCs\setminus \Class$.
	such that,
	for all $\le_{\msc'}$-maximal events $e$ of $\msc'$, removing $e$ (and its adjacent edge(s)) creates an MSC in $\Class$.
	We obtain such an MSC by successively removing maximal events.
	If $\msc'$ is the empty MSC, we are done, since then
	$\msc'\in (\Pref{\msc}\cap\stwMSCs{k+2})\setminus \Class$.
	Otherwise, let $e$ be $\le_{\msc'}$-maximal and
	%	$\lin$ be a linearization of $\msc'$, let $e$ be the last event
	%	of $\msc'$ wrt. $\lin$, and
	let $\msc''=\msc'\setminus\{e\}$.
	
	%	Note that this event
	%	$e$ exists unless $\msc'$ is the empty MSC, but in this degenerated case
	%	$\msc'\in (\Pref{\msc}\cap\stwMSCs{k+2})\setminus \Class$ and the proof
	%	is finished.

	Since $\msc'$ was taken minimal in terms of number of events,
	$\msc''\in \Class$.
	So Eve has a winning strategy with $k+1$ pebbles for $\msc''$.
	Let us design a winning strategy with $k+3$ pebbles for Eve for $\msc'$, which
	will show the claim.

	Observe that the event $e$ occurs at the end of the timeline of a process (say $p$), and it is part of at most two edges:
	\begin{itemize}
		\item one with the previous $p$-event (if any)
		\item one with the corresponding send event (if $e$ is a receive event)
	\end{itemize}
	Let $e_1,e_2$ be the two neighbors of $e$.
	The strategy of Eve is the following: in the first round, mark $e,e_1,e_2$,
	then erase the edges $(e_1,e)$ and $(e_2,e)$, then split the remaining graph
	in two parts: $\msc''$ on the one side, and the single node graph $\{e\}$ on
	the other side. Then Eve applies its winning strategy for $\msc''$, except
	that initially the two events $e_1,e_2$ are marked (so she may need up to $k+3$
	pebbles).
\end{proof}



\section{Proofs for Section~4}\label{appendix:section4}

\subsection{Proof of Proposition~\ref{prop:newweaklogiccg}}

\newweaklogiccg*

\begin{proof}
	$\implies$ Let $\msc$ be an MSC. If $\msc$ is weakly synchronous, then $\msc = \msc_1 \cdot \ldots \cdot \msc_n$ such that every $\msc_i$ is an exchange. Hence, for every vertex $v$ of the conflict graph, there is exactly one index $\iota(v) \in \{1,\ldots,n\}$ such that $\exists \lambda^{-1}(e) \in \msc_{\iota(v)}$, where $e \in Send(\plh, \plh, v)$. Note that if there is an edge from $v$ to $v'$ in the conflict
	graph, some action of $v$ must happen before some action of $v'$, i.e., $\iota(v) \leq \iota (v')$.
	 Furthermore, note that if $v \xrightarrow{RS} v'$ , then $\iota(v) < \iota(v')$, since within an exchange all the sends precede all the receives. So an RS edge cannot occur on a cyclic path.

	$\impliedby$ Let $\msc$ be an MSC. We assume now that the conflict graph of $\msc$ does not contain a cyclic path with an RS edge. Let $V_1, \ldots , V_n$ be the set of maximal SCCs of the conflict graph, listed in some
	topological order. For a fixed $i$, let $M_i = s_1 \ldots s_m r_1 \ldots r_{m'}$ be the enumeration of
	the actions of the message exchanges of $V_i$ defined by first taking all send actions
	of $V_i$ obeying the relation $\leq_M$, and then all the receive actions of
	$V_i$ in the same order as in $\leq_M$. Let $M' = M_1 \ldots M_n$. Then the conflict graph of $M'$ is the same as that of $M$, as the permutation of actions we defined could only postpone a receive after a send of a same SCC, therefore it could only replace some $v \xrightarrow{RS} v'$ edge with an $v \xrightarrow{SR} v'$ edge between two vertices $v,v'$ of a same SCC. However, since we assumed that the cycles (hence by extensions SCCs) do not contain RS edges, this cannot happen. Therefore $M$ and $M'$ have the same conflict graph, and correspond to the same MSC. Furthermore, since each $M_i$ is a downward-closed set, $M'$ is weakly synchronous, and so is $M$.
\end{proof}



\subsection{Proof of Corollary~\ref{cor:weak-sync-lcpdl}}

\weaksynclcpdl*

\begin{proof}
LCPDL can be used to express the graphical characterization of weakly synchronous MSCs. This follows from the formulas below. Here, we let $S = \bigvee_{a \in \sAct} a$ and $R = \bigvee_{a \in \rAct} a$.
\begin{align*}
	\xrightarrow{SS} & = \test{\neg R} \cdot \procrel^+ \cdot \test{\neg R} \\
	\xrightarrow{RR} & = \test{\neg R} \cdot \lhd \cdot \procrel^+ \cdot \inv{\lhd} \cdot \test{\neg R} \\
	\xrightarrow{RS} & = \test{\neg R} \cdot \lhd \cdot \procrel^+ \cdot \test{\neg R} \\
	\xrightarrow{SR} & = \test{\neg R} \cdot \procrel^+ \cdot \inv{\lhd} \cdot \test{\neg R} \\
	\xrightarrow{CG} & = (\xrightarrow{SS} + \xrightarrow{RR} + \xrightarrow{RS} + \xrightarrow{SR})
\end{align*}

The absence of RS edges in any cycle in the conflict graph can be expressed by the following formula:
\[
\Phi_{wsync} = \neg \Loop{(\xrightarrow{CG})^*\cdot \xrightarrow{RS} \cdot (\xrightarrow{CG})^*}
\qedhere
\]
\end{proof}


\subsection{Proof of Theorem~\ref{thm:p2p-weak-sync}}
\label{app:p2p-weak-sync}

\ppweaksynch*

\newcommand{\pcpSigma}{A}

\begin{proof}
We show that the control state reachability problem for p2p weakly
synchronizable systems is not decidable. This immediately shows
that the model-checking problem for p2p weak synchronizable systems
is not decidable. With some extra coding, it also shows that the
membership problem (decide whether a given system is p2p weakly synchronizable)
also is undecidable: indeed, it is enough to add a non weak
synchronizable behavior after the control states for which reachability is
undecidable: the system will be not weakly synchronizable iff the control
states are reached.

We reduce from Post correspondence problem (PCP).
Let us recall that a PCP instance consists of $N$ pairs $(u_i,v_i)$ of
finite words over an alphabet $\pcpSigma$, and that PCP undecidability holds already for
$N=7$ and $\pcpSigma=\{0,1\}$. We let the set of messages be
$\{1,\dots,N\}\uplus\pcpSigma\uplus\{\sharp\}$, and we consider a system with
four machines: Prover1, Prover2, Verifier1, and Verifier2. We have
unidirectional communication channels from provers to verifiers,
so the system is weakly synchronous by construction.

Informally, the system works as follows:
\begin{itemize}
\item Prover1 guesses a solution $u_{i_1}\dots u_{i_m}$ of the PCP instance,
and Prover2 also guesses the same solution $v_{i_1}...v_{i_m}$.
\item Prover1 sends $u_{i_1}\dots u_{i_n}$ to Verifier1 and sends
simultaneously $i_1\dots i_m$ to Verifier2
\item Prover2 sends $v_{i_1}\dots v_{i_m}$ to Verifier1 and sends
simultaneously $i_1\dots i_m$ to Verifier 2
\item Verifier1 checks that the two words are equal and Verifier2 checks that the sequences of indices are equal.
\end{itemize}

Let us now formally define these machines.
We describe them with regular expressions. For $w=a_1\cdots a_n$,
we write $\mathit{send}^*(p,q,w)$ (resp $\mathit{rec}^*(p,q,w)$)
for $\send{p}{q}{a_1}\cdots \send{p}{q}{a_n}$ (resp $\rec{p}{q}{a_1}\cdots\rec{p}{q}{a_n}$). We abbreviate Prover1 as P1, Prover2 as P2, Verifier1 as V1, and Verifier2 as V2

\begin{itemize}
\item Prover1 is $$\big(\sum_{i=1}^N\send{P_1}{V_1}{i}\mathit{send}^*(P_1,V_2,u_i)\big)^+\send{P_1}{V_1}{\sharp}\send{P_1}{V_2}{\sharp}$$

\item Prover2 is $$\big(\sum_{i=1}^N\send{P_2}{V_1}{i}\mathit{send}^*(P_2,V_2,v_i)\big)^+\send{P_2}{V_1}{\sharp}\send{P_2}{V_2}{\sharp}$$

\item Verifier1 is $$\big(\sum_{i=1}^N\rec{P_1}{V_1}{i}\rec{P_2}{V_1}{i}\big)^*\rec{P_1}{V_1}{\sharp}\rec{P_2}{V_1}{\sharp}$$

\item Verifier2 is $$\big(\sum_{a\in\Sigma}\rec{P_1}{V_2}{a}\rec{P_2}{V_2}{a}\big)^*\rec{P_1}{V_2}{\sharp}\rec{P_2}{V_2}{\sharp}$$

\end{itemize}

It can be checked that all machines reach their own final
state if and only if the PCP instance has a solution.
\end{proof}


\subsection{Proof of Proposition~\ref{prop:weak-logic-bounded}}


\weaklogicstw*

\begin{proof}
The formula for the property that there is no strongly connected component of size greater than $k$ in the conflict graph is as follows:
\begin{equation*}
	\nexists e_1, \ldots, e_{k+1} .\; [\bigwedge_{i \neq j} \XYformula{CG^*}(e_i, e_j)]
	\qedhere
\end{equation*}
\end{proof}


\subsection{Proof of Proposition~\ref{prop:kweakstw}}


\kweakstw*

\begin{proof}
	Let $M$ be a $k$-synchronous MSC. By definition, we know that $\msc = \msc_1 \cdot \ldots \cdot \msc_n$
	such that every $\msc_i$ is a $k$-exchange.

	Eve's strategy is to mark the vertices belonging to the set $M_1$. Hence, she marks at most $2k$ vertices. We can remove the edges between these vertices. Let the new marked MSC fragment be $(G, U)$, where $G$ is the new MSC fragment (with the edges between marked vertices removed), and $U$ the set of marked vertices.

	Notice  that $|U| \leq 2k$. Furthermore, since every vertex corresponding to a send message in $U$ is either unmatched or matched with a reception in $U$ (by definition), we can be sure that there are no message edges between vertices of $U$ and any other vertex. Moreover, we can also be sure that there are at most $|\Procs|$ process edges between vertices in $U$ and vertices outside this set. Let us mark these $|\Procs|$ vertices. We call these vertices $U'$. Let the new marked MSC fragment be $(G', U \cup U')$, where $G'$ is the new MSC fragment (with the edges between all vertices in $U \cup U'$ removed). Now, we see that there are no edges between any of the vertices in $U$ and any other vertex, i.e. all the vertices in $U$ are isolated. We can  divide the MSC fragment to consist of the vertices $U$ and $V \setminus U$ and the corresponding edges.

	Let the MSC fragment with vertices in $U$ be $(G_1, U_1)$. It consists of at most $2k$ isolated colored vertices. Let the MSC fragment with vertices in $V \setminus U$ be $(G_2, U_2)$. We observe that $|U_2|=n$. Adam  trivially loses if he chooses $(G_1, U_1)$, hence, he has to choose $(G_2, U_2)$. Now, we mark the vertices corresponding to $M_2$, which are again, at most $2k$. We have two possibilities for each vertex in $U_2$, either they belong to the set $M_2$ or belong to another set $M_p$ where $p>2$. However, if they belong to $M_p$, we can be sure that there is no other event on the same process that belongs to $M_2$ - this is because it was the successor of some event in $M_1$. Hence, we see once again, that marking all the vertices in $M_2$ and the immediate successors along each process will result in marked vertices of size at most $2k + |\Procs|$. And once again, we see that we can separate into MSC fragments $(G_1', U_1')$ and $(G_2', U_2')$ such that every vertex in $U_1'$ is isolated, and $|U_1'| \leq 2k$. We do this for all $i \in [n]$, and hence, we can effectively use $2k +|\Procs|$ colors. Therefore, set of MSCs over $|\Procs|$ processes which are $k$-synchronous have bounded special tree-width.
\end{proof}

\subsection{Proof of Proposition~\ref{prop:strong-logic-bounded}} \label{app:strong}
As was shown in \cite{DBLP:conf/fossacs/GiustoLL20}, in order to capture the mailbox semantics, we need  extended edges. We recall from \cite{DBLP:conf/fossacs/GiustoLL20} the extended edge relation $\xdashrightarrow{XY}$ with $X, Y \in \{\stype, \rtype\}$ in Figure~\ref{fig:extrules}.
%
We call the conflict graph along with the new extended edges the \emph{extended conflict graph} (ECG). This graph is also used to characterize some classes of MSCs in Section~\ref{sec:strong-sync}.


\begin{figure}[t]
$$\begin{array}{c}
		\infer[\text{(Rule 1)}]{v_1 \xdashrightarrow{XY} v_2}{v_1 \xrightarrow{XY} v_2}
	\qquad
		\infer[\text{(Rule 2)}]{v \xdashrightarrow{\stype\rtype} v}{v \in {\lhd} }
		\qquad
		\infer[\text{(Rule 3)}]{v_1 \xdashrightarrow{\stype\stype} v_2}{v_1 \xrightarrow{\rtype\rtype} v_2}
		\\ \\
		\infer[\text{(Rule 4)}]{v_1 \xdashrightarrow{XZ} v_2}{v_1 \xdashrightarrow{XY}\xdashrightarrow{YZ} v_2}
\qquad
	\infer[\text{(Rule 5)}]{\mexch(e_1) \xdashrightarrow{\stype\stype} e_2}{\stackanchor{$e_1 \in \Matched{\msc}  $ ~ $e_2 \in \Unm{\msc}$ }
		{
			$e_1 \in \mathit{Send}(p_1,q,\plh)$, $e_2 \in \mathit{Send}(p_2,q,\plh)$, $p_1,p_2,q \in \procSet$}}
\end{array}$$
\caption{Additional rules for extended conflict graph; $\xrightarrow{XY}$ refers to an edge in the conflict graph\label{fig:extrules}}
\end{figure}

%\todo{add some results on conflict graphs (characterization of conflict graph)}


\stronglogicstw*

%\ifx\islongversion\yes
%We first characterize this new notion of synchronizability graphically. This characterization is on the extended conflict graph of the system, as defined in \cite{DBLP:conf/fossacs/GiustoLL20}. This property can also be expressed in MSO logic. Furthermore, to show that this family has bounded STW, the proof nearly follows as for the weakly $k$-synchronizable case.

%!TEX root = icalp21.tex
%\subsection*{Decidability of strong \kSity{k}}\label{section:decidability_with_k}
%communication by unidirectionnal fifo channels or fifo mailbox
%
%definition for words or for MSCs ? \\

%For strong $k$-synchronizability, we use the extended conflict graph.
Similar to Theorem \ref{th:scccharactweak}, we now show the graphical characterization of strong synchronizability.

\begin{theorem}[Graphical Characterization of strongly $k$-synchronous MSCs]\label{theorem:graphical_characterisation_strong}
	Let $\msc \in \mbMSCs$. % be an MSC which satisfies causal delivery.
  $\msc$ is strongly $k$-synchronous iff every strongly connected component (SCC) in the ECG is of size at most $k$ and no RS edge occurs on any cycle in the ECG.
\end{theorem}

\begin{proof}
	($\implies$) Assume that we have an MSC $\msc$ that is strongly $k$-synchronous. Hence, we can divide $\msc = \msc_1 \ldots \msc_n$ such that each $\msc_i$ is a \kE{k}.
  By contradiction, suppose that there is an SCC of size $k' >k$ in $\ecgraph{\msc}$. As there are at most $k$ messages in each \kE{k}, there are $v,v'$ which belong to the SCC such that $v \in M_i$ and $v' \in M_j$, $1 \leq i<j \leq n$.
  Then, we have $v \dashrightarrow^* v' \dashrightarrow^* v$.

  By induction, we prove that $v' \dashrightarrow^* v$ implies that $j \leq i$.
  \begin{itemize}
    \item[Base] There are two cases.
    \begin{itemize}
      \item Suppose that $v' \xrightarrow{XY} v$ then an action of $v'$ is done by the same process than an action of $v$ and it is done before it. Then, $j \leq i$.
      \item Suppose that $v' \xdashrightarrow{SS} v$, built by Rule 5 (because others rules do not add any edges between vertices that are not already connected), then, $v'$ is matched and $v$ is unmatched, such that, $v \in Send(q,p,v)$ and $v' \in Send(q',p,v')$, $q,q' \in \procSet$. Then, the send of $v'$ has to be done before the send of $v$ and so $j \leq i$.
    \end{itemize}
    \item[Step]  By hypothesis, there is $v' \dashrightarrow^* v_1\dashrightarrow v$ such that $v_1 \in M_l$, $j \leq l \leq n$.
    There are also two cases.
    \begin{itemize}
      \item Either $v_1\xrightarrow{XY} v$. Then, an action of $v_1$ is done before and by the same process than an action of $v$. Then, $l \leq i$ and so $j \leq i$.
      \item Or $v_1\xdashrightarrow{SS} v$. Then, similarly as before, $v_1$ has to be sent before $v$ and so $l \leq i$. Therefore, $j\leq i$.
    \end{itemize}
  \end{itemize}
  Finally, we have that $v' \dashrightarrow^* v$ implies that $j \leq i$ and so there is a contradiction.

  Now, we show that there is no RS edge in any SCC.
  By contradiction, suppose that we have $v \xrightarrow{RS} v' \dashrightarrow^* v$ in the extended conflict graph. Then, as proved before, $v$ and $v'$ have to be in the same $\msc_i$, $1 \leq i \leq n$. However, $v \xrightarrow{RS} v'$ implies that the reception of $v$ has to be done before the send of $v'$, but a \kE{k} can, by definition, be linearized with all the sends followed by all the receptions. So we have a contradiction.
  %
	% To show that every SCC is of size at most $k$ in the ECG, we show that if we have vertices $v,v'$ in the ECG such that $v  \dashrightarrow^* v'   \dashrightarrow^* v$, then the events corresponding to the sends and receives (if any) of the messages $v,v'$ are in the same \kE{k}.
  %
  % We know that each SCC in the conflict graph of $M$ is of size at most $k$ since every strongly-$k$-synchronous MSC is weakly-$k$-synchronous. Hence, if we only have Rules 1 to 4 in the ECG, since we are not adding any edges between vertices that are not already connected, we have SCCs of the same size in the ECG as well. The only rule that is an exception is Rule 5. Let us consider this rule. W.l.o.g, let us assume that we have vertices $v,v', v_1$ such that $v \xdashrightarrow{SS} v'   \dashrightarrow^*  v_1 \xdashrightarrow{XY} v$, such that $XY$ is not $SS$. This condition can be met, because if we have a cycle of just SS edges, we have $v \xdashrightarrow{SS} v$, which violates causal delivery.
  %
  % By the definition of strong synchronizability, we know that the send corresponding to $v$ is now either in the same \kE{k}, or some preceding \kE{k}, as the send corresponding to $v'$. However, if we consider the path $ v_1  \xdashrightarrow{XY}   v$, it is either RR, RS or SR. However, it cannot be an RS edge because then we have $ v_1  \xdashrightarrow{RS}   v \xdashrightarrow{SR} v$, which reduces to $v_1 \xdashrightarrow{RR} v$, which reduces to $v_1 \xdashrightarrow{SS} v$. Similarly, we cannot have $v_1 \xdashrightarrow{RR} v$. Hence, it has to be $v_1 \xdashrightarrow{SR} v$, which implies that there is a vertex in the cycle such it is sent before $v$ is received. Hence, they are in the same k-exchange. Using this argument on all pairs of vertices in the SCC, we have that all the vertices have actions in the same k-exchange. Hence, we can have SCCs of size at most $k$, and every SCC corresponds to a k-exchange. Furthermore, since the sends and receives are ordered in every $k$-exchange, we cannot have an RS edge in a cycle.
%	\begin{enumerate}
%		\item  there exists a linearisation $e$ of $msc$ which is an execution of $\system$ and divisible into slices of size $k$
%		\begin{itemize}
%			\item $e = e_1 \cdots e_m$ where for all $ i\in [1,m]$,
%			%$e=a_1^1 \cdots a_{n_1}^1  \cdots a_{1}^m \cdots a_{n_m}^m$ where for all $i\in [1,m]$,
%			$e_i \in \sendSet^{\leq k} \cdot \receiveSet^{\leq k}$
%			%$a_1^i\cdots a_{n_i}^i \in \sendSet^{\leq k}\cdot \receiveSet^{\leq k}$
%			and such that $e \in E(\system)$,
%		\end{itemize}
%		\item the send and the reception of a matched message is in the same \kE{k}
%		\begin{itemize}
%			\item  for all $j,j'$ such that $a_j\matches a_{j'}$ holds in $e$, $\exists i, a_j, a_{j'} \in e_i$.
%		\end{itemize}
%	\end{enumerate}
	%There exists a linearisation $e$ of $M$ which is an execution of $\system$ and divisible into slices of size $k$.
	%Since every strongly-$k$-synchronous MSC is also weakly-${k}$-synchronous, we know that the conflict graph has SCCs of size at most $k$ and no RS edge on any cycle in the conflict graph. Therefore, we know that there is no RS edge on any cycle in the ECG since we can only have an RS edge between two nodes in the cycle if there is already an existing RS edge in the cycle (by definition of the rules). Furthermore, we know from mailbox semantics that the k-exchanges are a valid linearization of the MSC. Hence, we do not have vertices $v_1, v_2$ that could create a new SS edge as in Rule 5. So all new edges are only between connected vertices and the vertices in the SCCs do not change from the conflict graph. Therefore, in the ECG, there can be no RS edges in a cycle, and furthermore, every cycle has at most $k$ nodes.

	($\impliedby$) Conversely, assume that  every SCC in the extended conflict graph of $M$ is of size at most $k$ and no RS edge occurs on any cyclic path in the ECG. Then, we first show that every SCC in the extended conflict graph is $k$-synchronous.
  Let $C$ be an SCC formed of a set of nodes $v_1,\cdots, v_n$, for some $1 \leq n \leq k$  such that $s_i \in \msAct{v_i}$, for all $1 \leq i \leq n$.
  W.l.o.g., assume that the indexing of the
	nodes in $C$ is consistent with the edges labeled by SS (note that there is no cycle formed only of edges labeled by SS), i.e., for every $1 \leq i_1 < i_2 \leq n$, $C$ doesn’t contain an edge labeled by SS from $i_2$ to $i_1$, and for every $1\leq i <j <k \leq n$, if
  $s_i,s_k \in Send(p,\plh, \plh)$ for $p \in \procSet$ then $s_j \in Send(p,\plh, \plh)$.
  %$proc(s_i ) = proc(s_k)$, then $proc(s_i ) = proc(s_j )$.
  Let  $i_1,\cdots,i_m$ be the maximal subsequence of $1,\ldots ,n$ such that $r_\ell \in \mrAct{v_i}$ for every $\ell = i_j$ where $1 \leq j \leq m$.
  We have that $C$ is the graph of the execution $e = s_{i_1} \cdots  s_{i_n} r_{i_1} \cdots r_{i_m}$.
	The fact that all sends can be executed before the receives is a consequence of the fact that $C$ doesn’t contain edges labeled by RS.
  Then, the order between receives is consistent with the one between sends because $C$ satisfies causal delivery. By definition, $e$ is the label of an $n$-exchange transition, and therefore, $C$ is strongly $k$-synchronous.

	To complete the proof we proceed by induction on the number of strongly connected components of the extended conflict graph. The base case is for an MSC with a single SCC, which can be deduced from above. For the induction step, assume that the claim holds for every MSC whose extended conflict graph has at most $n$ strongly connected components, and let $M$ be a MSC with $n+1$ strongly connected components. Let $C$ be a strongly connected component of $M$ such that $C$ has no outgoing edges towards another strongly connected component of $M$. By the definition of the extended conflict-graph, $M= M'\cdot M''$ is the MSC corresponding to the nodes of $C$. We have shown above that $M''$ is $k$-synchronous, and by the induction hypothesis, $M'$ is also $k$-synchronous. As there is no outgoing edges from $M''$, we know that all messages in it have not to be done before a message of $M'$. Therefore, $M$ is strongly $k$-synchronous.
\end{proof}

%\subsection*{MSO definability of the Extended Conflict Graph}

For the extended conflict graph, we use the following MSO formulas to express the edge relation. For instance, the extended SR edge relation includes all SR %\comEtienne{SR?}
edges along with the set of self loops around each message that ensures that the sends are before the corresponding receives.

\begin{equation*}
\XYformula{ESR}(e_1, e_2) = \XYformula{SR}(e_1,e_2) \; \vee \; (\exists f_1.\; [e_1 \lhd f_1  \; \wedge \; (e_1 = e_2)  ])
\end{equation*}

Similarly, the extended send edge relation includes the SS edges along with the edges produced from Rule 3 and Rule 5.

	\begin{align*}
		\XYformula{ESS}(e_1, e_2) = \XYformula{SS}(e_1,e_2) \; \vee \;  \XYformula{RR}(e_1, e_2) \; \vee \; \biggl(  \exists f_1.\; [e_1 \lhd f_1 \; \wedge \; \nexists f_2. \; [e_2 \lhd f_2]] \; \\
		 \wedge \; \bigvee_{p, p',q \in \Procs}[(\lambda(e_1) =  \pqsAct{p}{q} \; \wedge \; \lambda(e_2) =  \pqsAct{p'}{q} ) ]\biggr)
	\end{align*}

The extended RR and RS edges are the same as in the conflict graph.
\begin{align*}
		\XYformula{ERR}(e_1, e_2) = \XYformula{RR}(e_1, e_2) \\
\XYformula{ERS}(e_1, e_2) = \XYformula{RS}(e_1, e_2)
\end{align*}

The transitive closure of each of these formulas is defined as follows. It essentially takes care of Rule 4. For all $X,Y,Z \in \{R, S\}$, we have:

\begin{align*}
	&\XYformula{EXY}(e_1, e_2)  \; \wedge \; \XYformula{EYZ}(e_2, e_3)  \implies \XYformula{EXZ^*}(e_1, e_3)\\
	&\XYformula{EXZ}(e_1, e_2) \implies \XYformula{EXZ^*}(e_1, e_2)
\end{align*}

We then extend the rest of the results, as in the case of the conflict graph in the previous section.

%\paragraph*{LCPDL formulas for the ECG.}
%
%We can express the above properties using LCPDL, as shown below. \begin{align*}
%	\xrightarrow{ESR} & = (\xrightarrow{SR} + (\test{\neg R} \cdot \lhd \cdot \test{R} \cdot \inv{\lhd} \cdot \test{\neg R})\\
%	\xrightarrow{ESS} & = (\xrightarrow{SS} + \xrightarrow{RR})\\
%	\xrightarrow{ERS} & = \xrightarrow{RS}\\
%	\xrightarrow{ESR} & = \xrightarrow{SR}\\
%\end{align*}

%\subsection*{$\Sync_k$ has bounded STW}

And finally, for the condition of the bounded STW, we observe that the set of strongly $k$-synchronizable MSCs are included in the set of weakly $k$-synchronizable MSCs. Hence, the decomposition strategy as used for the weakly $k$-synchronizable MSCs can be applied to the set of strongly $k$-synchronizable MSCs.

Therefore, the family $(\Sync_k)_{k \in \N}$ is MSO-definable and STW-bounded.

%\fi

\subsection{Proof of Proposition~\ref{prop:exist-k-mailbox-bounded}}
% \ifx\islongversion\yes
 We define the following relation in order to characterize $k$-mailbox-bounded MSCs.

 Let $k\geq 1$, and let $\msc$ be a fixed mailbox MSC. Let $\revb$ be the binary relation among events of $\msc$ defined as follows: $r\revb s$ if
\begin{enumerate}
\item $r$ is a receive event of a process $p$;
\item let $r'$ be the $k$-th receive event of process $p$ after $r$; then
$s\lhd r'$.
\end{enumerate}

\begin{lemma}\label{lem:exists-k-mailbox-acyclicity-condition}
$\msc$ is existential k-mailbox-bounded if and only if $\preceq_M\cup\revb$ is
acyclic.
\end{lemma}

\begin{proof}
Assume that $\msc$ is existential k-mailbox-bounded. Let $\linrel$ be a mailbox
linearisation of $\msc$ such that
for all $e \in \Matched{\msc}$,
say with $\lambda(e) = \sact{p}{q}{\msg}$,
\[\sametype{e}{\pqsAct{-}{q}}{\linrel} - \sametype{e}{\pqrAct{-}{q}}{\linrel}
\leq k\,.\]
Then $\linrel$ is also a linearisation of $(\preceq_M\cup\revb)^*$. Indeed,
if it was not the case, there would be a pair of events $r,s$ such
that $r\revb s$ and $s\linrel r$. But then we would have
\[\sametype{s}{\pqsAct{-}{q}}{\linrel} - \sametype{s}{\pqrAct{-}{q}}{\linrel}
> k\,,\]
and the contradiction. So $\linrel$ is
a linearisation of $(\preceq_M\cup\revb)^*$ and $\preceq_M\cup\revb$ is
acyclic.

Conversely, assume that $\preceq_M\cup\revb$, and let
$\linrel$ be a linearisation of $(\preceq_M\cup\revb)^*$.
In particular, $\linrel$ is a mailbox linearisation of $\msc$.
Let us show that
for all $s \in \Matched{\msc}$,
say with $\lambda(s) = \sact{p}{q}{\msg}$,
\[\sametype{s}{\pqsAct{-}{q}}{\linrel} - \sametype{s}{\pqrAct{-}{q}}{\linrel}
\leq k\,.\]
Let $s\in\Matched{\msc}$ be fixed, and let $r'$ be such that
$s\lhd r'$. There are two cases:
\begin{itemize}
\item $\sametype{r'}{\pqrAct{-}{q}}{\procrel}\leq k$. Then
\[\sametype{s}{\pqsAct{-}{q}}{\linrel}\leq k\,,\] because all sends before $s$
are matched. So \[\sametype{s}{\pqsAct{-}{q}}{\linrel} - \sametype{s}{\pqrAct{-}{q}}{\linrel}
\leq k\,,\]
\item $\sametype{r'}{\pqrAct{-}{q}}{\procrel}\leq k$. Then there is
$r$ on process $q$ such that $r\revb s$. So $r\linrel s$, and
there are at most $k$ messages in the buffer of $q$ at the time of event $s$, or in other words,
\[\sametype{e}{\pqsAct{-}{q}}{\linrel} - \sametype{e}{\pqrAct{-}{q}}{\linrel}
\leq k\,.\]
\end{itemize}
So $\linrel$ is a mailbox linearisation with $k$ bounded buffers,
and $\msc$ is existential k-mailbox-bounded.
\end{proof}

% \fi
% \begin{restatable}{proposition}{existkmailboxbounded}\label{prop:exist-k-mailbox-bounded}
% 	For all $k \in \N$, the set of existentially $k$-mailbox-bounded MSCs
% 	is MSO-definable and STW-bounded.
% \end{restatable}
\existkmailboxbounded*
% \ifx\islongversion\yes
%
 
\begin{proof}
Let $k\geq 1$ be fixed.
Since every existentially k-mailbox-bounded MSCs is also
existentially k-p2p-bounded, and since the class of existentially k-p2p-bounded MSCs is STW bounded (cf Proposition~\ref{prop:exists-k-p2p-bounded}),
the class of existentially k-mailbox-bounded MSCs is also STW bounded.

Let us show that it is moreover MSO definable.
%\comEtienne{we could also prove that it is FO definable with the same trick as in the p2p case. Maybe it is important complexity wise?}

By Lemma~\ref{lem:exists-k-mailbox-acyclicity-condition}, it is
enough to show that the acyclicity of $\preceq_M\cup\revb$ is
MSO definable, and since $\preceq_M$ was already shown MSO definable
and acyclicity is easily MSO definable, it is enough to show
that $\revb$ is MSO definable. It is indeed the case, as demonstrated by
this formula
$$
\phi(r,s)=\exists r_1,r_2,\dots,r_n. r\procrel r_1\procrel r_2\procrel\dots
\procrel r_n \wedge s\lhd r_n.
$$

Finally, let us show that existentially k-mailbox-bounded is also LCPDL
definable. This follows from the following formulas:

$$
\begin{array}{rl}
\prec_M= & (\lhd+\procrel)^+
\\
R = & \langle\lhd^{-1}\rangle\top
\\
\xrightarrow{\tiny{\mbox{next R}}} = & (\procrel\wedge\test{\neg R})^*\cdot (\procrel\wedge \test{R})
\\
\revb= & (\xrightarrow{\tiny{\mbox{next R}}})^k.(\lhd)^{-1}.
\\
\Phi_{\exists k~\mbox{mb-bounded}}=&\neg\Exists\Loop{(\prec_M+\revb)^+}
\end{array}
$$

\end{proof}

% \fi

% This extension is also valid for the \pp definition of existentially $k$-bounded MSCs. Finally, our framework can also be adapted to treat universally bounded systems \cite{HENRIKSEN20051,DBLP:conf/fossacs/LohreyM02}. Those extensions are available in Appendix~\ref{}.

\subsection{Existentially $k$-\pp-bounded MSCs}\label{appendix:existentiallyptop}
\begin{definition}
Let $\msc = (\Events,\procrel,\lhd,\lambda) \in \MSCs$ and $k \in \N$.
A linearization $\linrel$ of $\msc$ is called
$k$-\emph{\pp-bounded} if, for all $e \in \Matched{\msc}$, say with $\lambda(e) = \sact{p}{q}{\msg}$,
\\ $\sametype{e}{\pqsAct{p}{q}}{\linrel} - \sametype{e}{\pqrAct{p}{q}}{\linrel} \le k\,,$

We call $\msc$
\emph{existentially $k$-\pp-bounded} if
it has some \pp linearization that is $k$-\pp-bounded,
\end{definition}

 \begin{proposition}
 \label{prop:exists-k-p2p-bounded}
 For all $k \in \N$, the set of existentially $k$-\pp-bounded MSCs
 is MSO-definable and STW-bounded.
 \end{proposition}

 \begin{proof}
 The set of existentially $k$-\pp-bounded MSCs was shown to be MSO-definable
 (in fact, even FO-definable) in \cite{DBLP:journals/iandc/LohreyM04}. Note that there are minor differences
 in the definitions (in particular, the fact that we deal with unmatched messages),
 which, however, do not affect FO-definability.
 In \cite[Proposition 5.4, page 163]{DBLP:journals/corr/abs-1904-06942},\
 it was shown that their special tree-width is bounded by $k|\Procs|^2 + |\Procs|$.
 \end{proof}

% We obtain the following result as a corollary:
%
 \begin{theorem}\label{thm:exists-sync}
 For $\comsymb \in \{\ppsymb, \mbsymb\}$, the following problem is decidable:
 Given finite sets $\Procs$ and $\Msg$, a communicating system $\System$, and $k \in \N$,
 is every MSC in $\cL{\System}$ existentially $k$-\pp-bounded?
 \end{theorem}

 \begin{proof}
 Again, the proof follows exactly the same lines as that or Theorem~\ref{thm:weak-sync},
 now using Proposition~\ref{prop:exists-k-p2p-bounded}.
 \end{proof}
%
 Note that this is similar to the problem considered
 in \cite{GKM07,kuske2014communicating},
 though there is a subtle difference: in \cite{GKM07,kuske2014communicating},
 there are a notion of deadlock and distinguished final configurations.

\subsection{Definition of universally bounded MSCs}\label{appendix:universally}
\begin{definition}[Universally bounded MSC]
	Let $\msc = (\Events, \rightarrow, \lhd, \lambda)$ and $k \in \mathbb{N}$. We call $\msc$ universally $k$-\pp-bounded (resp., universally $k$-mailbox-bounded) if every p2p (resp., mailbox) linearization ${\lin} \subseteq \Events \times \Events$
is $k$-\pp-bounded (resp., $k$-mailbox-bounded).
\end{definition}




\section{Additional Material for Section~\ref{section:comparison}}\label{appendix:comparison}

%!TEX root = ../concur2021.tex

%\subsection{Proofs for section 5 } \label{sec:app-relations}

%\begin{theorem}\label{theorem:strong_equal_weak_p2p}
%	A peer-to-peer MSC is \skSous{k} iff it is weakly-$k$-synchronous.
%\end{theorem}

\subsection{Proofs}
\label{app:stronglykp2p}
%\strongequalweakptop*
%
Proof of the property from Remark~\ref{rem:stronglykp2p}:
\begin{restatable}{proposition}{strongequalweakptop}
\label{proposition:strong_equal_weak_p2p}
%	A \pp MSC is \skSo{k} iff it is \wkSo{k}.
Consider an MSC of the form
$\msc = \msc_1 \cdot \ldots \cdot \msc_n$ %(with $\msc_i$ an MSC)
such that every MSC $\msc_i = (\Events_i,\procrel_i,\lhd_i,\lambda_i)$ is a ($k$-)exchange.
Then, for all $(e,f) \in {\le}_\msc$, there are $1 \le i \le j \le n$
such that $e \in \Events_i$ and $f \in \Events_j$.
\end{restatable}

\begin{proof}
%	$\Rightarrow$ By definition.
%%\alain{I change the direction of implications}
%
%	$\Leftarrow$ Let $\msc \in \MSCs$ such that $\msc$ is \wkSo{k}.
%	By contradiction, suppose that $\msc$ is not \skSo{k}. Then:
%	\begin{itemize}
%		\item either there is no decomposition such that $M = M_1 \cdots M_n$, $M_i$ is a \kE{k}, $1 \le i \le n$, and so $M$ cannot be \wkSo{k}, which is a contradition;
%		\item or there is $e \le_M f $ such that $e \in \Events_i, f \in \Events_j$, $i>j$. If $ e \le_M f$, either $e \lhd f$ and so $i = j$, or there is $ p \in \procSet$ such that $e \rightarrow_p f$ and $e$ appears always before $f$ on $p$ thus $i \le j$, and so we have a contradiction. \qedhere
%	\end{itemize}
%	%\qed
If $e \le_\msc f$, then there is a sequence of events $e = e_0 \bowtie_1 e_1 \bowtie_2 \ldots
\bowtie_m e_m = f$ where $\bowtie$ is either $\to$ or $\lhd$. Clearly, for every
$\ell \in \{0,\ldots,m-1\}$, there are $1 \le i \le j \le n$
such that $e_\ell \in \Events_i$ and $e_{\ell+1} \in \Events_j$.
By transitivity, this proves the statement.
\qedhere
	%\qed
\end{proof}




\synchroinexists*
\begin{proof}
We begin by considering a \pp MSC.
	Let $\msc \in \MSCs$ be such that $M$ is \skSo{k}. Then, there is $M = M_1 \cdots M_n$, $M_i$ which is a \kE{k}, $1 \le i \le n$.
	By induction on $\msc$:
	\begin{itemize}
		\item[\textbf{1.}]\textbf{Base} $\msc = \msc_1$ then $\msc$ is a \kE{k} and by definition $|\Matched{\msc} \cup \Unm{\msc}| \leq k$.
		Then,  for all $e \in Matched(M)$ s.t. $\lambda(e) = \sact{p}{q}{\msg}$, we have
		\[\sametype{e}{\pqsAct{p}{q}}{\linrel} - \sametype{e}{\pqrAct{p}{q}}{\linrel} \le k.\]
		Then, $M$ is $k$-p2p-bounded.
		\item[\textbf{2.}]\textbf{Step}
		$M = M' \cdot M_n$ and we suppose that $M' = M_1 \cdots M_{n-1}$ is $k$-p2p-bounded.
		For all $1\leq i\leq {n}$, $M_i$ is a \kE{k}, and so an MSC, and by definition we know that any reception belongs to the same MSC as its matched send.
		Let $\lin$ be a linearization and $f \in Matched(M')$ s.t. $\lambda(f) = \sact{p}{q}{\msg}$ and, for all $e \in Matched(M')$ s.t. $\lambda(e) = \sact{p}{q}{\msg}$,  \[\sametype{f}{\pqsAct{p}{q}}{\linrel} > \sametype{e}{\pqsAct{p}{q}}{\linrel}.\]
		Then, we have:
		\[\sametype{f}{\pqsAct{p}{q}}{\linrel} - \sametype{f}{\pqrAct{p}{q}}{\linrel} = 0.\]
		Note that there is no unmatched message sent to $q$ before $f$ as $f$ is matched.
		As $M_n$ is a \kE{k}, we have for all  $e \in Matched(M_n)$ s.t. $\lambda(e) = \sact{p}{q}{\msg}$
		\[\sametype{e}{\pqsAct{p}{q}}{\linrel} - \sametype{e}{\pqrAct{p}{q}}{\linrel} \le k.\]

		Finally, for all  $e' \in Matched(M)$ s.t. $\lambda(e') = \sact{p}{q}{\msg}$, there is $e \in Matched(M_n)$ such that we can decompose:
		\[\sametype{e'}{\pqsAct{p}{q}}{\linrel} = \sametype{f}{\pqsAct{p}{q}}{\linrel} + \sametype{e}{\pqsAct{p}{q}}{\linrel} \]
		and
		\[\sametype{e'}{\pqrAct{p}{q}}{\linrel} =  \sametype{f}{\pqrAct{p}{q}}{\linrel} + \sametype{e}{\pqrAct{p}{q}}{\linrel}.\]
		Therefore \[ \sametype{e'}{\pqsAct{p}{q}}{\linrel} - \sametype{e'}{\pqrAct{p}{q}}{\linrel} \le k.\]
		Then, $M$ is $k$-p2p-bounded.
	\end{itemize}

Now, we move to  mailbox MSCs.
		Let $M \in \mbMSCs$ be a \skSo{k} MSC.
	By definition, $M = M_1 \cdots M_n$ such that every $M_i=(\Events_i,\procrel_i,\lhd_i,\lambda_i)$ is a \kE{k} and, for all $(e,f) \in {\sqsubset}_\msc$, there are indices $1 \leq i < j \leq n$ such that $e \in \Events_i$ and $f \in \Events_j$.

	By induction of $M$, we show that $M$ is $k$-mailbox-bounded.
	\begin{itemize}
		\item[\textbf{1.}]\textbf{Base}
		$M = M_1$ then $M$ is a \kE{k} and $|\Matched{\msc} \cup \Unm{\msc}| \le k$. Let $\lin$ be any linearization of $M$ and so, for all $e \in Matched(M)$ s.t. $\lambda(e) = \sact{p}{q}{\msg}$, we have
		$\sametype{e}{\pqsAct{\plh}{q}}{\linrel} - \sametype{e}{\pqrAct{\plh}{q}}{\linrel} \le k$. Then, $M$ is $k$-mailbox-bounded.
		\item[\textbf{2.}]\textbf{Step}
 $M = M' \cdot M_n$ and we suppose that $M' = M_1 \cdots M_{n-1}$ is $k$-mailbox-bounded.
		For all $1\leq i\leq {n}$, $M_i$ is a \kE{k}, and so an MSC, and by definition we know that any reception belongs to the same MSC as its matched send.
		Let $\lin$ be a linearization and $f \in Matched(M')$ s.t. $\lambda(f) = \sact{p}{q}{\msg}$ and, for all $e \in Matched(M')$ s.t. $\lambda(e) = \sact{p}{q}{\msg}$,
		\[\sametype{f}{\pqsAct{\plh}{q}}{\linrel} > \sametype{e}{\pqsAct{\plh}{q}}{\linrel}.\]
		Then, we have:
		\[\sametype{f}{\pqsAct{\plh}{q}}{\linrel} - \sametype{f}{\pqrAct{\plh}{q}}{\linrel} = 0.\]
		Note that there is no unmatched message sent to $q$ before $f$ as $f$ is matched.
		As $M_n$ is a \kE{k}, we have for all  $e \in Matched(M_n)$ s.t. $\lambda(e) = \sact{p}{q}{\msg}$
		\[\sametype{e}{\pqsAct{\plh}{q}}{\linrel} - \sametype{e}{\pqrAct{\plh}{q}}{\linrel} \le k.\]
		Finally, for all  $e' \in Matched(M)$ s.t. $\lambda(e') = \sact{p}{q}{\msg}$, there is $e \in Matched(M_n)$ such that we can decompose:
		\[\sametype{e'}{\pqsAct{\plh}{q}}{\linrel} = \sametype{f}{\pqsAct{\plh}{q}}{\linrel} + \sametype{e}{\pqsAct{\plh}{q}}{\linrel} \]
		and
		\[\sametype{e'}{\pqrAct{\plh}{q}}{\linrel} =  \sametype{f}{\pqrAct{\plh}{q}}{\linrel} + \sametype{e}{\pqrAct{\plh}{q}}{\linrel}.\]
		Therefore \[  \sametype{e'}{\pqsAct{\plh}{q}}{\linrel} - \sametype{e'}{\pqrAct{\plh}{q}}{\linrel} \le k.\]
		Then, $M$ is $k$-mailbox-bounded.
	\end{itemize}%\qed
\end{proof}


\weakuniveruweak*

\begin{proof}

	Let $\System$ be a system such that, for all $\msc \in \MSCs$, $\msc = \msc_1 \cdots \msc_n$ where $\msc_i$ is an exchange, $1\leq i \leq n$.
	Moreover, for all $e \in Matched(M)$ s.t. $\lambda(e) = \send{p}{q}{\msg}$,
	\begin{itemize}
		\item if $\msc \in \MSCs \setminus \mbMSCs$, for all $\lin \subseteq {\le}_\msc$, 	\[\sametype{e}{\pqsAct{p}{q}}{\linrel} - \sametype{e}{\pqrAct{p}{q}}{\linrel} \le k.\]
		\item if $\msc \in \mbMSCs$, for all $\lin \subseteq {\preceq}_\msc$,
		\[\sametype{e}{\pqsAct{\plh}{q}}{\linrel} - \sametype{e}{\pqrAct{\plh}{q}}{\linrel} \le k.\]
	\end{itemize}
	\begin{itemize}
		\item[\textbf{1.}]\textbf{Base}
		Suppose that $\msc = \msc_1$ so $\msc$ is an exchange.
		\begin{itemize}
			\item Either $\msc \in \MSCs \setminus \mbMSCs$ and, as $\msc $ is $k$-\pp-bounded, $\mid \msc \mid = k_1 \leq k \times \mid \procSet \mid^2 $.
			So $\msc$ is \wkSo{k_1}.
			\item Or $\msc \in \mbMSCs$ and,   as $\msc$ is $k$-mailbox-bounded,
			$\mid \msc \mid = k_2 \leq k \times \mid \procSet \mid $. 			So $\msc$ is \skSo{k_2}.
		\end{itemize}
		\item[\textbf{2.}]\textbf{Step}
		Suppose now that $\msc = M' \cdot  M''$ such that $M'$ is \wkSo{k} for a $k' \in \mathbb{N}$.  Then,
\begin{itemize}
	\item  if $\msc \in \MSCs \setminus \mbMSCs$,	for all linearizations $\lin \subseteq  \le_\msc$, let $f \in Matched(M')$ s.t. $\lambda(f) = \send{p}{q}{\msg}$ and, for all $e \in Matched(M')$ s.t. $\lambda(e) = \sact{p}{q}{\msg}$,
	\[\sametype{f}{\pqsAct{p}{q}}{\linrel} > \sametype{e}{\pqsAct{p}{q}}{\linrel}.\]
	Then, we have:
	\[\sametype{f}{\pqsAct{p}{q}}{\linrel} - \sametype{f}{\pqrAct{p}{q}}{\linrel} = 0.\]
	As $\System$ is \ukb{k}, $\mid M'' \mid \leq k_1$, and as $M''$ is an exchange, we know that $M''$ is a \kE{k_1}.
	\item  if $\msc \in \mbMSCs$,	for all linearizations $\lin \subseteq  \preceq_\msc$, let $f \in Matched(M')$ s.t. $\lambda(f) = \send{p}{q}{\msg}$ and, for all $e \in Matched(M')$ s.t. $\lambda(e) = \sact{p}{q}{\msg}$,
	\[\sametype{f}{\pqsAct{\plh}{q}}{\linrel} > \sametype{e}{\pqsAct{\plh}{q}}{\linrel}.\]
	Then, we have:
	\[\sametype{f}{\pqsAct{\plh}{q}}{\linrel} - \sametype{f}{\pqrAct{\plh}{q}}{\linrel} = 0.\]
	As $\System$ is \ukb{k}, $\mid M'' \mid \leq k_2$, and as $M''$ is an exchange, we know that $M''$ is a \kE{k_2}.
\end{itemize}
Then, $M$ is at least \wkSo{k_1} ($k_1 > k_2$).

	\end{itemize}
	Finally, as all MSCs are \wkSo{k_1}, $\System$ is \wks{k_1}.

	The equivalent proposition for strong properties can be shown in the same way. As an MSC is \sS, it can be divided while maintaining the mailbox order. In a recursive way, as MSC is \ukb{k}, we have that each exchange of the MSC is bounded. Finally, each MSC is \skSo{k'} for a $k'$  depending of $k$ and the number of channels, and so the system is \uskS{k'}.

\end{proof}



\subsection{Examples}


%\comL{to  move for arxiv version}



\subsubsection{\pp systems }
For p2p semantics,
\wS{} and \sS{} systems form a single class, as \sks{k} and \wks{k} form a single class too.
Indeed, as a consequence of Proposition~\ref{proposition:strong_equal_weak_p2p}, \wks{k}  systems are \sks{k}, and \wS{} are \sS{}.
Therefore, for this \pp part, we will only talk about weak classes.
We will see that this is not the case for mailbox systems.

However, \ub{} and \eb{} classes are not equal, an \ub{} system is \eb{} by definition but we can find a \eb{} system, as system $\systemexist$ in Fig.~\ref{fig:system_exist} where, as we can see in a corresponding MSC in Fig.~\ref{fig:msc_exist}, an unbounded number of $\msg_3$ can be sent before be read by $r$, which prevent the system to be \ub{}.

\begin{center}
%  \begin{minipage}[c]{7cm}
    


%\begin{figure}
  \begin{center}
      \begin{tikzpicture}[>=stealth,node distance=3.4cm,shorten >=1pt,
      every state/.style={text=black, scale =0.65}, semithick,
        font={\fontsize{8pt}{12}\selectfont}]
	% %P
  \begin{scope}[->]
     \node[state,initial,initial text={}] (q0)  {$\ell_p^0$};
     \node[state, right of=q0] (q1)  {$\ell_p^1$};
     \node[state, right of=q1] (q2) {$\ell_p^2$};

     \path (q0) edge node [above] {$\send{p}{q}{\msg_1}$} (q1);
     \path (q1) edge[bend left = 10] node [above] {$\send{p}{q}{\msg_1}$}(q2);
     \path (q2) edge[bend left = 10] node [below] {$\rec{q}{p}{\msg_2}$}(q1);

     \node[rectangle, thick, draw] at (-0.6,0.5) {$A_p$};
 \end{scope}

 \begin{scope}[->, shift={(6,0)}]
     \node[state,initial,initial text={}] (q0)  {$\ell_q^0$};
     \node[state, right of=q0] (q1)  {$\ell_q^1$};

     \path (q0) edge[bend left = 10] node [above] {$\send{q}{p}{\msg_2}$} (q1);
      \path (q1) edge[bend left = 10] node [below] {$\rec{p}{q}{\msg_1}$} (q0);
      \path (q0) edge [loop below] node [below]   {$~\send{q}{r}{\msg_3}$}(q0);
      \path (q1) edge [loop below] node [below]   {$~\send{q}{r}{\msg_3}$}(q1);


     \node[rectangle, thick, draw] at (-0.6,0.5) {$A_q$};
 \end{scope}

\begin{scope}[->, shift ={(11, 0)}]
   \node[state,initial,initial text={}] (q0)  {$\ell_r^0$};
   \path (q0) edge [loop below] node [below]   {$~\rec{q}{r}{\msg_3}$}(q0);
   \node[rectangle, thick, draw] at (-0.6,0.5) {$A_r$};

\end{scope}


		% \begin{scope}[shift = {(7.5,0.5)}, scale = 0.8]
    %   \draw (1.25, -4.75) node{\textbf{(b existe plus haut)}};
    %   \draw (-8, -4.75) node{\textbf{(a)}};
    %
		% 	%MACHINES
		% 	\draw (0,0) node{$p$} ;
		% 	\draw (1.25,0) node{$q$} ;
    %   \draw (2.5,0) node{$r$} ;
		% 	\draw (2.5, -0.25) -- (2.5, -3.75) ;
		% 	\draw (0,-0.25) -- (0,-3.75) ;
		% 	\draw (1.25,-0.25) -- (1.25,-3.75);
		% 	%MESSAGES
    %
    %   \draw[>=latex,->] (0, -0.5) -- (1.25, -1.25) node[pos=0.4, sloped, above] {$\amessage_1$};
    %   \draw[>=latex,->] (0, -1.5) -- (1.25, -2.25) node[pos=0.5, sloped, above] {$\amessage_1$};
    %   \draw[>=latex,->] (0, -2.5) -- (1.25, -3.25) node[pos=0.65, sloped, above] {$\amessage_1$};
    %
    %   \draw[>=latex,->] (1.25, -0.75) -- (0, -2) node[pos=0.5, sloped, above] {$\amessage_2$};
    %   \draw[>=latex,->] (1.25, -1.75) -- (0, -3) node[pos=0.5, sloped, above] {$\amessage_2$};
    %
    %   \draw[>=latex,->] (1.25, -1) -- (2.5, -1) node[midway, above] {$\amessage_3$};
    %   \draw[>=latex,->] (1.25, -1.5) -- (2.5, -1.5) node[midway, above] {$\amessage_3$};
    %   \draw[>=latex,->] (1.25, -2.5) -- (2.5, -2.5) node[midway, above] {$\amessage_3$};
    %
    %   \draw (0.6, -3.35) node{$\cdots$};
    %   \draw (1.9, -3.35) node{$\cdots$};
    %
    %
		% \end{scope}
\end{tikzpicture}
\captionof{figure}{System $\systemexist$}
\label{fig:system_exist}

\end{center}

%\end{figure}

%\end{minipage}
%\hspace*{1cm}
% \begin{minipage}[c]{3.5cm}
% \begin{center}
  \begin{tikzpicture}[>=stealth,node distance=3.4cm,shorten >=1pt,
  every state/.style={text=black, scale =0.7}, semithick,
    font={\fontsize{8pt}{12}\selectfont}]
\begin{scope}[shift = {(0,0)}, scale = 0.8]
  % \draw (1.25, -4.75) node{\textbf{(a)}};
  % \draw (4.5, -4.75) node{\textbf(b)};

  %MACHINES
  \draw (0,-0.1) node{$p$} ;
  \draw (1.25,-0.1) node{$q$} ;
  \draw (2.5,-0.1) node{$r$} ;
  \draw (2.5, -0.35) -- (2.5, -3.4) ;
  \draw (0,-0.35) -- (0,-3.4) ;
  \draw (1.25,-0.35) -- (1.25,-3.4);
  %MESSAGES

  \draw[>=latex,->] (0, -0.5) -- (1.25, -1.25) node[pos=0.4, sloped, above] {$\amessage_1$};
  \draw[>=latex,->] (0, -1.5) -- (1.25, -2.25) node[pos=0.55, sloped, above] {$\amessage_1$};
  \draw[>=latex,->, dashed] (0, -2.5) -- (1.25, -3.25) node[pos=0.65, sloped, above] {$\amessage_1$};

  \draw[>=latex,->] (1.25, -0.75) -- (0, -2) node[pos=0.5, sloped, above] {$\amessage_2$};
  \draw[>=latex,->] (1.25, -1.75) -- (0, -3) node[pos=0.5, sloped, above] {$\amessage_2$};


  \draw[>=latex,->] (1.25, -1) -- (2.5, -1) node[midway, above] {$\amessage_3$};
  \draw[>=latex,->] (1.25, -1.5) -- (2.5, -1.5) node[midway, above] {$\amessage_3$};
  \draw[>=latex,->] (1.25, -2.5) -- (2.5, -2.5) node[midway, above] {$\amessage_3$};

  %\draw (0.6, -3.35) node{$\cdots$};
  %\draw (1.9, -3.35) node{$\cdots$};
\end{scope}
      % \begin{scope}[shift = {(3,0)}, scale = 0.8]
      %   % \draw (0.5, -4) node{\textbf{(b)}};
      %   %MACHINES
      %   \draw (0,0) node{$p$} ;
      %   \draw (1,0) node{$q$} ;
      %   \draw (0,-0.25) -- (0,-3.5) ;
      %   \draw (1,-0.25) -- (1,-3.5);
      %   %MESSAGES
      %
      %    \draw[>=latex,->] (0, -0.75) -- (1, -0.75) node[midway, above] {$\amessage_1$};
      %    \draw[>=latex,->] (1, -1.5) -- (0, -1.5) node[midway, above] {$\amessage_2$};
      %    \draw[>=latex,->] (0, -2.25) -- (1, -2.25) node[midway, above] {$\amessage_1$};
      %    \draw[>=latex,->] (1, -3) -- (0, -3) node[midway, above] {$\amessage_2$};
      %
      %
      %   % \draw (0.5, -3.25) node{$\cdots$};
      % \end{scope}
        \end{tikzpicture}
\captionof{figure}{MSC $\mscexist$}
\label{fig:msc_exist}
      \end{center}

% \end{minipage}
 \end{center}


By definition, \wks{k} systems are \wS{}.
Also, \sks{k} systems are included into \ekb{k} systems, as proved by Proposition~\ref{proposition:synchro_in_exists}, so \wks{k} are included into \ekb{k}.

We can see that \wS{} systems and \eb{} systems are incomparable. System
$\systemWSexist$ in Fig.~\ref{fig:system_W_S_exist} is \wS{}
%(and so \sS{})
because we
can send all messages before read them, and \ekb{1} because each MSC of
$\ppL{\systemWSexist}$ has a linearization of the form $\send{q}{p}{\msg_2} \cdot
(\send{p}{q}{\msg_1} \cdot \rec{p}{q}{\msg_1})^* \rec{q}{p}{\msg_2}$,  allowing
to have in each channel only one pending message.

\begin{center}
  \begin{minipage}[c]{6cm}
    \begin{center}
\begin{tikzpicture}[>=stealth,node distance=3.4cm,shorten >=1pt,
    every state/.style={text=black, scale =0.7}, semithick,
      font={\fontsize{8pt}{12}\selectfont}]
      \begin{scope}[->, shift={(0,0)}]
          \node[state,initial,initial text={}] (q0)  {$\ell_p^0$};
          \node[state, right of=q0] (q1)  {$\ell_p^1$};

          \path (q0) edge [loop above] node [right]   {$~\send{p}{q}{\msg_1}$}(q0);
          \path (q0) edge node [below] {$\rec{q}{r}{\msg_2}$} (q1);

          \node[rectangle, thick, draw] at (-0.7,0.6) {$A_p$};
      \end{scope}
      \begin{scope}[->, shift={(0,-1.5)}]
  	      \node[state,initial,initial text={}] (q0)  {$\ell_q^0$};
  				\node[state, right of=q0] (q1)  {$\ell_q^1$};

  				\path (q0) edge node [above] {$\send{q}{r}{\msg_2}$} (q1);
          \path (q1) edge [loop right] node [right]   {$\rec{p}{q}{\msg_1}~$}(q1);


      		\node[rectangle, thick, draw] at (-0.7,0.6) {$A_q$};
  	  \end{scope}
      \end{tikzpicture}
\captionof{figure}{System $\systemWSexist$}
\label{fig:system_W_S_exist}
      \end{center}

\end{minipage}
\hspace*{1cm}
\begin{minipage}[c]{3.5cm}
  \input{Appendix-Sec5/msc_W_S_exist}
\end{minipage}
\end{center}


But, system $\systemWS$ in
Fig.\ref{fig:system_W_S} is only \wS{}. Indeed, for each execution we can add an
iteration of message $\msg_1$ or $\msg_2$, or both, and, as we can see in
Fig.~\ref{fig:msc_W_S}, we need to send all
messages before begin to read, and so each execution need a bigger channel than
the previous one.
\begin{center}
  \begin{minipage}[c]{6cm}
\begin{center}
\begin{tikzpicture}[>=stealth,node distance=3.4cm,shorten >=1pt,
    every state/.style={text=black, scale =0.7}, semithick,
      font={\fontsize{8pt}{12}\selectfont}]
      \begin{scope}[->, shift={(0,0)}]
          \node[state,initial,initial text={}] (q0)  {$\ell_p^0$};
          \node[state, right of=q0] (q1)  {$\ell_p^1$};

          \path (q0) edge [loop above] node [right]   {$\send{p}{q}{\msg_1}$}(q0);
          \path (q0) edge node [below] {$\send{p}{q}{\msg_1}$} (q1);
          \path (q1)  edge [loop above] node [right] {$\rec{q}{p}{\msg_2}$} (q1);

          \node[rectangle, thick, draw] at (-0.7,0.6) {$A_p$};
      \end{scope}
      \begin{scope}[->, shift={(0,-2)}]
  	      \node[state,initial,initial text={}] (q0)  {$\ell_q^0$};
  				\node[state, right of=q0] (q1)  {$\ell_q^1$};

          \path (q0) edge [loop above] node [right]   {$\send{q}{p}{\msg_2}$}(q0);
          \path (q0) edge node [below] {$\send{q}{p}{\msg_2}$} (q1);
          \path (q1)  edge [loop above] node [right] {$\rec{p}{q}{\msg_1}$} (q1);


      		\node[rectangle, thick, draw] at (-0.7,0.6) {$A_q$};
  	  \end{scope}
      \end{tikzpicture}
\captionof{figure}{System $\systemWS$}
\label{fig:system_W_S}
      \end{center}

\end{minipage}
\hspace*{1cm}
\begin{minipage}[c]{3.5cm}
\begin{center}
  \begin{tikzpicture}[>=stealth,node distance=3.4cm,shorten >=1pt,
  every state/.style={text=black, scale =0.7}, semithick,
    font={\fontsize{8pt}{12}\selectfont}]

\begin{scope}[shift = {(8,0.75)}, scale = 0.8]
%	\draw (0.75, -4) node{\textbf{(b)}};
  %MACHINES
  \draw (0,0) node{$p$} ;
  \draw (1.5,0) node{$q$} ;
  \draw (0,-0.25) -- (0,-3.5) ;
  \draw (1.5,-0.25) -- (1.5,-3.5);
  %MESSAGES

  \draw[>=latex,->] (0, -0.7) -- (1.5, -1.8) node[pos=0.2, sloped, above] {$\amessage_1$};
  \draw[>=latex,->] (0, -1.4) -- (1.5, -2.5) node[pos=0.1, sloped, above] {$\amessage_1$}; %{$\amessage_1'$};
  %\draw[>=latex,->, dashed] (0, -2.5) -- (1.25, -3.25) node[pos=0.55, sloped, above] {$\amessage_1''$};

  \draw[>=latex,->] (1.5, -0.5) -- (0, -2.2) node[pos=0.1, sloped, above] {$\amessage_2$};
  \draw[>=latex,->] (1.5, -1.2) -- (0, -2.9) node[pos=0.05, sloped, above] {$\amessage_2$};
  % \node[rotate = 90, left]at (1.13, -0.65) {$\cdots$};
  % \node[rotate = -90, right]at (0.1, -0.65) {$\cdots$};

\end{scope}

\end{tikzpicture}
\captionof{figure}{MSC $\mscWS$}
\label{fig:msc_W_S}
\end{center}

\end{minipage}
\end{center}

For system $\systemuniver$ in Fig.~\ref{fig:system_univer}, we
can see an example of MSC in Fig.~\ref{fig:msc_univer} which cannot be divided
into exchanges as send and receptions are intertwined and so $\systemuniver$ is not
\wS{}.

\begin{center}
  \begin{minipage}[c]{6cm}
    
  \begin{center}
      \begin{tikzpicture}[>=stealth,node distance=3.4cm,shorten >=1pt,
      every state/.style={text=black, scale =0.65}, semithick,
        font={\fontsize{8pt}{12}\selectfont}]
    \begin{scope}[->]
	      \node[state,initial,initial text={}] (q0)  {$\ell_p^0$};
	      \node[state, right of=q0] (q1)  {$\ell_p^1$};
				\node[state, right of=q1] (q2) {$\ell_p^2$};

	    	\path (q0) edge node [above] {$\send{p}{q}{\msg_1}$} (q1);
				\path (q1) edge[bend left = 10] node [above] {$\send{p}{q}{\msg_1}$}(q2);
				\path (q2) edge[bend left = 10] node [below] {$\rec{q}{p}{\msg_2}$}(q1);

	    	\node[rectangle, thick, draw] at (-0.7,0.5) {$A_p$};
	  \end{scope}

		\begin{scope}[->, shift={(0,-1.5)}]
	      \node[state,initial,initial text={}] (q0)  {$\ell_q^0$};
				\node[state, right of=q0] (q1)  {$\ell_q^1$};

				\path (q0) edge[bend left = 10] node [above] {$\send{q}{p}{\msg_2}$} (q1);
        \path (q1) edge[bend left = 10] node [below] {$\rec{p}{q}{\msg_1}$} (q0);

    		\node[rectangle, thick, draw] at (-0.7,0.5) {$A_q$};


	  \end{scope}

	\end{tikzpicture}
	\captionof{figure}{System $\systemuniver$ }
	\label{fig:system_univer}

  \end{center}

\end{minipage}
\hspace*{1cm}
\begin{minipage}[c]{3.5cm}
  \begin{center}
  \begin{tikzpicture}[>=stealth,node distance=3.4cm,shorten >=1pt,
  every state/.style={text=black, scale =0.7}, semithick,
    font={\fontsize{8pt}{12}\selectfont}]

\begin{scope}[shift = {(8,0.75)}, scale = 0.8]
%	\draw (0.75, -4) node{\textbf{(b)}};
  %MACHINES
  \draw (0,0) node{$p$} ;
  \draw (1.5,0) node{$q$} ;
  \draw (0,-0.25) -- (0,-3.5) ;
  \draw (1.5,-0.25) -- (1.5,-3.5);
  %MESSAGES

  \draw[>=latex,->] (0, -0.5) -- (1.5, -1.25) node[pos=0.4, sloped, above] {$\amessage_1$};
  \draw[>=latex,->] (0, -1.5) -- (1.5, -2.25) node[pos=0.55, sloped, above] {$\amessage_1$};
  \draw[>=latex,->, dashed] (0, -2.5) -- (1.5, -3.25) node[pos=0.55, sloped, above] {$\amessage_1$};


  \draw[>=latex,->] (1.5, -0.75) -- (0, -2) node[pos=0.5, sloped, above] {$\amessage_2$};
  \draw[>=latex,->] (1.5, -1.75) -- (0, -3) node[pos=0.55, sloped, above] {$\amessage_2$};
%  \draw (0.6, -3.25) node{$\cdots$};

\end{scope}

\end{tikzpicture}
\captionof{figure}{MSC $\mscuniver$}
\label{fig:msc_univer}
\end{center}

\end{minipage}
\end{center}




We can observe that $\systemuniver$ is \ukb{3} as each time a
message is sent, one is received and
the maximal number of messages in a buffer is 3. As we said, $\systemWS$ is \wS{} but not \eb{} and so not \ub{}, so \wS{} and \ub{} systems are incomparable.
However, as proved in Proposition~\ref{proposition:weak_univer_uweak}, a system which is both is also \wks{k} (not necessarly for the same $k$).

Finally, \wks{k} and \ukb{k} systems are incomparable. System $\systemweakuniver$ in
Fig.~\ref{fig:system_weak_univer} is  both weakly 1-synchronizable and
universally 1-bounded.   But,  we have, for example, system $\systemweakSexist$ below
which is \wks{1} but not \ukb{k} for any $k$.  Indeed,
each execution can be rescheduled to have all the receptions just after the
respective sends. Then each MSC can be divided into \kE{1}s  (for an example,
see  MSC $\mscweakSexist$ in Fig.~\ref{fig:msc_weak_S_exist}). However, as we can send an
unbounded number of $\msg_2$ messages before reading them, the size of channel
$c_p$ can also be unbounded thus the system is not universally bounded.
Conversely, as we have seen, system $\systemuniver$ is \ukb{3} but not \wks{k} for any $k$.

%\begin{figure}
  \begin{center}

  \begin{tikzpicture}[>=stealth,node distance=3.5cm,shorten >=1pt,
      every state/.style={text=black, scale =0.65}, semithick,
      font={\fontsize{8pt}{12}\selectfont}]
	% %P
	  \begin{scope}[->]
	      \node[state,initial,initial text={}] (q0)  {$\ell_p^0$};
	      \node[state, right of=q0] (q1)  {$\ell_p^1$};
				\node[state, right of=q1] (q2) {$\ell_p^2$};

	    	\path (q0) edge node [below] {$\send{p}{q}{\msg_1}$} (q1);
				\path (q1) edge node [below] {$\rec{q}{p}{\msg_2}$}(q2);
				\path (q2) edge [loop above] node [above]   {$~\rec{q}{p}{\msg_2}$}(q2);

	    	\node[rectangle, thick, draw] at (-0.7,0.4) {$A_p$};
	  \end{scope}

	  \begin{scope}[->, shift={(6.5,0)} ]
	      \node[state,initial,initial text={}] (q0)  {$\ell_q^0$};
				\node[state, right of=q0] (q1)  {$\ell_q^1$};
				\node[state, right of=q1] (q2) {$\ell_q^2$};

				\path (q0) edge node [below] {$\send{q}{p}{\msg_2}$} (q1);
				\path (q1) edge [loop above] node [above]   {$~\send{q}{p}{\msg_2}$}(q1);
				\path (q1) edge node [below] {$\rec{r}{q}{\msg_3}$}(q2);
				\node[rectangle, thick, draw] at (-0.7,0.4) {$A_q$};
	  \end{scope}

		\begin{scope}[->, shift={(4.5,-1.3)} ]
	      \node[state,initial,initial text={}] (q0)  {$\ell_r^0$};
				\node[state, right of=q0] (q1)  {$\ell_r^1$};

				\path (q0) edge node [below] {$\send{r}{q}{\msg_3}$} (q1);
				\node[rectangle, thick, draw] at (-0.7,0.4) {$A_r$};
			%\node at (1, -.7) {\textbf{(a)}};

	  \end{scope}
		% \begin{scope}[shift = {(8,0.5)}, scale = 0.8]
		% 	\draw (1, -4.25) node{\textbf{(b) existe plus haut}};
    %   \node at (-9, -4.25) {\textbf{(a)}};
    %
		% 	%MACHINES
		% 	\draw (0,0) node{$p$} ;
		% 	\draw (1,0) node{$q$} ;
		% 	\draw (2,0) node{$r$} ;
		% 	\draw (0,-0.25) -- (0,-3.5) ;
		% 	\draw (1,-0.25) -- (1,-3.5);
		% 	\draw (2, -0.25) -- (2, -3.5) ;
		% 	%MESSAGES
		% 	\draw[>=latex,->, dashed] (0,-0.75) -- (1, -0.75) node[midway,above]{$\amessage_1$};
    %
		% 	\draw[>=latex,->] (1, -1.5) -- (0, -1.5) node[midway, above] {$\amessage_2$};
		% 	\draw (0.5,-1.7) node{$\cdots$};
		% 	\draw[>=latex,->] (1, -2.25) -- (0, -2.25) node[midway, above] {$\amessage_2$};
    %
		% 	\draw[>=latex,->] (2,-3) -- (1,-3) node[midway, above] {$\amessage_3$};
		% \end{scope}
\end{tikzpicture}
\captionof{figure}{System $\systemweakSexist$}
\label{fig:system_weak_S_exist}

\end{center}
%\end{figure}



\subsubsection{Mailbox systems }

Now consider  mailbox semantics. As depicted in Fig. \ref{fig:diagram_mailbox_all}, we can now distinguish between weakly and strongly synchronizable systems.

Some inclusions are obvious, by definition of the classes:
\begin{itemize}
  \item \ub{} systems are \eb{}, but it is a proper inclusion. Indeed, a system, as system $\systemexist$ in Fig.~\ref{fig:system_exist}, can be \ekb{1}, but not \ub{}. In this case, as in \pp semantics, an unbounded number of message $\msg_3$ can be sent before be read, as we can see in MSC $\mscexist$ in Fig.~\ref{fig:msc_exist}.
  \item \sks{k} systems are \sS{}. As well, a system can be \sS{} without have a bound on the size of its exchange, as system $\systemWS$ in Fig~\ref{fig:system_W_S}. See an example of MSC of $\systemWS$ in Fig~\ref{fig:msc_W_S}, where we can always build a bigger exchange adding iterations of messages $\msg_1$ or $\msg_2$.
  \item \wks{k} systems are \wS{}. We can also find a \wS{} system without bound on the size of its exchange. Let see $\systemW$ in Fig.~\ref{fig:system_W} with MSC $\mscW$ in Fig.~\ref{fig:msc_W}, which can be divided into exchanges, each message can be in a separate exchange, except messages $\msg_2$ and $\msg_3$ that have to be all in the same exchange. As we can have as many repetitions of them, this exchange no have bound on its size and prevent the system to be \wks{k} for a precise $k$.
\end{itemize}




%
% \begin{center}
%   \begin{minipage}[c]{6cm}
     
\begin{center}
\begin{tikzpicture}[>=stealth,node distance=3.4cm,shorten >=1pt,
    every state/.style={text=black, scale =0.7}, semithick,
      font={\fontsize{8pt}{12}\selectfont}]
      \begin{scope}[->, shift={(0,0)}]
          \node[state,initial,initial text={}] (q0)  {$\ell_p^0$};
          \node[state, right of=q0] (q1)  {$\ell_p^1$};
          \node[state, right of=q1] (q2)  {$\ell_p^2$};


          \path (q0) edge node [above] {$\send{p}{q}{\msg_1}$} (q1);
          \path (q1) edge node [above]  {$\rec{q}{p}{\msg_2}$} (q2);

          \node[rectangle, thick, draw] at (-0.7,0.6) {$A_p$};
      \end{scope}
      \begin{scope}[->, shift={(7,0)}]
  	      \node[state,initial,initial text={}] (q0)  {$\ell_q^0$};
  				\node[state, right of=q0] (q1)  {$\ell_q^1$};
          \node[state, right of=q1] (q2)  {$\ell_q^2$};

          \path (q0) edge [loop above] node [right]   {$\send{q}{r}{\msg_3}$}(q0);
          \path (q0) edge node [above] {$\send{q}{r}{\msg_3}$} (q1);
          \path (q1) edge [loop above] node [right]   {$\rec{r}{q}{\msg_4}$}(q1);
          \path (q1) edge node [above] {$\rec{r}{q}{\msg_5}$} (q2);

      		\node[rectangle, thick, draw] at (-0.7,0.6) {$A_q$};
  	  \end{scope}

      \begin{scope}[->, shift={(3.5,-1.5)}]
          \node[state,initial,initial text={}] (q0)  {$\ell_q^0$};
          \node[state, right of=q0] (q1)  {$\ell_q^1$};
          \node[state, right of=q1] (q2)  {$\ell_q^2$};


          \path (q0) edge node [above] {$\send{r}{q}{\msg_4}$} (q1);
          \path (q0) edge [loop above] node [right]  {$\send{r}{q}{\msg_4}$}(q0);

          \path (q1) edge [loop above] node [right]  {$\rec{q}{r}{\msg_3}$}(q1);
          \path (q1) edge node [above] {$\send{r}{q}{\msg_5}$} (q2);

          \node[rectangle, thick, draw] at (-0.7,0.6) {$A_r$};
      \end{scope}
      \end{tikzpicture}
\captionof{figure}{System $\systemW$}
\label{fig:system_W}
      \end{center}

% \end{minipage}
% \hspace*{1cm}
% \begin{minipage}[c]{3.5cm}
%   \input{Appendix-Sec5/msc_W}
% \end{minipage}
% \end{center}


The classes of \wS{} and \wks{k} systems are incomparable with the classes of \eb{} and \ub{} systems.

Indeed, let see system $\systemuniver$ in Fig.~\ref{fig:system_univer} which is \ukb{3}. It cannot be \wS{} (and so \wks{k} for a $k$) as we cannot divide a corresponding MSC, as $\mscuniver$ in Fig.~\ref{fig:msc_univer}, into exchanges, where all sends have to be before all receptions, as in \pp semantics.
We see a difference with the \pp semantics looking at $\systemweak$ in
Fig.~\ref{fig:system_weak} which is \wS{} but not \eb{}. Indeed, we can see in
Fig.~\ref{fig:msc_weak} that $\mscweak$ can be divided easily into \kE{1} but,
as $\msg_1$ has to be sent before $\msg_4$, and $\msg_2$ has to be sent to send
$\msg_4$,  it means  that all messages $\msg_2$ have to wait the send of
$\msg_4$ to be read.  Then, each execution is $x$-mailbox-bounded, where $x$ is
the number of repetitions of $\msg_2$. Then, there is no bound on the buffers
and $\systemweak$ cannot be \eb{} (and so \ub{}).



\begin{center}
  \begin{minipage}[c]{6cm}
    
  \begin{center}

  \begin{tikzpicture}[>=stealth,node distance=3.4cm,shorten >=1pt,
      every state/.style={text=black, scale =0.65}, semithick,
        font={\fontsize{8pt}{12}\selectfont}]

        \begin{scope}[-> ]
            \node[state,initial,initial text={}] (q0)  {$\ell_p^0$};
            \node[state, right of=q0] (q1)  {$\ell_p^1$};
            \node[state, right of=q1] (q2) {$\ell_p^2$};

            \path (q0) edge node [above] {$\send{p}{q}{\msg_2}$} (q1);
            \path (q1) edge [loop above] node [right]   {$~\send{p}{q}{\msg_2}$}(q1);
            \path (q1) edge node [above] {$\send{p}{s}{\msg_3}$}(q2);
            \node[rectangle, thick, draw] at (-0.6,0.6) {$A_p$};
        \end{scope}


      \begin{scope}[->,  shift={(0,-1.1)}]
          \node[state,initial,initial text={}] (q0)  {$\ell_q^0$};
          \node[state, right of=q0] (q1)  {$\ell_q^1$};
          \node[state, right of=q1] (q2) {$\ell_q^2$};

          \path (q0) edge node [above] {$\send{q}{r}{\msg_1}$} (q1);
          \path (q1) edge node [above] {$\rec{p}{q}{\msg_2}$}(q2);
          \path (q2) edge [loop above] node [right]   {$~\rec{p}{q}{\msg_2}$}(q2);

          \node[rectangle, thick, draw] at (-0.6,0.6) {$A_q$};
      \end{scope}



      \begin{scope}[->, shift={(0,-2.2)} ]
          \node[state,initial,initial text={}] (q0)  {$\ell_r^0$};
          \node[state, right of=q0] (q1)  {$\ell_r^1$};

          \path (q0) edge node [above] {$\rec{s}{r}{\msg_4}$} (q1);

          \node[rectangle, thick, draw] at (-0.6,0.6) {$A_r$};
        %\node at (1, -1.5) {\textbf{(a)}};

      \end{scope}
      \begin{scope}[->, shift ={(0, -3.3)}]
         \node[state,initial,initial text={}] (q0)  {$\ell_s^0$};
         \node[state, right of=q0] (q1)  {$\ell_s^1$};
         \node[state, right of=q1] (q2) {$\ell_s^2$};

         \path (q0) edge node [above] {$\rec{p}{s}{\msg_3}$} (q1);
         \path (q1) edge node [above] {$\send{s}{r}{\msg_4}$} (q2);

         \node[rectangle, thick, draw] at (-0.6,0.6) {$A_s$};

      \end{scope}
      % \begin{scope}[shift = {(8.5,0)}, scale = 0.8]
      %   \draw (1.5, -4.75) node{\textbf{(b)}};
      %   \draw (-9.5, -4.75) node{\textbf{(a)}};
      %
      %   %MACHINES
      %   \draw (0,0) node{$p$} ;
    	% 	\draw (1,0) node{$q$} ;
    	% 	\draw (2,0) node{$r$} ;
      %   \draw (3,0) node{$s$} ;
    	% 	\draw (0,-0.25) -- (0,-3.75) ;
    	% 	\draw (1,-0.25) -- (1,-3.75);
    	% 	\draw (2, -0.25) -- (2, -3.75) ;
      %   \draw (3, -0.25) -- (3, -3.75) ;
      %
    	% 	%MESSAGES
    	% 	\draw[>=latex,->, dashed] (1,-0.7) -- (2, -0.7) node[midway,above]{$\amessage_1$};
      %
    	% 	\draw[>=latex,->] (0, -1.35) -- (1, -1.35) node[midway, above] {$\amessage_2$};
      %
    	% 	\draw[>=latex,->] (0,-2.1) -- (1,-2.1) node[midway, above] {$\amessage_2$};
      %
    	% 	\draw[>=latex,->] (0,-2.75) -- (3,-2.75) node[midway,above] {$\amessage_3$};
      %
      %   \draw[>=latex,->] (3,-3.4) -- (2,-3.4) node[midway,above] {$\amessage_4$};
      %
      % %  \draw[>=latex,->] (2,-3.25) -- (3,-3.25) node[midway,above] {$\amessage_5$};
      %
      %   \draw (0.5, -1.55) node{$\cdots$};
      %
      % \end{scope}


  \end{tikzpicture}
  \captionof{figure}{System $\systemweak$ }
  \label{fig:system_weak}

\end{center}

\end{minipage}
\hspace*{1cm}
\begin{minipage}[c]{3.5cm}
  \begin{center}


\begin{tikzpicture}[>=stealth,node distance=3.4cm,shorten >=1pt,
    every state/.style={text=black, scale =0.7}, semithick,
      font={\fontsize{8pt}{12}\selectfont}]

\begin{scope}[shift = {(8.5,0)}, scale = 0.8]

  %MACHINES
  \draw (0,0.3) node{$p$} ;
  \draw (1,0.3) node{$q$} ;
  \draw (2,0.3) node{$r$} ;
  \draw (3,0.3) node{$s$} ;
  \draw (0,0.05) -- (0,-4) ;
  \draw (1,0.05) -- (1,-4);
  \draw (2, 0.05) -- (2, -4) ;
  \draw (3, 0.05) -- (3, -4) ;

  %MESSAGES
  \draw[>=latex,->, dashed] (1,-0.7) -- (2, -0.7) node[midway,above]{$\amessage_1$};

  \draw[>=latex,->] (0, -1.35) -- (1, -1.35) node[midway, above] {$\amessage_2$};

  \draw[>=latex,->] (0,-2.1) -- (1,-2.1) node[midway, above] {$\amessage_2$};

  \draw[>=latex,->] (0,-2.75) -- (3,-2.75) node[midway,above] {$\amessage_3$};

  \draw[>=latex,->] (3,-3.4) -- (2,-3.4) node[midway,above] {$\amessage_4$};

%  \draw[>=latex,->] (2,-3.25) -- (3,-3.25) node[midway,above] {$\amessage_5$};

  %\draw (0.5, -1.55) node{$\cdots$};

\end{scope}
\end{tikzpicture}
\captionof{figure}{MSC $\mscweak$ }
\label{fig:msc_weak}
\end{center}

\end{minipage}
\end{center}


Finally, as proved in Proposition~\ref{proposition:weak_univer_uweak}, if a system is \wS{} and \ub{} system, there is a $k$ such that it is also \sks{k}. In all the other intersections, we can find a system:
\begin{itemize}
  \item \wS{} and \eb{} (but not \wks{k} or \ub{}): $\systemWexist$ in Fig.~\ref{fig:system_W_exist} is \ekb{1} but there is no bound on the size of the exchange containing messages $\msg_3$ and $\msg_4$ and so $\systemWexist$ cannot be \wks{k}, and repetitions of $\msg_3$ prevent it to be \ub{};
\end{itemize}
  \begin{center}
    \begin{minipage}[c]{6cm}
      \input{Appendix-Sec5/system_W_exist}
  \end{minipage}
  \hspace*{1cm}
  \begin{minipage}[c]{3.5cm}
    \begin{center}
  \begin{tikzpicture}[>=stealth,node distance=3.4cm,shorten >=1pt,
  every state/.style={text=black, scale =0.7}, semithick,
    font={\fontsize{8pt}{12}\selectfont}]

\begin{scope}[shift = {(8,0.75)}, scale = 0.8]
%	\draw (0.75, -4) node{\textbf{(b)}};
  %MACHINES
  \draw (0,1.25) node{$q$} ;
  \draw (1.25,1.25) node{$r$} ;
  \draw (-1.25,1.25) node{$p$} ;
  \draw (0,1) -- (0,-3.5) ;
  \draw (1.25,1) -- (1.25,-3.5);
  \draw (-1.25,1) -- (-1.25,-3.5);

  %MESSAGES
  \draw[>=latex,->, dashed] (-1.25, 0.5) -- (0, 0.5) node[ above, midway] {$\amessage_1$};
  \draw[>=latex,->] (0, 0) -- (-1.25, 0) node[ above, midway] {$\amessage_2$};


  \draw[>=latex,->] (0, -0.5) -- (1.25, -1.75) node[pos=0.1, sloped, above] {$\amessage_3$};
  \draw[>=latex,->] (0, -1.25) -- (1.25, -2.5) node[pos=0.1, sloped, above] {$\amessage_3$}; %{$\amessage_1'$};
  %\draw[>=latex,->, dashed] (0, -2.5) -- (1.25, -3.25) node[pos=0.55, sloped, above] {$\amessage_1''$};

  \draw[>=latex,->] (1.25, -0.5) -- (0, -2.2) node[pos=0.1, sloped, above] {$\amessage_4$};
%  \draw[>=latex,->] (1.25, -1.25) -- (0, -2.5) node[pos=0.55, sloped, above] {}; %{$\amessage_2'$};
%  \node[rotate = 90, left]at (1.13, -0.65) {$\cdots$};
  %\node[rotate = -90, right]at (0.1, -0.65) {$\cdots$};

  \draw[>=latex,->] (1.25, -3) -- (0, -3) node[ above, midway] {$\amessage_5$};


\end{scope}

\end{tikzpicture}
\captionof{figure}{MSC $\mscWexist$}
\label{fig:msc_W_exist}
\end{center}

  \end{minipage}
  \end{center}
  \begin{itemize}
      \item \wks{k} and \eb{} (but not \ub{}): $\systemweakexist$ in Fig.~\ref{fig:system_weak_exist} is \wks{1} and \ekb{1} but repetitions of $\msg_3$ prevent it from being \ub{};
  \end{itemize}
  \begin{center}
    \begin{minipage}[c]{6cm}
      \input{Appendix-Sec5/system_weak_exist}
  \end{minipage}
  \hspace*{1cm}
  \begin{minipage}[c]{3.5cm}
    \input{Appendix-Sec5/msc_weak_exist}
  \end{minipage}
  \end{center}
\begin{itemize}
  \item \wks{k} and \ub{} as $\systemuniver$ in Fig.~\ref{fig:system_weak_univer}.

\end{itemize}







Now, focus on the strong classes. We know by
Proposition~\ref{proposition:synchro_in_exists} that a \sks{k} is \eb{}. So let
compare \sks{k} systems with \ub{} ones.  We can see that there are incomparable.
$\systemstrongexist$ in Fig.~\ref{fig:system_strong_exist} is \sks{1}, as we can
see with $\mscstrongexist$ in Fig.~\ref{fig:msc_strong_exist}, but cannot be \ub{} as
the unbounded iterations of $\msg_1$ can be stored before begin to read.
As we see before $\System_7$ is only \ub{} and cannot have any synchronizable property. But, $\systemstronguniver$ in Fig.~\ref{fig:system_strong_univer} is both \sks{1} and \ukb{1}.

% \begin{center}
%   % \begin{minipage}[c]{6cm}
%
% % \end{minipage}
% % \hspace*{1cm}
% % \begin{minipage}[c]{3.5cm}
% %   \begin{center}

\begin{tikzpicture}[>=stealth,node distance=3.4cm,shorten >=1pt,
    every state/.style={text=black, scale =0.7}, semithick,
      font={\fontsize{8pt}{12}\selectfont}]

      \begin{scope}[shift = {(3.5,0.75)}, scale = 0.8]
           %\draw (0.5, -4) node{\textbf{(b)}};
          %MACHINES
          \draw (0,0) node{$p$} ;
          \draw (1,0) node{$q$} ;
          \draw (0,-0.25) -- (0,-3.5) ;
          \draw (1,-0.25) -- (1,-3.5);
          %MESSAGES

           \draw[>=latex,->] (0, -0.75) -- (1, -0.75) node[midway, above] {$\amessage_1$};
           \draw[>=latex,->] (0, -1.75) -- (1, -1.75) node[midway, above] {$\amessage_1$};
           \draw[>=latex,->] (0, -2.75) -- (1, -2.75) node[midway, above] {$\amessage_1$};

           %\draw (0.5, -3.25) node{$\cdots$};
      \end{scope}

\end{tikzpicture}
\captionof{figure}{MSC $\mscstrongexist$}
\label{fig:msc_strong_exist}
\end{center}

% % \end{minipage}
% \end{center}


% \begin{center}
%   \begin{center}

  \begin{tikzpicture}[>=stealth,node distance=3.4cm,shorten >=1pt,
      every state/.style={text=black, scale =0.65}, semithick,
        font={\fontsize{8pt}{12}\selectfont}]


      \begin{scope}[->, shift={(0,0)}]
          \node[state,initial,initial text={}] (q0)  {$\ell_p^0$};
           \path (q0) edge [loop right] node [right]   {$~\send{p}{q}{\msg_1}$}(q0);
          \node[rectangle, thick, draw] at (-0.6,0.6) {$A_p$};
      \end{scope}

     \begin{scope}[->, shift ={(4, 0)}]
        \node[state,initial,initial text={}] (q0)  {$\ell_q^0$};
        \path (q0) edge [loop right] node [right]   {$~\rec{p}{q}{\msg_1}$}(q0);
        \node[rectangle, thick, draw] at (-0.6,0.6) {$A_q$};
     \end{scope}
\end{tikzpicture}
\captionof{figure}{System $\systemstrongexist$}
\label{fig:system_strong_exist}
\end{center}

% \end{center}

\begin{center}
  \begin{minipage}[c]{8cm}
    \begin{center}

  \begin{tikzpicture}[>=stealth,node distance=3.4cm,shorten >=1pt,
      every state/.style={text=black, scale =0.65}, semithick,
        font={\fontsize{8pt}{12}\selectfont}]


      \begin{scope}[->, shift={(0,0)}]
          \node[state,initial,initial text={}] (q0)  {$\ell_p^0$};
           \path (q0) edge [loop right] node [right]   {$~\send{p}{q}{\msg_1}$}(q0);
          \node[rectangle, thick, draw] at (-0.6,0.6) {$A_p$};
      \end{scope}

     \begin{scope}[->, shift ={(4, 0)}]
        \node[state,initial,initial text={}] (q0)  {$\ell_q^0$};
        \path (q0) edge [loop right] node [right]   {$~\rec{p}{q}{\msg_1}$}(q0);
        \node[rectangle, thick, draw] at (-0.6,0.6) {$A_q$};
     \end{scope}
\end{tikzpicture}
\captionof{figure}{System $\systemstrongexist$}
\label{fig:system_strong_exist}
\end{center}

    
\begin{center}
\begin{tikzpicture}[>=stealth,node distance=3.4cm,shorten >=1pt,
    every state/.style={text=black, scale =0.7}, semithick,
      font={\fontsize{8pt}{12}\selectfont}]
      \begin{scope}[->, shift={(0,0)}]
          \node[state,initial,initial text={}] (q0)  {$\ell_p^0$};
          \node[state, right of=q0] (q1)  {$\ell_p^1$};

          \path (q0) edge[bend left = 10] node [above] {$\send{p}{q}{\msg_1}$} (q1);
          \path (q1) edge[bend left = 10] node [below] {$\rec{q}{r}{\msg_2}$} (q0);

          \node[rectangle, thick, draw] at (-0.7,0.6) {$A_p$};
      \end{scope}
      \begin{scope}[->, shift={(4,0)}]
  	      \node[state,initial,initial text={}] (q0)  {$\ell_q^0$};
  				\node[state, right of=q0] (q1)  {$\ell_q^1$};

  				\path (q0) edge[bend left = 10] node [above] {$\rec{p}{q}{\msg_1}$} (q1);
          \path (q1) edge[bend left = 10] node [below] {$\send{q}{p}{\msg_2}$} (q0);

      		\node[rectangle, thick, draw] at (-0.7,0.6) {$A_q$};
  	  \end{scope}
      \end{tikzpicture}
\captionof{figure}{System $\systemstronguniver$}
\label{fig:system_strong_univer}
      \end{center}

\end{minipage}
\hspace*{1cm}
\begin{minipage}[c]{3.5cm}
  



      \begin{center}
        \begin{tikzpicture}[>=stealth,node distance=3.4cm,shorten >=1pt,
            every state/.style={text=black, scale =0.7}, semithick,
              font={\fontsize{8pt}{12}\selectfont}]

      \begin{scope}[shift = {(10.5,0.75)}, scale = 0.8]
        % \draw (0.5, -4) node{\textbf{(b)}};
        %MACHINES
        \draw (0,0) node{$p$} ;
        \draw (1,0) node{$q$} ;
        \draw (0,-0.25) -- (0,-3.5) ;
        \draw (1,-0.25) -- (1,-3.5);
        %MESSAGES

         \draw[>=latex,->] (0, -0.75) -- (1, -0.75) node[midway, above] {$\amessage_1$};
         \draw[>=latex,->] (1, -1.5) -- (0, -1.5) node[midway, above] {$\amessage_2$};
         \draw[>=latex,->] (0, -2.25) -- (1, -2.25) node[midway, above] {$\amessage_1$};
         \draw[>=latex,->] (1, -3) -- (0, -3) node[midway, above] {$\amessage_2$};


         %\draw (0.5, -3.25) node{$\cdots$};


      \end{scope}


              \end{tikzpicture}
              \captionof{figure}{MSC $\mscstronguniver$}
              \label{fig:msc_strong_univer}
            \end{center}

\end{minipage}
\end{center}


Similarly, \sS{} systems are incomparable with \eb{} ones.
$\systemweakS$ in Fig.~\ref{fig:system_weak_S} is \sS{} (and also \wks{1})
but, to get an execution from $\mscweakS$ we need to send message $\msg_5$ before $\msg_1$ and, as we see with $\msc_6$ previously, the number of iterations of $\msg_2$ becomes the bound of the channels for this execution. As it is unbounded, the system cannot be \eb{}.

\begin{center}
  \begin{minipage}[c]{8cm}
    

\begin{center}
\begin{tikzpicture}[>=stealth,node distance=3.4cm,shorten >=1pt,
    every state/.style={text=black, scale =0.6}, semithick,
      font={\fontsize{8pt}{12}\selectfont}]
      \begin{scope}[->, shift={(7,0)}]
          \node[state,initial,initial text={}] (q0)  {$\ell_p^0$};
          \node[state, right of=q0] (q1)  {$\ell_p^1$};

          \path (q0) edge [loop above] node [right]   {$\send{p}{q}{\msg_2}$}(q0);
          \path (q0) edge node [above] {$\rec{s}{p}{\msg_4}$} (q1);

          \node[rectangle, thick, draw] at (-0.7,0.6) {$A_p$};
      \end{scope}
      \begin{scope}[->, shift={(7,-1.25)}]
  	      \node[state,initial,initial text={}] (q0)  {$\ell_q^0$};
  				\node[state, right of=q0] (q1)  {$\ell_q^1$};
          \node[state, right of=q1] (q2)  {$\ell_q^2$};

          \path (q0) edge node [above] {$\send{q}{r}{\msg_1}$} (q1);
          \path (q1) edge [loop above] node [right]   {$\rec{p}{q}{\msg_2}$}(q1);
          \path (q1) edge node [above] {$\rec{s}{q}{\msg_3}$} (q2);

      		\node[rectangle, thick, draw] at (-0.7,0.6) {$A_q$};
  	  \end{scope}

      \begin{scope}[->, shift={(7,-2.5)}]
          \node[state,initial,initial text={}] (q0)  {$\ell_q^0$};
          \node[state, right of=q0] (q1)  {$\ell_q^1$};

          \path (q0) edge node [above] {$\rec{s}{r}{\msg_5}$} (q1);

          \node[rectangle, thick, draw] at (-0.7,0.6) {$A_r$};
      \end{scope}
      \begin{scope}[->, shift={(7,-3.75)}]
          \node[state,initial,initial text={}] (q0)  {$\ell_q^0$};
          \node[state, right of=q0] (q1)  {$\ell_q^1$};
          \node[state, right of=q1] (q2)  {$\ell_q^2$};
          \node[state, right of=q2] (q3)  {$\ell_q^3$};

          \path (q0) edge node [above] {$\send{s}{q}{\msg_3}$} (q1);
          \path (q1) edge node [above] {$\send{s}{p}{\msg_4}$} (q2);
          \path (q2) edge node [above] {$\send{s}{r}{\msg_5}$} (q3);

          \node[rectangle, thick, draw] at (-0.7,0.6) {$A_s$};
      \end{scope}
      \end{tikzpicture}
\captionof{figure}{System $\systemweakS$}
\label{fig:system_weak_S}
      \end{center}

\end{minipage}
\hspace*{1cm}
\begin{minipage}[c]{3.5cm}
  \begin{center}


\begin{tikzpicture}[>=stealth,node distance=3.4cm,shorten >=1pt,
    every state/.style={text=black, scale =0.7}, semithick,
      font={\fontsize{8pt}{12}\selectfont}]

\begin{scope}[shift = {(8.5,0)}, scale = 0.8]

  %MACHINES
  \draw (0,0) node{$p$} ;
  \draw (1,0) node{$q$} ;
  \draw (2,0) node{$r$} ;
  \draw (3,0) node{$s$} ;
  \draw (0,-0.25) -- (0,-4.7) ;
  \draw (1,-0.25) -- (1,-4.7);
  \draw (2, -0.25) -- (2, -4.7) ;
  \draw (3, -0.25) -- (3, -4.7) ;

  %MESSAGES
  \draw[>=latex,->, dashed] (1,-0.7) -- (2, -0.7) node[midway,above]{$\amessage_1$};

  \draw[>=latex,->] (0, -1.4) -- (1, -1.4) node[midway, above] {$\amessage_2$};

  \draw[>=latex,->] (0,-2.1) -- (1,-2.1) node[midway, above] {$\amessage_2$};

  \draw[>=latex,->] (3,-2.8) -- (1,-2.8) node[pos = 0.3,above] {$\amessage_3$};

  \draw[>=latex,->] (3,-3.5) -- (0,-3.5) node[midway,above] {$\amessage_4$};

  \draw[>=latex,->] (3,-4.2) -- (2,-4.2) node[midway,above] {$\amessage_5$};

%  \draw[>=latex,->] (2,-3.25) -- (3,-3.25) node[midway,above] {$\amessage_5$};

  %\draw (0.5, -1.55) node{$\cdots$};

\end{scope}
\end{tikzpicture}
\captionof{figure}{MSC $\msc_{13}$ }
\label{fig:msc_weak_S}
\end{center}

\end{minipage}
\end{center}


$\systemWexist$ in Fig.~\ref{fig:system_W_exist} is \eb{} and \wS{}, but, again, message $\msg_5$ have to be sent before $\msg_1$, so all messages in $\mscWexist$ have to be in the same exchange. However, on process $r$, receptions of $\msg_3$ precedes send of $\msg_5$ and so this is not an exchange, and the system is not \sS{}.

%\input{system_W_exist}

For the intersection, as with weak, a \sS{} and \eb{} systems is always \sks{k} for a $k$, by Proposition~\ref{proposition:weak_univer_uweak}. We can have for example, $\systemW$ which is \sS{}, \ub{}, \wks{1}
 but, as an exchange have to contain all messages that we have in $\msc_3$, and we can repeat $\msg_2$ as many times we want, an exchange in mailbox is no bounded, and the system is not \sks{k} for any $k$.
 We can also have $\systemWSexist$ in Fig.~\ref{fig:system_W_S_exist} which is \sS{} and \eb{} but, as one execution is always an exchange, and it can grow without limits, we cannot be neither \sks{k} nor \wks{k} for any $k$.

 %\begin{center}
\begin{tikzpicture}[>=stealth,node distance=3.4cm,shorten >=1pt,
    every state/.style={text=black, scale =0.7}, semithick,
      font={\fontsize{8pt}{12}\selectfont}]
      \begin{scope}[->, shift={(0,0)}]
          \node[state,initial,initial text={}] (q0)  {$\ell_p^0$};
          \node[state, right of=q0] (q1)  {$\ell_p^1$};

          \path (q0) edge [loop above] node [right]   {$~\send{p}{q}{\msg_1}$}(q0);
          \path (q0) edge node [below] {$\rec{q}{r}{\msg_2}$} (q1);

          \node[rectangle, thick, draw] at (-0.7,0.6) {$A_p$};
      \end{scope}
      \begin{scope}[->, shift={(0,-1.5)}]
  	      \node[state,initial,initial text={}] (q0)  {$\ell_q^0$};
  				\node[state, right of=q0] (q1)  {$\ell_q^1$};

  				\path (q0) edge node [above] {$\send{q}{r}{\msg_2}$} (q1);
          \path (q1) edge [loop right] node [right]   {$\rec{p}{q}{\msg_1}~$}(q1);


      		\node[rectangle, thick, draw] at (-0.7,0.6) {$A_q$};
  	  \end{scope}
      \end{tikzpicture}
\captionof{figure}{System $\systemWSexist$}
\label{fig:system_W_S_exist}
      \end{center}



%, and weakly synchronizable systems become incomparable with respect to existentially bounded.

%\input{diagram-mailbox-classes}

%Proposition \ref{theorem:synchro_in_exists_p2p} needs to be adapted to mailbox semantics.

%
%
% \begin{restatable}{proposition}{stronginexistsmailbox}\label{theorem:strong_in_exists_mailbox}
% 	Every \skSous{k} mailbox MSC is existentially $k$-mailbox-bounded.
% \end{restatable}
% ---------------------OLD VERSION
%
%
% By definition, \sS mailbox systems are \wS, as each MSC can be divided into exchange.
% In the same way, any \sks{k} mailbox system is \wks{k}, as each MSC can be decomposed into \kE{k}s. Note that if a system is \sks{k} with the smallest possible $k$, it can be \wks{k'} with  $k'<k$.
% However, this  is a proper inclusion. Indeed  system $\System_2$ in Fig.~\ref{fig:system_weak_exist} (when we consider the mailbox semantics)  is \wks{1} but not \sks{k} for any $k$. Messages $\msg_1$ and $\msg_3$ are constrained to be in the same \kE{k} but as we can have an unbounded number of sends and reception of $\msg_2$, the size of the \kE{k} is unbounded as well. Notice that this system is also \ekb{1}.
%
% %MSC $M_2$ in Fig.~\ref{fig:msc_weak_exist} is strongly-$3$-synchronous. But, if we add an iteration of message $\msg_2$, the corresponding MSC is strongly-$4$-synchronous. Finally, the size of the \kE{k}s is not bounded.
%
%
% Similarly, any \ukb{k} mailbox system is also \ekb{k} as all linearizations are bounded. However, the bound can  be smaller and so the system can sometimes be \ekb{k'} with $k'<k$.
% This is also a proper inclusion. Take, for instance, system $\System_5$ in Fig.~\ref{fig:system_exist}, %and an example of MSC with $M_5$ in Fig.~\ref{fig:msc_exist}
% which %, if we imagine an unbounded number of repetions of $\msg_3$,
% is \ekb{1} but not \ukb{k} for any $k$, as channel $c_r$  can be unbounded  because of the loops for  $\send{q}{r}{\msg_3}$.
% 


%\begin{figure}
  \begin{center}
      \begin{tikzpicture}[>=stealth,node distance=3.4cm,shorten >=1pt,
      every state/.style={text=black, scale =0.65}, semithick,
        font={\fontsize{8pt}{12}\selectfont}]
	% %P
  \begin{scope}[->]
     \node[state,initial,initial text={}] (q0)  {$\ell_p^0$};
     \node[state, right of=q0] (q1)  {$\ell_p^1$};
     \node[state, right of=q1] (q2) {$\ell_p^2$};

     \path (q0) edge node [above] {$\send{p}{q}{\msg_1}$} (q1);
     \path (q1) edge[bend left = 10] node [above] {$\send{p}{q}{\msg_1}$}(q2);
     \path (q2) edge[bend left = 10] node [below] {$\rec{q}{p}{\msg_2}$}(q1);

     \node[rectangle, thick, draw] at (-0.6,0.5) {$A_p$};
 \end{scope}

 \begin{scope}[->, shift={(6,0)}]
     \node[state,initial,initial text={}] (q0)  {$\ell_q^0$};
     \node[state, right of=q0] (q1)  {$\ell_q^1$};

     \path (q0) edge[bend left = 10] node [above] {$\send{q}{p}{\msg_2}$} (q1);
      \path (q1) edge[bend left = 10] node [below] {$\rec{p}{q}{\msg_1}$} (q0);
      \path (q0) edge [loop below] node [below]   {$~\send{q}{r}{\msg_3}$}(q0);
      \path (q1) edge [loop below] node [below]   {$~\send{q}{r}{\msg_3}$}(q1);


     \node[rectangle, thick, draw] at (-0.6,0.5) {$A_q$};
 \end{scope}

\begin{scope}[->, shift ={(11, 0)}]
   \node[state,initial,initial text={}] (q0)  {$\ell_r^0$};
   \path (q0) edge [loop below] node [below]   {$~\rec{q}{r}{\msg_3}$}(q0);
   \node[rectangle, thick, draw] at (-0.6,0.5) {$A_r$};

\end{scope}


		% \begin{scope}[shift = {(7.5,0.5)}, scale = 0.8]
    %   \draw (1.25, -4.75) node{\textbf{(b existe plus haut)}};
    %   \draw (-8, -4.75) node{\textbf{(a)}};
    %
		% 	%MACHINES
		% 	\draw (0,0) node{$p$} ;
		% 	\draw (1.25,0) node{$q$} ;
    %   \draw (2.5,0) node{$r$} ;
		% 	\draw (2.5, -0.25) -- (2.5, -3.75) ;
		% 	\draw (0,-0.25) -- (0,-3.75) ;
		% 	\draw (1.25,-0.25) -- (1.25,-3.75);
		% 	%MESSAGES
    %
    %   \draw[>=latex,->] (0, -0.5) -- (1.25, -1.25) node[pos=0.4, sloped, above] {$\amessage_1$};
    %   \draw[>=latex,->] (0, -1.5) -- (1.25, -2.25) node[pos=0.5, sloped, above] {$\amessage_1$};
    %   \draw[>=latex,->] (0, -2.5) -- (1.25, -3.25) node[pos=0.65, sloped, above] {$\amessage_1$};
    %
    %   \draw[>=latex,->] (1.25, -0.75) -- (0, -2) node[pos=0.5, sloped, above] {$\amessage_2$};
    %   \draw[>=latex,->] (1.25, -1.75) -- (0, -3) node[pos=0.5, sloped, above] {$\amessage_2$};
    %
    %   \draw[>=latex,->] (1.25, -1) -- (2.5, -1) node[midway, above] {$\amessage_3$};
    %   \draw[>=latex,->] (1.25, -1.5) -- (2.5, -1.5) node[midway, above] {$\amessage_3$};
    %   \draw[>=latex,->] (1.25, -2.5) -- (2.5, -2.5) node[midway, above] {$\amessage_3$};
    %
    %   \draw (0.6, -3.35) node{$\cdots$};
    %   \draw (1.9, -3.35) node{$\cdots$};
    %
    %
		% \end{scope}
\end{tikzpicture}
\captionof{figure}{System $\systemexist$}
\label{fig:system_exist}

\end{center}

%\end{figure}

%
%
%
% The class of weakly synchronizable systems is incomparable with the class of existentially bounded. Indeed,   system $\System_6$ below is \wks{1} but  not \ekb{k} for any $k$. As exemplified in Fig. \ref{fig:msc_weak}, MSCs can be  divided into \kE{1}s.  But when we linearize them, $\msg_4$ has to be sent before $\msg_1$ and  all messages $\msg_2$ have to be sent before  $\msg_4$. The reception of  $\msg_2$ can start only when $\msg_1$ has been sent. Hence  we can have an unbounded number of $\msg_2$ and therefore the size of channel $c_q$ is also unbounded.
% \begin{center}
%  \noindent
%  \begin{minipage}[b]{7.75cm}
% 
  \begin{center}

  \begin{tikzpicture}[>=stealth,node distance=3.4cm,shorten >=1pt,
      every state/.style={text=black, scale =0.65}, semithick,
        font={\fontsize{8pt}{12}\selectfont}]

        \begin{scope}[-> ]
            \node[state,initial,initial text={}] (q0)  {$\ell_p^0$};
            \node[state, right of=q0] (q1)  {$\ell_p^1$};
            \node[state, right of=q1] (q2) {$\ell_p^2$};

            \path (q0) edge node [above] {$\send{p}{q}{\msg_2}$} (q1);
            \path (q1) edge [loop above] node [right]   {$~\send{p}{q}{\msg_2}$}(q1);
            \path (q1) edge node [above] {$\send{p}{s}{\msg_3}$}(q2);
            \node[rectangle, thick, draw] at (-0.6,0.6) {$A_p$};
        \end{scope}


      \begin{scope}[->,  shift={(0,-1.1)}]
          \node[state,initial,initial text={}] (q0)  {$\ell_q^0$};
          \node[state, right of=q0] (q1)  {$\ell_q^1$};
          \node[state, right of=q1] (q2) {$\ell_q^2$};

          \path (q0) edge node [above] {$\send{q}{r}{\msg_1}$} (q1);
          \path (q1) edge node [above] {$\rec{p}{q}{\msg_2}$}(q2);
          \path (q2) edge [loop above] node [right]   {$~\rec{p}{q}{\msg_2}$}(q2);

          \node[rectangle, thick, draw] at (-0.6,0.6) {$A_q$};
      \end{scope}



      \begin{scope}[->, shift={(0,-2.2)} ]
          \node[state,initial,initial text={}] (q0)  {$\ell_r^0$};
          \node[state, right of=q0] (q1)  {$\ell_r^1$};

          \path (q0) edge node [above] {$\rec{s}{r}{\msg_4}$} (q1);

          \node[rectangle, thick, draw] at (-0.6,0.6) {$A_r$};
        %\node at (1, -1.5) {\textbf{(a)}};

      \end{scope}
      \begin{scope}[->, shift ={(0, -3.3)}]
         \node[state,initial,initial text={}] (q0)  {$\ell_s^0$};
         \node[state, right of=q0] (q1)  {$\ell_s^1$};
         \node[state, right of=q1] (q2) {$\ell_s^2$};

         \path (q0) edge node [above] {$\rec{p}{s}{\msg_3}$} (q1);
         \path (q1) edge node [above] {$\send{s}{r}{\msg_4}$} (q2);

         \node[rectangle, thick, draw] at (-0.6,0.6) {$A_s$};

      \end{scope}
      % \begin{scope}[shift = {(8.5,0)}, scale = 0.8]
      %   \draw (1.5, -4.75) node{\textbf{(b)}};
      %   \draw (-9.5, -4.75) node{\textbf{(a)}};
      %
      %   %MACHINES
      %   \draw (0,0) node{$p$} ;
    	% 	\draw (1,0) node{$q$} ;
    	% 	\draw (2,0) node{$r$} ;
      %   \draw (3,0) node{$s$} ;
    	% 	\draw (0,-0.25) -- (0,-3.75) ;
    	% 	\draw (1,-0.25) -- (1,-3.75);
    	% 	\draw (2, -0.25) -- (2, -3.75) ;
      %   \draw (3, -0.25) -- (3, -3.75) ;
      %
    	% 	%MESSAGES
    	% 	\draw[>=latex,->, dashed] (1,-0.7) -- (2, -0.7) node[midway,above]{$\amessage_1$};
      %
    	% 	\draw[>=latex,->] (0, -1.35) -- (1, -1.35) node[midway, above] {$\amessage_2$};
      %
    	% 	\draw[>=latex,->] (0,-2.1) -- (1,-2.1) node[midway, above] {$\amessage_2$};
      %
    	% 	\draw[>=latex,->] (0,-2.75) -- (3,-2.75) node[midway,above] {$\amessage_3$};
      %
      %   \draw[>=latex,->] (3,-3.4) -- (2,-3.4) node[midway,above] {$\amessage_4$};
      %
      % %  \draw[>=latex,->] (2,-3.25) -- (3,-3.25) node[midway,above] {$\amessage_5$};
      %
      %   \draw (0.5, -1.55) node{$\cdots$};
      %
      % \end{scope}


  \end{tikzpicture}
  \captionof{figure}{System $\systemweak$ }
  \label{fig:system_weak}

\end{center}

% \end{minipage}
%  \begin{minipage}[b]{4cm}
%    \begin{center}


\begin{tikzpicture}[>=stealth,node distance=3.4cm,shorten >=1pt,
    every state/.style={text=black, scale =0.7}, semithick,
      font={\fontsize{8pt}{12}\selectfont}]

\begin{scope}[shift = {(8.5,0)}, scale = 0.8]

  %MACHINES
  \draw (0,0.3) node{$p$} ;
  \draw (1,0.3) node{$q$} ;
  \draw (2,0.3) node{$r$} ;
  \draw (3,0.3) node{$s$} ;
  \draw (0,0.05) -- (0,-4) ;
  \draw (1,0.05) -- (1,-4);
  \draw (2, 0.05) -- (2, -4) ;
  \draw (3, 0.05) -- (3, -4) ;

  %MESSAGES
  \draw[>=latex,->, dashed] (1,-0.7) -- (2, -0.7) node[midway,above]{$\amessage_1$};

  \draw[>=latex,->] (0, -1.35) -- (1, -1.35) node[midway, above] {$\amessage_2$};

  \draw[>=latex,->] (0,-2.1) -- (1,-2.1) node[midway, above] {$\amessage_2$};

  \draw[>=latex,->] (0,-2.75) -- (3,-2.75) node[midway,above] {$\amessage_3$};

  \draw[>=latex,->] (3,-3.4) -- (2,-3.4) node[midway,above] {$\amessage_4$};

%  \draw[>=latex,->] (2,-3.25) -- (3,-3.25) node[midway,above] {$\amessage_5$};

  %\draw (0.5, -1.55) node{$\cdots$};

\end{scope}
\end{tikzpicture}
\captionof{figure}{MSC $\mscweak$ }
\label{fig:msc_weak}
\end{center}

%  \end{minipage}
%
%  \end{center}
%
%
% As for \pp semantics,  universally bounded systems are incomparable with the weakly synchronizable ones.
% There are systems in both classes like $\System_1$ in Fig.~\ref{fig:system_weak_univer} which is \ukb{1} and \wks{1}. Some systems are \ukb{k} but not \wks{k'} for any $k'$ as $\System_3$ in Fig.~\ref{fig:system_univer}:  notice that a system, with two machines only, behaves in the same way in the \pp and mailbox semantics, thus the discussion presented above works for the mailbox semantics as well.
% %The MSC $M_3$ in Fig.~\ref{fig:system_univer} shows a possible behaviour of this system which cannot be divided into \kE{k}s, all messages of the execution have to be in the same \kE{k} but we cannot put all sends before all receptions. On the other hand, we see that each executions is at most $2$-bounded, as we need to read to send an other message.
% System $\System_4$ below is \wks{1} (and \sks{1}) but not \ukb{k} for any $k$. Indeed, channel $c_p$ can be filled by an unbounded number of messages $\msg_1$.
% \begin{center}

  \begin{tikzpicture}[>=stealth,node distance=3.4cm,shorten >=1pt,
      every state/.style={text=black, scale =0.65}, semithick,
        font={\fontsize{8pt}{12}\selectfont}]


      \begin{scope}[->, shift={(0,0)}]
          \node[state,initial,initial text={}] (q0)  {$\ell_p^0$};
           \path (q0) edge [loop right] node [right]   {$~\send{p}{q}{\msg_1}$}(q0);
          \node[rectangle, thick, draw] at (-0.6,0.6) {$A_p$};
      \end{scope}

     \begin{scope}[->, shift ={(4, 0)}]
        \node[state,initial,initial text={}] (q0)  {$\ell_q^0$};
        \path (q0) edge [loop right] node [right]   {$~\rec{p}{q}{\msg_1}$}(q0);
        \node[rectangle, thick, draw] at (-0.6,0.6) {$A_q$};
     \end{scope}
\end{tikzpicture}
\captionof{figure}{System $\systemstrongexist$}
\label{fig:system_strong_exist}
\end{center}

% The class of universally bounded systems is also incomparable with the classes of strongly synchronizable systems.
% As for \pp, system $\System_3$ is not \sks{k}  for any $k$ but  it is  \ukb{3}.  Still the properties are not incompatible,  for instance, system $\System_7$ below  is both \ukb{1} and \sks{1}.
% \begin{center}
%  \noindent
%  \begin{minipage}[b]{5cm}
% 
\begin{center}
\begin{tikzpicture}[>=stealth,node distance=3.4cm,shorten >=1pt,
    every state/.style={text=black, scale =0.7}, semithick,
      font={\fontsize{8pt}{12}\selectfont}]
      \begin{scope}[->, shift={(0,0)}]
          \node[state,initial,initial text={}] (q0)  {$\ell_p^0$};
          \node[state, right of=q0] (q1)  {$\ell_p^1$};

          \path (q0) edge[bend left = 10] node [above] {$\send{p}{q}{\msg_1}$} (q1);
          \path (q1) edge[bend left = 10] node [below] {$\rec{q}{r}{\msg_2}$} (q0);

          \node[rectangle, thick, draw] at (-0.7,0.6) {$A_p$};
      \end{scope}
      \begin{scope}[->, shift={(4,0)}]
  	      \node[state,initial,initial text={}] (q0)  {$\ell_q^0$};
  				\node[state, right of=q0] (q1)  {$\ell_q^1$};

  				\path (q0) edge[bend left = 10] node [above] {$\rec{p}{q}{\msg_1}$} (q1);
          \path (q1) edge[bend left = 10] node [below] {$\send{q}{p}{\msg_2}$} (q0);

      		\node[rectangle, thick, draw] at (-0.7,0.6) {$A_q$};
  	  \end{scope}
      \end{tikzpicture}
\captionof{figure}{System $\systemstronguniver$}
\label{fig:system_strong_univer}
      \end{center}

%  \end{minipage}
%  \begin{minipage}[b]{5cm}
% 



      \begin{center}
        \begin{tikzpicture}[>=stealth,node distance=3.4cm,shorten >=1pt,
            every state/.style={text=black, scale =0.7}, semithick,
              font={\fontsize{8pt}{12}\selectfont}]

      \begin{scope}[shift = {(10.5,0.75)}, scale = 0.8]
        % \draw (0.5, -4) node{\textbf{(b)}};
        %MACHINES
        \draw (0,0) node{$p$} ;
        \draw (1,0) node{$q$} ;
        \draw (0,-0.25) -- (0,-3.5) ;
        \draw (1,-0.25) -- (1,-3.5);
        %MESSAGES

         \draw[>=latex,->] (0, -0.75) -- (1, -0.75) node[midway, above] {$\amessage_1$};
         \draw[>=latex,->] (1, -1.5) -- (0, -1.5) node[midway, above] {$\amessage_2$};
         \draw[>=latex,->] (0, -2.25) -- (1, -2.25) node[midway, above] {$\amessage_1$};
         \draw[>=latex,->] (1, -3) -- (0, -3) node[midway, above] {$\amessage_2$};


         %\draw (0.5, -3.25) node{$\cdots$};


      \end{scope}


              \end{tikzpicture}
              \captionof{figure}{MSC $\mscstronguniver$}
              \label{fig:msc_strong_univer}
            \end{center}

%  \end{minipage}
%  \end{center}


\end{document}
